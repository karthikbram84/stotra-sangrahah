% !TeX program = XeLaTeX
% !TeX root = stotramanjari-balapatha.tex
\makeatletter
\def\@cline#1-#2\@nil{%
  \omit
  \@multicnt#1%
  \advance\@multispan\m@ne
  \ifnum\@multicnt=\@ne\@firstofone{&\omit}\fi
  \@multicnt#2%
  \advance\@multicnt-#1%
  \advance\@multispan\@ne
  \leaders\hrule\@height\arrayrulewidth\hfill
  \cr
  \noalign{\nobreak\vskip-\arrayrulewidth}}
\makeatother
\newcommand{\dngfamily}{\fontspec[Script=Devanagari,Scale=1,AutoFakeBold=2]{Sanskrit 2003}}
\newcommand{\tamfamily}{\fontspec[Script=Tamil,FakeStretch=0.8]{Noto Serif Tamil}}
\newcommand{\tam}[1]{%
{\fontspec[Script=Tamil,FakeStretch=0.9,Scale=0.8]{Noto Serif Tamil}  #1}}
\setlength{\columnseprule}{0pt}
\renewcommand{\labelenumi}{\devanumber\theenumi.}
% \fontsize{15}{19}\selectfont
\sect{संवत्सराः षष्टिः}
\twolineshloka
{प्रभवो विभवः शुक्लः प्रमोदोऽथ प्रजापतिः}
{अङ्गिराः श्रीमुखो भावो युवा धाता तथैव च}

\twolineshloka
{ईश्वरो बहुधान्यश्च प्रमाथी विक्रमो वृषः}
{चित्रभानुः सुभानुश्च तारणः पार्थिवो व्ययः}

\twolineshloka
{सर्वजित्सर्वधारी च विरोधी विकृतिः खरः}
{नन्दनो विजयश्चैव जयो मन्मथदुर्मुखौ}

\twolineshloka
{हेमलम्बो विलम्बोऽथ विकारी शार्वरी प्लवः}
{शुभकृच्छोभनः क्रोधी विश्वावसुपराभवौ}

\twolineshloka
{प्लवङ्गः कीलकः सौम्यः साधारणविरोधिकृत्}
{परिधावी प्रमादी च आनन्दो राक्षसो नलः}

\twolineshloka
{पिङ्गलः कालयुक्तश्च सिद्धार्थी रौद्रदुर्मती}
{दुन्दुभी रुधिरोद्गारी रक्ताक्षी क्रोधनः क्षयः}

\begin{multicols}{3}
  \begin{enumerate}\itemsep-1ex
    \item प्रभवः
    \item विभवः
    \item शुक्लः
    \item प्रमोदः
    \item प्रजापतिः
    \item अङ्गिराः
    \item श्रीमुखः
    \item भावः
    \item युवा
    \item धाता
    \item ईश्वरः
    \item बहुधान्यः
    \item प्रमाथी
    \item विक्रमः
    \item वृषः
    \item चित्रभानुः
    \item सुभानुः
    \item तारणः
    \item पार्थिवः
    \item व्ययः
    \item सर्वजित्
    \item सर्वधारी
    \item विरोधी
    \item विकृतिः
    \item खरः
    \item नन्दनः
    \item विजयः
    \item जयः
    \item मन्मथः
    \item दुर्मुखः
    \item हेमलम्बः
    \item विलम्बः
    \item विकारी
    \item शार्वरी
    \item प्लवः
    \item शुभकृत्
    \item शोभनः
    \item क्रोधी
    \item विश्वावसुः
    \item पराभवः
    \item प्लवङ्गः
    \item कीलकः
    \item सौम्यः
    \item साधारणः
    \item विरोधिकृत्
    \item परिधावी/ परितापी
    \item प्रमादी
    \item आनन्दः
    \item राक्षसः
    \item नलः
    \item पिङ्गलः
    \item कालयुक्तिः
    \item सिद्धार्थी
    \item रौद्रः
    \item दुर्मतिः
    \item दुन्दुभिः
    \item रुधिरोद्गारी
    \item रक्ताक्षः
    \item क्रोधनः
    \item क्षयः
  \end{enumerate}
\end{multicols}

\sect{अयने द्वे}
१. उत्तरम् (उत्तरायणम्)\hspace{2em}
२. दक्षिणम् (दक्षिणायनम्)


\sect{ऋतवः षट्}

१. वसन्तः २. ग्रीष्मः
३. वर्षाः ४. शरत्
५. हेमन्तः ६. शिशिरः

\tamfamily
\sect{\tam{பருவங்கள் ஆறு}}
\nopagebreak[4]

\begin{multicols}{2}
  \renewcommand{\labelenumi}{\theenumi.}
  \begin{enumerate}\itemsep-1ex
    \item இளவேனில்காலம்
    \item  முதுவேனில்காலம்
    \item  கார்காலம்
    \item  கூதிர்காலம்
    \item  முன்பனிக்காலம்
    \item  பின்பனிக்காலம்
  \end{enumerate}
\end{multicols}

\dngfamily

\sect{मासाः द्वादश}

\begin{longtable}{rcc}
  & सौरमासाः  राशयश्च & \tam{ஸௌர மாதங்கள்} \\\endhead
  १. & मेषः              & \tam{சித்திரை}    \\
  २. & वृषभः             & \tam{வைகாசி}     \\
  ३. & मिथुनम्            & \tam{ஆனி}        \\
  ४. & कटकः             & \tam{ஆடி}        \\
  ५. & सिंहः             & \tam{ஆவணி}       \\
  ६. & कन्या             & \tam{புரட்டாசி}   \\
  ७.  & तुला              & \tam{ஐப்பசி}      \\
  ८.  & वृश्चिकः           & \tam{கார்த்திகை}   \\
  ९.  & धनुः              & \tam{மார்கழி}     \\
  १०. & मकरः             & \tam{தை}         \\
  ११. & कुम्भः             & \tam{மாசி}       \\
  १२. & मीनः             & \tam{பங்குனி}     \\
\end{longtable}

\medskip


% ऋतुमासाः चान्द्रमासाः ऋतवश्च
\nopagebreak[4]


\begin{longtable}{lllll}
  ऋतुमासाः & सौरमासाः & ऋतुः    & चान्द्रमासाः & \textsf{\normalsize Gregorian} \\
  \multicolumn{5}{c}{\small (सौरमासेन ऋतुमासस्य चान्द्रमासस्य च अन्तयोगः, आधुनिकस्य आदियोगः)}\\\hline\endhead
  मधुः     & मेषः      & \multirow{2}{*}{वसन्तः}  & चैत्रः       & \textsf{\normalsize April}    \\*
  माधवः   & वृषभः     &   & वैशाखः      & \textsf{\normalsize May}       \\\hline
  शुक्रः    & मिथुनम्    & \multirow{2}{*}{ग्रीष्मः} & ज्यैष्ठः      & \textsf{\normalsize June}    \\*
  शुचिः    & कटकः     &  & आषाढः      & \textsf{\normalsize July}      \\\hline
  नभाः    & सिंहः     & \multirow{2}{*}{वर्षाः}  & श्रावणः     & \textsf{\normalsize August}     \\*
  नभस्यः   & कन्या     &   & भाद्रपदः    & \textsf{\normalsize September} \\\hline
  इषः     & तुला      & \multirow{2}{*}{शरत्}    & आश्वयुजः     & \textsf{\normalsize October}\\*
  ऊर्जः    & वृश्चिकः   &     & कार्त्तिकः   & \textsf{\normalsize November}  \\\hline
  सहाः    & धनुः      & \multirow{2}{*}{हेमन्तः}  & मार्गशीर्षः  & \textsf{\normalsize December}  \\*
  सहस्यः   & मकरः     &   & पौषः       & \textsf{\normalsize January}   \\\hline
  तपाः    & कुम्भः     & \multirow{2}{*}{शिशिरः} & माघः       & \textsf{\normalsize February}  \\*
  तपस्यः   & मीनः     &  & फाल्गुनः     & \textsf{\normalsize March}     \\\hline
\end{longtable}

\sect{पक्षौ द्वौ}

१. शुक्लः \hspace{2em} २. कृष्णः

\sect{तिथयः पञ्चदश}

\twolineshloka*
{प्रतिपत्पूर्णिमान्तश्च शुक्लपक्षः प्रकीर्तितः}
{पूर्णिमायाः प्रतिपदश्चामावास्यान्त एव च}

\twolineshloka*
{कृष्णपक्षस्तु विज्ञेयो वेदविद्भिर्निरूपितः}
{द्वितीया च तृतीया च चतुर्थी पञ्चमी तथा}

\threelineshloka*
{षष्ठी च सप्तमी चैव ह्यष्टमी नवमी तथा}
{दशम्येकादशी चापि द्वादशी च त्रयोदशी}
{चतुर्दशी कुहूर्यावद्दिनं तु गणनं स्मृतम्}

{\fontsize{10}{4}\selectfont ---श्रीब्रह्मवैवर्त-महापुराणे श्रीकृष्णजन्मखण्डे उत्तरार्धे नारदनारायणसंवादे राधोद्धवसंवादे कालनिरूपणं नाम षण्ण्वतितमेऽध्याये श्लोकाः ६५-ः७.५}

\begin{multicols}{2}
  \begin{enumerate}\itemsep-0.8ex
    \item प्रथमा/प्रतिपत्
    \item द्वितीया
    \item तृतीया
    \item चतुर्थी
    \item पञ्चमी
    \item षष्ठी
    \item सप्तमी
    \item अष्टमी
    \item नवमी
    \item दशमी
    \item एकादशी
    \item द्वादशी
    \item त्रयोदशी
    \item चतुर्दशी
    \item पूर्णिमा
    \item अमावास्या/कुहुः
  \end{enumerate}
\end{multicols}

\sect{वासराः सप्त \tam{(கிழமைகள் ஏழு)}}

\begin{tabular}{lll@{\hspace{3ex}}lll}
  १. & भानुः  & \tam{ஞாயிறு} & ५. & गुरुः   & \tam{வியாழன்} \\
  २. & इन्दुः  & \tam{திங்கள்}  & ६. & भृगुः   & \tam{வெள்ளி}  \\
  ३. & भौमः  & \tam{செவ்வாய்} & ७. & स्थिरः & \tam{சனி}    \\
  ४. & सौम्यः & \tam{புதன்}                               \\
\end{tabular}


\sect{नक्षत्राणि सप्तविंशतिः \mbox{\tam{நக்ஷத்ரங்கள் இருபத்தேழு}}}


% \twolineshloka
% {अश्विनी भरणी चैव कृत्तिका रोहिणी मृगः}
% {आर्द्रा पुनर्वसुः पुष्यस्ततोऽश्रेषा मघास्तथा}

% \twolineshloka
% {पूर्वफाल्गुनिका तस्मादुत्तराफल्गुनी ततः}
% {हस्तश्चित्रा ततः स्वाती विशाखा तदननतरम्}

% \twolineshloka
% {अनूराधा ततो जयेष्ठा ततो मूलं निगद्यते}
% {पूर्वाषाढोतराषाढा त्वभिजिछ्रवणस्ततः}

% \twolineshloka
% {धनिष्ठा शतताराख्य पूर्वा भाद्रपदा ततः}
% {उत्तरा भाद्रपदा चैव रेवत्येतानि भानि च}

\threelineshloka*
{अश्विनी भरणी चापि कृत्तिका रोहिणी तथा}
{मृगशिरस्तथाऽऽर्द्रा च नक्षत्रे द्वे पुनर्वसू}
{पुष्याश्लेषे मघा चैव पूर्वा चोत्तरफाल्गुनी}

\twolineshloka*
{हस्तचित्रे तथा स्वाती विशाखा चानुराधिका}
{ज्येष्ठा मूलं तथा ज्ञेया पूर्वाषाढोत्तरा तथा}

\twolineshloka*
{श्रवणाभिजिती चैव धनिष्ठा च प्रकीर्तिता}
{ततः शतभिषा ज्ञेया पूर्वाभाद्रपदा तथा}

\twolineshloka*
{तथोत्तरा तु विज्ञेया रेवती चरमा स्मृता}
{अष्टाविंशति नक्षत्रं कलत्रं शशिनस्तथा}
{\fontsize{10}{4}\selectfont ---श्रीब्रह्मवैवर्त-महापुराणे श्रीकृष्णजन्मखण्डे उत्तरार्धे नारदनारायणसंवादे राधोद्धवसंवादे कालनिरूपणं नाम षण्ण्वतितमेऽध्याये श्लोकाः ६८-७२}

\begin{longtable}{lll}
  १. & अश्विनी  & \tam{அஶ்வினி}     \\
  २. & भरणी    & \tam{பரணி}       \\
  ३. & कृत्तिका  & \tam{க்ருத்திகை}   \\
  ४. & रोहिणी  & \tam{ரோஹிணி}     \\
  ५. & मृगशीर्षम् & \tam{ம்ருகஶீர்ஷம்}   \\
  ६. & आर्द्रा   & \tam{திருவாதிரை} \\
  ७. & पुनर्वसुः  & \tam{புனர்பூசம்}   \\
  ८. & पुष्यः    & \tam{பூசம்}       \\
  ९. & आश्लेषा   & \tam{ஆயில்யம்}     \\
  १०. & मघा                   & \tam{மகம்}        \\
  ११. & पूर्वफल्गुनी              & \tam{பூரம்}       \\
  १२. & उत्तरफल्गुनी             & \tam{உத்திரம்}     \\
  १३. & हस्तः                  & \tam{ஹஸ்தம்}       \\
  १४. & चित्रा                 & \tam{சித்திரை}    \\
  १५. & स्वाती                 & \tam{ஸ்வாதி}      \\
  १६. & विशाखा                & \tam{விஶாகம்}     \\
  १७. & अनुराधा                & \tam{அனுஷம்}      \\
  १८. & ज्येष्ठा                 & \tam{கேட்டை}      \\
  १९. & मूलम्                   & \tam{மூலம்}       \\
  २०. & पूर्वाषाढा              & \tam{பூராடம்}     \\
  २१. & उत्तराषाढा             & \tam{உத்திராடம்}   \\
  २२. & श्रवणम्                 & \tam{திருவோணம்}   \\
  २३. & श्रविष्ठा/धनिष्ठा        & \tam{அவிட்டம்}     \\
  २४. & शतभिषक्                & \tam{சதயம்}       \\
  २५. & पूर्व-प्रोष्ठपदा/भाद्रपदा  & \tam{பூரட்டாதி}   \\
  २६. & उत्तर-प्रोष्ठपदा/भाद्रपदा & \tam{உத்திரட்டாதி} \\
  २७. & रेवती                  & \tam{ரேவதி}      \\
\end{longtable}

\nopagebreak[4]
\dnsub{जन्मानुजन्म-नक्षत्राणि}

{\normalsize
\begin{tabular}{p{0.27\linewidth}p{0.27\linewidth}p{0.33\linewidth}}
  अश्विनी  & मघा       & मूलम्                   \\
  भरणी    & पूर्वफल्गुनी  & पूर्वाषाढा              \\
  कृत्तिका  & उत्तरफल्गुनी & उत्तराषाढा             \\
  रोहिणी  & हस्तः      & श्रवणम्                 \\
  मृगशीर्षम् & चित्रा     & श्रविष्ठा/धनिष्ठा        \\
  आर्द्रा   & स्वाती     & शतभिषक्                \\
  पुनर्वसुः  & विशाखा    & पूर्व-प्रोष्ठपदा/भाद्रपदा  \\
  पुष्यः    & अनुराधा    & उत्तर-प्रोष्ठपदा/भाद्रपदा \\
  आश्लेषा   & ज्येष्ठा     & रेवती                  \\
\end{tabular}

}
\clearpage
नक्षत्राणां राशयः

\begingroup
\small\renewcommand{\arraystretch}{0.333}
\begin{longtable}{|l@{~}l|>{\tiny}c|c|}
  \hline
  & नक्षत्रम्                   & \small{पादः} & राशिः                  \\\hline
  \endhead
  \multirow{4}{*}{ १.}            & \multirow{4}{*}{अश्विनी}  & १            & \multirow{9}{*}{मेषः}   \\*
  &                          & २            &                        \\*
  &                          & ३            &                        \\*
  &                          & ४            &                        \\*
  \cline{1-3}\multirow{4}{*}{ २.} & \multirow{4}{*}{भरणी}    & १            &                        \\*
  &                          & २            &                        \\*
  &                          & ३            &                        \\*
  &                          & ४            &                        \\*
  \cline{1-3}\multirow{4}{*}{ ३.} & \multirow{4}{*}{कृत्तिका}  & १            &                        \\*
  \cline{3-4}                     &                          & २            & \multirow{9}{*}{वृषभः}  \\*
  &                          & ३            &                        \\*
  &                          & ४            &                        \\*
  \cline{1-3}\multirow{4}{*}{ ४.} & \multirow{4}{*}{रोहिणी}  & १            &                        \\*
  &                          & २            &                        \\*
  &                          & ३            &                        \\*
  &                          & ४            &                        \\*
  \cline{1-3}\multirow{4}{*}{ ५.} & \multirow{4}{*}{मृगशीर्षम्} & १            &                        \\*
  &                          & २            &                        \\*
  \cline{3-4}                     &                          & ३            & \multirow{9}{*}{मिथुनम्} \\*
  &                          & ४            &                        \\*
  \cline{1-3}\multirow{4}{*}{ ६.} & \multirow{4}{*}{आर्द्रा}   & १            &                        \\*
  &                          & २            &                        \\*
  &                          & ३            &                        \\*
  &                          & ४            &                        \\*
  \cline{1-3}\multirow{4}{*}{ ७.} & \multirow{4}{*}{पुनर्वसुः}  & १            &                        \\*
  &                          & २            &                        \\*
  &                          & ३            &                        \\*
  \cline{3-4}                     &                          & ४            & \multirow{9}{*}{कटकः}  \\*
  \cline{1-3}\multirow{4}{*}{ ८.} & \multirow{4}{*}{पुष्यः}    & १            &                        \\*
  &                          & २            &                        \\*
  &                          & ३            &                        \\*
  &                          & ४            &                        \\*
  \cline{1-3}\multirow{4}{*}{ ९.} & \multirow{4}{*}{आश्लेषा}   & १            &                        \\*
  &                          & २            &                        \\*
  &                          & ३            &                        \\*
  &                          & ४            &                        \\\hline
  \multirow{4}{*}{१०.}            & \multirow{4}{*}{मघा}                   & १            & \multirow{9}{*}{सिंहः}   \\*
  &                                        & २            &                         \\*
  &                                        & ३            &                         \\*
  &                                        & ४            &                         \\*
  \cline{1-3}\multirow{4}{*}{११.} & \multirow{4}{*}{पूर्वफल्गुनी}              & १            &                         \\*
  &                                        & २            &                         \\*
  &                                        & ३            &                         \\*
  &                                        & ४            &                         \\*
  \cline{1-3}\multirow{4}{*}{१२.} & \multirow{4}{*}{उत्तरफल्गुनी}             & १            &                         \\*
  \cline{3-4}                     &                                        & २            & \multirow{9}{*}{कन्या}   \\*
  &                                        & ३            &                         \\*
  &                                        & ४            &                         \\*
  \cline{1-3}\multirow{4}{*}{१३.} & \multirow{4}{*}{हस्तः}                  & १            &                         \\*
  &                                        & २            &                         \\*
  &                                        & ३            &                         \\*
  &                                        & ४            &                         \\*
  \cline{1-3}\multirow{4}{*}{१४.} & \multirow{4}{*}{चित्रा}                 & १            &                         \\*
  &                                        & २            &                         \\*
  \cline{3-4}                     &                                        & ३            & \multirow{9}{*}{तुला}    \\*
  &                                        & ४            &                         \\*
  \cline{1-3}\multirow{4}{*}{१५.} & \multirow{4}{*}{स्वाती}                 & १            &                         \\*
  &                                        & २            &                         \\*
  &                                        & ३            &                         \\*
  &                                        & ४            &                         \\*
  \cline{1-3}\multirow{4}{*}{१६.} & \multirow{4}{*}{विशाखा}                & १            &                         \\*
  &                                        & २            &                         \\*
  &                                        & ३            &                         \\*
  \cline{3-4}                     &                                        & ४            & \multirow{9}{*}{वृश्चिकः} \\*
  \cline{1-3}\multirow{4}{*}{१७.} & \multirow{4}{*}{अनुराधा}                & १            &                         \\*
  &                                        & २            &                         \\*
  &                                        & ३            &                         \\*
  &                                        & ४            &                         \\*
  \cline{1-3}\multirow{4}{*}{१८.} & \multirow{4}{*}{ज्येष्ठा}                 & १            &                         \\*
  &                                        & २            &                         \\*
  &                                        & ३            &                         \\*
  &                                        & ४            &                         \\\hline
  \multirow{4}{*}{१९.}            & \multirow{4}{*}{मूलम्}                   & १            & \multirow{9}{*}{धनुः}    \\*
  &                                        & २            &                         \\*
  &                                        & ३            &                         \\*
  &                                        & ४            &                         \\*
  \cline{1-3}\multirow{4}{*}{२०.} & \multirow{4}{*}{पूर्वाषाढा}              & १            &                         \\*
  &                                        & २            &                         \\*
  &                                        & ३            &                         \\*
  &                                        & ४            &                         \\*
  \cline{1-3}\multirow{4}{*}{२१.} & \multirow{4}{*}{उत्तराषाढा}             & १            &                         \\*
  \cline{3-4}                     &                                        & २            & \multirow{9}{*}{मकरः}   \\*
  &                                        & ३            &                         \\*
  &                                        & ४            &                         \\*
  \cline{1-3}\multirow{4}{*}{२२.} & \multirow{4}{*}{श्रवणम्}                 & १            &                         \\*
  &                                        & २            &                         \\*
  &                                        & ३            &                         \\*
  &                                        & ४            &                         \\*
  \cline{1-3}\multirow{4}{*}{२३.} & \multirow{4}{*}{श्रविष्ठा/धनिष्ठा}        & १            &                         \\*
  &                                        & २            &                         \\*
  \cline{3-4}                     &                                        & ३            & \multirow{9}{*}{कुम्भः}   \\*
  &                                        & ४            &                         \\*
  \cline{1-3}\multirow{4}{*}{२४.} & \multirow{4}{*}{शतभिषक्}                & १            &                         \\*
  &                                        & २            &                         \\*
  &                                        & ३            &                         \\*
  &                                        & ४            &                         \\*
  \cline{1-3}\multirow{4}{*}{२५.} & \multirow{4}{*}{पूर्व-प्रोष्ठपदा/भाद्रपदा}  & १            &                         \\*
  &                                        & २            &                         \\*
  &                                        & ३            &                         \\*
  \cline{3-4}                     &                                        & ४            & \multirow{9}{*}{मीनः}   \\*
  \cline{1-3}\multirow{4}{*}{२६.} & \multirow{4}{*}{उत्तर-प्रोष्ठपदा/भाद्रपदा} & १            &                         \\*
  &                                        & २            &                         \\*
  &                                        & ३            &                         \\*
  &                                        & ४            &                         \\*
  \cline{1-3}\multirow{4}{*}{२७.} & \multirow{4}{*}{रेवती}                  & १            &                         \\*
  &                                        & २            &                         \\*
  &                                        & ३            &                         \\*
  &                                        & ४            &                         \\\hline
\end{longtable}
\endgroup

\sect{योगाः सप्तविंशतिः}

\twolineshloka*
{विष्कम्भः प्रीतिरायुष्मान् सौभाग्यं शोभनस्तथा}
{अतिगण्डः सुकर्मा च धृतिः शूलस्तथैव च}

\threelineshloka*
{गण्डो वृद्धिर्ध्रुवश्चैव व्याघातो हर्षणस्तथा}
{वज्रः सिद्धिर्व्यतीपातो वरीयान् परिघः शिवः}
{सिद्धः साध्यः शुभः शुभ्रो ब्राह्मो माहेन्द्र-वैधृती}


\begin{multicols}{3}
  \begin{enumerate}\itemsep-1ex
    
    \item विष्कम्भः
    \item प्रीतिः
    \item आयुष्मान्
    \item सौभाग्यम्
    \item शोभनः
    \item अतिगण्डः
    \item सुकर्मा
    \item धृतिः
    \item शूलः
    \item गण्डः
    \item वृद्धिः
    \item ध्रुवः
    \item व्याघातः
    \item हर्षणः
    \item वज्रः
    \item सिद्धिः
    \item व्यतीपातः
    \item वरीयान्
    \item परिघः
    \item शिवः
    \item सिद्धः
    \item साध्यः
    \item शुभः
    \item शुभ्रः
    \item ब्राह्मः
    \item माहेन्द्रः
    \item वैधृतिः
    
  \end{enumerate}
\end{multicols}

\sect{करणानि एकादश}

\threelineshloka*
{बवश्च बालवश्चैव कौलवस्तैतिलस्तथा}
{गरश्च वणिजश्चापि विष्टिश्च शकुनिस्तथा}
{चतुष्पाच्चापि नागश्च किंस्तुघ्न इति कीर्तितम्}

{\fontsize{10}{4}\selectfont ---श्रीब्रह्मवैवर्त-महापुराणे श्रीकृष्णजन्मखण्डे उत्तरार्धे नारदनारायणसंवादे राधोद्धवसंवादे कालनिरूपणं नाम षण्ण्वतितमेऽध्याये श्लोकः ८२}


{चराणि सप्त}

१. बवम् \hspace{2ex} २. बालवम् \hspace{2ex} ३. कौलवम् \hspace{2ex} ४. तैतिलम्

५. गरजा \hspace{2ex} ६. वणिजा  \hspace{2ex}७. भद्रा

{स्थिराणि चत्वारि}

१. शकुनिः \hspace{2ex} २. चतुष्पात् \hspace{2ex} ३. नागवान् \hspace{2ex} ४. किंस्तुघ्नम्

\pagebreak[4]

\dnsub{तिथीनां पूर्वोत्तरार्ध-करणानि}
\begingroup
\small
\begin{longtable}{llcc}
  & तिथिः       & पूर्वार्ध-करणम् & उत्तरार्ध-करणम् \\\endhead
  १.  & शुक्ल-प्रथमा   & किंस्तुघ्नम्     & बवम्          \\
  २.  & शुक्ल-द्वितीया & बालवम्       & कौलवम्        \\
  ३.  & शुक्ल-तृतीया   & तैतिलम्       & गरजा         \\
  ४.  & शुक्ल-चतुर्थी   & वणिजा       & भद्रा         \\
  ५.  & शुक्ल-पञ्चमी   & बवम्         & बालवम्        \\
  ६.  & शुक्ल-षष्ठी    & कौलवम्       & तैतिलम्        \\
  ७.  & शुक्ल-सप्तमी   & गरजा        & वणिजा        \\
  ८.  & शुक्ल-अष्टमी   & भद्रा        & बवम्          \\
  ९.  & शुक्ल-नवमी    & बालवम्       & कौलवम्        \\
  १०. & शुक्ल-दशमी    & तैतिलम्       & गरजा         \\
  ११. & शुक्ल-एकादशी  & वणिजा       & भद्रा         \\
  १२. & शुक्ल-द्वादशी  & बवम्         & बालवम्        \\
  १३. & शुक्ल-त्रयोदशी & कौलवम्       & तैतिलम्        \\
  १४. & शुक्ल-चतुर्दशी  & गरजा        & वणिजा        \\
  १५. & पौर्णमासी    & भद्रा        & बवम्          \\
  १६. & कृष्ण-प्रथमा   & बालवम्       & कौलवम्        \\
  १७. & कृष्ण-द्वितीया & तैतिलम्       & गरजा         \\
  १८. & कृष्ण-तृतीया   & वणिजा       & भद्रा         \\
  १९. & कृष्ण-चतुर्थी   & बवम्         & बालवम्        \\
  २०. & कृष्ण-पञ्चमी   & कौलवम्       & तैतिलम्        \\
  २१. & कृष्ण-षष्ठी    & गरजा        & वणिजा        \\
  २२. & कृष्ण-सप्तमी   & भद्रा        & बवम्          \\
  २३. & कृष्ण-अष्टमी   & बालवम्       & कौलवम्        \\
  २४. & कृष्ण-नवमी    & तैतिलम्       & गरजा         \\
  २५. & कृष्ण-दशमी    & वणिजा       & भद्रा         \\
  २६. & कृष्ण-एकादशी  & बवम्         & बालवम्        \\
  २७. & कृष्ण-द्वादशी  & कौलवम्       & तैतिलम्        \\
  २८. & कृष्ण-त्रयोदशी & गरजा        & वणिजा        \\
  २९. & कृष्ण-चतुर्दशी  & भद्रा        & शकुनिः        \\
  ३०. & अमावास्या    & चतुष्पात्      & नागवान्       \\
\end{longtable}
\endgroup

\sect{ग्रहाः नव}

\twolineshloka*
{आदित्याय च सोमाय मङ्गलाय बुधाय च}
{गुरुशुक्रशनिभ्यश्च राहवे केतवे नमः}

\begin{minipage}{\linewidth}
\begin{enumerate}\itemsep-1ex
  
    \item सूर्यः/आदित्यः
    \item चन्द्रः/सोमः/इन्दुः
    \item मङ्गलः/अङ्गारकः/कुजः/भौमः
    \item बुधः/सौम्यः
    \item गुरुः/बृहस्पतिः
    \item शुक्रः/भृगुः/काव्यः
    \item शनैश्चरः/मन्दः
    \item राहुः
    \item केतुः
    
  \end{enumerate}
\end{minipage}
  
  \fourlineindentedshloka*
  {आरोग्यं प्रददातु नो दिनकरश्चन्द्रो यशो निर्मलम्}
  {भूतिं भूमिसुतः सुधांशुतनयः प्रज्ञां गुरुर्गौरवम्‌}
  {काव्यः कोमलवाग्विलासमतुलं मन्दो मुदं सर्वदा}
  {राहुर्बाहुबलं विरोधशमनं केतुः कुलस्योन्नतिम्}
  























