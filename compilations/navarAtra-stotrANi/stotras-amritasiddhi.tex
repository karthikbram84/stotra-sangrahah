% !TeX program = XeLaTeX
% !TeX root = ./amritasiddhi-kindle.tex
\setcounter{page}{0}
\pagenumbering{arabic}
\sectionmark{\mbox{}}
\renewcommand{\chaptermark}[1]{%
\markboth{\large #1}{}}
\begin{center}
% \phantomsection\addcontentsline{toc}{chapter}{शक्तिस्तोत्राणि}
\sect{सङ्कल्पः}

ममोपात्त + प्रीत्यर्थं भगवत्याः जगदम्बायाः प्रसादेन -
\begin{itemize}\addtolength{\itemsep}{-1ex}
\item इदानीं लोके सर्वत्र प्रसृतस्य साङ्क्रामिक-रोग-विशेषस्य निश्शेषम् उन्मूलनार्थम्,
\item अस्मद्-देशीयानां विदेशीयानां चापि सर्वेषां व्याधि-भय-निवृत्त्यर्थं,
\item सञ्चार-प्रतिषेधात् सञ्जातस्य शुभकार्य-प्रतिबन्धस्य उद्योगादि-प्रतिबन्धस्य तज्जन्यायाः आर्थिक-दुःस्थितेश्च  परिहारार्थं,
\item सर्वेषां धार्मिक-अनुष्ठानानां, मन्दिरादिषु भगवतः पूजा-उत्सवानां च यथापूर्वं शीघ्रमेव प्रवृत्त्यर्थम्,
\item जनानां दुर्विचार-निवृत्ति-पूर्वक-सद्विचार-अभिवृद्ध्यर्थं,
\item साधूनां धार्मिकाणां च धैर्य-विश्वास-पुष्टि-सिद्ध्यर्थम्, आधर्मिक-शक्तीनां विनाशार्थं,
\item तद्-द्वारा सर्वलोक-क्षेमार्थं
\end{itemize}
\_\_\_\_ पुण्यकाले 
दुर्गासप्तशती-ललितासहस्रनाम-सौन्दर्यलहरी-विराटपर्वदुर्गास्तुति-दुर्गाचन्द्रकलास्तुति-महिषासुरमर्दिनीस्तोत्र-अभिराम्यन्तादिस्तोत्र-पारायणं करिष्ये।



% !TeX program = XeLaTeX
% !TeX root = ../../shloka.tex

\sect{ललितापञ्चरत्नम्}
\fourlineindentedshloka
{प्रातः स्मरामि ललितावदनारविन्दम्}
{बिम्बाधरं पृथुलमौक्तिकशोभिनासम्}
{आकर्णदीर्घनयनं मणिकुण्डलाढ्यम्}
{मन्दस्मितं मृगमदोज्ज्वलभालदेशम्}

\fourlineindentedshloka
{प्रातर्भजामि ललिताभुजकल्पवल्लीम्}
{रत्नाङ्गुलीयलसदङ्गुलिपल्लवाढ्याम्}
{माणिक्यहेमवलयाङ्गदशोभमानाम्}
{पुण्ड्रेक्षुचापकुसुमेषुसृणीर्दधानाम्}

\fourlineindentedshloka
{प्रातर्नमामि ललिताचरणारविन्दम्}
{भक्तेष्टदाननिरतं भवसिन्धुपोतम्}
{पद्मासनादिसुरनायकपूजनीयम्}
{पद्माङ्कुशध्वजसुदर्शनलाञ्छनाढ्यम्}

\fourlineindentedshloka
{प्रातः स्तुवे परशिवां ललितां भवानीम्}
{त्रय्यन्तवेद्यविभवां करुणानवद्याम्}
{विश्वस्य सृष्टिविलयस्थितिहेतुभूताम्}
{विश्वेश्वरीं निगमवाङ्मनसातिदूराम्}

\fourlineindentedshloka
{प्रातर्वदामि ललिते तव पुण्यनाम}
{कामेश्वरीति कमलेति महेश्वरीति}
{श्रीशाम्भवीति जगतां जननी परेति}
{वाग्देवतेति वचसा त्रिपुरेश्वरीति}

\fourlineindentedshloka*
{यः श्लोकपञ्चकमिदं ललिताम्बिकायाः}
{सौभाग्यदं सुललितं पठति प्रभाते}
{तस्मै ददाति ललिता झटिति प्रसन्ना}
{विद्यां श्रियं विमलसौख्यमनन्तकीर्तिम्}
॥इति श्रीमच्छङ्कराचार्यविरचितं  श्री-ललितापञ्चरत्नं सम्पूर्णम्॥
% !TeX program = XeLaTeX
% !TeX root = ../../shloka.tex

\sect{दुर्गापञ्चरत्नम्}

\fourlineindentedshloka
{ते ध्यान-योगानुगता अपश्यन्}
{त्वामेव देवीं स्वगुणैर्निगूढाम्}
{त्वमेव शक्तिः परमेश्वरस्य}
{मां पाहि सर्वेश्वरि मोक्षदात्रि}%॥ १॥

\fourlineindentedshloka
{देवात्मशक्तिः श्रुतिवाक्यगीता}
{महर्षि लोकस्य पुरः प्रसन्ना}
{गुहा परं व्योम सतः प्रतिष्ठा}
{मां पाहि सर्वेश्वरि मोक्षदात्रि}%॥ २॥

\fourlineindentedshloka
{परास्य शक्तिर्विविधैव श्रूयसे}
{श्वेताश्व-वाक्योदित-देवि दुर्गे}
{स्वाभाविकी ज्ञानबलक्रिया ते}
{मां पाहि सर्वेश्वरि मोक्षदात्रि}%॥ ३॥

\fourlineindentedshloka
{देवात्मशब्देन शिवात्मभूता}
{यत्कूर्मवायव्यवचो विवृत्या}
{त्वं पाशविच्छेदकरी प्रसिद्धा}
{मां पाहि सर्वेश्वरि मोक्षदात्रि}%॥ ४॥

\fourlineindentedshloka
{त्वं ब्रह्मपुच्छा विविधा मयूरी}
{ब्रह्म-प्रतिष्ठाऽस्युपदिष्ट-गीता}
{ज्ञानस्वरूपात्मतयाऽखिलानाम्}
{मां पाहि सर्वेश्वरि मोक्षदात्रि}%॥ ५॥

{॥इति श्री-काञ्चीपुरजगद्गुरुणा~श्रीमच्चन्द्रशेखरेन्द्र-\\सरस्वती-स्वामिना विरचितं श्री-दुर्गापञ्चरत्नं सम्पूर्णम्॥}


% !TeX program = XeLaTeX
% !TeX root = ../../shloka.tex

\sect{दुर्गास्तोत्रम् (युधिष्ठिरकृतम्)}

\uvacha{वैशम्पायन उवाच}

\twolineshloka
{विराटनगरं रम्यं गच्छमानो युधिष्ठिरः}
{अस्तुवन्मनसा देवीं दुर्गां त्रिभुवनेश्वरीम्}% ॥१॥

\twolineshloka
{यशोदागर्भसम्भूतां नारायणवरप्रियाम्}
{नन्दगोपकुले जातां मङ्गल्यां कुलवर्धनीम्}% ॥२॥

\twolineshloka
{कंसविद्रावणकरीमसुराणां क्षयङ्करीम्}
{शिलातटविनिक्षिप्तामाकाशं प्रति गामिनीम्}% ॥३॥

\twolineshloka
{वासुदेवस्य भगिनीं दिव्यमाल्यविभूषिताम्}
{दिव्याम्बरधरां देवीं खङ्गखेटकधारिणीम्}% ॥४॥

\twolineshloka
{भारावतरणे पुण्ये ये स्मरन्ति सदा शिवाम्}
{तान्वै तारयते पापात्पङ्के गामिव दुर्बलाम्}% ॥५॥

\twolineshloka
{स्तोतुं प्रचक्रमे भूयो विविधैः स्तोत्रसम्भवैः}
{आमन्त्र्य दर्शनाकाङ्क्षी राजा देवीं सहानुजः}% ॥६॥

\twolineshloka
{नमोस्तु वरदे कृष्णे कुमारि ब्रह्मचारिणि}
{बालार्कसदृशाकारे पूर्णचन्द्रनिभानने}% ॥७॥

\twolineshloka
{चतुर्भुजे चतुर्वक्रे पीनश्रोणिपयोधरे}
{मयूरपिच्छवलये केयूराङ्गदधारिणि}% ॥८॥

\twolineshloka
{भासि देवि यथा पद्मा नारायणपरिग्रहः}
{स्वरूपं ब्रह्मचर्यं च विशदं तव खेचरि}% ॥९॥

\twolineshloka
{कृष्णच्छविसमा कृष्णा सङ्कर्षणसमानना}
{बिभ्रती विपुलौ बाहु शक्रध्वजसमुच्छ्रयौ}% ॥१०॥

\twolineshloka
{पात्री च पङ्कजी घण्टी स्त्री विशुद्धा च या भुवि}
{पाशं धनुर्महाचक्रं विविधान्यायुधानि च}% ॥११॥

\twolineshloka
{कुण्डलाभ्यां सुपूर्णाभ्यां कर्णाभ्यां च विभूषिता}
{चन्द्रविस्पर्धिना देवि मुखेन त्वं विराजसे}% ॥१२॥

\twolineshloka
{मुकुटेन विचित्रेण केशबन्धेन शोभिना}
{भुजङ्गाभोगवासेन श्रोणिसूत्रेण राजता}% ॥१३॥

\twolineshloka
{विभ्राजसे चाऽऽबद्धेन भोगेनेवेह मन्दरः}
{ध्वजेन शिखिपिच्छानामुच्छ्रितेन विराजसे}% ॥१४॥

\twolineshloka
{कौमारं व्रतमास्थाय त्रिदिवं पावितं त्वया}
{तेन त्वं स्तूयसे देवि त्रिदशैः पूज्यसेऽपि च}% ॥१५॥

\twolineshloka
{त्रैलोक्यरक्षणार्थाय महिषासुरनाशिनि}
{प्रसन्ना मे सुरश्रेष्ठे दयां कुरु शिवा भव}% ॥१६॥

\twolineshloka
{जया त्वं विजया चैव सङ्ग्रामे च जयप्रदा}
{ममापि विजयं देहि वरदा त्वं च साम्प्रतम्}% ॥१७॥

\twolineshloka
{विन्ध्ये चैव नगश्रेष्ठे तव स्थानं हि शाश्वतम्}
{कालि कालि महाकालि शीधुमांसपशुप्रिये}% ॥१८॥

\twolineshloka
{कृतानुयात्रा भूतैस्त्वं वरदा कामचारिणी}
{भारावतारे ये च त्वां संस्मरिष्यन्ति मानवाः}% ॥१९॥

\twolineshloka
{प्रणमन्ति च ये त्वां हि प्रभाते तु नरा भुवि}
{न तेषां दुर्लभं किञ्चित्पुत्रतो धनतोपि वा}% ॥२०॥

\threelineshloka
{दुर्गात्तारयसे दुर्गे तत्त्वं दुर्गा स्मृता जनैः}
{कान्तारेष्ववसन्नानां मग्रानां च महार्णवे}
{दस्युभिर्वा निरुद्धानां त्वं गतिः परमा नृणाम्}% ॥२१॥

\twolineshloka
{जलप्रतरणे चैव कान्तारेष्वटवीषु च}
{ये स्मरन्ति महादेवि न च सीदन्ति ते नराः}% ॥२२॥

\twolineshloka
{त्वं कीर्तिः श्रीर्धृतिः सिद्धिर्ह्रीर्विद्या सन्ततिर्मतिः}
{सन्ध्या रात्रिः प्रभा निद्रा ज्योत्स्ना कान्तिःक्षमादया}% ॥२३॥

\twolineshloka
{नृणां च बन्धनं मोहं पुत्रनाशं धनक्षयम्}
{व्याधिं मृत्युं भयं चैव पूजिता नाशयिष्यसि}% ॥२४॥

\twolineshloka
{सोहं राज्यात्परिभ्रष्टः शरणं त्वां प्रपन्नवान्}
{प्रणतश्च यथा मूर्ध्ना तव देवि सुरेश्वरि}% ॥२५॥

\twolineshloka
{त्राहि मां पद्मपत्राक्षि सत्ये सत्या भवस्व नः}
{शरणं भव मे दुर्गे शरण्ये भक्तवत्सले}% ॥२६॥

\twolineshloka
{एवं स्तुता हि सा देवी दर्शयामास पाण्डवम्}
{उपगम्य तु राजानमिदं वचनमब्रवीत्}% ॥२७॥

\uvacha{देव्युवाच}

\twolineshloka
{शृणु राजन्महाबाहो मदीयं वचनं प्रभो}
{भविष्यत्यचिरादेव सङ्ग्रामे विजयस्तव}% ॥२८॥

\twolineshloka
{मम प्रसादान्निर्जित्य हत्वा कौरववाहिनीम्}
{राज्यं निष्कण्टकं कृत्वा भोक्ष्यसे मेदिनीं पुनः}% ॥२९॥

\twolineshloka
{भ्रातृभिः सहितो राजन्प्रीतिं प्राप्स्यसि पुष्कलाम्}
{मत्प्रसादाच्च ते सौख्यमारोग्यं च भविष्यति}% ॥३०॥

\twolineshloka
{ये च सङ्कीर्तयिष्यन्ति लोके विगतकल्मषाः}
{तेषां तुष्टा प्रदास्यामि राज्यमायुर्वपुः सुतम्}% ॥३१॥

\twolineshloka
{प्रवासे नगरे वाऽपि सङ्ग्रामे शत्रुसङ्कटे}
{अटव्यां दुर्गकान्तारे सागरे गहने गिरौ}% ॥३२॥

\twolineshloka
{ये स्मरिष्यन्ति मां राजन्यथाऽहं भवता स्मृता}
{न तेषां दुर्लभं किञ्चिदस्मिँल्लोके भविष्यति}% ॥३३॥

\twolineshloka
{इदं स्तोत्रवरं भक्त्या शृणुयाद्वा पठेत वा}
{तस्य सर्वाणि कार्याणि सिद्धिं यास्यन्ति पाण्डवाः}% ॥३४॥

\twolineshloka
{मत्प्रसादाच्च वः सर्वान्विराटनगरे स्थितान्}
{न प्रज्ञास्यन्ति कुरको नरा वा तन्निवासिनः}% ॥३५॥

\twolineshloka
{इत्युक्त्वा वरदा देवी युधिष्ठिरमरिन्दमम्}
{रक्षां कृत्वा च पाण्डूनां तत्रैवान्तरधीयत}% ॥३६॥

{॥इति श्रीमन्महाभारते विराटपर्वणि पाण्डवप्रवेशपर्वणि अष्टमोऽध्यायः॥}

% !TeX program = XeLaTeX
% !TeX root = ../../shloka.tex

\sect{दुर्गा-स्तुतिः (अर्जुन-कृता)}

\uvacha{सञ्जय उवाच}

\twolineshloka
{धार्तराष्ट्रबलं दृष्ट्वा युद्धाय समुपस्थितम्}
{अर्जुनस्य हितार्थाय कृष्णो वचनमब्रवीत्}

\uvacha{श्रीभगवानुवाच}

\twolineshloka
{शुचिर्भूत्वा महाबाहो सङ्ग्रामाभिमुखे स्थितः}
{पराजयाय शत्रूणां दुर्गास्तोत्रमुदीरय}

\uvacha{सञ्जय उवाच}

\twolineshloka
{एवमुक्तोऽर्जुनः सङ्ख्ये वासुदेवेन धीमता}
{अवतीर्य रथात्पार्थः स्तोत्रमाह कृताञ्जलिः}

\uvacha{अर्जुन उवाच}

\twolineshloka
{नमस्ते सिद्धसेनानि आर्ये मन्दरवासिनि}
{कुमारि कालि कापालि कपिले कृष्णपिङ्गले}

\twolineshloka
{भद्रकालि नमस्तुभ्यं महाकालि नमोऽस्तु ते}
{चण्डि चण्डे नमस्तुभ्यं तारिणि वरवर्णिनि}

\twolineshloka
{कात्यायनि महाभागे करालि विजये जये}
{शिखिपिच्छध्वजधरे नानाभरणभूषिते}

\twolineshloka
{अट्टशूलप्रहरणे खड्गखेटकधारिणि}
{गोपेन्द्रस्यानुजे ज्येष्ठे नन्दगोपकुलोद्भवे}

\twolineshloka
{महिषासृक्प्रिये नित्यं कौशिकि पीतवासिनि}
{अट्टहासे कोकमुखे नमस्तेऽस्तु रणप्रिये}

\twolineshloka
{उमे शाकम्भरि श्वेते कृष्णे कैटभनाशिनि}
{हिरण्याक्षि विरूपाक्षि सुधूम्राक्षि नमोऽस्तु ते}

\twolineshloka
{वेदश्रुति महापुण्ये ब्रह्मण्ये जातवेदसि}
{जम्बूकटकचैत्येषु नित्यं सन्निहितालये}

\twolineshloka
{त्वं ब्रह्मविद्या विद्यानां महानिद्रा च देहिनाम्}
{स्कन्दमातर्भगवति दुर्गे कान्तारवासिनि}

\twolineshloka
{स्वाहाकारः स्वधा चैव कला काष्ठा सरस्वती}
{सावित्री वेदमाता च तथा वेदान्त उच्यते}

\twolineshloka
{स्तुताऽसि त्वं महादेवि विशुद्धेनान्तरात्मना}
{जयो भवतु मे नित्यं त्वत्प्रसादाद्रणाजिरे}

\twolineshloka
{कान्तारभयदुर्गेषु भक्तानां चाऽऽलयेषु च}
{नित्यं वससि पाताले युद्धे जयसि दानवान्}

\twolineshloka
{त्वं जम्भनी मोहिनी च माया ह्रीः श्रीस्तथैव च}
{सन्ध्या प्रभावती चैव सावित्री जननी तथा}

\twolineshloka
{तुष्टिः पुष्टिर्धृतिर्दीप्तिश्चन्द्रादित्यविवर्धिनी}
{भूतिर्भूतिमतां सङ्ख्ये वीक्ष्यसे सिद्धचारणैः}

\uvacha{सञ्जय उवाच}

\twolineshloka
{ततः पार्थस्य विज्ञाय भक्तिं मानववत्सला}
{अन्तरिक्षगतोवाच गोविन्दस्याग्रतः स्थिता}

\uvacha{देव्युवाच}

\twolineshloka
{स्वल्पेनैव तु कालेन शत्रूञ्जेष्यसि पाण्डव}
{नरस्त्वमसि दुर्धर्ष नारायणसहायवान्}

\twolineshloka
{अजेयस्त्वं रणेऽरीणामपि वज्रभृतः स्वयम्}
{इत्येवमुक्त्वा वरदा क्षणेनान्तरधीयत}

\twolineshloka
{लब्ध्वा वरं तु कौन्तेयो मेने विजयमात्मनः}
{आरुरोह ततः पार्थो रथं परमसम्मतम्}

\twolineshloka
{कृष्णार्जुनावेकरथौ दिव्यौ शङ्खौ प्रदध्मतुः}
{य इदं पठते स्तोत्रं कल्य उत्थाय मानवः}

\twolineshloka
{यक्षरक्षःपिशाचेभ्यो न भयं विद्यते सदा}
{न चापि रिपवस्तेभ्यः सर्पाद्या ये च दंष्ट्रिणः}

\twolineshloka
{न भयं विद्यते तस्य सदा राजकुलादपि}
{विवादे जयमाप्नोति बद्धो मुच्यति बन्धनात्}

\twolineshloka
{दुर्गं तरति चावश्यं तथा चोरैर्विमुच्यते}
{सङ्ग्रामे विजयेन्नित्यं लक्ष्मीं प्राप्नोति केवलाम्}

\twolineshloka
{आरोग्यबलसम्पन्नो जीवेद्वर्षशतं तथा}
{एतद्दृष्टं प्रसादात्तु मया व्यासस्य धीमतः}

\twolineshloka
{मोहादेतौ न जानन्ति नरनारायणावृषी}
{तव पुत्रा दुरात्मानः सर्वे मन्युवशानुगाः}


\threelineshloka
{प्राप्तकालमिदं वाक्यं कालपाशेन कुण्ठिताः}
{द्वैपायनो नारदश्च कण्वो रामस्तथाऽनघः}
{अवारयंस्तव सुतं न चासौ तद्गृहीतवान्}

\twolineshloka
{यत्र धर्मो द्युतिः कान्तिर्यत्र ह्रीः श्रीस्तथा मतिः}
{यतो धर्मस्ततः कृष्णो यतः कृष्णस्ततो जयः}

॥इति श्रीमन्महाभरते भीष्मपर्वणि श्रीमद्भगवद्गीतापर्वणि
त्रयोविंशोऽध्यायः॥

\input{"../../stotras/shakti/DurgaChandrakalaStuti.tex"}
% !TeX program = XeLaTeX
% !TeX root = ../../shloka.tex

\sect{गौरीदशकम्}


\fourlineindentedshloka
{लीलारब्धस्थापितलुप्ताखिललोकां}
{लोकातीतैर्योगिभिरन्तश्चिरमृग्याम्}
{बालादित्यश्रेणिसमानद्युतिपुञ्जां}
{गौरीमम्बामम्बुरुहाक्षीमहमीडे}

\fourlineindentedshloka
{प्रत्याहारध्यानसमाधिस्थितिभाजां}
{नित्यं चित्ते निर्वृतिकाष्ठां कलयन्तीम्}
{सत्यज्ञानानन्दमयीं तां तनुरूपां}
{गौरीमम्बामम्बुरुहाक्षीमहमीडे}

\fourlineindentedshloka
{चन्द्रापीडानन्दितमन्दस्मितवक्त्रां}
{चन्द्रापीडालङ्कृतनीलालकभाराम्}
{इन्द्रोपेन्द्राद्यर्चितपादाम्बुजयुग्मां}
{गौरीमम्बामम्बुरुहाक्षीमहमीडे}

\fourlineindentedshloka
{आदिक्षान्तामक्षरमूर्त्या विलसन्तीं}
{भूते भूते भूतकदम्बप्रसवित्रीम्}
{शब्दब्रह्मानन्दमयीं तां तडिदाभां}
{गौरीमम्बामम्बुरुहाक्षीमहमीडे}

\fourlineindentedshloka
{मूलाधारादुत्थितवीथ्या विधिरन्ध्रं}
{सौरं चान्द्रं व्याप्य विहारज्वलिताङ्गीम्}
{येयं सूक्ष्मात् सूक्ष्मतनुस्तां सुखरूपां}
{गौरीमम्बामम्बुरुहाक्षीमहमीडे}

\fourlineindentedshloka
{नित्यः शुद्धो निष्कल एको जगदीशः}
{साक्षी यस्याः सर्गविधौ संहरणे च}
{विश्वत्राणक्रीडनलोलां शिवपत्नीं}
{गौरीमम्बामम्बुरुहाक्षीमहमीडे}

\fourlineindentedshloka
{यस्याः कुक्षौ लीनमखण्डं जगदण्डं}
{भूयो भूयः प्रादुरभूदुत्थितमेव}
{पत्या सार्धं तां रजताद्रौ विहरन्तीं}
{गौरीमम्बामम्बुरुहाक्षीमहमीडे}

\fourlineindentedshloka
{यस्यामोतं प्रोतमशेषं मणिमाला}
{सूत्रे यद्वत् क्वापि चरं चाप्यचरं च}
{तामध्यात्मज्ञानपदव्या गमनीयां}
{गौरीमम्बामम्बुरुहाक्षीमहमीडे}

\fourlineindentedshloka
{नानाकारैः शक्तिकदम्बैर्भुवनानि}
{व्याप्य स्वैरं क्रीडति येयं स्वयमेका}
{कल्याणीं तां कल्पलतामानतिभाजां}
{गौरीमम्बामम्बुरुहाक्षीमहमीडे}

\fourlineindentedshloka
{आशापाशक्लेशविनाशं विदधानां}
{पादाम्भोजध्यानपराणां पुरुषाणाम्}
{ईशामीशार्धाङ्गहरां तामभिरामां}
{गौरीमम्बामम्बुरुहाक्षीमहमीडे}

\fourlineindentedshloka
{प्रातःकाले भावविशुद्धः प्रणिधानाद्}
{भक्त्या नित्यं जल्पति गौरीदशकं यः}
{वाचां सिद्धिं सम्पदमग्र्यां शिवभक्तिं}
{तस्यावश्यं पर्वतपुत्री विदधाति}

॥इति श्रीमत्परमहंसपरिव्राजकाचार्यस्य श्री-गोविन्द-भगवत्पूज्य-पाद-शिष्यस्य 
श्रीमच्छङ्करभगवतः कृतौ श्री-गौरीदशकं सम्पूर्णम्॥
% !TeX program = XeLaTeX
% !TeX root = ../../shloka.tex

\sect{महिषासुरमर्दिनि स्तोत्रम्}
\setlength{\shlokaspaceskip}{10pt}
\fourlineindentedshloka
{अयि गिरिनन्दिनि नन्दितमेदिनि विश्वविनोदिनि नन्दिनुते}
{गिरिवर-विन्ध्य-शिरोधिनिवासिनि विष्णुविलासिनि जिष्णुनुते}
{भगवति हे शितिकण्ठकुटुम्बिनि भूरिकुटुम्बिनि भूरिकृते}
{जय जय हे महिषासुरमर्दिनि रम्यकपर्दिनि शैलसुते}

\fourlineindentedshloka
{सुरवरवर्षिणि दुर्धरधर्षिणि दुर्मुखमर्षिणि हर्षरते}
{त्रिभुवनपोषिणि शङ्करतोषिणि किल्बिषमोषिणि घोषरते}
{दनुज-निरोषिणि दितिसुत-रोषिणि दुर्मद-शोषिणि सिन्धुसुते}
{जय जय हे महिषासुरमर्दिनि रम्यकपर्दिनि शैलसुते}

\fourlineindentedshloka
{अयि जगदम्ब-मदम्ब-कदम्ब-वनप्रिय-वासिनि हासरते}
{शिखरि शिरोमणि तुङ्ग-हिमालय-शृङ्ग-निजालय-मध्यगते}
{मधु-मधुरे मधु-कैटभ-गञ्जिनि कैटभ-भञ्जिनि रासरते}
{जय जय हे महिषासुरमर्दिनि रम्यकपर्दिनि शैलसुते}

\fourlineindentedshloka
{अयि शतखण्ड-विखण्डित-रुण्ड-वितुण्डित-शुण्ड-गजाधिपते}
{रिपु-गज-गण्ड-विदारण-चण्ड-पराक्रम-शुण्ड-मृगाधिपते}
{निज-भुज-दण्ड-निपातित-खण्ड-विपातित-मुण्ड-भटाधिपते}
{जय जय हे महिषासुरमर्दिनि रम्यकपर्दिनि शैलसुते}

\fourlineindentedshloka
{अयि रण-दुर्मद-शत्रु-वधोदित-दुर्धर-निर्जर-शक्तिभृते}
{चतुर-विचार-धुरीण-महाशिव-दूतकृत-प्रमथाधिपते}
{दुरित-दुरीह-दुराशय-दुर्मति-दानवदूत-कृतान्तमते}
{जय जय हे महिषासुरमर्दिनि रम्यकपर्दिनि शैलसुते}

\fourlineindentedshloka
{अयि शरणागत-वैरि-वधूवर-वीर-वराभय-दायकरे}
{त्रिभुवन-मस्तक-शूल-विरोधि शिरोधि कृतामल-शूलकरे}
{दुमिदुमि-तामर-दुन्दुभिनाद-महो-मुखरीकृत-तिग्मकरे}
{जय जय हे महिषासुरमर्दिनि रम्यकपर्दिनि शैलसुते}

\fourlineindentedshloka
{अयि निज-हुङ्कृति मात्र-निराकृत-धूम्रविलोचन-धूम्रशते}
{समर-विशोषित-शोणित-बीज-समुद्भव-शोणित-बीजलते}
{शिव-शिव-शुम्भ-निशुम्भ-महाहव-तर्पित-भूत-पिशाचरते}
{जय जय हे महिषासुरमर्दिनि रम्यकपर्दिनि शैलसुते}

\fourlineindentedshloka
{धनुरनु-सङ्ग-रणक्षणसङ्ग-परिस्फुर-दङ्ग-नटत्कटके}
{कनक-पिशङ्ग-पृषत्क-निषङ्ग-रसद्भट-शृङ्ग-हतावटुके}
{कृत-चतुरङ्ग-बलक्षिति-रङ्ग-घटद्बहुरङ्ग-रटद्बटुके}
{जय जय हे महिषासुरमर्दिनि रम्यकपर्दिनि शैलसुते}

\fourlineindentedshloka
{जय जय जप्य-जयेजय-शब्द-परस्तुति-तत्पर-विश्वनुते}
{भण-भण-भिञ्जिमि-भिङ्कृत-नूपुर-सिञ्जित-मोहित-भूतपते}
{नटित-नटार्ध-नटीनट-नायक-नाटित-नाट्य-सुगानरते}
{जय जय हे महिषासुरमर्दिनि रम्यकपर्दिनि शैलसुते}

\fourlineindentedshloka
{अयि सुमनः सुमनः सुमनः सुमनः सुमनोहर-कान्तियुते}
{श्रित-रजनी-रजनी-रजनी-रजनी-रजनीकर-वक्त्रवृते}
{सुनयन-विभ्रमर-भ्रमर-भ्रमर-भ्रमर-भ्रमराधिपते}
{जय जय हे महिषासुरमर्दिनि रम्यकपर्दिनि शैलसुते}

\fourlineindentedshloka
{सहित-महाहव-मल्लम-तल्लिक-मल्लित-रल्लक-मल्लरते}
{विरचित-वल्लिक-पल्लिक-मल्लिक-भिल्लिक-भिल्लिक-वर्गवृते}
{सितकृत-फुल्लसमुल्ल-सितारुण-तल्लज-पल्लव-सल्ललिते}
{जय जय हे महिषासुरमर्दिनि रम्यकपर्दिनि शैलसुते}

\fourlineindentedshloka
{अविरल-गण्ड-गलन्मद-मेदुर-मत्त-मतङ्गज-राजपते}
{त्रिभुवन-भूषण-भूत-कलानिधि रूप-पयोनिधि राजसुते}
{अयि सुद-तीजन-लालसमानस-मोहन-मन्मथ-राजसुते}
{जय जय हे महिषासुरमर्दिनि रम्यकपर्दिनि शैलसुते}

\fourlineindentedshloka
{कमल-दलामल-कोमल-कान्ति कलाकलितामल-भाललते}
{सकल-विलास-कलानिलयक्रम-केलि-चलत्कल-हंसकुले}
{अलिकुल-सङ्कुल-कुवलय-मण्डल-मौलिमिलद्भकुलालि-कुले}
{जय जय हे महिषासुरमर्दिनि रम्यकपर्दिनि शैलसुते}

\fourlineindentedshloka
{करमुरली-रव-वीजित-कूजित-लज्जित-कोकिल-मञ्जुमते}
{मिलित-पुलिन्द-मनोहर-गुञ्जित-रञ्जितशैल-निकुञ्जगते}
{निजगुणभूत-महाशबरीगण-सद्गुण-सम्भृत-केलितले}
{जय जय हे महिषासुरमर्दिनि रम्यकपर्दिनि शैलसुते}

\fourlineindentedshloka
{कटितट-पीत-दुकूल-विचित्र-मयूख-तिरस्कृत-चन्द्ररुचे}
{प्रणत-सुरासुर-मौलिमणिस्फुर-दंशुल-सन्नख-चन्द्ररुचे}
{जित-कनकाचल-मौलिपदोर्जित-निर्भर-कुञ्जर-कुम्भकुचे}
{जय जय हे महिषासुरमर्दिनि रम्यकपर्दिनि शैलसुते}

%\begin{flushright}
\fourlineindentedshloka
{विजित-सहस्रकरैक-सहस्रकरैक-सहस्रकरैकनुते}
{कृतसुरतारक-सङ्गरतारक-सङ्गरतारक-सूनुसुते}
{सुरथ-समाधि समानसमाधि समाधिसमाधि सुजातरते}
{जय जय हे महिषासुरमर्दिनि रम्यकपर्दिनि शैलसुते}

\fourlineindentedshloka
{पदकमलं करुणानिलये वरिवस्यति योऽनुदिनं स शिवे}
{अयि कमले कमलानिलये कमलानिलयः स कथं न भवेत्}
{तव पदमेव परम्पदमित्यनुशीलयतो मम किं न शिवे}
{जय जय हे महिषासुरमर्दिनि रम्यकपर्दिनि शैलसुते}

\fourlineindentedshloka
{कनकलसत्कल-सिन्धुजलैरनुसिञ्चिनुते गुण-रङ्गभुवम्}
{भजति स किं न शचीकुच-कुम्भ-तटी-परिरम्भ-सुखानुभवम्}
{तव चरणं शरणं करवाणि नतामरवाणि निवासि शिवम्}
{जय जय हे महिषासुरमर्दिनि रम्यकपर्दिनि शैलसुते}

\fourlineindentedshloka
{तव विमलेन्दुकुलं वदनेन्दुमलं सकलं ननु कूलयते}
{किमु पुरुहूत-पुरीन्दुमुखी-सुमुखीभिरसौ विमुखी क्रियते}
{मम तु मतं शिवनामधने भवती कृपया किमुत क्रियते}
{जय जय हे महिषासुरमर्दिनि रम्यकपर्दिनि शैलसुते}

\fourlineindentedshloka
{अयि मयि दीनदयालुतया कृपयैव त्वया भवितव्यमुमे}
{अयि जगतो जननी कृपयाऽसि यथाऽसि तथाऽनुमितासिरते}
{यदुचितमत्र भवत्युररि कुरुतादुरुतापमपाकुरुते}
{जय जय हे महिषासुरमर्दिनि रम्यकपर्दिनि शैलसुते}
%\end{flushright}
॥इति~श्रीमच्छङ्कराचार्यविरचितं श्री-महिषासुरमर्दिनि-स्तोत्रं~सम्पूर्णम्॥
\setlength{\shlokaspaceskip}{24pt}
% !TeX program = XeLaTeX
% !TeX root = ../../shloka.tex

\sect{दुर्गास्तोत्रम्}

\uvacha{श्री-नारायण उवाच}
\twolineshloka*
{स्तोत्रं च श्रूयतां ब्रह्मन् सर्वविघ्नविनाशकम्}
{सुखदं मोक्षदं सारं भवसन्तारकारणम्}

\uvacha{श्रीकृष्ण उवाच}
\twolineshloka
{त्वमेव सर्वजननी मूलप्रकृतिरीश्वरी}
{त्वमेवाऽऽद्या सृष्टिविधौ स्वेच्छया त्रिगुणात्मिका}

\twolineshloka
{कार्यार्थे सगुणा त्वं च वस्तुतो निर्गुणा स्वयम्}
{परब्रह्मस्वरूपा त्वं सत्या नित्या सनातनी}

\twolineshloka
{तेजःस्वरूपा परमा भक्तानुग्रहविग्रहा}
{सर्वस्वरूपा सर्वेशा सर्वाधारा परात्परा}

\twolineshloka
{सर्वबीजस्वरूपा च सर्वपूज्या निराश्रया}
{सर्वज्ञा सर्वतोभद्रा सर्वमङ्गलमङ्गला}

\twolineshloka
{सर्वबुद्धिस्वरूपा च सर्वशक्तिस्वरूपिणी}
{सर्वज्ञानप्रदा देवी सर्वज्ञा सर्वभाविनी}

\twolineshloka
{त्वं स्वाहा देवदाने च पितृदाने स्वधा स्वयम्}
{दक्षिणा सर्वदाने च सर्वशक्तिस्वरूपिणी}

\twolineshloka
{निद्रा त्वं च दया त्वं च तृष्णा त्वं चाऽऽत्मनः प्रिया}
{क्षुत्क्षान्तिः शान्तिरीशा च कान्तिस्तुष्टिश्च शाश्वती}

\twolineshloka
{श्रद्धा पुष्टिश्च तन्द्रा च लज्जा शोभा प्रभा तथा}
{सतां सम्पत्स्वरूपा श्रीर्विपत्तिरसतामिह}

\twolineshloka
{प्रीतिरूपा पुण्यवतां पापिनां कलहाङ्कुरा}
{शश्वत्कर्ममयी शक्तिः सर्वदा सर्वजीविनाम्}

\twolineshloka
{देवेभ्यः स्वपदो दात्री धातुर्धात्री कृपामयी}
{हिताय सर्वदेवानां सर्वासुरविनाशिनी}

\twolineshloka
{योगनिद्रा योगरूपा योगदात्री च योगिनाम्}
{सिद्धिस्वरूपा सिद्धानां सिद्धिदा सिद्धियोगिनी}

\twolineshloka
{माहेश्वरी च ब्रह्माणी विष्णुमाया च वैष्णवी}
{भद्रदा भद्रकाली च सर्वलोकभयङ्करी}

\twolineshloka
{ग्रामे ग्रामे ग्रामदेवी गृहदेवी गृहे गृहे}
{सतां कीर्तिः प्रतिष्ठा च निन्दा त्वमसतां सदा}

\twolineshloka
{महायुद्धे महामारी दुष्टसंहाररूपिणी}
{रक्षास्वरूपा शिष्टानां मातेव हितकारिणी}

\twolineshloka
{वन्द्या पूज्या स्तुता त्वं च ब्रह्मादीनां च सर्वदा}
{ब्राह्मण्यरूपा विप्राणां तपस्या च तपस्विनाम्}

\twolineshloka
{विद्या विद्यावतां त्वं च बुद्धिर्बुद्धिमतां सताम्}
{मेधा स्मृतिस्वरूपा च प्रतिभा प्रतिभावताम्}

\twolineshloka
{राज्ञां प्रतापरूपा च विशां वाणिज्यरूपिणी}
{सृष्टौ सृष्टिस्वरूपा त्वं रक्षारूपा च पालने}

\twolineshloka
{तथाऽन्ते त्वं महामारी विश्वे विश्वैश्च पूजिते}
{कालरात्रिर्महारात्रिर्मोहरात्रिश्च मोहिनी}

\twolineshloka
{दुरत्यया मे माया त्वं यया सम्मोहितं जगत्}
{यया मुग्धो हि विद्वांश्च मोक्षमार्गं न पश्यति}

\dnsub{फलश्रुतिः}

\twolineshloka
{इत्यात्मना कृतं स्तोत्रं दुर्गाया दुर्गनाशनम्}
{पूजाकाले पठेद्यो हि सिद्धिर्भवति वाञ्छिता}

\twolineshloka
{वन्ध्या च काकवन्ध्या च मृतवत्सा च दुर्भगा}
{श्रुत्वा स्तोत्रं वर्षमेकं सुपुत्रं लभते ध्रुवम्}

\twolineshloka
{कारागारे महाघोरे यो बद्धो दृढबन्धने}
{श्रुत्वा स्तोत्रं मासमेकं बन्धनान्मुच्यते ध्रुवम्}

\twolineshloka
{यक्ष्मग्रस्तो गलत्कुष्ठी महाशूली महाज्वरी}
{श्रुत्वा स्तोत्रं वर्षमेकं सद्यो रोगात् प्रमुच्यते}

\twolineshloka
{पुत्रभेदे प्रजाभेदे पत्‍‌नीभेदे च दुर्गतः}
{श्रुत्वा स्तोत्रं मासमेकं लभते नात्र संशयः}

\twolineshloka
{राजद्वारे श्मशाने च महारण्ये रणस्थले}
{हिंस्रजन्तुसमीपे च श्रुत्वा स्तोत्रं प्रमुच्यते}

\twolineshloka
{गृहदाहे च दावाग्नौ दस्युसैन्यसमन्विते}
{स्तोत्रश्रवणमात्रेण लभते नात्र संशयः}

\twolineshloka
{महादरिद्रो मूर्खश्च वर्षं स्तोत्रं पठेत्तु यः}
{विद्यावान् धनवांश्चैव स भवेन्नात्र संशयः}

{॥इति~श्रीब्रह्मवैवर्तमहापुराणे~प्रकृतिखण्डे षट्षष्टितमेऽध्याये श्री-नारद-नारायण-संवादे दुर्गोपाख्याने श्री-कृष्णविरचितं श्री-दुर्गास्तोत्रं सम्पूर्णम्॥}
% !TeX program = XeLaTeX
% !TeX root = ../../shloka.tex

\sect{परशुरामकृत-दुर्गास्तोत्रम्}
\uvacha{श्री-परशुराम उवाच}

\twolineshloka
{श्रीकृष्णस्य च गोलोके परिपूर्णतमस्य च}
{आविर्भूता विग्रहतः परा सृष्ट्युन्मुखस्य च}

\twolineshloka
{सूर्यकोटिप्रभायुक्ता वस्त्रालङ्कारभूषिता}
{वह्निशुद्धांशुकाधाना सुस्मिता सुमनोहरा}

\twolineshloka
{नवयौवनसम्पन्ना सिन्दूरारुण्यशोभिता}
{ललितं कबरीभारं मालतीमाल्यमण्डितम्}

\twolineshloka
{अहोऽनिर्वचनीया त्वं चारुमूर्तिं च बिभ्रती}
{मोक्षप्रदा मुमुक्षूणां महाविष्णोर्विधिः स्वयम्}

\twolineshloka
{मुमोह क्षणमात्रेण दृष्ट्वा त्वां सर्वमोहिनीम्}
{बालैः सम्भूय सहसा सस्मिता धाविता पुरा}

\twolineshloka
{सद्भिः ख्याता तेन राधा मूलप्रकृतिरीश्वरी}
{कृष्णस्त्वां सहसा भीतो वीर्याधानं चकार ह}

\twolineshloka
{ततो डिम्भं महज्जज्ञे ततो जातो महान्विराट्}
{यस्यैव लोमकूपेषु ब्रह्माण्डान्यखिलानि च}

\twolineshloka
{राधारतिक्रमेणैव तन्निःश्वासो बभूव ह}
{स निःश्वासो महावायुः स विराड्\mbox{}विश्वधारकः}

\twolineshloka
{भयघर्मजलेनैव पुप्लुवे विश्वगोलकम्}
{स विराड् विश्वनिलयो जलराशिर्बभूव ह}

\twolineshloka
{ततस्त्वं पञ्चधा भूय पञ्चमूर्तीश्च बिभ्रती}
{प्राणाधिष्ठातृमूर्तिर्या कृष्णस्य परमात्मनः}

\twolineshloka
{कृष्णप्राणाधिकां राधां तां वदन्ति पुराविदः}
{वेदाधिष्ठातृमूर्तिर्या वेदशास्त्रप्रसूरपि}

\twolineshloka
{तं सावित्रीं शुद्धरूपां प्रवदन्ति मनीषिणः}
{ऐश्वर्याधिष्ठातृमूर्तिः शान्तिस्त्वं शान्तरूपिणी}

\twolineshloka
{लक्ष्मीं वदन्ति सन्तस्तां शुद्धां सत्त्‍‌वस्वरूपिणीम्}
{रागाधिष्ठात्री या देवी शुक्लमूर्तिः सतां प्रसूः}

\twolineshloka
{सरस्वतीं तां शास्त्रज्ञां शास्त्रज्ञाः प्रवदन्त्यहो}
{बुद्धिर्विद्या सर्वशक्तेर्या मूर्तिरधिदेवता}

\twolineshloka
{सर्वमङ्गलदा सन्तो वदन्ति सर्वमङ्गलाम्}
{सर्वमङ्गलमङ्गल्या सर्वमङ्गलरूपिणी}

\twolineshloka
{सर्वमङ्गलबीजस्य शिवस्य निलयेऽधुना}
{शिवे शिवास्वरूपा त्वं लक्ष्मीर्नारायणान्तिके}

\twolineshloka
{सरस्वती च सावित्री वेदसू‌र्ब्रह्मणः प्रिया}
{राधा रासेश्वरस्यैव परिपूर्णतमस्य च}

\twolineshloka
{परमानन्दरूपस्य परमानन्दरूपिणी}
{त्वत्कलांशांशकलया देवानामपि योषितः}

\twolineshloka
{त्वं विद्या योषितः सर्वाः सर्वेषां बीजरूपिणी}
{छाया सूर्यस्य चन्द्रस्य रोहिणी सर्वमोहिनी}

\twolineshloka
{शची शक्रस्य कामस्य कामिनी रतिरीश्वरी}
{वरुणानी जलेशस्य वायोः स्त्रीः प्राणवल्लभा}

\twolineshloka
{वह्नेः प्रिया हि स्वाहा च कुबेरस्य च सुन्दरी}
{यमस्य तु सुशीला च नैर्ऋतस्य च कैटभी}

\twolineshloka
{ऐशानी स्याच्छशिकला शतरूपा मनोः प्रिया}
{देवहूतिः कर्दमस्य वसिष्ठस्याप्यरुन्धती}

\twolineshloka
{लोपामुद्राऽप्यगस्त्यस्य देवमाताऽदितिस्तथा}
{अहल्या गौतमस्यापि सर्वाधारा वसुन्धरा}

\twolineshloka
{गङ्गा च तुलसी चापि पृथिव्यां या सरिद्वरा}
{एताः सर्वाश्च या ह्यन्या सर्वास्त्वत्कलयाऽम्बिके}

\twolineshloka
{गृहलक्ष्मीर्गृहे नॄणां राजलक्ष्मीश्च राजसु}
{तपस्विनां तपस्या त्वं गायत्री ब्राह्मणस्य च}

\twolineshloka
{सतां सत्त्‍‌वस्वरूपा त्वमसतां कलहाङ्कुरा}
{ज्योतीरूपा निर्गुणस्य शक्तिस्त्वं सगुणस्य च}

\twolineshloka
{सूर्ये प्रभास्वरूपा त्वं दाहिका च हुताशने}
{जले शैत्यस्वरूपा च शोभारूपा निशाकरे}

\twolineshloka
{त्वं भूमौ गन्धरूपा च आकाशे शब्दरूपिणी}
{क्षुत्पिपासादयस्त्वं च जीविनां सर्वशक्तयः}

\twolineshloka
{सर्वबीजस्वरूपा त्वं संसारे साररूपिणी}
{स्मृतिर्मेधा च बुद्धिर्वा ज्ञानशक्तिर्विपश्चिताम्}

\twolineshloka
{कृष्णेन विद्या या दत्ता सर्वज्ञानप्रसूः शुभा}
{शूलिने कृपया सा त्वं यया मृत्युञ्जयः शिवः}

\twolineshloka
{सृष्टिपालनसंहारशक्तयस्त्रिविधाश्च याः}
{ब्रह्मविष्णुमहेशानां सा त्वमेव नमोऽस्तु ते}

\twolineshloka
{मधुकैटभभीत्या च त्रस्तो धाता प्रकम्पितः}
{स्तुत्वा मुक्तश्च यां देवीं तां मूर्ध्ना प्रणमाम्यहम्}

\twolineshloka
{मधुकैटभयोर्युद्धे त्राताऽसौ विष्णुरीश्वरीम्}
{बभूव शक्तिमान् स्तुत्वा तां दुर्गां प्रणमाम्यहम्}

\twolineshloka
{त्रिपुरस्य महायुद्धे सरथे पतिते शिवे}
{यां तुष्टुवुः सुराः सर्वे तां दुर्गां प्रणमाम्यहम्}

\twolineshloka
{विष्णुना वृषरूपेण स्वयं शम्भुः समुत्थितः}
{जघान त्रिपुरं स्तुत्वा तां दुर्गां प्रणमाम्यहम्}

\twolineshloka
{यदाज्ञया वाति वातः सूर्यस्तपति सन्ततम्}
{वर्षतीन्द्रो दहत्यग्निस्तां दुर्गां प्रणमाम्यहम्}

\twolineshloka
{यदाज्ञया हि कालश्च शश्वद्-भ्रमति वेगतः}
{मृत्युश्चरति जन्तूनां तां दुर्गां प्रणमाम्यहम्}

\twolineshloka
{स्त्रष्टा सृजति सृष्टिं च पाता पाति यदाज्ञया}
{संहर्ता संहरेत् काले तां दुर्गां प्रणमाम्यहम्}

\twolineshloka
{ज्योतिःस्वरूपो भगवाञ्छ्रीकृष्णो निर्गुणः स्वयम्}
{यया विना न शक्तश्च सृष्टिं कर्तुं नमामि ताम्}

\twolineshloka
{रक्ष रक्ष जगन्मातरपराधं क्षमस्व मे}
{शिशूनामपराधेन कुतो माता हि कुप्यति}

\twolineshloka
{इत्युक्त्वा परशुरामश्च नत्वा तां च रुरोद ह}
{तुष्टा दुर्गा सम्भ्रमेण चाभयं च वरं ददौ}

\twolineshloka
{अमरो भव हे पुत्र वत्स सुस्थिरतां व्रज}
{शर्वप्रसादात् सर्वत्र जयोऽस्तु तव सन्ततम्}


\twolineshloka
{सर्वान्तरात्मा भगवांस्तुष्टः स्यात्सन्ततं हरिः}
{भक्तिर्भवतु ते कृष्णे शिवदे च शिवे गुरौ}

\twolineshloka
{इष्टदेवे गुरौ यस्य भक्तिर्भवति शाश्वती}
{तं हन्तुं न हि शक्ताश्च रुष्टा वा सर्वदेवताः}

\twolineshloka
{श्रीकृष्णस्य च भक्तस्त्वं शिष्यो वै शङ्करस्य च}
{गुरुपत्‍‌नीं स्तौषि यस्मात् कस्त्वां हन्तुमिहेश्वरः}

\twolineshloka
{अहो न कृष्णभक्तानामशुभं विद्यते क्वचित्}
{अन्यदेवेषु ये भक्ता न भक्ता वा निरङ्कुशाः}

\twolineshloka
{चन्द्रमा बलवांस्तुष्टो येषां भाग्यवतां भृगो}
{तेषां तारागणा रुष्टाः किं कुर्वन्ति च दुर्बलाः}

\twolineshloka
{यस्मै तुष्टः पालयति नरदेवो महान् सुखी}
{तस्य किं वा करिष्यन्ति रुष्टा भृत्याश्च दुर्बलाः}

\twolineshloka
{इत्युक्त्वा पार्वती तुष्टा दत्त्‍‌वा रामाय चाऽऽशिषम्}
{जगामान्तःपुरं तूर्णं हर्षशब्दो बभूव ह}

\dnsub{फलश्रुतिः}

\twolineshloka
{स्तोत्रं वै काण्वशाखोक्तं पूजाकाले च यः पठेत्}
{यात्राकाले च प्रातर्वा वाञ्छितार्थं लभेद्ध्रुवम्}

\twolineshloka
{पुत्रार्थी लभते पुत्रं कन्यार्थी कन्यकां लभेत्}
{विद्यार्थी लभते विद्यां प्रजार्थी चाऽऽप्नुयात् प्रजाम्}

\twolineshloka
{भ्रष्टराज्यो लभेद्राज्यं नष्टवित्तो धनं लभेत्}
{यस्य रुष्टो गुरुर्देवो राजा वा बान्धवोऽथवा}

\twolineshloka
{तस्मै तुष्टश्च वरदः स्तोत्रराजप्रसादतः}
{दस्युग्रस्तो फणिग्रस्तः शत्रुग्रस्तो भयानकः}

\twolineshloka
{व्याधिग्रस्तो भवेन्मुक्तः स्तोत्रस्मरणमात्रतः}
{राजद्वारे श्मशाने च कारागारे च बन्धने}

\twolineshloka
{जलराशौ निमग्नश्च मुक्तस्तत्स्मृतिमात्रतः}
{स्वामिभेदे पुत्रभेदे मित्रभेदे च दारुणे}

\twolineshloka
{स्तोत्रस्मरणमात्रेण वाञ्छितार्थं लभेद्ध्रुवम्}
{कृत्वा हविष्यं वर्षं च स्तोत्रराजं श्रृणोति या}

\threelineshloka
{भक्त्या दुर्गां च सम्पूज्य महावन्ध्या प्रसूयते}
{लभते सा दिव्यपुत्रं ज्ञानिनं चिरजीविनम्}
{असौभाग्या च सौभाग्यं षण्मासश्रवणाल्लभेत्}

\twolineshloka
{नवमासं काकवन्ध्या मृतवत्सा च भक्तितः}
{स्तोत्रराजं या श्रृणोति सा पुत्रं लभते ध्रुवम्}

\twolineshloka
{कन्यामाता पुत्रहीना पञ्चमासं श्रृणोति या}
{घटे सम्पूज्य दुर्गां च सा पुत्रं लभते ध्रुवम्}


{॥इति~श्री-ब्रह्मवैवर्तमहापुराणे~गणपतिखण्डे पञ्चचत्वारिंशेऽध्याये श्री-नारद-नारायण-संवादे श्री-परशुरामकृतं श्री-दुर्गास्तोत्रं सम्पूर्णम्॥}
% !TeX program = XeLaTeX
% !TeX root = ../../shloka.tex

\sect{सौन्दर्यलहरी}
\dnsub{आनन्दलहरी}

\fourlineindentedshloka
{शिवः शक्त्या युक्तो यदि भवति शक्तः प्रभवितुम्}
{न चेदेवं देवो न खलु कुशलः स्पन्दितुमपि}
{अतस्त्वामाराध्यां हरिहरविरिञ्चादिभिरपि}
{प्रणन्तुं स्तोतुं वा कथमकृतपुण्यः प्रभवति}%1

\fourlineindentedshloka
{तनीयांसं पांसुं तव चरणपङ्केरुहभवम्}
{विरिञ्चिः सञ्चिन्वन् विरचयति लोकानविकलम्}
{वहत्येनं शौरिः कथमपि सहस्रेण शिरसाम्}
{हरः सङ्क्षुद्यैनं भजति भसितोद्धूलनविधिम्}%2

\fourlineindentedshloka
{अविद्यानामन्तस्तिमिर-मिहिरद्वीपनगरी}
{जडानां चैतन्य-स्तबक-मकरन्द-स्रुतिझरी}
{दरिद्राणां चिन्तामणिगुणनिका जन्मजलधौ}
{निमग्नानां दंष्ट्रा मुररिपु-वराहस्य भवती}%3

\fourlineindentedshloka
{त्वदन्यः पाणिभ्यामभयवरदो दैवतगणः}
{त्वमेका नैवासि प्रकटितवराभीत्यभिनया}
{भयात् त्रातुं दातुं फलमपि च वाञ्छासमधिकम्}
{शरण्ये लोकानां तव हि चरणावेव निपुणौ}%4

\fourlineindentedshloka
{हरिस्त्वामाराध्य प्रणतजनसौभाग्यजननीम्}
{पुरा नारी भूत्वा पुररिपुमपि क्षोभमनयत्}
{स्मरोऽपि त्वां नत्वा रतिनयनलेह्येन वपुषा}
{मुनीनामप्यन्तः प्रभवति हि मोहाय महताम्}%5

\fourlineindentedshloka
{धनुः पौष्पं मौर्वी मधुकरमयी पञ्च विशिखाः}
{वसन्तः सामन्तो मलयमरुदायोधनरथः}
{तथाऽप्येकः सर्वं हिमगिरिसुते कामपि कृपाम्}
{अपाङ्गात्ते लब्ध्वा जगदिदमनङ्गो विजयते}%6

\fourlineindentedshloka
{क्वणत्काञ्चीदामा करिकलभकुम्भस्तननता}
{परिक्षीणा मध्ये परिणतशरच्चन्द्रवदना}
{धनुर्बाणान् पाशं सृणिमपि दधाना करतलैः}
{पुरस्तादास्तां नः पुरमथितुराहोपुरुषिका}%7

\fourlineindentedshloka
{सुधासिन्धोर्मध्ये सुरविटपिवाटीपरिवृते}
{मणिद्वीपे नीपोपवनवति चिन्तामणिगृहे}
{शिवाकारे मञ्चे परमशिवपर्यङ्कनिलयाम्}
{भजन्ति त्वां धन्याः कतिचन चिदानन्दलहरीम्}%8

\fourlineindentedshloka
{महीं मूलाधारे कमपि मणिपूरे हुतवहम्}
{स्थितं स्वाधिष्ठाने हृदि मरुतमाकाशमुपरि}
{मनोऽपि भ्रूमध्ये सकलमपि भित्वा कुलपथम्}
{सहस्रारे पद्मे सह रहसि पत्या विहरसे}%9

\fourlineindentedshloka
{सुधाधारासारैश्चरणयुगलान्तर्विगलितैः}
{प्रपञ्चं सिञ्चन्ती पुनरपि रसाम्नायमहसः}
{अवाप्य स्वां भूमिं भुजगनिभमध्युष्टवलयम्}
{स्वमात्मानं कृत्वा स्वपिषि कुलकुण्डे कुहरिणि}%10

\fourlineindentedshloka
{चतुर्भिः श्रीकण्ठैः शिवयुवतिभिः पञ्चभिरपि}
{प्रभिन्नाभिः शम्भोर्नवभिरपि मूलप्रकृतिभिः}
{चतुश्चत्वारिंशद्वसुदलकलाश्रत्रिवलय}
{त्रिरेखाभिः सार्धं तव शरणकोणाः परिणताः}%11

\fourlineindentedshloka
{त्वदीयं सौन्दर्यं तुहिनगिरिकन्ये तुलयितुम्}
{कवीन्द्राः कल्पन्ते कथमपि विरिञ्चिप्रभृतयः}
{यदालोकौत्सुक्यादमरललना यान्ति मनसा}
{तपोभिर्दुष्प्रापामपि गिरिशसायुज्यपदवीम्}%12

\fourlineindentedshloka
{नरं वर्षीयांसं नयनविरसं नर्मसु जडम्}
{तवापाङ्गालोके पतितमनुधावन्ति शतशः}
{गलद्वेणीबन्धाः कुचकलशविस्रस्तसिचया}
{हठात् त्रुट्यत्काञ्च्यो विगलितदुकूला युवतयः}%13

\fourlineindentedshloka
{क्षितौ षट्पञ्चाशद्-द्विसमधिकपञ्चाशदुदके}
{हुताशे द्वाषष्टिश्चतुरधिकपञ्चाशदनिले}
{दिवि द्विष्षट्त्रिंशन्मनसि च चतुष्षष्टिरिति ये}
{मयूखास्तेषामप्युपरि तव पादाम्बुजयुगम्}%14

\fourlineindentedshloka
{शरज्ज्योत्स्नाशुद्धां शशियुतजटाजूटमुकुटाम्}
{वरत्रासत्राणस्फटिकघुटिकापुस्तककराम्}
{सकृन्न त्वा नत्वा कथमिव सतां सन्निदधते}
{मधुक्षीरद्राक्षामधुरिमधुरीणाः कणितयः}%15

\fourlineindentedshloka
{कवीन्द्राणां चेतःकमलवनबालातपरुचिम्}
{भजन्ते ये सन्तः कतिचिदरुणामेव भवतीम्}
{विरिञ्चिप्रेयस्यास्तरुणतरशृङ्गारलहरी}
{गभीराभिर्वाग्भिर्विदधति सतां रञ्जनममी}%16

\fourlineindentedshloka
{सवित्रीभिर्वाचां शशिमणिशिलाभङ्गरुचिभिः}
{वशिन्याद्याभिस्त्वां सह जननि सञ्चिन्तयति यः}
{स कर्ता काव्यानां भवति महतां भङ्गिरुचिभिः}
{वचोभिर्वाग्देवीवदनकमलामोदमधुरैः}%17

\fourlineindentedshloka
{तनुच्छायाभिस्ते तरुणतरणिश्रीसरणिभिः}
{दिवं सर्वामुर्वीमरुणिमनिमग्नां स्मरति यः}
{भवन्त्यस्य त्रस्यद्वनहरिणशालीननयनाः}
{सहोर्वश्या वश्याः कति कति न गीर्वाणगणिकाः}%18

\fourlineindentedshloka
{मुखं बिन्दुं कृत्वा कुचयुगमधस्तस्य तदधो}
{हरार्धं ध्यायेद्योहरमहिषि ते मन्मथकलाम्}
{स सद्यः सङ्क्षोभं नयति वनिता इत्यतिलघु}
{त्रिलोकीमप्याशु भ्रमयति रवीन्दुस्तनयुगाम्}%19

\fourlineindentedshloka
{किरन्तीमङ्गेभ्यः किरणनिकुरुम्बामृतरसम्}
{हृदि त्वामाधत्ते हिमकरशिलामूर्तिमिव यः}
{स सर्पाणां दर्पं शमयति शकुन्ताधिप इव}
{ज्वरप्लुष्टान् दृष्ट्या सुखयति सुधाऽऽधारसिरया}%20

\fourlineindentedshloka
{तटिल्लेखातन्वीं तपनशशिवैश्वानरमयीम्}
{निषण्णां षण्णामप्युपरि कमलानां तव कलाम्}
{महापद्माटव्यां मृदितमलमायेन मनसा}
{महान्तः पश्यन्तो दधति परमाह्लादलहरीम्}%21

\fourlineindentedshloka
{भवानि त्वं दासे मयि वितर दृष्टिं सकरुणाम्}
{इति स्तोतुं वाञ्छन् कथयति भवानि त्वमिति यः}
{तदैव त्वं तस्मै दिशसि निजसायुज्यपदवीम्}
{मुकुन्दब्रह्मेन्द्रस्फुटमुकुटनीराजितपदाम्}%22

\fourlineindentedshloka
{त्वया हृत्वा वामं वपुरपरितृप्तेन मनसा}
{शरीरार्धं शम्भोरपरमपि शङ्के हृतमभूत्}
{यदेतत्त्वद्रूपं सकलमरुणाभं त्रिनयनम्}
{कुचाभ्यामानम्रं कुटिलशशिचूडालमुकुटम्}%23

\fourlineindentedshloka
{जगत्सूते धाता हरिरवति रुद्रः क्षपयते}
{तिरस्कुर्वन्नेतत्स्वमपि वपुरीशस्तिरयति}
{सदापूर्वः सर्वं तदिदमनुगृह्णाति च शिवः}
{तवाऽऽज्ञामालम्ब्य क्षणचलितयोर्भ्रूलतिकयोः}%24

\fourlineindentedshloka
{त्रयाणां देवानां त्रिगुणजनितानां तव शिवे}
{भवेत् पूजा पूजा तव चरणयोर्या विरचिता}
{तथा हि त्वत्पादोद्वहनमणिपीठस्य निकटे}
{स्थिता ह्येते शश्वन् मुकुलितकरोत्तंसमकुटाः}%25

\fourlineindentedshloka
{विरिञ्चिः पञ्चत्वं व्रजति हरिराप्नोति विरतिम्}
{विनाशं कीनाशो भजति धनदो याति निधनम्}
{वितन्द्री माहेन्द्री विततिरपि सम्मीलित-दृशाम्}
{महासंहारेऽस्मिन् विहरति सति त्वत्पतिरसौ}%26

\fourlineindentedshloka
{जपो जल्पः शिल्पं सकलमपि मुद्राविरचना}
{गतिः प्रादक्षिण्यक्रमणमशनाद्याहुतिविधिः}
{प्रणामः संवेशः सुखमखिलमात्मार्पणदृशा}
{सपर्यापर्यायस्तव भवतु यन्मे विलसितम्}%27

\fourlineindentedshloka
{सुधामप्यास्वाद्य प्रतिभयजरामृत्युहरिणीम्}
{विपद्यन्ते विश्वे विधिशतमखाद्या दिविषदः}
{करालं यत्क्ष्वेलं कबलितवतः कालकलना}
{न शम्भोस्तन्मूलं तव जननि ताटङ्कमहिमा}%28

\fourlineindentedshloka
{किरीटं वैरिञ्चं परिहर पुरः कैटभभिदः}
{कठोरे कोटीरे स्खलसि जहि जम्भारिमकुटम्}
{प्रणम्रेष्वेतेषु प्रसभमुपयातस्य भवनम्}
{भवस्याभ्युत्थाने तव परिजनोक्तिर्विजयते}%29

\fourlineindentedshloka
{स्वदेहोद्भूताभिर्घृणिभिरणिमाऽऽद्याभिरभितो}
{निषेव्ये नित्ये त्वामहमिति सदा भावयति यः}
{किमाश्चर्यं तस्य त्रिनयनसमृद्धिं तृणयतो}
{महासंवर्ताग्निर्विरचयति नीराजनविधिम्}%30

\fourlineindentedshloka
{चतुष्षष्ट्या तन्त्रैः सकलमतिसन्धाय भुवनम्}
{स्थितस्तत् तत्  सिद्धिप्रसवपरतन्त्रैः पशुपतिः}
{पुनस्त्वन्निर्बन्धादखिलपुरुषार्थैकघटना}
{स्वतन्त्रं ते तन्त्रं क्षितितलमवातीतरदिदम्}%31

\fourlineindentedshloka
{शिवः शक्तिः कामः क्षितिरथ रविः शीतकिरणः}
{स्मरो हंसः शक्रस्तदनु च परामारहरयः}
{अमी हृल्लेखाभिस्तिसृभिरवसानेषु घटिता}
{भजन्ते वर्णास्ते तव जननि नामावयवताम्}%32

\fourlineindentedshloka
{स्मरं योनिं लक्ष्मीं त्रितयमिदमादौ तव मनोः}
{निधायैके नित्ये निरवधिमहाभोगरसिकाः}
{भजन्ति त्वां चिन्तामणिगुणनिबद्धाक्षवलयाः}
{शिवाऽग्नौ जुह्वन्तः सुरभिघृतधाराऽऽहुतिशतैः}%33

\fourlineindentedshloka
{शरीरं त्वं शम्भोः शशिमिहिरवक्षोरुहयुगम्}
{तवाऽऽत्मानं मन्ये भगवति नवात्मानमनघम्}
{अतः शेषः शेषीत्ययमुभयसाधारणतया}
{स्थितः सम्बन्धो वां समरसपरानन्दपरयोः}%34

\fourlineindentedshloka
{मनस्त्वं व्योम त्वं मरुदसि मरुत्सारथिरसि}
{त्वमापस्त्वं भूमिस्त्वयि परिणतायां न हि परम्}
{त्वमेव स्वात्मानं परिणमयितुं विश्ववपुषा}
{चिदानन्दाकारं शिवयुवति भावेन बिभृषे}%35

\fourlineindentedshloka
{तवाऽऽज्ञाचक्रस्थं तपनशशिकोटिद्युतिधरम्}
{परं शम्भुं वन्दे परिमिलितपार्श्वं परचिता}
{यमाराध्यन् भक्त्या रविशशिशुचीनामविषये}
{निरालोकेऽलोके निवसति हि भालोकभवने}%36

\fourlineindentedshloka
{विशुद्धौ ते शुद्धस्फटिकविशदं व्योमजनकम्}
{शिवं सेवे देवीमपि शिवसमानव्यवसिताम्}
{ययोः कान्त्या यान्त्या शशिकिरणसारूप्यसरणिम्}
{विधूतान्तर्ध्वान्ता विलसति चकोरीव जगती}%37

\fourlineindentedshloka
{समुन्मीलत् संवित् कमलमकरन्दैकरसिकम्}
{भजे हंसद्वन्द्वं किमपि महतां मानसचरम्}
{यदालापादष्टादशगुणितविद्यापरिणतिः}
{यदादत्ते दोषाद्-गुणमखिलमद्भ्यः पय इव}%38

\fourlineindentedshloka
{तव स्वाधिष्ठाने हुतवहमधिष्ठाय निरतम्}
{तमीडे संवर्तं जननि महतीं तां च समयाम्}
{यदालोके लोकान् दहति महति क्रोधकलिते}
{दयार्द्रा यद्दृष्टिः शिशिरमुपचारं रचयति}%39

\fourlineindentedshloka
{तटित्त्वन्तं शक्त्या तिमिरपरिपन्थिस्फुरणया}
{स्फुरन्नानारत्नाभरणपरिणद्धेन्द्रधनुषम्}
{तव श्यामं मेघं कमपि मणिपूरैकशरणम्}
{निषेवे वर्षन्तं हरमिहिरतप्तं त्रिभुवनम्}%40

\fourlineindentedshloka
{तवाऽऽधारे मूले सह समयया लास्यपरया}
{नवात्मानं मन्ये नवरसमहाताण्डवनटम्}
{उभाभ्यामेताभ्यामुदयविधिमुद्दिश्य दयया}
{सनाथाभ्यां जज्ञे जनकजननीमज्जगदिदम्}%41

\mbox{}\\
\dnsub{सौन्दर्यलहरी}

\fourlineindentedshloka
{गतैर्माणिक्यत्वं गगनमणिभिः सान्द्रघटितम्}
{किरीटं ते हैमं हिमगिरिसुते कीर्तयति यः}
{स नीडेयच्छायाच्छुरणशबलं चन्द्रशकलम्}
{धनुः शौनासीरं किमिति न निबध्नाति धिषणाम्}%42

\fourlineindentedshloka
{धुनोतु ध्वान्तं नस्तुलितदलितेन्दीवरवनम्}
{घनस्निग्धश्लक्ष्णं चिकुरनिकुरम्बं तव शिवे}
{यदीयं सौरभ्यं सहजमुपलब्धुं सुमनसो}
{वसन्त्यस्मिन् मन्ये वलमथनवाटीविटपिनाम्}%43

\fourlineindentedshloka
{तनोतु क्षेमं नस्तव वदनसौन्दर्यलहरी}
{परीवाहस्रोतःसरणिरिव सीमन्तसरणिः}
{वहन्ती सिन्दूरं प्रबलकबरीभारतिमिर}
{द्विषां बृन्दैर्बन्दीकृतमिव नवीनार्ककिरणम्}%44

\fourlineindentedshloka
{अरालैः स्वाभाव्यादलिकलभसश्रीभिरलकैः}
{परीतं ते वक्त्रं परिहसति पङ्केरुहरुचिम्}
{दरस्मेरे यस्मिन् दशनरुचिकिञ्जल्करुचिरे}
{सुगन्धौ माद्यन्ति स्मरदहनचक्षुर्मधुलिहः}%45

\fourlineindentedshloka
{ललाटं लावण्यद्युतिविमलमाभाति तव यत्}
{द्वितीयं तन्मन्ये मकुटघटितं चन्द्रशकलम्}
{विपर्यासन्यासादुभयमपि सम्भूय च मिथः}
{सुधालेपस्यूतिः परिणमति राकाहिमकरः}%46

\fourlineindentedshloka
{भ्रुवौ भुग्ने  किञ्चिद्भुवनभयभङ्गव्यसनिनि}
{त्वदीये नेत्राभ्यां मधुकररुचिभ्यां धृतगुणम्}
{धनुर्मन्ये सव्येतरकरगृहीतं रतिपतेः}
{प्रकोष्ठे मुष्टौ च स्थगयति निगूढान्तरमुमे}%47

\fourlineindentedshloka
{अहः सूते सव्यं तव नयनमर्कात्मकतया}
{त्रियामां वामं ते सृजति रजनीनायकतया}
{तृतीया ते दृष्टिर्दरदलितहेमाम्बुजरुचिः}
{समाधत्ते सन्ध्यां दिवसनिशयोरन्तरचरीम्}%48

\fourlineindentedshloka
{विशाला कल्याणी स्फुटरुचिरयोध्या कुवलयैः}
{कृपाधाराधारा किमपि मधुरा भोगवतिका}
{अवन्ती दृष्टिस्ते बहुनगरविस्तारविजया}
{ध्रुवं तत्तन्नामव्यवहरणयोग्या विजयते}%49

\fourlineindentedshloka
{कवीनां सन्दर्भस्तबकमकरन्दैकरसिकम्}
{कटाक्षव्याक्षेपभ्रमरकलभौ कर्णयुगलम्}
{अमुञ्चन्तौ दृष्ट्वा तव नवरसास्वादतरलौ}
{असूयासंसर्गादलिकनयनं किञ्चिदरुणम्}%50

\fourlineindentedshloka
{शिवे शृङ्गारार्द्रा तदितरजने कुत्सनपरा}
{सरोषा गङ्गायां गिरिशचरिते विस्मयवती}
{हराहिभ्यो भीता सरसिरुहसौभाग्यजननी}
{सखीषु स्मेरा ते मयि जननि दृष्टिः सकरुणा}%51

\fourlineindentedshloka
{गते कर्णाभ्यर्णं गरुत इव पक्ष्माणि दधती}
{पुरां भेत्तुश्चित्तप्रशमरसविद्रावणफले}
{इमे नेत्रे गोत्राधरपतिकुलोत्तंसकलिके}
{तवाकर्णाकृष्टस्मरशरविलासं कलयतः}%52

\fourlineindentedshloka
{विभक्तत्रैवर्ण्यं व्यतिकरितलीलाञ्जनतया}
{विभाति त्वन्नेत्रत्रितयमिदमीशानदयिते}
{पुनः स्रष्टुं देवान् द्रुहिणहरिरुद्रानुपरतान्}
{रजः सत्त्वं बिभ्रत् तम इति गुणानां त्रयमिव}%53

\fourlineindentedshloka
{पवित्रीकर्तुं नः पशुपतिपराधीनहृदये}
{दयामित्रैर्नेत्रैररुणधवलश्यामरुचिभिः}
{नदः शोणो गङ्गा तपनतनयेति ध्रुवममुम्}
{त्रयाणां तीर्थानामुपनयसि सम्भेदमनघम्}%54

\fourlineindentedshloka
{निमेषोन्मेषाभ्यां प्रलयमुदयं याति जगती}
{तवेत्याहुः सन्तो धरणिधरराजन्यतनये}
{त्वदुन्मेषाज्जातं जगदिदमशेषं प्रलयतः}
{परित्रातुं शङ्के परिहृतनिमेषास्तव दृशः}%55

\fourlineindentedshloka
{तवापर्णे कर्णेजपनयनपैशुन्यचकिता}
{निलीयन्ते तोये नियतमनिमेषाः शफरिकाः}
{इयं च श्रीर्बद्धच्छदपुटकवाटं कुवलयम्}
{जहाति प्रत्यूषे निशि च विघटय्य प्रविशति}%56

\fourlineindentedshloka
{दृशा द्राघीयस्या दरदलितनीलोत्पलरुचा}
{दवीयांसं दीनं स्नपय कृपया मामपि शिवे}
{अनेनायं धन्यो भवति न च ते हानिरियता}
{वने वा हर्म्ये वा समकरनिपातो हिमकरः}%57

\fourlineindentedshloka
{अरालं ते पालीयुगलमगराजन्यतनये}
{न केषामाधत्ते कुसुमशरकोदण्डकुतुकम्}
{तिरश्चीनो यत्र श्रवणपथमुल्लङ्घ्य विलसन्}
{अपाङ्गव्यासङ्गो दिशति शरसन्धानधिषणाम्}%58

\fourlineindentedshloka
{स्फुरद्गण्डाभोगप्रतिफलितताटङ्कयुगलम्}
{चतुश्चक्रं मन्ये तव मुखमिदं मन्मथरथम्}
{यमारुह्य द्रुह्यत्यवनिरथम् अर्केन्दुचरणम्}
{महावीरो मारः प्रमथपतये सज्जितवते}%59

\fourlineindentedshloka
{सरस्वत्याः सूक्तीरमृतलहरीकौशलहरीः}
{पिबन्त्याः शर्वाणि श्रवणचुलुकाभ्यामविरलम्}
{चमत्कारश्लाघाचलितशिरसः कुण्डलगणो}
{झणत्कारैस्तारैः प्रतिवचनमाचष्ट इव ते}%60

\fourlineindentedshloka
{असौ नासावंशस्तुहिनगिरिवंशध्वजपटि}
{त्वदीयो नेदीयः फलतु फलमस्माकमुचितम्}
{वहत्यन्तर्मुक्ताः शिशिरकरनिश्वासगलितम्}
{समृद्ध्या यस्तासां बहिरपि च मुक्तामणिधरः}%61

\fourlineindentedshloka
{प्रकृत्याऽऽरक्तायास्तव सुदति दन्तच्छदरुचेः}
{प्रवक्ष्ये सादृश्यं जनयतु फलं विद्रुमलता}
{न बिम्बं त्वद्बिम्बप्रतिफलनरागादरुणितम्}
{तुलामध्यारोढुं कथमिव विलज्जेत कलया}%62

\fourlineindentedshloka
{स्मितज्योत्स्नाजालं तव वदनचन्द्रस्य पिबताम्}
{चकोराणामासीदतिरसतया चञ्चुजडिमा}
{अतस्ते शीतांशोरमृतलहरीमाम्लरुचयः}
{पिबन्ति स्वच्छन्दं निशि निशि भृशं काञ्जिकधिया}%63

\fourlineindentedshloka
{अविश्रान्तं पत्युर्गुणगणकथाऽऽम्रेडनजपा}
{जपापुष्पच्छाया तव जननि जिह्वा जयति सा}
{यदग्रासीनायाः स्फटिकदृषदच्छच्छविमयी}
{सरस्वत्या मूर्तिः परिणमति माणिक्यवपुषा}%64

\fourlineindentedshloka
{रणे जित्वा दैत्यानपहृतशिरस्त्रैः कवचिभिः}
{निवृत्तैश्चण्डांशत्रिपुरहरनिर्माल्यविमुखैः}
{विशाखेन्द्रोपेन्द्रैः शशिविशदकर्पूरशकला}
{विलीयन्ते मातस्तव वदनताम्बूलकबलाः}%65

\fourlineindentedshloka
{विपञ्च्या गायन्ती विविधमपदानं पशुपतेः}
{त्वयाऽऽरब्धे वक्तुं चलितशिरसा साधुवचने}
{तदीयैर्माधुर्यैरपलपिततन्त्रीकलरवाम्}
{निजां वीणां वाणी निचुलयति चोलेन निभृतम्}%66

\fourlineindentedshloka
{कराग्रेण स्पृष्टं तुहिनगिरिणा वत्सलतया}
{गिरीशेनोदस्तं मुहुरधरपानाकुलतया}
{करग्राह्यं शम्भोर्मुखमुकुरवृन्तं गिरिसुते}
{कथङ्कारं ब्रूमस्तव चुबुकमौपम्यरहितम्}%67

\fourlineindentedshloka
{भुजाश्लेषान् नित्यं पुरदमयितुः कण्टकवती}
{तव ग्रीवा धत्ते मुखकमलनालश्रियमियम्}
{स्वतः श्वेता कालागरुबहुलजम्बालमलिना}
{मृणालीलालित्यम् वहति यदधो हारलतिका}%68

\fourlineindentedshloka
{गले रेखास्तिस्रो गतिगमकगीतैकनिपुणे}
{विवाहव्यानद्धप्रगुणगुणसङ्ख्याप्रतिभुवः}
{विराजन्ते नानाविधमधुररागाकरभुवाम्}
{त्रयाणां ग्रामाणां स्थितिनियमसीमान इव ते}%69

\fourlineindentedshloka
{मृणालीमृद्वीनां तव भुजलतानां चतसृणाम्}
{चतुर्भिः सौन्दर्यं सरसिजभवः स्तौति वदनैः}
{नखेभ्यः सन्त्रस्यन् प्रथममथनादन्धकरिपोः}
{चतुर्णां शीर्षाणां सममभयहस्तार्पणधिया}%70

\fourlineindentedshloka
{नखानामुद्द्योतैर्नवनलिनरागं विहसताम्}
{कराणां ते कान्तिं कथय कथयामः कथमुमे}
{कयाचिद्वा साम्यं भजतु कलया हन्त कमलम्}
{यदि क्रीडल्लक्ष्मीचरणतललाक्षारसचणम्}%71

\fourlineindentedshloka
{समं देवि स्कन्दद्विपवदनपीतं स्तनयुगम्}
{तवेदं नः खेदं हरतु सततं प्रस्नुतमुखम्}
{यदालोक्याशङ्काऽऽकुलितहृदयो हासजनकः}
{स्वकुम्भौ हेरम्बः परिमृशति हस्तेन झटिति}%72

\fourlineindentedshloka
{अमू ते वक्षोजावमृतरसमाणिक्यकुतुपौ}
{न सन्देहस्पन्दो नगपतिपताके मनसि नः}
{पिबन्तौ तौ यस्मादविदितवधूसङ्गरसिकौ}
{कुमारावद्यापि द्विरदवदनक्रौञ्चदलनौ}%73

\fourlineindentedshloka
{वहत्यम्ब स्तम्बेरमदनुजकुम्भप्रकृतिभिः}
{समारब्धां मुक्तामणिभिरमलां हारलतिकाम्}
{कुचाभोगो बिम्बाधररुचिभिरन्तः शबलिताम्}
{प्रतापव्यामिश्रां पुरदमयितुः कीर्तिमिव ते}%74

\fourlineindentedshloka
{तव स्तन्यं मन्ये धरणिधरकन्ये हृदयतः}
{पयःपारावारः परिवहति सारस्वतमिव}
{दयावत्या दत्तं द्रविडशिशुरास्वाद्य तव यत्}
{कवीनां प्रौढानामजनि कमनीयः कवयिता}%75

\fourlineindentedshloka
{हरक्रोधज्वालाऽऽवलिभिरवलीढेन वपुषा}
{गभीरे ते नाभीसरसि कृतसङ्गो मनसिजः}
{समुत्तस्थौ तस्मादचलतनये धूमलतिका}
{जनस्तां जानीते तव जननि रोमावलिरिति}%76

\fourlineindentedshloka
{यदेतत् कालिन्दीतनुतरतरङ्गाकृति शिवे}
{कृशे मध्ये किञ्चिज्जननि तव तद्भाति सुधियाम्}
{विमर्दादन्योऽन्यं कुचकलशयोरन्तरगतम्}
{तनूभूतं व्योम प्रविशदिव नाभिं कुहरिणीम्}%77

\fourlineindentedshloka
{स्थिरो गङ्गावर्तः स्तनमुकुलरोमावलिलता}
{निजावालं कुण्डं कुसुमशरतेजोहुतभुजः}
{रतेर्लीलागारं किमपि तव नाभिर्गिरिसुते}
{बिलद्वारं सिद्धेर्गिरिशनयनानां विजयते}%78

\fourlineindentedshloka
{निसर्गक्षीणस्य स्तनतटभरेण क्लमजुषो}
{नमन्मूर्तेर्नारीतिलक शनकैस्त्रुट्यत इव}
{चिरं ते मध्यस्य त्रुटिततटिनीतीरतरुणा}
{समावस्थास्थेम्नो भवतु कुशलं शैलतनये}%79

\fourlineindentedshloka
{कुचौ सद्यःस्विद्यत्तटघटितकूर्पासभिदुरौ}
{कषन्तौ दोर्मूले कनककलशाभौ कलयता}
{तव त्रातुं भङ्गादलमिति वलग्नं तनुभुवा}
{त्रिधा नद्धं देवि त्रिवलि लवलीवल्लिभिरिव}%80

\fourlineindentedshloka
{गुरुत्वं विस्तारं क्षितिधरपतिः पार्वति निजात्}
{नितम्बादाच्छिद्य त्वयि हरणरूपेण निदधे}
{अतस्ते विस्तीर्णो गुरुरयमशेषां वसुमतीम्}
{नितम्बप्राग्भारः स्थगयति लघुत्वं नयति च}%81

\fourlineindentedshloka
{करीन्द्राणां शुण्डान् कनककदलीकाण्डपटलीम्}
{उभाभ्यामूरुभ्यामुभयमपि निर्जित्य भवति}
{सुवृत्ताभ्यां पत्युः प्रणतिकठिनाभ्यां गिरिसुते}
{विधिज्ञे जानुभ्यां विबुधकरिकुम्भद्वयमसि}%82

\fourlineindentedshloka
{पराजेतुं रुद्रं द्विगुणशरगर्भौ गिरिसुते}
{निषङ्गौ जङ्घे ते विषमविशिखो बाढमकृत}
{यदग्रे दृश्यन्ते दश शरफलाः पादयुगली}
{नखाग्रच्छद्मानः सुरमकुटशाणैकनिशिताः}%83

\fourlineindentedshloka
{श्रुतीनां मूर्धानो दधति तव यौ शेखरतया}
{ममाप्येतौ मातः शिरसि दयया धेहि चरणौ}
{ययोः पाद्यं पाथः पशुपतिजटाजूटतटिनी}
{ययोर्लाक्षालक्ष्मीररुणहरिचूडामणिरुचिः}%84

\fourlineindentedshloka
{नमोवाकं ब्रूमो नयनरमणीयाय पदयोः}
{तवास्मै द्वन्द्वाय स्फुटरुचिरसालक्तकवते}
{असूयत्यत्यन्तं यदभिहननाय स्पृहयते}
{पशूनामीशानः प्रमदवनकङ्केलितरवे}%85

\fourlineindentedshloka
{मृषा कृत्वा गोत्रस्खलनमथ वैलक्ष्यनमितम्}
{ललाटे भर्तारं चरणकमले ताडयति ते}
{चिरादन्तःशल्यं दहनकृतमुन्मूलितवता}
{तुलाकोटिक्वाणैः किलिकिलितमीशानरिपुणा}%86

\fourlineindentedshloka
{हिमानीहन्तव्यं हिमगिरिनिवासैकचतुरौ}
{निशायां निद्राणां निशि चरमभागे च विशदौ}
{वरं लक्ष्मीपात्रं श्रियमतिसृजन्तौ समयिनाम्}
{सरोजं त्वत्पादौ जननि जयतश्चित्रमिह किम्}%87

\fourlineindentedshloka
{पदं ते कीर्तीनां प्रपदमपदं देवि विपदाम्}
{कथं नीतं सद्भिः कठिनकमठीकर्परतुलाम्}
{कथं वा बाहुभ्यामुपयमनकाले पुरभिदा}
{यदादाय न्यस्तं दृषदि दयमानेन मनसा}%88

\fourlineindentedshloka
{नखैर्नाकस्त्रीणां करकमलसङ्कोचशशिभिः}
{तरूणां दिव्यानां हसत इव ते चण्डि चरणौ}
{फलानि स्वःस्थेभ्यः किसलयकराग्रेण ददताम्}
{दरिद्रेभ्यो भद्रां श्रियमनिशमह्नाय ददतौ}%89

\fourlineindentedshloka
{ददाने दीनेभ्यः श्रियमनिशमाशानुसदृशीम्}
{अमन्दं सौन्दर्यप्रकरमकरन्दं विकिरति}
{तवास्मिन् मन्दारस्तबकसुभगे यातु चरणे}
{निमज्जन् मज्जीवः करणचरणः षट्चरणताम्}%90

\fourlineindentedshloka
{पदन्यासक्रीडापरिचयमिवारब्धुमनसः}
{स्खलन्तस्ते खेलं भवनकलहंसा न जहति}
{अतस्तेषां शिक्षां सुभगमणिमञ्जीररणित-}
{च्छलादाचक्षाणं चरणकमलं चारुचरिते}%91

\fourlineindentedshloka
{गतास्ते मञ्चत्वं द्रुहिणहरिरुद्रेश्वरभृतः}
{शिवः स्वच्छच्छायाघटितकपटप्रच्छदपटः}
{त्वदीयानां भासां प्रतिफलनरागारुणतया}
{शरीरी शृङ्गारो रस इव दृशां दोग्धि कुतुकम्}%92

\fourlineindentedshloka
{अराला केशेषु प्रकृतिसरला मन्दहसिते}
{शिरीषाभा चित्ते दृषदुपलशोभा कुचतटे}
{भृशं तन्वी मध्ये पृथुरुरसिजारोहविषये}
{जगत्त्रातुं शम्भोर्जयति करुणा काचिदरुणा}%93

\fourlineindentedshloka
{कलङ्कः कस्तूरी रजनिकरबिम्बं जलमयम्}
{कलाभिः कर्पूरैर्मरकतकरण्डं निबिडितम्}
{अतस्त्वद्भोगेन प्रतिदिनमिदं रिक्तकुहरम्}
{विधिर्भूयो भूयो निबिडयति नूनं तव कृते}%94

\fourlineindentedshloka
{पुरारातेरन्तःपुरमसि ततस्त्वच्चरणयोः}
{सपर्यामर्यादा तरलकरणानामसुलभा}
{तथा ह्येते नीताः शतमखमुखाः सिद्धिमतुलाम्}
{तव द्वारोपान्तस्थितिभिरणिमाद्याभिरमराः}%95

\fourlineindentedshloka
{कलत्रं वैधात्रं कति कति भजन्ते न कवयः}
{श्रियो देव्याः को वा न भवति पतिः कैरपि धनैः}
{महादेवं हित्वा तव सति सतीनामचरमे}
{कुचाभ्यामासङ्गः कुरवकतरोरप्यसुलभः}%96

\fourlineindentedshloka
{गिरामाहुर्देवीं द्रुहिणगृहिणीमागमविदो}
{हरेः पत्नीं पद्मां हरसहचरीमद्रितनयाम्}
{तुरीया काऽपि त्वं दुरधिगमनिःसीममहिमा}
{महामाया विश्वं भ्रमयसि परब्रह्ममहिषि}%97

\fourlineindentedshloka
{कदा काले मातः कथय कलितालक्तकरसम्}
{पिबेयं विद्यार्थी तव चरणनिर्णेजनजलम्}
{प्रकृत्या मूकानामपि च कविताकारणतया}
{कदा धत्ते वाणीमुखकमलताम्बूलरसताम्}%98

\fourlineindentedshloka
{सरस्वत्या लक्ष्म्या विधिहरिसपत्नो विहरते}
{रतेः पातिव्रत्यं शिथिलयति रम्येण वपुषा}
{चिरं जीवन्नेव क्षपितपशुपाशव्यतिकरः}
{परानन्दाभिख्यं रसयति रसं त्वद्भजनवान्}%99

\fourlineindentedshloka
{प्रदीपज्वालाभिर्दिवसकरनीराजनविधिः}
{सुधासूतेश्चन्द्रोपलजललवैरर्घ्यरचना}
{स्वकीयैरम्भोभिः सलिलनिधिसौहित्यकरणम्}
{त्वदीयाभिर्वाग्भिस्तव जननि वाचां स्तुतिरियम्}%103

{॥ इति श्रीमच्छङ्कराचार्यविरचिता सौन्दर्यलहरी सम्पूर्णा॥}

\closesection

\fourlineindentedshloka*
{समानीतः पद्‌भ्यां मणिमुकुरतामम्बरमणिः}
{भयादास्यादन्तःस्तिमितकिरणश्रेणिमसृणः}
{दधाति त्वद्वक्त्रप्रतिफलनमश्रान्तविकचम्}
{निरातङ्कं चन्द्रान्निजहृदयपङ्केरुहमिव}%100

\fourlineindentedshloka*
{समुद्भूतस्थूलस्तनभरमुरश्चारु हसितम्}
{कटाक्षे कन्दर्पः कतिचन कदम्बद्युति वपुः}
{हरस्य त्वद्भ्रान्तिं मनसि जनयन्तः समयिनो}
{भवत्या ये भक्ताः परिणतिरमीषामियमुमे}%101

\fourlineindentedshloka*
{निधे नित्यस्मेरे निरवधिगुणे नीतिनिपुणे}
{निराघाटज्ञाने नियमपरचित्तैकनिलये}
{नियत्या निर्मुक्ते निखिलनिगमान्तस्तुतपदे}
{निरातङ्के नित्ये निगमय ममापि स्तुतिमिमाम्}%102

\closesection
\end{center}
