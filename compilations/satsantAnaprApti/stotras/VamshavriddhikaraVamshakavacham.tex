\sect{वंशवृद्धिकरं वंशकवचम्}

\centerline{श्री-शनैश्चर उवाच}

\twolineshloka
{भगवन् देवदेवेश कृपया त्वं जगत्प्रभो}
{वंशाख्यकवचं ब्रूहि मह्यं शिष्याय तेऽनघ}%१॥
{यस्य प्रभावाद्देवेश वंशवृद्धिर्हि जायते॥}

\centerline{श्री-सूर्य उवाच}

\twolineshloka
{शृणु पुत्र प्रवक्ष्यामि वंशाख्यं कवचं शुभम्}
{सन्तानवृद्धिर्यत्पाठाद्गर्भरक्षा सदा नृणाम्}%२॥

\twolineshloka
{वन्ध्याऽपि लभते पुत्रं काकवन्ध्या सुतैर्युता}
{मृतवत्सा सपुत्रा स्यात् स्रवद्गर्भा स्थिरप्रजा}%३॥

\twolineshloka
{अपुष्पा पुष्पिणी यस्य धारणाञ्च सुखप्रसूः}
{कन्याप्रजा पुत्रिणी स्यादेतत्स्तोत्रप्रभावतः}%४॥


\twolineshloka
{भूतप्रेतादिजा बाधा या बाधा कुलदोषजा}
{ग्रहबाधा देवबाधा बाधा शत्रुकृता च या}%५॥

\twolineshloka
{भस्मी भवन्ति सर्वास्ताः कवचस्य प्रभावतः}
{सर्वे रोगा विनश्यन्ति सर्वे बालग्रहाश्च ये}%६॥

\centerline{॥कवचम्॥}

\twolineshloka
{पूर्वे रक्षतु वाराही चाऽऽग्नेय्यामम्बिका स्वयम्}
{दक्षिणे चण्डिका रक्षेन्नैरृत्यां शववाहिनी}%७॥

\twolineshloka
{वाराही पश्चिमे रक्षेद्वायव्यां च महेश्वरी}
{उत्तरे वैष्णवी रक्षेदीशाने सिंहवाहिनी}%८॥

\twolineshloka
{ऊर्ध्वे तु शारदा रक्षेदधो रक्षतु पार्वती}
{शाकम्भरी शिरो रक्षेन्मुखं रक्षतु भैरवी}%९॥

\twolineshloka
{कण्ठं रक्षतु चामुण्डा हृदयं रक्षताच्छिवा}
{ईशानी च भुजौ रक्षेत् कुक्षिं नाभिं च कालिका}%१०॥

\twolineshloka
{अपर्णा ह्युदरं रक्षेत् कटिं बस्तिं शिवप्रिया}
{ऊरू रक्षतु कौमारी जया जानुद्वयं तथा}%११॥

\twolineshloka
{गुल्फौ पादौ सदा रक्षेद्ब्रह्माणी परमेश्वरी}
{सर्वाङ्गानि सदा रक्षेद्दुर्गा दुर्गार्तिनाशिनी}%१२॥

\twolineshloka
{नमो देव्यै महादेव्यै दुर्गायै सततं नमः}
{पुत्रसौख्यं देहि देहि गर्भरक्षां कुरुष्व नः}%१३॥

ॐ ह्रीं ह्रीं ह्रीं श्रीं श्रीं श्रीं ऐं ऐं ऐं महाकाली-महालक्ष्मी-\\
महासरस्वती-रूपायै नवकोटि-मूर्त्यै दुर्गायै नमः। ह्रीं ह्रीं ह्रीं\\
दुर्गार्तिनाशिनि सन्तानसौख्यं देहि देहि वन्ध्यत्वं मृतवत्सत्वं च\\
हर हर गर्भरक्षां कुरु कुरु सकलां बाधां कुलजां बाह्यजाम्\\
कृतामकृतां च नाशय नाशय सर्वगात्राणि रक्ष रक्ष \\
गर्भे पोषय पोषय सर्वोपद्रवं शोषय शोषय स्वाहा॥

\twolineshloka
{अनेन कवचेनाङ्गं सप्तवाराभिमन्त्रित्रम्}
{ऋतुस्नाता जलं पीत्वा भवेद्गर्भवती ध्रुवम्}%१४॥

\twolineshloka
{गर्भपातभये पीत्वा दृढगर्भा प्रजायते}
{अनेन कवचेनाथ मार्जिताया निशागमे}%१५॥

\twolineshloka
{सर्वबाधा विनिर्मुक्ता गर्भिणी स्यान्न संशयः}
{अनेन कवचेनेह ग्रन्थितं रक्तदोरकम्}%१६॥

\twolineshloka
{कटिदेशे धारयन्ती सुपुत्रसुखभागिनी}
{असूतपुत्रमिन्द्राणी जयन्तं यत्प्रभावतः}%१७॥

\threelineshloka
{गुरूपदिष्टं वंशाख्यं कवचं तदिदं सखे}
{गुह्याद्गुह्यतरं चेदं न प्रकाश्यं हि सर्वतः}
{धारणात् पठनादस्य वंशच्छेदो न जायते}%१८॥

\fourlineindentedshloka
{बाला विनश्यन्ति पतन्ति गर्भाः}
{तत्राबलाः कष्टयुताश्च वन्ध्याः}
{बालग्रहैर्भूतगणैश्च रोगैः}
{न यत्र धर्माचरणं गृहे स्यात्}

{॥इति~श्रीज्ञानभास्करे वंशवृद्धिकरं वंशकवचं सम्पूर्णम्॥}