% !TeX program = XeLaTeX
% !TeX root = ../../shloka.tex

\sect{अगस्त्यकृत-रामस्तुतिः}


\addtocounter{shlokacount}{9}
\uvacha{अगस्त्य उवाच}

\twolineshloka
{शीघ्रमानय भद्रं ते रामं मम हृदिस्थितम्}
{तमेव ध्यायमानोऽहं काङ्क्षमाणोऽत्र संस्थितः} %3-10

\twolineshloka
{इत्युक्त्वा स्वयमुत्थाय मुनिभिः सहितो द्रुतम्}
{अभ्यगात्परया भक्त्या गत्वा राममथाब्रवीत्} %3-11

\twolineshloka
{आगच्छ राम भद्रं ते दिष्ट्या तेऽद्य समागमः}
{प्रियातिथिर्मम प्राप्तोऽस्यद्य मे सफलं दिनम्} %3-12

\twolineshloka
{रामोऽपि मुनिमायान्तं दृष्ट्वा हर्षसमाकुलः}
{सीतया लक्ष्मणेनापि दण्डवत्पतितो भुवि} %3-13

\twolineshloka
{द्रुतमुत्थाप्य मुनिराड् राममालिङ्ग्य भक्तितः}
{तद्गात्रस्पर्शजाह्लादस्रवन्नेत्रजलाकुलः} %3-14

\twolineshloka
{गृहीत्वा करमेकेन करेण रघुनन्दनम्}
{जगाम स्वाश्रमं हृष्टो मनसा मुनिपुङ्गवः} %3-15

\twolineshloka
{सुखोपविष्टं सम्पूज्य पूजया बहुविस्तरम्}
{भोजयित्वा यथान्यायं भोज्यैर्वन्यैरनेकधा} %3-16

\twolineshloka
{सुखोपविष्टमेकान्ते रामं शशिनिभाननम्}
{कृताञ्जलिरुवाचेदमगस्त्यो भगवानृषिः} %3-17

\twolineshloka
{त्वदागमनमेवाहं प्रतीक्षन् समवस्थितः}
{यदा क्षीरसमुद्रान्ते ब्रह्मणा प्रार्थितः पुरा} %3-18

\threelineshloka
{भूमेर्भारापनुत्त्यर्थं रावणस्य वधाय च}
{तदादि दर्शनाकाङ्क्षी तव राम तपश्चरन्}
{वसामि मुनिभिः सार्धं त्वामेव परिचिन्तयन्} %3-19

\twolineshloka
{सृष्टेः प्रागेक एवासीर्निर्विकल्पोऽनुपाधिकः}
{त्वदाश्रया त्वद्विषया माया ते शक्तिरुच्यते} %3-20

\twolineshloka
{त्वामेव निर्गुणं शक्तिरावृणोति यदा तदा}
{अव्याकृतमिति प्राहुर्वेदान्तपरिनिष्ठिताः} %3-21

\twolineshloka
{मूलप्रकृतिरित्येके प्राहुर्मायेति केचन}
{अविद्या संसृतिर्बन्ध इत्यादि बहुधोच्यते} %3-22

\twolineshloka
{त्वया सङ्क्षोभ्यमाणा सा महत्तत्त्वं प्रसूयते}
{महत्तत्त्वादहङ्कारस्त्वया सञ्चोदितादभूत्} %3-23

\twolineshloka
{अहङ्कारो महत्तत्त्वसंवृतस्त्रिविधोऽभवत्}
{सात्त्विको राजसश्चैव तामसश्चेति भण्यते} %3-24

\twolineshloka
{तामसात्सूक्ष्मतन्मात्राण्यासन् भूतान्यतः परम्}
{स्थूलानि क्रमशो राम क्रमोत्तरगुणानि ह} %3-25

\twolineshloka
{राजसानीन्द्रियाण्येव सात्त्विका देवता मनः}
{तेभ्योऽभवत्सूत्ररूपं लिङ्गं सर्वगतं महत्} %3-26

\twolineshloka
{ततो विराट् समुत्पन्नः स्थूलाद्भूतकदम्बकात्}
{विराजः पुरुषात्सर्वं जगत्स्थावरजङ्गमम्} %3-27

\twolineshloka
{देवतिर्यङ्मनुष्याश्च कालकर्मक्रमेण तु}
{त्वं रजोगुणतो ब्रह्मा जगतः सर्वकारणम्} %3-28

\twolineshloka
{सत्त्वाद्विष्णुस्त्वमेवास्य पालकः सद्भिरुच्यते}
{लये रुद्रस्त्वमेवास्य त्वन्मायागुणभेदतः} %3-29

\twolineshloka
{जाग्रत्स्वप्नसुषुप्त्याख्या वृत्तयो बुद्धिजैर्गुणैः}
{तासां विलक्षणो राम त्वं साक्षी चिन्मयोऽव्ययः} %3-30

\twolineshloka
{सृष्टिलीलां यदा कर्तुमीहसे रघुनन्दन}
{अङ्गीकरोषि मायां त्वं तदा वै गुणवानिव} %3-31

\threelineshloka
{राम माया द्विधा भाति विद्याऽविद्येति ते सदा}
{प्रवृत्तिमार्गनिरता अविद्यावशवर्तिनः}
{निवृत्तिमार्गनिरता वेदान्तार्थविचारकाः} %3-32

\threelineshloka
{त्वद्भक्तिनिरता ये च ते वै विद्यामयाः स्मृताः}
{अविद्यावशगा ये तु नित्यं संसारिणश्च ते}
{विद्याभ्यासरता ये तु नित्यमुक्तास्त एव हि} %3-33

\twolineshloka
{लोके त्वद्भक्तिनिरतास्त्वन्मन्त्रोपासकाश्च ये}
{विद्या प्रादुर्भवेत्तेषां नेतरेषां कदाचन} %3-34

\twolineshloka
{अतस्त्वद्भक्तिसम्पन्ना मुक्ता एव न संशयः}
{त्वद्भक्त्यमृतहीनानां मोक्षः स्वप्नेऽपि नो भवेत्} %3-35

\twolineshloka
{किं राम बहुनोक्तेन सारं किञ्चिद्ब्रवीमि ते}
{साधुसङ्गतिरेवात्र मोक्षहेतुरुदाहृता} %3-36

\twolineshloka
{साधवः समचित्ता ये निःस्पृहा विगतैषिणः}
{दान्ताः प्रशान्तास्त्वद्भक्ता निवृत्ताखिलकामनाः} %3-37

\twolineshloka
{इष्टप्राप्तिविपत्त्योश्च समाः सङ्गविवर्जिताः}
{सन्न्यस्ताखिलकर्माणः सर्वदा ब्रह्मतत्पराः} %3-38

\twolineshloka
{यमादिगुणसम्पन्नाः सन्तुष्टा येन केनचित्}
{सत्सङ्गमो भवेद्यर्हि त्वत्कथाश्रवणे रतिः} %3-39

\twolineshloka
{समुदेति ततो भक्तिस्त्वयि राम सनातने}
{त्वद्भक्तावुपपन्नायां विज्ञानं विपुलं स्फुटम्} %3-40

\twolineshloka
{उदेति मुक्तिमार्गोऽयमाद्यश्चतुरसेवितः}
{तस्माद्राघव सद्भक्तिस्त्वयि मे प्रेमलक्षणा} %3-41

\twolineshloka
{सदा भूयाद्धरे सङ्गस्त्वद्भक्तेषु विशेषतः}
{अद्य मे सफलं जन्म भवत्सन्दर्शनादभूत्} %3-42

\threelineshloka
{अद्य मे क्रतवः सर्वे बभूवुः सफलाः प्रभो}
{दीर्घकालं मया तप्तमनन्यमतिना तपः}
{तस्येह तपसो राम फलं तव यदर्चनम्} %3-43

\twolineshloka
{सदा मे सीतया सार्धं हृदये वस राघव}
{गच्छतस्तिष्ठतो वाऽपि स्मृतिः स्यान्मे सदा त्वयि} %3-44

\twolineshloka
{इति स्तुत्वा रमानाथमगस्त्यो मुनिसत्तमः}
{ददौ चापं महेन्द्रेण रामार्थे स्थापितं पुरा} %3-45

\twolineshloka
{अक्षय्यौ बाणतूणीरौ खड्गो रत्नविभूषितः}
{जहि राघव भूभारभूतं राक्षसमण्डलम्} %3-46

\twolineshloka
{यदर्थमवतीर्णोऽसि मायया मनुजाकृतिः}
{इतो योजनयुग्मे तु पुण्यकाननमण्डितः} %3-47

\twolineshloka
{अस्ति पञ्चवटीनाम्ना आश्रमो गौतमीतटे}
{नेतव्यस्तत्र ते कालः शेषो रघुकुलोद्वह} %3-48

\onelineshloka
{तत्रैव बहुकार्याणि देवानां कुरु सत्पते} %3-49


\fourlineindentedshloka
{श्रुत्वा तदाऽगस्त्यसुभाषितं वचः}
{स्तोत्रं च तत्त्वार्थसमन्वितं विभुः}
{मुनिं समाभाष्य मुदान्वितो ययौ}
{प्रदर्शितं मार्गमशेषविद्धरिः} %3-50

{॥इति श्रीमदध्यात्मरामायणे उमामहेश्वरसंवादे अरण्यकाण्डे
तृतीये सर्गे अगस्त्यकृतं श्री-रामस्तुतिः सम्पूर्णा॥}
