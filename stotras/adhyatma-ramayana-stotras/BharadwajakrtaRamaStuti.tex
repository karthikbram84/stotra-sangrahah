% !TeX program = XeLaTeX
% !TeX root = ../../shloka.tex

\sect{भरद्वाजकृत-रामस्तुतिः}

\uvacha{श्री-महादेव उवाच}
\addtocounter{shlokacount}{9}

\twolineshloka
{त्वं ब्रह्म परमं साक्षादादिमध्यान्तवर्जितः}
{त्वमग्रे सलिलं सृष्ट्वा तत्र सुप्तोऽसि भूतकृत्} %14-21

\twolineshloka
{नारायणोऽसि विश्वात्मन्नराणामन्तरात्मकः}
{त्वन्नाभिकमलोत्पन्नो ब्रह्मा लोकपितामहः} %14-22

\twolineshloka
{अतस्त्वं जगतामीशः सर्वलोकनमस्कृतः}
{त्वं विष्णुर्जानकी लक्ष्मीः शेषोऽयं लक्ष्मणाभिधः} %14-23

\twolineshloka
{आत्मना सृजसीदं त्वमात्मन्येवाऽऽत्ममायया}
{न सज्जसे नभोवत्त्वं चिच्छक्त्या सर्वसाक्षिकः} %14-24

\twolineshloka
{बहिरन्तश्च भूतानां त्वमेव रघुनन्दन}
{पूर्णोऽपि मूढदृष्टीनां विच्छिन्न इव लक्ष्यसे} %14-25

\twolineshloka
{जगत्त्वं जगदाधारस्त्वमेव परिपालकः}
{त्वमेव सर्वभूतानां भोक्ता भोज्यं जगत्पते} %14-26

\twolineshloka
{दृश्यते श्रूयते यद्यत्स्मर्यते वा रघूत्तम}
{त्वमेव सर्वमखिलं त्वद्विनाऽन्यन्न किञ्चन} %14-27

\twolineshloka
{माया सृजति लोकांश्च स्वगुणैरहमादिभिः}
{त्वच्छक्तिप्रेरिता राम तस्मात्त्वय्युपचर्यते} %14-28

\twolineshloka
{यथा चुम्बकसान्निध्याच्चलन्त्येवायसादयः}
{जडास्तथा त्वया दृष्टा माया सृजति वै जगत्} %14-29

\twolineshloka
{देहद्वयमदेहस्य तव विश्वं रिरक्षिषोः}
{विराट् स्थूलं शरीरं ते सूत्रं सूक्ष्ममुदाहृतम्} %14-30

\twolineshloka
{विराजः सम्भवन्त्येते अवताराः सहस्रशः}
{कार्यान्ते प्रविशन्त्येव विराजं रघुनन्दन} %14-31

\twolineshloka
{अवतारकथां लोके ये गायन्ति गृणन्ति च}
{अनन्यमनसो मुक्तिस्तेषामेव रघूत्तम} %14-32

\twolineshloka
{त्वं ब्रह्मणा पुरा भूमेर्भारहाराय राघव}
{प्रार्थितस्तपसा तुष्टस्त्वं जातोऽसि रघोः कुले} %14-33

\twolineshloka
{देवकार्यमशेषेण कृतं ते राम दुष्करम्}
{बहुवर्षसहस्राणि मानुषं देहमाश्रितः} %14-34

\twolineshloka
{कुर्वन् दुष्करकर्माणि लोकद्वयहिताय च}
{पापहारीणि भुवनं यशसा पूरयिष्यसि} %14-35

\twolineshloka
{प्रार्थयामि जगन्नाथ पवित्रं कुरु मे गृहम्}
{स्थित्वाऽद्य भुक्त्वा सबलः श्वो गमिष्यसि पत्तनम्} %14-36


{॥इति श्रीमदध्यात्मरामायणे उमामहेश्वरसंवादे युद्धकाण्डे चतुर्दशे  सर्गे 
भरद्वाजकृत-रामस्तुतिः सम्पूर्णः॥}
