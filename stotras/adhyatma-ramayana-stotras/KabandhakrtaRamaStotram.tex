% !TeX program = XeLaTeX
% !TeX root = ../../shloka.tex

\sect{कबन्धकृत-रामस्तोत्रम्}


\addtocounter{shlokacount}{29}
\uvacha{गन्धर्व उवाच}

\twolineshloka
{स्तोतुमुत्सहते मेऽद्य मनो रामातिसम्भ्रमात्}
{त्वामनन्तमनाद्यन्तं मनोवाचामगोचरम्} %9-30

\threelineshloka
{सूक्ष्मं ते रूपमव्यक्तं देहद्वयविलक्षणम्}
{दृग्रूपमितरत्सर्वं दृश्यं जडमनात्मकम्}
{तत्कथं त्वां विजानीयाद्व्यतिरिक्तं मनः प्रभो} %9-31

\twolineshloka
{बुद्ध्यात्माभासयोरैक्यं जीव इत्यभिधीयते}
{बुद्ध्यादि साक्षी ब्रह्मैव तस्मिन्निर्विषयेऽखिलम्} %9-32

\twolineshloka
{आरोप्यतेऽज्ञानवशान्निर्विकारेऽखिलात्मनि}
{हिरण्यगर्भस्ते सूक्ष्मं देहं स्थूलं विराट् स्मृतम्} %9-33

\twolineshloka
{भावनाविषयो राम सूक्ष्मं ते ध्यातृमङ्गलम्}
{भूतं भव्यं भविष्यच्च यत्रेदं दृश्यते जगत्} %9-34

\twolineshloka
{स्थूलेऽण्डकोशे देहे ते महदादिभिरावृते}
{सप्तभिरुत्तरगुणैर्वैराजो धारणाश्रयः} %9-35

\twolineshloka
{त्वमेव सर्वकैवल्यं लोकास्तेऽवयवाः स्मृताः}
{पातालं ते पादमूलं पार्ष्णिस्तव महातलम्} %9-36

\twolineshloka
{रसातलं ते गुल्फौ तु तलातलमितीर्यते}
{जानुनी सुतलं राम ऊरू ते वितलं तथा} %9-37

\twolineshloka
{अतलं च मही राम जघनं नाभिगं नभः}
{उरःस्थलं ते ज्योतींषि ग्रीवा ते मह उच्यते} %9-38

\twolineshloka
{वदनं जनलोकस्ते तपस्ते शङ्खदेशगम्}
{सत्यलोको रघुश्रेष्ठ शीर्षण्यास्ते सदा प्रभो} %9-39

\twolineshloka
{इन्द्रादयो लोकपाला बाहवस्ते दिशः श्रुती}
{अश्विनौ नासिके राम वक्त्रं तेऽग्निरुदाहृतः} %9-40

\twolineshloka
{चक्षुस्ते सविता राम मनश्चन्द्र उदाहृतः}
{भ्रूभङ्ग एव कालस्ते बुद्धिस्ते वाक्पतिर्भवेत्} %9-41

\twolineshloka
{रुद्रोऽहङ्काररूपस्ते वाचश्छन्दांसि तेऽव्यय}
{यमस्ते दंष्ट्रदेशस्थो नक्षत्राणि द्विजालयः} %9-42

\twolineshloka
{हासो मोहकरी माया सृष्टिस्तेऽपाङ्गमोक्षणम्}
{धर्मः पुरस्तेऽधर्मश्च पृष्ठभाग उदीरितः} %9-43

\twolineshloka
{निमिषोन्मेषणे रात्रिर्दिवा चैव रघूत्तम}
{समुद्राः सप्त ते कुक्षिर्नाड्यो नद्यस्तव प्रभो} %9-44

\twolineshloka
{रोमाणि वृक्षौषधयो रेतो वृष्टिस्तव प्रभो}
{महिमा ज्ञानशक्तिस्ते एवं स्थूलं वपुस्तव} %9-45

\twolineshloka
{यदस्मिन् स्थूलरूपे ते मनः सन्धार्यते नरैः}
{अनायासेन मुक्तिः स्यादतोऽन्यन्नहि किञ्चन} %9-46

\twolineshloka
{अतोऽहं राम रूपं ते स्थूलमेवानुभावये}
{यस्मिन् ध्याते प्रेमरसः सरोमपुलको भवेत्} %9-47

\twolineshloka
{तदैव मुक्तिः स्याद्राम यदा ते स्थूलभावकः}
{तदप्यास्तां तवैवाहमेतद्रूपं विचिन्तये} %9-48

\twolineshloka
{धनुर्बाणधरं श्यामं जटावल्कलभूषितम्}
{अपीच्यवयसं सीतां विचिन्वन्तं सलक्ष्मणम्} %9-49

\onelineshloka
{इदमेव सदा मे स्यान्मानसे रघुनन्दन} %9-50


\threelineshloka
{सर्वज्ञः शङ्करः साक्षात्पार्वत्या सहितः सदा}
{त्वद्रूपमेव सततं ध्यायन्नास्ते रघूत्तम}
{मुमूर्षूणां तदा काश्यां तारकं ब्रह्मवाचकम्} %9-51

\twolineshloka
{रामरामेत्युपदिशन् सदा सन्तुष्टमानसः}
{अतस्त्वं जानकीनाथ परमात्मा सुनिश्चितः} %9-52

\twolineshloka
{सर्वे ते मायया मूढास्त्वां न जानन्ति तत्त्वतः}
{नमस्ते रामभद्राय वेधसे परमात्मने} %9-53

\twolineshloka
{अयोध्याधिपते तुभ्यं नमः सौमित्रिसेवित}
{त्राहि त्राहि जगन्नाथ मां माया नावृणोतु ते} %9-54

\uvacha{श्री-राम उवाच}

\twolineshloka
{तुष्टोऽहं देवगन्धर्व भक्त्या स्तुत्या च तेऽनघ}
{याहि मे परमं स्थानं योगिगम्यं सनातनम्} %9-55

\fourlineindentedshloka
{जपन्ति ये नित्यमनन्यबुद्ध्या}
{भक्त्या त्वदुक्तं स्तवमागमोक्तम्}
{तेऽज्ञानसम्भूतभवं विहाय}
{मां यान्ति नित्यानुभवानुमेयम्} %9-56

{॥इति श्रीमदध्यात्मरामयणे उमामहेश्वरसंवादे
अरण्यकाण्डे नवमे  सर्गे  कबन्धकृतं  श्री-रामस्तोत्रं  सम्पूर्णम्॥}
