% !TeX program = XeLaTeX
% !TeX root = ../../shloka.tex

\sect{कौसल्याकृत-रामस्तोत्रम्}

\uvacha{कौसल्योवाच}
\addtocounter{shlokacount}{19}
\twolineshloka
{देवदेव नमस्तेऽस्तु शङ्खचक्रगदाधर}
{परमात्माऽच्युतोऽनन्तः पूर्णस्त्वं पुरुषोत्तमः} %3-20

\twolineshloka
{वदन्त्यगोचरं वाचां बुद्ध्यादीनामतीन्द्रियम्}
{त्वां वेदवादिनः सत्तामात्रं ज्ञानैकविग्रहम्} %3-21

\twolineshloka
{त्वमेव मायया विश्वं सृजस्यवसि हंसि च}
{सत्त्वादिगुणसंयुक्तस्तुर्य एवामलः सदा} %3-22

\twolineshloka
{करोषीव न कर्ता त्वं गच्छसीव न गच्छसि}
{शृणोषि न शृणोषीव पश्यसीव न पश्यसि} %3-23

\twolineshloka
{अप्राणो ह्यमनाः शुद्ध इत्यादि श्रुतिरब्रवीत्}
{समः सर्वेषु भूतेषु तिष्ठन्नपि न लक्ष्यसे} %3-24

\twolineshloka
{अज्ञानध्वान्तचित्तानां व्यक्त एव सुमेधसाम्}
{जठरे तव दृश्यन्ते ब्रह्माण्डाः परमाणवः} %3-25

\twolineshloka
{त्वं ममोदरसम्भूत इति लोकान् विडम्बसे}
{भक्तेषु पारवश्यं ते दृष्टं मेऽद्य रघूत्तम} %3-26

\twolineshloka
{संसारसागरे मग्ना पतिपुत्रधनादिषु}
{भ्रमामि मायया तेऽद्य पादमूलमुपागता} %3-27

\twolineshloka
{देव त्वद्रूपमेतन्मे सदा तिष्ठतु मानसे}
{आवृणोतु न मां माया तव विश्वविमोहिनी} %3-28

\threelineshloka
{उपसंहर विश्वात्मन्नदो रूपमलौकिकम्}
{दर्शयस्व महानन्दबालभावं सुकोमलम्}
{ललितालिङ्गनालापैस्तरिष्याम्युत्कटं तमः} %3-29

\uvacha{श्री-भगवानुवाच}

\onelineshloka
{यद्यदिष्टं तवास्त्यम्ब तत्तद्भवतु नान्यथा} %3-30

\twolineshloka
{अहं तु ब्रह्मणा पूर्वं भूमेर्भारापनुत्तये}
{प्रार्थितो रावणं हन्तुं मानुषत्वमुपागतः} %3-31

\twolineshloka
{त्वया दशरथेनाहं तपसाराधितः पुरा}
{मत्पुत्रत्वाभिकाङ्क्षिण्या तथा कृतमनिन्दिते} %3-32

\twolineshloka
{रूपमेतत्त्वया दृष्टं प्राक्तनं तपसः फलम्}
{मद्दर्शनं विमोक्षाय कल्पते ह्यन्यदुर्लभम्} %3-33

\twolineshloka
{संवादमावयोर्यस्तु पठेद्वा शृणुयादपि}
{स याति मम सारूप्यं मरणे मत्स्मृतिं लभेत्} %3-34

{॥इति श्रीमदध्यात्मरामायणे उमामहेश्वरसंवादे बालकाण्डे
तृतीये सर्गे श्री-कौसल्याविरचितं श्री-रामस्तोत्रं सम्पूर्णम्॥}
