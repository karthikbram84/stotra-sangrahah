% !TeX program = XeLaTeX
% !TeX root = ../../shloka.tex

\sect{क्रियायोगः}


\addtocounter{shlokacount}{7}
\uvacha{श्री-लक्ष्मण उवाच}
\twolineshloka
{इदानीं श्रोतुमिच्छामि क्रियामार्गेण राघव}
{भवदाराधनं लोके यथा कुर्वन्ति योगिनः} %4-8

\twolineshloka
{इदमेव सदा प्राहुर्योगिनो मुक्तिसाधनम्}
{नारदोऽपि तथा व्यासो ब्रह्मा कमलसम्भवः} %4-9

\threelineshloka
{ब्रह्मक्षत्रादिवर्णानामाश्रमाणां च मोक्षदम्}
{स्त्रीशूद्राणां च राजेन्द्र सुलभं मुक्तिसाधनम्}
{तव भक्ताय मे भ्रात्रे ब्रूहि लोकोपकारकम्} %4-10

\uvacha{श्री-राम उवाच}

\twolineshloka
{मम पूजाविधानस्य नान्तोऽस्ति रघुनन्दन}
{तथाऽपि वक्ष्ये सङ्क्षेपाद्यथावदनुपूर्वशः} %4-11

\twolineshloka
{स्वगृह्योक्तप्रकारेण द्विजत्वं प्राप्य मानवः}
{सकाशात्सद्गुरोर्मन्त्रं लब्ध्वा मद्भक्तिसंयुतः} %4-12

\twolineshloka
{तेन सन्दर्शितविधिर्मामेवाराधयेत्सुधीः}
{हृदये वाऽनले वार्चेत्प्रतिमादौ विभावसौ} %4-13

\twolineshloka
{शालग्रामशिलायां वा पूजयेन्मामतन्द्रितः}
{प्रातःस्नानं प्रकुर्वीत प्रथमं देहशुद्धये} %4-14

\twolineshloka
{वेदतन्त्रोदितैर्मन्त्रैर्मृल्लेपनविधानतः}
{सन्ध्यादि कर्म यन्नित्यं तत्कुर्याद्विधिना बुधः} %4-15

\twolineshloka
{सङ्कल्पमादौ कुर्वीत सिद्ध्यर्थं कर्मणां सुधीः}
{स्वगुरुं पूजयेद्भक्त्या मद्बुद्ध्या पूजको मम} %4-16

\twolineshloka
{शिलायां स्नपनं कुर्यात्प्रतिमासु प्रमार्जनम्}
{प्रसिद्धैर्गन्धपुष्पाद्यैर्मत्पूजा सिद्धिदायिका} %4-17

\twolineshloka
{अमायिकोऽनुवृत्त्या मां पूजयेन्नियतव्रतः}
{प्रतिमादिष्वलङ्कारः प्रियो मे कुलनन्दन} %4-18

\twolineshloka
{अग्नौ यजेत हविषा भास्करे स्थण्डिले यजेत्}
{भक्तेनोपहृतं प्रीत्यै श्रद्धया मम वार्यपि} %4-19

\twolineshloka
{किं पुनर्भक्ष्यभोज्यादि गन्धपुष्पाक्षतादिकम्}
{पूजाद्रव्याणि सर्वाणि सम्पाद्यैवं समारभेत्} %4-20

\twolineshloka
{चैलाजिनकुशैः सम्यगासनं परिकल्पयेत्}
{तत्रोपविश्य देवस्य सम्मुखे शुद्धमानसः} %4-21

\twolineshloka
{ततो न्यासं प्रकुर्वीत मातृकाबहिरान्तरम्}
{केशवादि ततः कुर्यात्तत्त्वन्यासं ततः परम्} %4-22

\twolineshloka
{मन्मूर्तिपञ्जरन्यासं मन्त्रन्यासं ततो न्यसेत्}
{प्रतिमादावपि तथा कुर्यान्नित्यमतन्द्रितः} %4-23

\twolineshloka
{कलशं स्वपुरो वामे क्षिपेत्पुष्पादि दक्षिणे}
{अर्घ्यपाद्यप्रदानार्थं मधुपर्कार्थमेव च} %4-24

\twolineshloka
{तथैवाचमनार्थं तु न्यसेत्पात्रचतुष्टयम्}
{हृत्पद्मे भानुविमले मत्कलां जीवसंज्ञिताम्} %4-25

\twolineshloka
{ध्यायेत्स्वदेहमखिलं तया व्याप्तमरिन्दम}
{तामेवावाहयेन्नित्यं प्रतिमादिषु मत्कलाम्} %4-26

\twolineshloka
{पाद्यार्घ्याचमनीयाद्यैः स्नानवस्त्रविभूषणैः}
{यावच्छक्योपचारैर्वा त्वर्चयेन्माममायया} %4-27

\twolineshloka
{विभवे सति कर्पूरकुङ्कुमागरुचन्दनैः}
{अर्चयेन्मन्त्रवन्नित्यं सुगन्धकुसुमैः शुभैः} %4-28

\twolineshloka
{दशावरणपूजां वै ह्यागमोक्तां प्रकारयेत्}
{नीराजनैर्धूपदीपैर्नैवेद्यैर्बहुविस्तरैः} %4-29

\twolineshloka
{श्रद्धयोपहरेन्नित्यं श्रद्धाभुगहमीश्वरः}
{होमं कुर्यात्प्रयत्नेन विधिना मन्त्रकोविदः} %4-30

\twolineshloka
{अगस्त्येनोक्तमार्गेण कुण्डेनागमवित्तमः}
{जुहुयान्मूलमन्त्रेण पुंसूक्तेनाथवा बुधः} %4-31

\twolineshloka
{अथवौपासनाग्नौ वा चरुणा हविषा तथा}
{तप्तजाम्बूनदप्रख्यं दिव्याभरणभूषितम्} %4-32

\twolineshloka
{ध्यायेदनलमध्यस्थं होमकाले सदा बुधः}
{पार्षदेभ्यो बलिं दत्त्वा होमशेषं समापयेत्} %4-33

\twolineshloka
{ततो जपं प्रकुर्वीत ध्यायेन्मां यतवाक् स्मरन्}
{मुखवासं च ताम्बूलं दत्त्वा प्रीतिसमन्वितः} %4-34

\twolineshloka
{मदर्थे नृत्यगीतादि स्तुतिपाठादि कारयेत्}
{प्रणमेद्दण्डवद्भूमौ हृदये मां निधाय च} %4-35

\twolineshloka
{शिरस्याधाय मद्दत्तं प्रसादं भावनामयम्}
{पाणिभ्यां मत्पदे मूर्ध्नि गृहीत्वा भक्तिसंयुतः} %4-36

\twolineshloka
{रक्ष मां घोरसंसारादित्युक्त्वा प्रणमेत्सुधीः}
{उद्वासयेद्यथापूर्वं प्रत्यग्ज्योतिषि संस्मरन्} %4-37

\twolineshloka
{एवमुक्तप्रकारेण पूजयेद्विधिवद्यदि}
{इहामुत्र च संसिद्धिं प्राप्नोति मदनुग्रहात्} %4-38

\twolineshloka
{मद्भक्तो यदि मामेवं पूजां चैव दिने दिने}
{करोति मम सारूप्यं प्राप्नोत्येव न संशयः} %4-39

\fourlineindentedshloka
{इदं रहस्यं परमं च पावनम्}
{मयैव साक्षात्कथितं सनातनम्}
{पठत्यजस्रं यदि वा शृणोति यः}
{स सर्वपूजाफलभाङ्न संशयः} %4-40

\twolineshloka
{एवं परात्मा श्रीरामः क्रियायोगमनुत्तमम्}
{पृष्टः प्राह स्वभक्ताय शेषांशाय महात्मने} %4-41



{॥इति श्रीमदध्यात्मरामयणे उमामहेश्वरसंवादे
कीष्किन्धाकाण्डे चतुर्थे सर्गे  तारायै  रामस्य  तत्त्वोपदेशः॥}
