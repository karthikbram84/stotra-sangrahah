% !TeX program = XeLaTeX
% !TeX root = ../../shloka.tex

\sect{लक्ष्मणोपदेशः}


\addtocounter{shlokacount}{9}

\uvacha{श्री-लक्ष्मण  उवाच}

\twolineshloka
{शोकेन महताऽऽविष्टं सौमित्रिरिदमब्रवीत्}
{यं शोचसि त्वं दुःखेन कोऽयं तव विभीषण} %12-10

\twolineshloka
{त्वं वास्य कतमः सृष्टेः पुरेदानीमतः परम्}
{यद्वत्तोयौघपतिताः सिकता यान्ति तद्वशाः} %12-11

\twolineshloka
{संयुज्यन्ते वियुज्यन्ते तथा कालेन देहिनः}
{यथा धानासु वै धाना भवन्ति न भवन्ति च} %12-12

\twolineshloka
{एवं भूतेषु भूतानि प्रेरितानीशमायया}
{त्वं चेमे वयमन्ये च तुल्याः कालवशोद्भवाः} %12-13

\twolineshloka
{जन्ममृत्यू यदा यस्मात्तदा तस्माद्भविष्यतः}
{ईश्वरः सर्वभूतानि भूतैः सृजति हन्त्यजः} %12-14

\twolineshloka
{आत्मसृष्टैरस्वतन्त्रैर्निरपेक्षोऽपि बालवत्}
{देहेन देहिनो जीवा देहाद्देहोऽभिजायते} %12-15

\twolineshloka
{बीजादेव यथा बीजं देहान्य इव शाश्वतः}
{देहिदेहविभागोऽयमविवेककृतः पुरा} %12-16

\twolineshloka
{नानात्वं जन्म नाशश्च क्षयो वृद्धिः क्रियाफलम्}
{द्रष्टुराभान्त्यतद्धर्मा यथाग्नेर्दारुविक्रियाः} %12-17

\twolineshloka
{त इमे देहसंयोगादात्मना भान्त्यसद्ग्रहात्}
{यथा यथा तथा चान्यद्ध्यायतोऽसत्सदाग्रहात्} %12-18

\twolineshloka
{प्रसुप्तस्यानहम्भावात्तदा भाति न संसृतिः}
{जीवतोऽपि तथा तद्वद्विमुक्तस्यानहङ्कृतेः} %12-19

\twolineshloka
{तस्मान्मायामनोधर्मं जह्यहम्ममताभ्रमम्}
{रामभद्रे भगवति मनो धेह्यात्मनीश्वरे} %12-20

\twolineshloka
{सर्वभूतात्मनि परे मायामानुषरूपिणि}
{बाह्येन्द्रियार्थसम्बन्धात्त्याजयित्वा मनः शनैः} %12-21

\twolineshloka
{तत्र दोषान् दर्शयित्वा रामानन्दे नियोजय}
{देहबुद्ध्या भवेद्भ्राता पिता माता सुहृत्प्रियः} %12-22

\twolineshloka
{विलक्षणं यदा देहाज्जानात्यात्मानमात्मना}
{तदा कः कस्य वा बन्धुर्भ्राता माता पिता सुहृत्} %12-23

\twolineshloka
{मिथ्याज्ञानवशाज्जाता दारागारादयः सदा}
{शब्दादयश्च विषया विविधाश्चैव सम्पदः} %12-24

\twolineshloka
{बलं कोशो भृत्यवर्गो राज्यं भूमिः सुतादयः}
{अज्ञानजत्वात्सर्वे ते क्षणसङ्गमभङ्गुराः} %12-25

\twolineshloka
{अथोत्तिष्ठ हृदा रामं भावयन् भक्तिभावितम्}
{अनुवर्तस्व राज्यादि भुञ्जन् प्रारब्धमन्वहम्} %12-26

\twolineshloka
{भूतं भविष्यदभजन् वर्तमानमथाचरन्}
{विहरस्व यथान्यायं भवदोषैर्न लिप्यसे} %12-27

{॥इति श्रीमदध्यात्मरामायणे उमामहेश्वरसंवादे युद्धकाण्डे द्वादशे  सर्गे  लक्ष्मणोपदेशः  सम्पूर्णः}
