% !TeX program = XeLaTeX
% !TeX root = ../../shloka.tex

\sect{नवविधभक्तिवर्णनम्}


\addtocounter{shlokacount}{19}

\uvacha{श्री- राम उवाच}

\twolineshloka
{पुंस्त्वे स्त्रीत्वे विशेषो वा जातिनामाश्रमादयः}
{न कारणं मद्भजने भक्तिरेव हि कारणम्} %10-20

\twolineshloka
{यज्ञदानतपोभिर्वा वेदाध्ययनकर्मभिः}
{नैव द्रष्टुमहं शक्यो मद्भक्तिविमुखैः सदा} %10-21

\onelineshloka
{तस्माद्भामिनि सङ्क्षेपाद्वक्ष्येऽहं भक्तिसाधनम्} %10-22


\threelineshloka
{सतां सङ्गतिरेवात्र साधनं प्रथमं स्मृतम्}
{द्वितीयं मत्कथालापस्तृतीयं मद्गुणेरणम्}
{व्याख्यातृत्वं मद्वचसां चतुर्थं साधनं भवेत्} %10-23

\twolineshloka
{आचार्योपासनं भद्रे सद्बुद्ध्याऽमायया सदा}
{पञ्चमं पुण्यशीलत्वं यमादि नियमादि च} %10-24

\twolineshloka
{निष्ठा मत्पूजने नित्यं षष्ठं साधनमीरितम्}
{मम मन्त्रोपासकत्वं साङ्गं सप्तममुच्यते} %10-25

\twolineshloka
{मद्भक्तेष्वधिका पूजा सर्वभूतेषु मन्मतिः}
{बाह्यार्थेषु विरागित्वं शमादिसहितं तथा} %10-26

\twolineshloka
{अष्टमं नवमं तत्त्वविचारो मम भामिनि}
{एवं नवविधा भक्तिः साधनं यस्य कस्य वा} %10-27

\twolineshloka
{स्त्रियो वा पुरुषस्यापि तिर्यग्योनिगतस्य वा}
{भक्तिः सञ्जायते प्रेमलक्षणा शुभलक्षणे} %10-28

\twolineshloka
{भक्तौ सञ्जातमात्रायां मत्तत्त्वानुभवस्तदा}
{ममानुभवसिद्धस्य मुक्तिस्तत्रैव जन्मनि} %10-29

\twolineshloka
{स्यात्तस्मात्कारणं भक्तिर्मोक्षस्येति सुनिश्चितम्}
{प्रथमं साधनं यस्य भवेत्तस्य क्रमेण तु} %10-30

\twolineshloka
{भवेत्सर्वं ततो भक्तिर्मुक्तिरेव सुनिश्चितम्}
{यस्मान्मद्भक्तियुक्ता त्वं ततोऽहं त्वामुपस्थितः} %10-31


{इतो मद्दर्शनान्मुक्तिस्तव नास्त्यत्र संशयः।}

{॥इति श्रीमदध्यात्मरामयणे उमामहेश्वरसंवादे
अरण्यकाण्डे दशमे सर्गे श्रीरामोक्त  नवविधभक्तिवर्णनं  सम्पूर्णम्॥}
