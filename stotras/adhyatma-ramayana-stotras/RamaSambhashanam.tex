% !TeX program = XeLaTeX
% !TeX root = ../../shloka.tex

\sect{राम-सम्भाषणम्}
\addtocounter{shlokacount}{18}

\uvacha{श्री-राम  उवाच}

\twolineshloka
{यदिदं दृश्यते सर्वं राज्यं देहादिकं च यत्}
{यदि सत्यं भवेत्तत्र आयासः सफलश्च ते} %4-19

\twolineshloka
{भोगा मेघवितानस्थविद्युल्लेखेव चञ्चलाः}
{आयुरप्यग्निसन्तप्तलोहस्थजलबिन्दुवत्} %4-20

\twolineshloka
{यथा व्यालगलस्थोऽपि भेको दंशानपेक्षते}
{तथा कालाहिना ग्रस्तो लोको भोगानशाश्वतान्} %4-21

\fourlineindentedshloka
{करोति दुःखेन हि कर्मतन्त्रम्}
{शरीरभोगार्थमहर्निशं नरः}
{देहस्तु भिन्नः पुरुषात्समीक्ष्यते}
{को वात्र भोगः पुरुषेण भुज्यते} %4-22

\twolineshloka
{पितृमातृसुतभ्रातृदारबन्ध्वादिसङ्गमः}
{प्रपायामिव जन्तूनां नद्यां काष्ठौघवच्चलः} %4-23

\fourlineindentedshloka
{छायेव लक्ष्मीश्चपला प्रतीता}
{तारुण्यमम्बूर्मिवदध्रुवं च}
{स्वप्नोपमं स्त्रीसुखमायुरल्पम्}
{तथाऽपि जन्तोरभिमान एषः} %4-24

\twolineshloka
{संसृतिः स्वप्नसदृशी सदा रोगादिसङ्कुला}
{गन्धर्वनगरप्रख्या मूढस्तामनुवर्तते} %4-25

\twolineshloka
{आयुष्यं क्षीयते यस्मादादित्यस्य गतागतैः}
{दृष्ट्वाऽन्येषां जरामृत्यू कथञ्चिन्नैव बुध्यते} %4-26

\twolineshloka
{स एव दिवसः सैव रात्रिरित्येव मूढधीः}
{भोगाननुपतत्येव कालवेगं न पश्यति} %4-27

\twolineshloka
{प्रतिक्षणं क्षरत्येतदायुरामघटाम्बुवत्}
{सपत्ना इव रोगौघाः शरीरं प्रहरन्त्यहो} %4-28

\twolineshloka
{जरा व्याघ्रीव पुरतस्तर्जयन्त्यवतिष्ठते}
{मृत्युः सहैव यात्येष समयं सम्प्रतीक्षते} %4-29

\twolineshloka
{देहेऽहम्भावमापन्नो राजाहं लोकविश्रुतः}
{इत्यस्मिन्मनुते जन्तुः कृमिविड्भस्मसंज्ञिते} %4-30

\twolineshloka
{त्वगस्थिमान्सविण्मूत्ररेतोरक्तादिसंयुतः}
{विकारी परिणामी च देह आत्मा कथं वद} %4-31

\twolineshloka
{यमास्थाय भवान्ल्लोकं दग्धुमिच्छति लक्ष्मण}
{देहाभिमानिनः सर्वे दोषाः प्रादुर्भवन्ति हि} %4-32

\twolineshloka
{देहोऽहमिति या बुद्धिरविद्या सा प्रकीर्तिता}
{नाहं देहश्चिदात्मेति बुद्धिर्विद्येति भण्यते} %4-33

\threelineshloka
{अविद्या संसृतेर्हेतुर्विद्या तस्या निवर्तिका}
{तस्माद्यत्नः सदा कार्यो विद्याभ्यासे मुमुक्षुभिः}
{कामक्रोधादयस्तत्र शत्रवः शत्रुसूदन} %4-34

\twolineshloka
{तत्रापि क्रोध एवालं मोक्षविघ्नाय सर्वदा}
{येनाविष्टः पुमान् हन्ति पितृभ्रातृसुहृत्सखीन्} %4-35

\twolineshloka
{क्रोधमूलो मनस्तापः क्रोधः संसारबन्धनम्}
{धर्मक्षयकरः क्रोधस्तस्मात्क्रोधं परित्यज} %4-36

\twolineshloka
{क्रोध एष महान् शत्रुस्तृष्णा वैतरणी नदी}
{सन्तोषो नन्दनवनं शान्तिरेव हि कामधुक्} %4-37

\twolineshloka
{तस्माच्छान्तिं भजस्वाद्य शत्रुरेवं भवेन्न ते}
{देहेन्द्रियमनःप्राणबुद्ध्यादिभ्यो विलक्षणः} %4-38

\twolineshloka
{आत्मा शुद्धः स्वयञ्ज्योतिरविकारी निराकृतिः}
{यावद्देहेन्द्रियप्राणैर्भिन्नत्वं नात्मनो विदुः} %4-39

\twolineshloka
{तावत्संसारदुःखौघैः पीड्यन्ते मृत्युसंयुताः}
{तस्मात्त्वं सर्वदा भिन्नमात्मानं हृदि भावय} %4-40

\twolineshloka
{बुद्ध्यादिभ्यो बहिः सर्वमनुवर्तस्व मा खिदः}
{भुञ्जन् प्रारब्धमखिलं सुखं वा दुःखमेव वा} %4-41

\twolineshloka
{प्रवाहपतितं कार्यं कुर्वन्नपि न लिप्यसे}
{बाह्ये सर्वत्र कर्तृत्वमावहन्नपि राघव} %4-42

\twolineshloka
{अन्तःशुद्धस्वभावस्त्वं लिप्यसे न च कर्मभिः}
{एतन्मयोदितं कृत्स्नं हृदि भावय सर्वदा} %4-43

\twolineshloka
{संसारदुःखैरखिलैर्बाध्यसे न कदाचन}
{त्वमप्यम्ब ममाऽऽदिष्टं हृदि भावय नित्यदा} %4-44

\twolineshloka
{समागमं प्रतीक्षस्व न दुःखैः पीड्यसे चिरम्}
{न सदैकत्र संवासः कर्ममार्गानुवर्तिनाम्} %4-45

\twolineshloka
{यथा प्रवाहपतितप्लवानां सरितां तथा}
{चतुर्दशसमा सङ्ख्या क्षणार्द्धमिव जायते} %4-46

\twolineshloka
{अनुमन्यस्व मामम्ब दुःखं सन्त्यज्य दूरतः}
{एवं चेत्सुखसंवासो भविष्यति वने मम} %4-47

\twolineshloka
{इत्युक्त्वा दण्डवन्मातुः पादयोरपतच्चिरम्}
{उत्थाप्याङ्के समावेश्य आशीर्भिरभ्यनन्दयत्} %4-48

{॥इति श्रीमदध्यात्मरामायणे उमामहेश्वरसंवादे
अयोध्याकाण्डे चतुर्थे सर्गे  राम-सम्भाषणं  सम्पूर्णम्॥}
