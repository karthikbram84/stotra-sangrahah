% !TeX program = XeLaTeX
% !TeX root = ../../shloka.tex

\sect{सुग्रीवकृत-रामस्तोत्रम्}

\uvacha{सुग्रीव  उवाच}
\addtocounter{shlokacount}{75}
\twolineshloka
{देव त्वं जगतां नाथः परमात्मा न संशयः}
{मत्पूर्वकृतपुण्यौघैः सङ्गतोऽद्य मया सह} %1-76

\twolineshloka
{त्वां भजन्ति महात्मानः संसारविनिवृत्तये}
{त्वां प्राप्य मोक्षसचिवं प्रार्थयेऽहं कथं भवम्} %1-77

\twolineshloka
{दाराः पुत्रा धनं राज्यं सर्वं त्वन्मायया कृतम्}
{अतोऽहं देवदेवेश नाकाङ्क्षेऽन्यत्प्रसीद मे} %1-78

\twolineshloka
{आनन्दानुभवं त्वाऽद्य प्राप्तोऽहं भाग्यगौरवात्}
{मृदर्थं यतमानेन निधानमिव सत्पते} %1-79

\twolineshloka
{अनाद्यविद्यासंसिद्धं बन्धनं छिन्नमद्य नः}
{यज्ञदानतपःकर्मपूर्तेष्टादिभिरप्यसौ} %1-80

\twolineshloka
{न जीर्यते पुनर्दार्ढ्यं भजते संसृतिः प्रभो}
{त्वत्पाददर्शनात्सद्यो नाशमेति न संशयः} %1-81

\twolineshloka
{क्षणार्धमपि यच्चित्तं त्वयि तिष्ठत्यचञ्चलम्}
{तस्याज्ञानमनर्थानां मूलं नश्यति तत्क्षणात्} %1-82

\onelineshloka
{तत्तिष्ठतु मनो राम त्वयि नान्यत्र मे सदा} %1-83

\twolineshloka
{रामरामेति यद्वाणी मधुरं गायति क्षणम्}
{स ब्रह्महा सुरापो वा मुच्यते सर्वपातकैः} %1-84

\twolineshloka
{न काङ्क्षे विजयं राम न च दारसुखादिकम्}
{भक्तिमेव सदा काङ्क्षे त्वयि बन्धविमोचनीम्} %1-85

\twolineshloka
{त्वन्मायाकृतसंसारस्त्वदंशोऽहं रघूत्तम}
{स्वपादभक्तिमादिश्य त्राहि मां भवसङ्कटात्} %1-86

\twolineshloka
{पूर्वं मित्रार्युदासीनास्त्वन्मायावृतचेतसः}
{आसन्मेऽद्य भवत्पाददर्शनादेव राघव} %1-87

\twolineshloka
{सर्वं ब्रह्मैव मे भाति क्व मित्रं क्व च मे रिपुः}
{यावत्त्वन्मायया बद्धस्तावद्गुणविशेषता} %1-88

\twolineshloka
{सा यावदस्ति नानात्वं तावद्भवति नान्यथा}
{यावन्नानात्वमज्ञानात्तावत्कालकृतं भयम्} %1-89

\threelineshloka
{अतोऽविद्यामुपास्ते यः सोऽन्धे तमसि मज्जति}
{मायामूलमिदं सर्वं पुत्रदारादिबन्धनम्}
{तदुत्सारय मायां त्वं दासीं तव रघूत्तम} %1-90

\fourlineindentedshloka
{त्वत्पादपद्मार्पितचित्तवृत्ति\-}
{स्त्वन्नामसङ्गीतकथासु वाणी}
{त्वद्भक्तसेवानिरतौ करौ मे}
{त्वदङ्गसङ्गं लभतां मदङ्गम्} %1-91

\fourlineindentedshloka
{त्वन्मूर्तिभक्तान् स्वगुरुं च चक्षुः}
{पश्यत्वजस्रं स शृणोतु कर्णः}
{त्वज्जन्मकर्माणि च पादयुग्मम्}
{व्रजत्वजस्रं तव मन्दिराणि} %1-92

\fourlineindentedshloka
{अङ्गानि ते पादरजोविमिश्र-}
{तीर्थानि बिभ्रत्वहिशत्रुकेतो}
{शिरस्त्वदीयं भवपद्मजाद्यैर्-}
{जुष्टं पदं राम नमत्वजस्रम्} %1-93

{॥इति श्रीमदध्यात्मरामायणे उमामहेश्वरसंवादे किष्किन्धाकाण्डे
प्रथमे सर्गे  सुग्रीवकृतं श्री-रामस्तोत्रं सम्पूर्णम्॥}
