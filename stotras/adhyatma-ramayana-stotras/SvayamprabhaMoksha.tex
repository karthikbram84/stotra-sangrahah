% !TeX program = XeLaTeX
% !TeX root = ../../shloka.tex

\sect{स्वयम्प्रभा-मोक्षः}


\addtocounter{shlokacount}{57}

\twolineshloka
{यूयं पिदध्वमक्षीणि गमिष्यथ बहिर्गुहाम्}
{तथैव चक्रुस्ते वेगाद्गताः पूर्वस्थितं वनम्} %6-58

\twolineshloka
{साऽपि त्यक्त्वा गुहां शीघ्रं ययौ राघवसन्निधिम्}
{तत्र रामं ससुग्रीवं लक्ष्मणं च ददर्श ह} %6-59

\twolineshloka
{कृत्वा प्रदक्षिणं रामं प्रणम्य बहुशः सुधीः}
{आह गद्गदया वाचा रोमाञ्चिततनूरुहा} %6-60

\uvacha{स्वयम्प्रभोवाच}
\twolineshloka
{दासी तवाहं राजेन्द्र दर्शनार्थमिहाऽऽगता}
{बहुवर्षसहस्राणि तप्तं मे दुश्चरं तपः} %6-61

\twolineshloka
{गुहायां दर्शनार्थं ते फलितं मेऽद्य तत्तपः}
{अद्य हि त्वां नमस्यामि मायायाः परतः स्थितम्} %6-62

\twolineshloka
{सर्वभूतेषु चालक्ष्यं बहिरन्तरवस्थितम्}
{योगमायाजवनिकाच्छन्नो मानुषविग्रहः} %6-63

\twolineshloka
{न लक्ष्यसेऽज्ञानदृशां शैलूष इव रूपधृक्}
{महाभागवतानां त्वं भक्तियोगविधित्सया} %6-64

\twolineshloka
{अवतीर्णोऽसि भगवन् कथं जानामि तामसी}
{लोके जानातु यः कश्चित्तव तत्त्वं रघूत्तम} %6-65

\twolineshloka
{ममैतदेव रूपं ते सदा भातु हृदालये}
{राम ते पादयुगलं दर्शितं मोक्षदर्शनम्} %6-66

\threelineshloka
{अदर्शनं भवार्णानां सन्मार्गपरिदर्शनम्}
{धनपुत्रकलत्रादिविभूतिपरिदर्पितः}
{अकिञ्चनधनं त्वाऽद्य नाभिधातुं जनोऽर्हति} %6-67

\onelineshloka
{निवृत्तगुणमार्गाय निष्किञ्चनधनाय ते॥६८॥} %6-68

\twolineshloka
{नमः स्वात्माभिरामाय निर्गुणाय गुणात्मने}
{कालरूपिणमीशानमादिमध्यान्तवर्जितम्} %6-69

\twolineshloka
{समं चरन्तं सर्वत्र मन्ये त्वां पुरुषं परम्}
{देव ते चेष्टितं कश्चिन्न वेद नृविडम्बनम्} %6-70

\twolineshloka
{न तेऽस्ति कश्चिद्दयितो द्वेष्यो वाऽपर एव च}
{त्वन्मायापिहितात्मानस्त्वां पश्यन्ति तथाविधम्} %6-71

\twolineshloka
{अजस्याकर्तुरीशस्य देवतिर्यङ्नरादिषु}
{जन्मकर्मादिकं यद्यत्तदत्यन्तविडम्बनम्} %6-72

\twolineshloka
{त्वामाहुरक्षरं जातं कथाश्रवणसिद्धये}
{केचित्कोसलराजस्य तपसः फलसिद्धये} %6-73

\twolineshloka
{कौसल्यया प्रार्थ्यमानं जातमाहुः परे जनाः}
{दुष्टराक्षसभूभारहरणायार्थितो विभुः} %6-74

\twolineshloka
{ब्रह्मणा नररूपेण जातोऽयमिति केचन}
{शृण्वन्ति गायन्ति च ये कथास्ते रघुनन्दन} %6-75

\twolineshloka
{पश्यन्ति तव पादाब्जं भवार्णवसुतारणम्}
{त्वन्मायागुणबद्धाहं व्यतिरिक्तं गुणाश्रयम्} %6-76

\threelineshloka
{कथं त्वां देव जानीयां स्तोतुं वाऽविषयं विभुम्}
{नमस्यामि रघुश्रेष्ठं बाणासनशरान्वितम्}
{लक्ष्मणेन सह भ्रात्रा सुग्रीवादिभिरन्वितम्} %6-77

\twolineshloka
{एवं स्तुतो रघुश्रेष्ठः प्रसन्नः प्रणताघहृत्}
{उवाच योगिनीं भक्तां किं ते मनसि काङ्क्षितम्} %6-78

\twolineshloka
{सा प्राह राघवं भक्त्या भक्तिं ते भक्तवत्सल}
{यत्र कुत्रापि जाताया निश्चलां देहि मे प्रभो} %6-79

\twolineshloka
{त्वद्भक्तेषु सदा सङ्गो भूयान्मे प्राकृतेषु न}
{जिह्वा मे रामरामेति भक्त्या वदतु सर्वदा} %6-80

\twolineshloka
{मानसं श्यामलं रूपं सीतालक्ष्मणसंयुतम्}
{धनुर्बाणधरं पीतवाससं मुकुटोज्ज्वलम्} %6-81

\twolineshloka
{अङ्गदैर्नूपुरैर्मुक्ताहारैः कौस्तुभकुण्डलैः}
{भान्तं स्मरतु मे राम वरं नान्यं वृणे प्रभो} %6-82

\uvacha{श्री- राम उवाच}

\threelineshloka
{भवत्वेवं महाभागे गच्छ त्वं बदरीवनम्}
{तत्रैव मां स्मरन्ती त्वं त्यक्त्वेदं भूतपञ्चकम्}
{मामेव परमात्मानमचिरात्प्रतिपद्यसे} %6-83

\fourlineindentedshloka
{श्रुत्वा रघूत्तमवचोऽमृतसारकल्पम्}
{गत्वा तदैव बदरीतरुखण्डजुष्टम्}
{तीर्थं तदा रघुपतिं मनसा स्मरन्ती}
{त्यक्त्वा कलेवरमवाप परं पदं सा} %6-84

{॥इति श्रीमदध्यात्मरामायणे उमामहेश्वरसंवादे किष्किन्धाकाण्डे
षष्ठे सर्गे स्वयम्प्रभा-मोक्षः सम्पूर्णः॥}
