% !TeX program = XeLaTeX
% !TeX root = ../../shloka.tex

\sect{तारायै  रामस्य  तत्त्वोपदेशः}


\addtocounter{shlokacount}{12}
\uvacha{श्री-राम उवाच}

\twolineshloka
{किं भीरु शोचसि व्यर्थं शोकस्याविषयं पतिम्}
{पतिस्तवायं देहो वा जीवो वा वद तत्त्वतः} %3-13

\twolineshloka
{पञ्चात्मको जडो देहस्त्वङ्मांसरुधिरास्थिमान्}
{कालकर्मगुणोत्पन्नः सोऽप्यास्तेऽद्यापि ते पुरः} %3-14

\twolineshloka
{मन्यसे जीवमात्मानं जीवस्तर्हि निरामयः}
{न जायते न म्रियते न तिष्ठति न गच्छति} %3-15

\threelineshloka
{न स्त्री पुमान्वा षण्ढो वा जीवः सर्वगतोऽव्ययः}
{एक एवाद्वितीयोऽयमाकाशवदलेपकः}
{नित्यो ज्ञानमयः शुद्धः स कथं शोकमर्हति} %3-16

\uvacha{तारोवाच}

\twolineshloka
{देहोऽचित्काष्ठवद्राम जीवो नित्यश्चिदात्मकः}
{सुखदुःखादिसम्बन्धः कस्य स्याद्राम मे वद} %3-17

\uvacha{श्री-राम उवाच}

\twolineshloka
{अहङ्कारादिसम्बन्धो यावद्देहेन्द्रियैः सह}
{संसारस्तावदेव स्यादात्मनस्त्वविवेकिनः} %3-18

\twolineshloka
{मिथ्यारोपितसंसारो न स्वयं विनिवर्तते}
{विषयान् ध्यायमानस्य स्वप्ने मिथ्यागमो यथा} %3-19

\twolineshloka
{अनाद्यविद्यासम्बन्धात्तत्कार्याहङ्कृतेस्तथा}
{संसारोऽपार्थकोऽपि स्याद्रागद्वेषादिसङ्कुलः} %3-20

\twolineshloka
{मन एव हि संसारो बन्धश्चैव मनः शुभे}
{आत्मा मनःसमानत्वमेत्य तद्गतबन्धभाक्} %3-21

\twolineshloka
{यथा विशुद्धः स्फटिकोऽलक्तकादिसमीपगः}
{तत्तद्वर्णयुगाभाति वस्तुतो नास्ति रञ्जनम्} %3-22

\twolineshloka
{बुद्धीन्द्रियादिसामीप्यादात्मनः संसृतिर्बलात्}
{आत्मा स्वलिङ्गं तु मनः परिगृह्य तदुद्भवान्} %3-23

\twolineshloka
{कामान् जुषन् गुणैर्बद्धः संसारे वर्ततेऽवशः}
{आदौ मनोगुणान् सृष्ट्वा ततः कर्माण्यनेकधा} %3-24

\twolineshloka
{शुक्ललोहितकृष्णानि गतयस्तत्समानतः}
{एवं कर्मवशाज्जीवो भ्रमत्याभूतसम्प्लवम्} %3-25

\twolineshloka
{सर्वोपसंहृतौ जीवो वासनाभिः स्वकर्मभिः}
{अनाद्यविद्यावशगस्तिष्ठत्यभिनिवेशतः} %3-26

\twolineshloka
{सृष्टिकाले पुनः पूर्ववासनामानसैः सह}
{जायते पुनरप्येवं घटीयन्त्रमिवावशः} %3-27

\twolineshloka
{यदा पुण्यविशेषेण लभते सङ्गतिं सताम्}
{मद्भक्तानां सुशान्तानां तदा मद्विषया मतिः} %3-28

\twolineshloka
{मत्कथाश्रवणे श्रद्धा दुर्लभा जायते ततः}
{ततः स्वरूपविज्ञानमनायासेन जायते} %3-29

\twolineshloka
{तदाऽऽचार्यप्रसादेन वाक्यार्थज्ञानतः क्षणात्}
{देहेन्द्रियमनःप्राणाहङ्कृतिभ्यः पृथक् स्थितम्} %3-30

\twolineshloka
{स्वात्मानुभवतः सत्यमानन्दात्मानमद्वयम्}
{ज्ञात्वा सद्यो भवेन्मुक्तः सत्यमेव मयोदितम्} %3-31

\twolineshloka
{एवं मयोदितं सम्यगालोचयति योऽनिशम्}
{तस्य संसारदुःखानि न स्पृशन्ति कदाचन} %3-32

\twolineshloka
{त्वमप्येतन्मया प्रोक्तमालोचय विशुद्धधीः}
{न स्पृश्यसे दुःखजालैः कर्मबन्धाद्विमोक्ष्यसे} %3-33

\twolineshloka
{पूर्वजन्मनि ते सुभ्रु कृता मद्भक्तिरुत्तमा}
{अतस्तव विमोक्षाय रूपं मे दर्शितं शुभे} %3-34

\twolineshloka
{ध्यात्वा मद्रूपमनिशमालोचय मयोदितम्}
{प्रवाहपतितं कार्यं कुर्वन्त्यपि न लिप्यसे} %3-35

\twolineshloka
{श्रीरामेणोदितं सर्वं श्रुत्वा ताराऽतिविस्मिता}
{देहाभिमानजं शोकं त्यक्त्वा नत्वा रघूत्तमम्} %3-36

\twolineshloka
{आत्मानुभवसन्तुष्टा जीवन्मुक्ता बभूव ह}
{क्षणसङ्गममात्रेण रामेण परमात्मना} %3-37

\twolineshloka
{अनादिबन्धं निर्धूय मुक्ता साऽपि विकल्मषा}
{सुग्रीवोऽपि च तच्छ्रुत्वा रामवक्त्रात्समीरितम्} %3-38

{॥इति श्रीमदध्यात्मरामयणे उमामहेश्वरसंवादे
कीष्किन्धाकाण्डे तृतीये सर्गे  तारायै  रामस्य  तत्त्वोपदेशः॥}
