% !TeX program = XeLaTeX
% !TeX root = ../../shloka.tex

\sect{वाल्मीकीरित-रामस्थानवर्णनम्}
\addtocounter{shlokacount}{51}
\uvacha{श्री-वाल्मीकिरुवाच}


\twolineshloka
{त्वमेव सर्वलोकानां निवासस्थानमुत्तमम्}
{तवापि सर्वभूतानि निवाससदनानि हि} %6-52

\threelineshloka
{एवं साधारणं स्थानमुक्तं ते रघुनन्दन}
{सीतया सहितस्येति विशेषं पृच्छतस्तव}
{तद्वक्ष्यामि रघुश्रेष्ठ यत्ते नियतमन्दिरम्} %6-53

\twolineshloka
{शान्तानां समदृष्टीनामद्वेष्टॄणां च जन्तुषु}
{त्वामेव भजतां नित्यं हृदयं तेऽधिमन्दिरम्} %6-54

\twolineshloka
{धर्माधर्मान् परित्यज्य त्वामेव भजतोऽनिशम्}
{सीतया सह ते राम तस्य हृत्सुखमन्दिरम्} %6-55

\twolineshloka
{त्वन्मन्त्रजापको यस्तु त्वामेव शरणं गतः}
{निर्द्वन्द्वो निःस्पृहस्तस्य हृदयं ते सुमन्दिरम्} %6-56

\twolineshloka
{निरहङ्कारिणः शान्ता ये रागद्वेषवर्जिताः}
{समलोष्टाश्मकनकास्तेषां ते हृदयं गृहम्} %6-57

\twolineshloka
{त्वयि दत्तमनोबुद्धिर्यः सन्तुष्टः सदा भवेत्}
{त्वयि सन्त्यक्तकर्मा यस्तन्मनस्ते शुभं गृहम्} %6-58

\twolineshloka
{यो न द्वेष्ट्यप्रियं प्राप्य प्रियं प्राप्य न हृष्यति}
{सर्वं मायेति निश्चित्य त्वां भजेत्तन्मनो गृहम्} %6-59

\twolineshloka
{षड्भावादिविकारान् यो देहे पश्यति नात्मनि}
{क्षुत्तृट् सुखं भयं दुःखं प्राणबुद्ध्योर्निरीक्षते} %6-60

\onelineshloka
{संसारधर्मैर्निर्मुक्तस्तस्य ते मानसं गृहम्} %6-61

\fourlineindentedshloka
{पश्यन्ति ये सर्वगुहाशयस्थम्}
{त्वां चिद्घनं सत्यमनन्तमेकम्}
{अलेपकं सर्वगतं वरेण्यम्}
{तेषां हृदब्जे सह सीतया वस} %6-62

\fourlineindentedshloka
{निरन्तराभ्यासदृढीकृतात्मनाम्}
{त्वत्पादसेवापरिनिष्ठितानाम्}
{त्वन्नामकीर्त्या हतकल्मषाणाम्}
{सीतासमेतस्य गृहं हृदब्जे} %6-63

\twolineshloka
{राम त्वन्नाममहिमा वर्ण्यते केन वा कथम्}
{यत्प्रभावादहं राम ब्रह्मर्षित्वमवाप्तवान्} %6-64

{॥इति श्रीमदध्यात्मरामायणे उमामहेश्वरसंवादे
अयोध्याकाण्डे षष्ठे सर्गे  वाल्मीकीरित-रामस्थानवर्णनं  सम्पूर्णम्॥}
