% !TeX program = XeLaTeX
% !TeX root = ../../shloka.tex

\sect{विभीषणकृतं  रामस्तोत्रम्}


% \addtocounter{shlokacount}{14}

\uvacha{श्रीमहादेव उवाच}

\twolineshloka
{विभीषणो महाभागश्चतुर्भिर्मन्त्रिभिः सह}
{आगत्य गगने रामसम्मुखे समवस्थितः} %3-1

\twolineshloka
{उच्चैरुवाच भोः स्वामिन् राम राजीवलोचन}
{रावणस्यानुजोऽहं ते दारहर्तुर्विभीषणः} %3-2

\twolineshloka
{नाम्ना भ्रात्रा निरस्तोऽहं त्वामेव शरणं गतः}
{हितमुक्तं मया देव तस्य चाविदितात्मनः} %3-3

\twolineshloka
{सीतां रामाय वैदेहीं प्रेषयेति पुनः पुनः}
{उक्तोऽपि न शृणोत्येव कालपाशवशं गतः} %3-4

\twolineshloka
{हन्तुं मां खड्गमादाय प्राद्रवद्राक्षसाधमः}
{ततोऽचिरेण सचिवैश्चतुर्भिः सहितो भयात्} %3-5

\twolineshloka
{त्वामेव भवमोक्षाय मुमुक्षुः शरणं गतः}
{विभीषणवचः श्रुत्वा सुग्रीवो वाक्यमब्रवीत्} %3-6

\twolineshloka
{विश्वासार्हो न ते राम मायावी राक्षसाधमः}
{सीताहर्तुर्विशेषेण रावणस्यानुजो बली} %3-7

\twolineshloka
{मन्त्रिभिः सायुधैरस्मान् विवरे निहनिष्यति}
{तदाज्ञापय मे देव वानरैर्हन्यतामयम्} %3-8

\twolineshloka
{ममैवं भाति मे राम बुद्ध्या किं निश्चितं वद}
{श्रुत्वा सुग्रीववचनं रामः सस्मितमब्रवीत्} %3-9

\twolineshloka
{यदीच्छामि कपिश्रेष्ठ लोकान् सर्वान् सहेश्वरान्}
{निमिषार्धेन संहन्यां सृजामि निमिषार्धतः} %3-10

\onelineshloka
{अतो मयाऽभयं दत्तं शीघ्रमानय राक्षसम्} %3-11

\twolineshloka
{सकृदेव प्रपन्नाय तवास्मीति च याचते}
{अभयं सर्वभूतेभ्यो ददाम्येतद्व्रतं मम} %3-12

\twolineshloka
{रामस्य वचनं श्रुत्वा सुग्रीवो हृष्टमानसः}
{विभीषणमथानाय्य दर्शयामास राघवम्} %3-13

\twolineshloka
{विभीषणस्तु साष्टाङ्गं प्रणिपत्य रघूत्तमम्}
{हर्षगद्गदया वाचा भक्त्या च परयान्वितः} %3-14

\twolineshloka
{रामं श्यामं विशालाक्षं प्रसन्नमुखपङ्कजम्}
{धनुर्बाणधरं शान्तं लक्ष्मणेन समन्वितम्} %3-15

\onelineshloka
{कृताञ्जलिपुटो भूत्वा स्तोतुं समुपचक्रमे} %3-16


\uvacha{विभीषण उवाच}

\twolineshloka
{नमस्ते राम राजेन्द्र नमः सीतामनोरम}
{नमस्ते चण्डकोदण्ड नमस्ते भक्तवत्सल} %3-17

\twolineshloka
{नमोऽनन्ताय शान्ताय रामायामिततेजसे}
{सुग्रीवमित्राय च ते रघूणां पतये नमः} %3-18

\twolineshloka
{जगदुत्पत्तिनाशानां कारणाय महात्मने}
{त्रैलोक्यगुरवेऽनादिगृहस्थाय नमो नमः} %3-19

\twolineshloka
{त्वमादिर्जगतां राम त्वमेव स्थितिकारणम्}
{त्वमन्ते निधनस्थानं स्वेच्छाचारस्त्वमेव हि} %3-20

\twolineshloka
{चराचराणां भूतानां बहिरन्तश्च राघव}
{व्याप्यव्यापकरूपेण भवान् भाति जगन्मयः} %3-21

\twolineshloka
{त्वन्मायया हृतज्ञाना नष्टात्मानो विचेतसः}
{गतागतं प्रपद्यन्ते पापपुण्यवशात्सदा} %3-22

\twolineshloka
{तावत्सत्यं जगद्भाति शुक्तिकारजतं यथा}
{यावन्न ज्ञायते ज्ञानं चेतसाऽनन्यगामिना} %3-23

\twolineshloka
{त्वदज्ञानात्सदा युक्ताः पुत्रदारगृहादिषु}
{रमन्ते विषयान् सर्वानन्ते दुःखप्रदान् विभो} %3-24

\twolineshloka
{त्वमिन्द्रोऽग्निर्यमो रक्षो वरुणश्च तथाऽनिलः}
{कुबेरश्च तथा रुद्रस्त्वमेव पुरुषोत्तम} %3-25

\twolineshloka
{त्वमणोरप्यणीयांश्च स्थूलात् स्थूलतरः प्रभो}
{त्वं पिता सर्वलोकानां माता धाता त्वमेव हि} %3-26

\twolineshloka
{आदिमध्यान्तरहितः परिपूर्णोऽच्युतोऽव्ययः}
{त्वं पाणिपादरहितश्चक्षुःश्रोत्रविवर्जितः} %3-27

\twolineshloka
{श्रोता द्रष्टा ग्रहीता च जवनस्त्वं खरान्तक}
{कोशेभ्यो व्यतिरिक्तस्त्वं निर्गुणो निरुपाश्रयः} %3-28

\twolineshloka
{निर्विकल्पो निर्विकारो निराकारो निरीश्वरः}
{षड्भावरहितोऽनादिः पुरुषः प्रकृतेः परः} %3-29

\twolineshloka
{मायया गृह्यमाणस्त्वं मनुष्य इव भाव्यसे}
{ज्ञात्वा त्वां निर्गुणमजं वैष्णवा मोक्षगामिनः} %3-30

\twolineshloka
{अहं त्वत्पादसद्भक्तिनिःश्रेणीं प्राप्य राघव}
{इच्छामि ज्ञानयोगाख्यं सौधमारोढुमीश्वर} %3-31

\twolineshloka
{नमः सीतापते राम नमः कारुणिकोत्तम}
{रावणारे नमस्तुभ्यं त्राहि मां भवसागरात्} %3-32

\twolineshloka
{ततः प्रसन्नः प्रोवाच श्रीरामो भक्तवत्सलः}
{वरं वृणीष्व भद्रं ते वाञ्छितं वरदोऽस्म्यहम्} %3-33

\uvacha{विभीषण उवाच}

\twolineshloka
{धन्योऽस्मि कृतकृत्योऽस्मि कृतकार्योऽस्मि राघव}
{त्वत्पाददर्शनादेव विमुक्तोऽस्मि न संशयः} %3-34

\twolineshloka
{नास्ति मत्सदृशो धन्यो नास्ति मत्सदृशः शुचिः}
{नास्ति मत्सदृशो लोके राम त्वन्मूर्तिदर्शनात्} %3-35

\twolineshloka
{कर्मबन्धविनाशाय त्वज्ज्ञानं भक्तिलक्षणम्}
{त्वद्ध्यानं परमार्थं च देहि मे रघुनन्दन} %3-36

\twolineshloka
{न याचे राम राजेन्द्र सुखं विषयसम्भवम्}
{त्वत्पादकमले सक्ता भक्तिरेव सदास्तु मे} %3-37

\twolineshloka
{ओमित्युक्त्वा पुनः प्रीतो रामः प्रोवाच राक्षसम्}
{शृणु वक्ष्यामि ते भद्रं रहस्यं मम निश्चितम्} %3-38

\twolineshloka
{मद्भक्तानां प्रशान्तानां योगिनां वीतरागिणाम्}
{हृदये सीतया नित्यं वसाम्यत्र न संशयः} %3-39

\twolineshloka
{तस्मात्त्वं सर्वदा शान्तः सर्वकल्मषवर्जितः}
{मां ध्यात्वा मोक्ष्यसे नित्यं घोरसंसारसागरात्} %3-40

\twolineshloka
{स्तोत्रमेतत्पठेद्यस्तु लिखेद्यः शृणुयादपि}
{मत्प्रीतये ममाभीष्टं सारूप्यं समवाप्नुयात्} %3-41

\twolineshloka
{इत्युक्त्वा लक्ष्मणं प्राह श्रीरामो भक्तभक्तिमान्}
{पश्यत्विदानीमेवैष मम सन्दर्शने फलम्} %3-42

\twolineshloka
{लङ्काराज्येऽभिषेक्ष्यामि जलमानय सागरात्}
{यावच्चन्द्रश्च सूर्यश्च यावत्तिष्ठति मेदिनी} %3-43

\twolineshloka
{यावन्मम कथा लोके तावद्राज्यं करोत्वसौ}
{इत्युक्त्वा लक्ष्मणेनाम्बु ह्यानाय्य कलशेन तम्} %3-44

\twolineshloka
{लङ्काराज्याधिपत्यार्थमभिषेकं रमापतिः}
{कारयामास सचिवैर्लक्ष्मणेन विशेषतः} %3-45

% \twolineshloka
{साधु साध्विति ते सर्वे वानरास्तुष्टुवुर्भृशम्।}

{॥इति श्रीमदध्यात्मरामायणे उमामहेश्वरसंवादे युद्धकाण्डे
तृतीये सर्गे विभीषणकृतं  श्री-रामस्तोत्रं सम्पूर्णम्॥}
