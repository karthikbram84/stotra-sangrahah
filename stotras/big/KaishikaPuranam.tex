% !TeX program = XeLaTeX
% !TeX root = ../../shloka.tex

%\chapter[विष्णुसहस्रनामस्तोत्रम्]{॥विष्णुसहस्रनामस्तोत्रम्॥}
\sect{श्री-कैशिकपुराणम्}

\twolineshloka*
{शुक्लाम्बरधरं विष्णुं शशिवर्णं चतुर्भुजम्}
{प्रसन्नवदनं ध्यायेत् सर्वविघ्नोपशान्तये}

\twolineshloka*
{वागीशाद्याः सुमनसः सर्वार्थानामुपक्रमे}
{यं नत्वा कृतकृत्याः स्युस्तं नमामि गजाननम्}

\dnsub{श्री-गुरु-प्रार्थना}

\twolineshloka*
{गुरुर्ब्रह्मा गुरुर्विष्णुर्गुरुर्देवो महेश्वरः}
{गुरुः साक्षात् परं ब्रह्म तस्मै श्री-गुरवे नमः}

\twolineshloka*
{सदाशिवसमारम्भां शङ्कराचार्यमध्यमाम्}
{अस्मदाचार्यपर्यन्तां वन्दे गुरुपरम्पराम्}

\twolineshloka*
{अखण्डमण्डलाकारं व्याप्तं येन चराचरम्}
{तत्पदं दर्शितं येन तस्मै श्री-गुरवे नमः}

\dnsub{श्री-सरस्वती-प्रार्थना}
\fourlineindentedshloka*
{दोर्भिर्युक्ता चतुर्भिः स्फटिकमणिनिभैरक्षमालां दधाना}
{हस्तेनैकेन पद्मं सितमपि च शुकं पुस्तकं चापरेण}
{भासा कुन्देन्दुशङ्खस्फटिकमणिनिभा भासमानाऽसमाना}
{सा मे वाग्देवतेयं निवसतु वदने सर्वदा सुप्रसन्ना}


\dnsub{पुराण मङ्गल-श्लोकाः}

\twolineshloka*
{नमस्तेऽस्तु वराहाय लीलयोद्धरते महीम्}
{खुरमध्यगतो यस्य मेरुः कणकणायते}


\twolineshloka*
{प्रलयोदन्वदुत्तीर्णां प्रपद्येऽहं वसुन्धराम्}
{महावराह-दंष्ट्राग्र-मल्लीकोश-मधुव्रताम्}



\uvacha{श्री-वराह उवाच}


\twolineshloka
{जागरे तु विशालाक्षि जानतो वाऽप्यजानतः}
{यो मे प्रगायते गेयं मम भक्त्या व्यवस्थितः}% ॥ 1 ॥

\twolineshloka
{यावन्ति त्वक्षराण्यस्य गीयमाने यशस्विनि}
{तावद्वर्षसहस्राणि स्वर्गलोके महीयते}% ॥ 2 ॥

\twolineshloka
{रूपवान् गुणवान् शुद्धः सर्वधर्मभृतां वरः}
{नित्यं पश्यति वै शक्रं वज्रहस्तं न संशयः}% ॥ 3 ॥

\twolineshloka
{मद्भक्तश्चापि जायेत इन्द्रेणैकपदे स्थितः}
{सर्वधर्मगुणश्रेष्ठस्तत्रापि मम लुब्धकः}% ॥ 4 ॥

\twolineshloka
{इन्द्रलोकात् परिभ्रष्टो मम गेयपरायणः}
{प्रमुक्तः सर्वसंसारैर्मम लोकं च गच्छति}% ॥ 5 ॥

\twolineshloka
{एवं तु वचनं श्रुत्वा तत्प्रसादाद्वसुन्धरा}
{वराहरूपिणं देवं प्रत्युवाच शुभानना}% ॥ 6 ॥

\uvacha{धरोवाच}

\twolineshloka
{अहो गीतप्रभावो वै यस्त्वया परिकीर्तितः}
{कश्च गीतप्रभावेन सिद्धिं प्राप्तो महातपाः}% ॥ 7 ॥


\uvacha{श्री-वराह उवाच}

\twolineshloka
{शृणु तत्त्वेन ते देवि कथ्यमानं यशस्विनम्}
{यस्तु गीतप्रभावेन सिद्धिं प्राप्तो महातपाः}% ॥ 8 ॥

\twolineshloka
{अस्ति दक्षिणदिग्भागे महेन्द्रो नाम पर्वतः}
{तत्र क्षीरनदी पुण्या दक्षिणे सागरङ्गमा}% ॥ 9 ॥

\twolineshloka
{तत्र सिद्धाश्रमे भद्रे चण्डालः कृतनिश्चयः}
{दूराज्जागरणे गाति मम भक्त्या व्यवस्थितः}% ॥ 10 ॥

\twolineshloka
{एवं तु गायमानस्य गताः संवत्सरा दश}
{श्वपाकस्य गुणज्ञस्य मद्भक्तस्य वसुन्धरे}% ॥ 11 ॥

\twolineshloka
{कौमुदस्य तु मासस्य द्वादश्यां शुक्लपक्षके}
{सुप्ते जने गते यामे वीणामादाय निर्ययौ}% ॥ 12 ॥

\twolineshloka
{ततो वर्त्मनि चण्डालो गृहीतो ब्रह्मरक्षसा}
{अल्पप्राणः श्वपाको वै बलवान् ब्रह्मराक्षसः}% ॥ 13 ॥

\twolineshloka
{दुःखेन स तु सन्तप्तो न च शक्तो विचेष्टितुम्}
{उवाच वचनं मन्दं मातङ्गो ब्रह्मराक्षसम्}% ॥ 14 ॥

\twolineshloka
{गच्छामि सन्तोषयितुमहं जागरणे हरिम्}
{गानेन पुण्डरीकाक्षं ब्रह्मराक्षस मुञ्च माम्}% ॥ 15 ॥

\twolineshloka
{एवमुक्तः श्वपाकेन बलवान् ब्रह्मराक्षसः}
{अमर्षवशमापन्नो न च किञ्चित् तमब्रवीत्}% ॥ 16 ॥

\twolineshloka
{आत्मानं प्रतिधावन्तं चण्डालः ब्रह्मराक्षसम्}
{किं त्वया चेष्टितव्यं मे य एवं परिधावसि}% ॥ 17 ॥

\twolineshloka
{श्वपाकवचनं श्रुत्वा ततो वै ब्रह्मराक्षसः}
{उवाच वचनं घोरं मानुषाहारलोलुपः}% ॥ 18 ॥

\twolineshloka
{अद्य मे दशरात्रं वै निराहारस्य गच्छतः}
{धात्रा त्वं विहितो मह्यमाहारः परितो मम}% ॥ 19 ॥

\twolineshloka
{अद्य त्वां भक्षयिष्यामि सवसामांसशोणितम्}
{तर्पयित्वा यथान्यायं यास्यामि च यथेप्सितम्}% ॥ 20 ॥

\twolineshloka
{ब्रह्मरक्षोवचः श्रुत्वा श्वपाको गीतलालसः}
{राक्षसं छन्दयामास मम भक्त्या व्यवस्थितः}% ॥ 21 ॥

\twolineshloka
{शृणु तत्त्वं महाभाग भक्ष्योऽहं समुपागतः}
{अवश्यमेतत् कर्तव्यं धात्रा दत्तं यथा तव}% ॥ 22 ॥

\twolineshloka
{पश्चात्खादसि मां रक्षो जागरे विनिवर्तिते}
{विष्णोः सन्तोषणार्थाय ममैतद् व्रतमुत्तमम्}% ॥ 23 ॥

\twolineshloka
{रक्ष मां व्रतभङ्गाद्वै देवं नारायणं प्रति}
{जागरे विनिवृत्ते तु मां भक्षय यथेप्सितम्}% ॥ 24 ॥

\twolineshloka
{श्वपाकस्य वचः श्रुत्वा ब्रह्मरक्षः क्षुधाऽर्दितम्}
{उवाच मधुरं वाक्यं श्वपाकं तदनन्तरम्}% ॥ 25 ॥

\twolineshloka
{मोघं भाषसि चण्डाल पुनरेष्याम्यहं त्विति}
{को हि रक्षोमुखाद् भ्रष्टस्तन्मुखायाभिवर्धते}% ॥ 26 ॥

\twolineshloka
{बहवः सन्ति पन्थानो देशाश्च बहवस्तथा}
{आत्मदेशं परित्यज्य परेषां गन्तुमिच्छसि}% ॥ 27 ॥

\twolineshloka
{स्वशरीरविनाशाय न चाऽऽगच्छति कश्चन}
{रक्षसो मुखविभ्रष्टः पुनरागन्तुमिच्छसि}% ॥ 28 ॥

\twolineshloka
{राक्षसस्य वचः श्रुत्वा चण्डालो धर्मसंश्रितम्}
{उवाच मधुरं वाक्यं राक्षसं पिशिताशनम्}% ॥ 29 ॥

\twolineshloka
{यद्यप्यहं हि चण्डालः पूर्वकर्मविदूषितः}
{प्राप्तोऽहं मानुषं भावं विदितेनान्तरात्मना}% ॥ 30 ॥

\twolineshloka
{शृणु तत्समयं रक्षो येनाऽऽगच्छाम्यहं पुनः}
{दूराज्जागरणं कृत्वा लोकनाथस्य तृप्तये}% ॥ 31 ॥

\twolineshloka
{सत्यमूलं जगत्सर्वं लोकः सत्ये प्रतिष्ठितः}
{नाहं मिथ्या प्रवक्ष्यामि सत्यमेव वदाम्यहम्}% ॥ 32 ॥

\twolineshloka
{अद्य मे समयस्तत्र ब्रह्मराक्षस तं शृणु}
{शपामि सत्येन गतो यद्यहं नाऽऽगमे पुनः}% ॥ 33 ॥

\twolineshloka
{यो गच्छेत् परदारांश्च काममोहप्रपीडितः}
{तस्य पापेन लिप्येयं यद्यहं नाऽऽगमे पुनः}% ॥ 34 ॥

\twolineshloka
{पाकभेदं तु यः कुर्यादात्मनश्चोपभुञ्जतः}
{तस्य पापेन लिप्येयं यद्यहं नाऽऽगमे पुनः}% ॥ 35 ॥

\twolineshloka
{दत्त्वा वै भूमिदानं तु पुनराच्छिन्दतीह यः}
{तस्य पापेन लिप्येयं यद्यहं नाऽऽगमे पुनः}% ॥ 36 ॥

\twolineshloka
{स्त्रियं भुक्त्वा रूपवतीं पुनर्यस्तां विनिन्दति}
{तस्य पापेन लिप्येयं यद्यहं नाऽऽगमे पुनः}% ॥ 37 ॥

\twolineshloka
{योऽमावास्यां विशालाक्षि श्राद्धं कृत्वा स्त्रियं व्रजेत्}
{तस्य पापेन लिप्येयं यद्यहं नाऽऽगमे पुनः}% ॥ 38 ॥

\twolineshloka
{भुक्त्वा परस्य चान्नानि यस्तं निन्दति निर्घृणः}
{तस्य पापेन लिप्येयं यद्यहं नाऽऽगमे पुनः}% ॥ 39 ॥

\twolineshloka
{यस्तु कन्यां ददामीति पुनस्तां न प्रयच्छति}
{तस्य पापेन लिप्येयं यद्यहं नाऽऽगमे पुनः}% ॥ 40 ॥

\twolineshloka
{षष्ठ्यष्टम्योरमावास्याचतुर्दश्योश्च नित्यशः}
{अस्नातानां गतिं गच्छे यद्यहं नाऽऽगमे पुनः}% ॥ 41 ॥

\twolineshloka
{दास्यामीति प्रतिश्रुत्य न च यस्तत्प्रयच्छति}
{गतिं तस्य प्रपद्ये वै यद्यहं नाऽऽगमे पुनः}% ॥ 42 ॥

\twolineshloka
{मित्रभार्यां तु यो गच्छेत् कामबाणवशानुगः}
{तस्य पापेन लिप्येयं यद्यहं नाऽऽगमे पुनः}% ॥ 43 ॥

\twolineshloka
{गुरुपत्नीं राजपत्नीं ये तु गच्छन्ति मोहिताः}
{तेषां गतिं प्रपद्ये वै यद्यहं नाऽऽगमे पुनः}% ॥ 44 ॥

\twolineshloka
{यो वै दारद्वयं कृत्वा एकस्यां प्रीतिमान् भवेत्}
{गतिं तस्य प्रपद्ये वै यद्यहं नाऽऽगमे पुनः}% ॥ 45 ॥

\twolineshloka
{अनन्यशरणां भार्यां यौवने यः परित्यजेत्}
{तस्य पापेन लिप्येयं यद्यहं नाऽऽगमे पुनः}% ॥ 46 ॥

\twolineshloka
{गोकुलस्य तृषार्तस्य जलार्थमभिधावतः}
{विघ्नमाचरते यस्तु तत्पापं स्यादनागमे}% ॥ 47 ॥

\twolineshloka
{ब्रह्मघ्ने च सुरापे च चोरे भग्नव्रते तथा}
{या गतिर्विहिता सद्भिः तत्पापं स्यादनागमे}% ॥ 48 ॥

\twolineshloka
{वासुदेवं परित्यज्य येऽन्यं देवमुपासते}
{तेषां गतिं प्रपद्ये वै यद्यहं नाऽऽगमे पुनः}% ॥ 49 ॥

\twolineshloka
{नारायणमथान्यैस्तु देवैस्तुल्यं करोति यः}
{तस्य पापेन लिप्येयं यद्यहं नाऽऽगमे पुनः}% ॥ 50 ॥

\twolineshloka
{चण्डालवचनं श्रुत्वा परितुष्टस्तु राक्षसः}
{उवाच मधुरं वाक्यं गच्छ शीघ्रं नमोऽस्तु ते}% ॥ 51 ॥

\twolineshloka
{राक्षसेन विनिर्मुक्तश्चण्डालः कृतनिश्चयः}
{पुनर्गायति मह्यं वै मम भक्त्या व्यवस्थितः}% ॥ 52 ॥

\twolineshloka
{अथ प्रभाते विमले विनिवृत्ते तु जागरे}
{नमो नारायणेत्युक्त्वा श्वपाकः पुनरागमत्}% ॥ 53 ॥

\twolineshloka
{गच्छतस्त्वरितं तस्य पुरुषः पुरतः स्थितः}
{उवाच मधुरं वाक्यं श्वपाकं तदनन्तरम्}% ॥ 54 ॥

\twolineshloka
{कुतो गच्छसि चण्डाल द्रुतं गमननिश्चितम्}
{एतदाचक्ष्व तत्त्वेन यत्र ते वर्तते मनः}% ॥ 55 ॥

\twolineshloka
{तस्य तद्वचनं श्रुत्वा श्वपाकः सत्यसम्मतः}
{उवाच मधुरं वाक्यं पुरुषं तदनन्तरम्}% ॥ 56 ॥

\twolineshloka
{समयो मे कृतो यत्र ब्रह्मराक्षससन्निधौ}
{तत्राहं गन्तुमिच्छामि यत्रासौ ब्रह्मराक्षसः}% ॥ 57 ॥

\twolineshloka
{श्वपाकवचनं श्रुत्वा पुरुषो भावशोधकः}
{उवाच मधुरं वाक्यं श्वपाकं तदनन्तरम्}% ॥ 58 ॥

\twolineshloka
{न तत्र गच्छ चण्डाल मार्गेणानेन सुव्रत}
{तत्रासौ राक्षसः पापः पिशिताशी दुरासदः}% ॥ 59 ॥

\twolineshloka
{पुरुषस्य वचः श्रुत्वा श्वपाकः सत्यसङ्गरः}
{मरणं तत्र निश्चित्य मधुरं वाक्यमब्रवीत्}% ॥ 60 ॥

\twolineshloka
{नाहमेवं करिष्यामि यन्मां त्वं परिपृच्छसि}
{अहं सत्ये प्रवृत्तो वै शीलं सत्ये प्रतिष्ठितम्}% ॥ 61 ॥

\twolineshloka
{ततः स पद्मपत्राक्षः श्वपाकं प्रत्युवाच ह}
{यद्येवं निश्चयस्तात स्वस्ति तेऽस्तु गमिष्यतः}% ॥ 62 ॥

\twolineshloka
{ब्रह्मरक्षोन्तिकं प्राप्य सत्येऽसौ कृतनिश्चयः}
{उवाच मधुरं वाक्यं राक्षसं पिशिताशनम्}% ॥ 63 ॥

\twolineshloka
{भवता समनुज्ञातो गानं कृत्वा यथेप्सया}
{विष्णवे लोकनाथाय मम पूर्णो मनोरथः}% ॥ 64 ॥

\twolineshloka
{एतानि मम चाङ्गानि भक्षयस्व यथेच्छया}
{श्वपाकवचनं श्रुत्वा ब्रह्मरक्षो भयानकम्}% ॥ 65 ॥

\twolineshloka
{उवाच मधुरं वाक्यं श्वपाकं संशितव्रतम्}
{त्वमद्य रात्रौ चण्डाल विष्णोर्जागरणं प्रति}% ॥ 66 ॥

\twolineshloka
{फलं गीतस्य मे देहि जीवितं यदि चेच्छसि}
{ब्रह्मरक्षो वचः श्रुत्वा श्वपाकः पुनरब्रवीत्}% ॥ 67 ॥

\twolineshloka
{यत् त्वया भाषितं पूर्वं मया सत्यं च यत्कृतम्}
{भक्षयस्व यथेच्छं मां दद्यां गीतफलं न तु}% ॥ 68 ॥

\twolineshloka
{चण्डालस्य वचः श्रुत्वा हेतुयुक्तमनन्तरम्}
{उवाच मधुरं वाक्यं चण्डालं ब्रह्मराक्षसः}% ॥ 69 ॥

\twolineshloka
{अथवाऽर्धं तु मे देहि पुण्यं गीतस्य यत्फलम्}
{ततो मोक्ष्यामि कल्याण भक्षादस्माद्विभीषणात्}% ॥ 70 ॥

\twolineshloka
{ब्रह्मरक्षो वचः श्रुत्वा श्वपाकः संशितव्रतः}
{वाणीं श्लक्ष्णां समादाय ब्रह्मराक्षसमब्रवीत्}% ॥ 71 ॥

\twolineshloka
{भक्षयामीति संश्रुत्य गीतमन्यत् किमिच्छसि}
{श्वपाकस्य वचः श्रुत्वा ब्रह्मरक्षो भयावहम्}% ॥ 72 ॥

\twolineshloka
{उवाच मधुरं वाक्यं श्वपाकं संशितव्रतम्}
{एकयामस्य मे देहि पुण्यं गीतस्य यत्फलम्}% ॥ 73 ॥

\twolineshloka
{ततो यास्यसि कल्याण सङ्गमं पुत्रदारकैः}
{श्रुत्वा राक्षस वाक्यानि चण्डालो गीतलालसः}% ॥ 74 ॥

\twolineshloka
{उवाच मधुरं वाक्यं राक्षसं कृतनिश्चयः}
{न यामस्य फलं दद्यां ब्रह्मरक्षस्तवेप्सितम्}% ॥ 75 ॥

\twolineshloka
{पिबस्व शोणितं मह्यं यत् त्वया पूर्वभाषितम्}
{श्वपाकस्य वचः श्रुत्वा राक्षसः पिशिताशनः}% ॥ 76 ॥

\twolineshloka
{सत्यवन्तं गुणज्ञं च चण्डालमिदमब्रवीत्}
{एकं गीतस्य मे देहि यत् त्वया विष्णुसंसदि}% ॥ 77 ॥

\twolineshloka
{निग्रहात्तारयास्माद्वै तेन गीतफलेन माम्}
{एवमुक्त्वा तु चण्डालं राक्षसः शरणं गतः}% ॥ 78 ॥

\twolineshloka
{श्रुत्वा राक्षसवाक्यानि श्वपाकः संशितव्रतः}
{उवाच मधुरं वाक्यं राक्षसं पिशिताशनम्}% ॥ 79 ॥

\twolineshloka
{किं त्वया दुष्कृतं कर्म कृतपूर्वं तु राक्षसः}
{कर्मणो यस्य दोषेण राक्षसीं योनिमाश्रितः}% ॥ 80 ॥

\twolineshloka
{एवमुक्तः श्वपाकेन पूर्ववृत्तमनुस्मरन्}
{राक्षसः शरणं गत्वा श्वपाकमिदमब्रवीत्}% ॥ 81 ॥

\twolineshloka
{नाम्ना वै सोमशर्माऽहं चरको ब्रह्मयोनिजः}
{सूत्रमन्त्रपरिभ्रष्टो यूपकर्मण्यधिष्ठितः}% ॥ 82 ॥

\twolineshloka
{ततोऽहं कारये यज्ञं लोभमोहप्रपीडितः}
{यज्ञे प्रवर्तमाने तु शूलदोषस्त्वजायत}% ॥ 83 ॥

\twolineshloka
{अथ पञ्चमरात्रे तु असमाप्ते क्रतावहम्}
{अकृत्वा विमलं कर्म ततः पञ्चत्वमागतः}% ॥ 84 ॥

\twolineshloka
{तस्य यज्ञस्य दोषेण मातङ्ग शृणु यन्मम}
{जातोऽस्मि राक्षसस्तत्र ब्राह्मणो ब्रह्मराक्षसः}% ॥ 85 ॥

\twolineshloka
{एवं तु यज्ञदोषेण वपुः प्राप्तमिदं मम}
{इत्युक्त्वा तु तदा रक्षः श्वपाकं शरणं गतम्}% ॥ 86 ॥

\twolineshloka
{ब्रह्मरक्षो वचः श्रुत्वा श्वपाकः संशितव्रतः}
{बाढमित्यब्रवीद्वाक्यं ब्रह्मराक्षसचोदितः}% ॥ 87 ॥

\twolineshloka
{यन्मया पश्चिमं गीतं स्वरं कैशिकमुत्तमम्}
{फलेन तस्य भद्रं ते मोक्षयिष्यामि किल्बिषात्}% ॥ 88 ॥

\uvacha{श्री-वराह उवाच}

\twolineshloka
{यस्तु गायति भक्त्या वै कैशिकं मम संसदि}
{स तारयति दुर्गाणि श्वपाको राक्षसं यथा}% ॥ 89 ॥

\twolineshloka
{एवं तत्र वरं गृह्य राक्षसो ब्रह्मसंस्थितः}
{जातस्तु विमले वंशे मम लोकं च गच्छति}% ॥ 90 ॥

\twolineshloka
{श्वपाकश्चापि सुश्रोणि मम चैवोपगायकः}
{कृत्वा तु विपुलं कर्म स ब्रह्मत्वमुपागतः}% ॥ 91 ॥

\twolineshloka
{एतद्गीतफलं देवि कौमुदद्वादशीं पुनः}
{यस्तु गायति स श्रीमान् मम लोकं च गच्छति}% ॥ 92 ॥

॥ इति श्री-वाराह-पुराणे भूमि-वराह-संवादे श्री-कैशिक-माहात्म्यं नाम अष्टचत्वारिंशोऽध्यायः॥

\twolineshloka*
{कायेन वाचा मनसेन्द्रियैर्वा बुद्‌ध्याऽऽत्मना वा प्रकृतेः स्वभावात्}
{करोमि यद्यत् सकलं परस्मै नारायणायेति समर्पयामि}
