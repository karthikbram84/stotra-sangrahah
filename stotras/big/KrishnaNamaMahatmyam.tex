% !TeX program = XeLaTeX
% !TeX root = ../../shloka.tex

\sect{श्री-स्कान्दपुराणान्तर्गत-मार्गशीर्ष-माहात्म्ये कृष्ण-नाम-माहात्म्यम्}

\addtocounter{shlokacount}{31}


\twolineshloka
{वक्तव्यमेव यत्प्रोक्तं तच्छृणुष्व समाहितः}
{कथयिष्ये तव प्रीत्या अपि गुह्यतरं मम} % ३२॥

\twolineshloka
{मम नाम प्रवक्तव्यं सहे चैव विशेषतः}
{कृष्णकृष्णेति वक्तव्यं मम प्रीतिकरं परम्} % ३३॥

\twolineshloka
{प्रतिज्ञैषा च मे पुत्र न जानन्ति सुरासुराः}
{मनसा कर्मणा वाचा यो मे शरणमागतः} % ३४॥

\twolineshloka
{स हि सर्वामवाप्नोति कामनामिह लौकिकीम्}
{सर्वोत्कृष्टं च वैकुण्ठं मत्प्रियां कमलामपि} % ३५॥

\twolineshloka
{कृष्णकृष्णेति कृष्णेति यो मां स्मरति नित्यशः}
{जलं भित्त्वा यथा पद्मं नरकादुद्धराम्यहम्} % ३६॥

\twolineshloka
{विनोदेनापि दम्भेन मौढ्याल्लोभाच्छलादपि}
{यो मां भजत्यसौ वत्स मद्भक्तो नावसीदति} % ३७॥

\twolineshloka
{ये वै पठन्ति कृष्णेति मरणे पर्युपस्थिते}
{यदि पापयुताः पुत्र न पश्यंति यमं क्वचित्} % ३८॥

\twolineshloka
{पूर्वे वयसि पापानि कृतान्यपि च कृत्स्नशः}
{अन्तकाले च कृष्णेति स्मृत्वा मामेत्यसंशयम्} % ३९॥

\twolineshloka
{नमः कृष्णाय महते विवशोऽपि वदेद्यदि}
{ध्रुवं पदमवाप्नोति मरणे पर्युपस्थिते} % ४०॥

\twolineshloka
{श्रीकृष्णेति कृतोच्चारैः प्राणैर्यदि वियुज्यते}
{दूरस्थः पश्यति च तं स्वर्गतं प्रेतनायकः} % ४१॥

\twolineshloka
{श्मशाने यदि रथ्यायां कृष्णकृष्णेति जल्पति}
{म्रियते यदि चेत्पुत्र मामेवैति न संशयः} % ४२॥

\twolineshloka
{दर्शनान्मम भक्तानां मृत्युमाप्नोति यः क्वचित्}
{विना मत्स्मरणात्पुत्र मुक्तिमेति स मानवः} % ४३॥

\twolineshloka
{पापानलस्य दीप्तस्य भयं मा कुरु पुत्रक}
{श्रीकृष्णनाममेघोत्थैः सिच्यते नीरबिंदुभिः} % ४४॥

\twolineshloka
{कलिकालभुजङ्गस्य तीक्ष्णदंष्ट्रस्य किं भयम्}
{श्रीकृष्णनामदारूत्थवह्निदग्धः स नश्यति} % ४८॥

\twolineshloka
{पापपावकदग्धानां कर्मचेष्टावियोगिनाम्}
{भेषजं नास्ति मर्त्यानां श्रीकृष्णस्मरणं विना} % ४६॥

\twolineshloka
{प्रयागे वै यथा गङ्गा शुक्लतीर्थे च नर्मदा}
{सरस्वती कुरुक्षेत्रे तद्वच्छ्रीकृष्णकीर्तनम्} % ४७॥

\twolineshloka
{भवाम्भोधिनिमग्नानां महापापोर्मिपातिनाम्}
{न गतिर्मानवानां च श्रीकृष्णस्मरणं विना} % ४८॥

\twolineshloka
{मृत्युकालेऽपि मर्त्यानां पापिनां तदनिच्छताम्}
{गच्छतां नाऽस्ति पाथेयं श्रीकृष्णस्मरणं विना} % ४९॥

\twolineshloka
{तत्र पुत्र गया काशी पुष्करं कुरुजाङ्गलम्}
{प्रत्यहं मन्दिरे यस्य कृष्णकृष्णेति कीर्तनम्} % ५०॥

\twolineshloka
{जीवितं जन्मसाफल्यं सुखं तस्यैव सार्थकम्}
{सततं रसना यस्य कृष्णकृष्णेति जल्पति} % ५१॥

\twolineshloka
{सकृदुच्चरितं येन हरिरित्यक्षरद्वयम्}
{बद्धः परिकरस्तेन मोक्षाय गमनं प्रति} % ५२॥

\twolineshloka
{नाम्नोऽस्य यावती शक्तिः पापनिर्दहने मम}
{तावत्कर्तुं न शक्नोति पातकं पातकी जनः} % ५३॥

\twolineshloka
{नाऽपविद्धं भवेत् तस्य शरीरं नैव मानसम्}
{न पापं न च वैक्लव्यं कृष्णकृष्णेति कीर्तनात्} % ५४॥

\twolineshloka
{श्रीकृष्णेति वचः पथ्यं न त्यजेद् यः कलौ नरः}
{पापामयो वै न भवेत्कलौ तस्यैव मानसे} % ५५॥

\twolineshloka
{श्रीकृष्णेति प्रजल्पन्तं दक्षिणाशापतिर्नरम्}
{श्रुत्वा मार्जयते पापं तस्य जन्मशतार्जितम्} % ५६॥

\twolineshloka
{चान्द्रायणशतैः पापं पराकाणां सहस्रकैः}
{यन्नापयाति तद्याति कृष्णकृष्णेति कीर्तनात्} % ५७॥

\twolineshloka
{नान्याभिर्नामकोटीभिस्तोषो मम भवेत् क्वचित्}
{श्रीकृष्णेति कृतोच्चारे प्रीतिरेवाधिकाधिका} % ५८॥

\twolineshloka
{चन्द्रसूर्योपरागैस्तु कोटीभिर्यत्फलं स्मृतम्}
{तत्फलं समवाप्नोति कृष्णकृष्णेति कीर्तनात्} % ५९॥

\twolineshloka
{गुरुदाराभिगमनं हेमस्तेयादि पातकम्}
{श्रीकृष्णकीर्तनाद् याति घर्मतप्तं हिमं यथा} % ६०॥

\twolineshloka
{युक्तो यदि महापापैरगम्यागमनादिभिः}
{मुच्यते चान्तकालेऽपि सकृच्छ्रीकृष्णकीर्तनात्} % ६१॥

\twolineshloka
{अविशुद्धमना यस्तु विनाप्याचारवर्तनात्}
{प्रेतत्वं सोऽपि नाप्नोति अन्ते श्रीकृष्णकीर्तनात्} % ६२॥

\twolineshloka
{मुखे भवतु मा जिह्वाऽसती यातु रसातलम्}
{न सा चेत् कलिकाले या श्रीकृष्णगुणवादिनी} % ६३॥

\twolineshloka
{स्ववक्त्रे परवक्त्रे च वन्द्या जिह्वा प्रयत्नतः}
{कुरुते या कलौ पुत्र श्रीकृष्ण-गुणकीर्तनम्} % ६४॥

\twolineshloka
{पापवल्ली मुखे तस्य जिह्वारूपेण कीर्त्यते}
{या न वक्ति दिवारात्रौ श्रीकृष्णगुणकीर्तनम्} % ६५॥

\twolineshloka
{पततां शतखण्डा तु सा जिह्वा रोगरूपिणी}
{श्रीकृष्णकृष्णकृष्णेति श्रीकृष्णेति न जल्पति} % ६६॥

\twolineshloka
{श्रीकृष्णनाममाहात्म्यं प्रातरुत्थाय यः पठेत्}
{तस्याऽहं श्रेयसां दाता भवाम्येव न संशयः} % ६७॥

\twolineshloka
{श्रीकृष्णनाममाहात्म्यं त्रिसन्ध्यं हि पठेत् तु यः}
{सर्वान्कामानवाप्नोति स मृतः परमां गतिम्} % ६८॥


॥इति श्रीस्कान्दे महापुराणे एकाशीतिसाहस्र्यां संहितायां द्वितीये वैष्णवखण्डे ब्रह्मविष्णुसंवादे मार्गशीर्षमासमाहात्म्ये श्रीकृष्णनाममाहात्म्यवर्णनं नाम पञ्चदशोऽध्यायः॥
