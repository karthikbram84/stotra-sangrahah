% !TeX program = XeLaTeX
% !TeX root = ../../shloka.tex

\sect{वक्रतुण्डमहागणपतिसहस्रनामस्तोत्रम्}

\uvacha{मुनिरुवाच}
\twolineshloka
{कथं नाम्नां सहस्रं तं गणेश उपदिष्टवान्}
{शिवदं तन्ममाचक्ष्व लोकानुग्रहतत्पर}

\uvacha{ब्रह्मोवाच}

\twolineshloka
{देवः पूर्वं पुरारातिः पुरत्रयजयोद्यमे}
{अनर्चनाद्गणेशस्य जातो विघ्नाकुलः किल}

\twolineshloka
{मनसा स विनिर्धार्य ददृशे विघ्नकारणम्}
{महागणपतिं भक्त्या समभ्यर्च्य यथाविधि}

\twolineshloka
{विघ्नप्रशमनोपायमपृच्छदपरिश्रमम्}
{सन्तुष्टः पूजया शम्भोर्महागणपतिः स्वयम्}

\twolineshloka
{सर्वविघ्नप्रशमनं सर्वकामफलप्रदम्}
{ततस्तस्मै स्वयं नाम्नां सहस्रमिदमब्रवीत्}

अस्य श्रीवक्रतुण्डमहागणपतिसहस्रनामस्तोत्रमन्त्रस्य\\
महागणपतिर्ऋषिः। अनुष्टुप् छन्दः। महागणपतिर्देवता।\\
गं बीजम्। हुं शक्तिः। स्वाहा कीलकम्।\\
चतुर्विधपुरुषार्थसिद्ध्यर्थे जपे विनियोगः।

\dnsub{करन्यासः}
{गणेश्वरो गणक्रीड इत्यङ्गुष्ठाभ्यां नमः।}\\
{कुमारगुरुरीशान इति तर्जनीभ्यां नमः।}\\
{ब्रह्माण्डकुम्भश्चिद्व्योमेति मध्यमाभ्यां नमः।}\\
{रक्तो रक्ताम्बरधर इत्यनामिकाभ्यां नमः।}\\
{सर्वसद्गुरुसंसेव्य इति कनिष्ठिकाभ्यां नमः।}\\
{लुप्तविघ्नः स्वभक्तानामिति करतलकरपृष्ठाभ्यां नमः।}\\


\dnsub{हृदयादिन्यासः}
छन्दश्छन्दोद्भव इति हृदयाय नमः।\\
निष्कलो निर्मल इति शिरसे स्वाहा।\\
सृष्टिस्थितिलयक्रीड इति शिखायै वषट्।\\
ज्ञानं विज्ञानमानन्द इति कवचाय हुम्।\\
अष्टाङ्गयोगफलभृदिति नेत्रत्रयाय वौषट्।\\
अनन्तशक्तिसहित इत्यस्त्राय फट्।\\
भूर्भुवः स्वरोम् इति दिग्बन्धः।\\


\dnsub{ध्यानम्}
\fourlineindentedshloka*
{गजवदनमचिन्त्यं तीक्ष्णदंष्ट्रं त्रिनेत्रम्}
{बृहदुदरमशेषं भूतिराजं पुराणम्}
{अमरवरसुपूज्यं रक्तवर्णं सुरेशम्}
{पशुपतिसुतमीशं विघ्नराजं नमामि}

\fourlineindentedshloka*
{खर्वं स्थूलतनुं गजेन्द्रवदनं लम्बोदरं सुन्दरम्}
{प्रस्यन्दन्मदगन्धलुब्धमधुपव्यालोलगण्डस्थलम्}
{दन्ताघातविदारितारिरुधिरैः सिन्दूरशोभाकरम्}
{वन्दे शैलसुतासुतं गणपतिं सिद्धिप्रदं कामदम्}

सकलविघ्नविनाशनद्वारा श्रीमहागणपतिप्रसादसिद्ध्यर्थे जपे विनियोगः।


\dnsub{स्तोत्रम्}

\uvacha{श्रीगणपतिरुवाच}
\resetShloka
\twolineshloka
{ॐ गणेश्वरो गणक्रीडो गणनाथो गणाधिपः}
{एकदन्तो वक्रतुण्डो गजवक्त्रो महोदरः}

\twolineshloka
{लम्बोदरो धूम्रवर्णो विकटो विघ्ननाशनः}
{सुमुखो दुर्मुखो बुद्धो विघ्नराजो गजाननः}

\twolineshloka
{भीमः प्रमोद आमोदः सुरानन्दो मदोत्कटः}
{हेरम्बः शम्बरः शम्भुर्लम्बकर्णो महाबलः}

\twolineshloka
{नन्दनो लम्पटो भीमो मेघनादो गणञ्जयः}
{विनायको विरूपाक्षो वीरः शूरवरप्रदः}

\twolineshloka
{महागणपतिर्बुद्धिप्रियः क्षिप्रप्रसादनः}
{रुद्रप्रियो गणाध्यक्ष उमापुत्रोऽघनाशनः}

\twolineshloka
{कुमारगुरुरीशानपुत्रो मूषकवाहनः}
{सिद्धिप्रियः सिद्धिपतिः सिद्धः सिद्धिविनायकः}

\twolineshloka
{अविघ्नस्तुम्बुरुः सिंहवाहनो मोहिनीप्रियः}
{कटङ्कटो राजपुत्रः शाकलः सम्मितोऽमितः}

\twolineshloka
{कूष्माण्डसामसम्भूतिर्दुर्जयो धूर्जयो जयः}
{भूपतिर्भुवनपतिर्भूतानां पतिरव्ययः}

\twolineshloka
{विश्वकर्ता विश्वमुखो विश्वरूपो निधिर्गुणः}
{कविः कवीनामृषभो ब्रह्मण्यो ब्रह्मवित्प्रियः}

\twolineshloka
{ज्येष्ठराजो निधिपतिर्निधिप्रियपतिप्रियः}
{हिरण्मयपुरान्तःस्थः सूर्यमण्डलमध्यगः}

\twolineshloka
{कराहतिध्वस्तसिन्धुसलिलः पूषदन्तभित्}
{उमाङ्ककेलिकुतुकी मुक्तिदः कुलपावनः}

\twolineshloka
{किरीटी कुण्डली हारी वनमाली मनोमयः}
{वैमुख्यहतदैत्यश्रीः पादाहतिजितक्षितिः}

\twolineshloka
{सद्योजातः स्वर्णमुञ्जमेखली दुर्निमित्तहृत्}
{दुःस्वप्नहृत्प्रसहनो गुणी नादप्रतिष्ठितः}

\twolineshloka
{सुरूपः सर्वनेत्राधिवासो वीरासनाश्रयः}
{पीताम्बरः खण्डरदः खण्डवैशाखसंस्थितः}

\twolineshloka
{चित्राङ्गः श्यामदशनो भालचन्द्रो हविर्भुजः}
{योगाधिपस्तारकस्थः पुरुषो गजकर्णकः}

\twolineshloka
{गणाधिराजो विजयः स्थिरो गजपतिर्ध्वजी}
{देवदेवः स्मरः प्राणदीपको वायुकीलकः}

\twolineshloka
{विपश्चिद्वरदो नादो नादभिन्नमहाचलः}
{वराहरदनो मृत्युञ्जयो व्याघ्राजिनाम्बरः}

\twolineshloka
{इच्छाशक्तिभवो देवत्राता दैत्यविमर्दनः}
{शम्भुवक्त्रोद्भवः शम्भुकोपहा शम्भुहास्यभूः}

\twolineshloka
{शम्भुतेजाः शिवाशोकहारी गौरीसुखावहः}
{उमाङ्गमलजो गौरीतेजोभूः स्वर्धुनीभवः}

\twolineshloka
{यज्ञकायो महानादो गिरिवर्ष्मा शुभाननः}
{सर्वात्मा सर्वदेवात्मा ब्रह्ममूर्धा ककुप्श्रुतिः}

\twolineshloka
{ब्रह्माण्डकुम्भश्चिद्व्योमभालःसत्यशिरोरुहः}
{जगज्जन्मलयोन्मेषनिमेषोऽग्न्यर्कसोमदृक्}

\twolineshloka
{गिरीन्द्रैकरदो धर्माधर्मोष्ठः सामबृंहितः}
{ग्रहर्क्षदशनो वाणीजिह्वो वासवनासिकः}

\twolineshloka
{भ्रूमध्यसंस्थितकरो ब्रह्मविद्यामदोदकः}
{कुलाचलांसः सोमार्कघण्टो रुद्रशिरोधरः}

\twolineshloka
{नदीनदभुजः सर्पाङ्गुलीकस्तारकानखः}
{व्योमनाभिः श्रीहृदयो मेरुपृष्ठोऽर्णवोदरः}

\twolineshloka
{कुक्षिस्थयक्षगन्धर्वरक्षःकिन्नरमानुषः}
{पृथ्वीकटिः सृष्टिलिङ्गः शैलोरुर्दस्रजानुकः}

\twolineshloka
{पातालजङ्घो मुनिपात्कालाङ्गुष्ठस्त्रयीतनुः}
{ज्योतिर्मण्डललाङ्गूलो हृदयालाननिश्चलः}

\twolineshloka
{हृत्पद्मकर्णिकाशाली वियत्केलिसरोवरः}
{सद्भक्तध्याननिगडः पूजावारिनिवारितः}

\twolineshloka
{प्रतापी काश्यपो मन्ता गणको विष्टपी बली}
{यशस्वी धार्मिको जेता प्रथमः प्रमथेश्वरः}

\twolineshloka
{चिन्तामणिर्द्वीपपतिः कल्पद्रुमवनालयः}
{रत्नमण्डपमध्यस्थो रत्नसिंहासनाश्रयः}

\twolineshloka
{तीव्राशिरोद्धृतपदो ज्वालिनीमौलिलालितः}
{नन्दानन्दितपीठश्रीर्भोगदो भूषितासनः}

\twolineshloka
{सकामदायिनीपीठः स्फुरदुग्रासनाश्रयः}
{तेजोवतीशिरोरत्नं सत्यानित्यावतंसितः}

\twolineshloka
{सविघ्ननाशिनीपीठः सर्वशक्त्यम्बुजालयः}
{लिपिपद्मासनाधारो वह्निधामत्रयालयः}

\twolineshloka
{उन्नतप्रपदो गूढगुल्फः संवृतपार्ष्णिकः}
{पीनजङ्घः श्लिष्टजानुः स्थूलोरुः प्रोन्नमत्कटिः}

\twolineshloka
{निम्ननाभिः स्थूलकुक्षिः पीनवक्षा बृहद्भुजः}
{पीनस्कन्धः कम्बुकण्ठो लम्बोष्ठो लम्बनासिकः}

\twolineshloka
{भग्नवामरदस्तुङ्गसव्यदन्तो महाहनुः}
{ह्रस्वनेत्रत्रयः शूर्पकर्णो निबिडमस्तकः}

\twolineshloka
{स्तबकाकारकुम्भाग्रो रत्नमौलिर्निरङ्कुशः}
{सर्पहारकटीसूत्रः सर्पयज्ञोपवीतवान्}

\twolineshloka
{सर्पकोटीरकटकः सर्पग्रैवेयकाङ्गदः}
{सर्पकक्षोदराबन्धः सर्पराजोत्तरच्छदः}

\twolineshloka
{रक्तो रक्ताम्बरधरो रक्तमालाविभूषणः}
{रक्तेक्षणो रक्तकरो रक्तताल्वोष्ठपल्लवः}

\twolineshloka
{श्वेतः श्वेताम्बरधरः श्वेतमालाविभूषणः}
{श्वेतातपत्ररुचिरः श्वेतचामरवीजितः}

\twolineshloka
{सर्वावयवसम्पूर्णः सर्वलक्षणलक्षितः}
{सर्वाभरणशोभाढ्यः सर्वशोभासमन्वितः}

\twolineshloka
{सर्वमङ्गलमाङ्गल्यः सर्वकारणकारणम्}
{सर्वदेववरः शार्ङ्गी बीजपूरी गदाधरः}

% \twolineshloka
% {शुभाङ्गो लोकसारङ्गः सुतन्तुस्तन्तुवर्धनः}
% {किरीटी कुण्डली हारी वनमाली शुभाङ्गदः}

\twolineshloka
{इक्षुचापधरः शूली चक्रपाणिः सरोजभृत्}
{पाशी धृतोत्पलः शालिमञ्जरीभृत्स्वदन्तभृत्}

\twolineshloka
{कल्पवल्लीधरो विश्वाभयदैककरो वशी}
{अक्षमालाधरो ज्ञानमुद्रावान् मुद्गरायुधः}

\twolineshloka
{पूर्णपात्री कम्बुधरो विधृताङ्कुशमूलकः}
{करस्थाम्रफलश्चूतकलिकाभृत्कुठारवान्}

\twolineshloka
{पुष्करस्थस्वर्णघटीपूर्णरत्नाभिवर्षकः}
{भारतीसुन्दरीनाथो विनायकरतिप्रियः}

\twolineshloka
{महालक्ष्मीप्रियतमः सिद्धलक्ष्मीमनोरमः}
{रमारमेशपूर्वाङ्गो दक्षिणोमामहेश्वरः}

\twolineshloka
{महीवराहवामाङ्गो रतिकन्दर्पपश्चिमः}
{आमोदमोदजननः सम्प्रमोदप्रमोदनः}

\twolineshloka
{संवर्धितमहावृद्धिरृद्धिसिद्धिप्रवर्धनः}
{दन्तसौमुख्यसुमुखः कान्तिकन्दलिताश्रयः}

\twolineshloka
{मदनावत्याश्रिताङ्घ्रिः कृतवैमुख्यदुर्मुखः}
{विघ्नसम्पल्लवः पद्मः सर्वोन्नतमदद्रवः}

\twolineshloka
{विघ्नकृन्निम्नचरणो द्राविणीशक्तिसत्कृतः}
{तीव्राप्रसन्ननयनो ज्वालिनीपालितैकदृक्}

\twolineshloka
{मोहिनीमोहनो भोगदायिनीकान्तिमण्डनः}
{कामिनीकान्तवक्त्रश्रीरधिष्ठितवसुन्धरः}

\twolineshloka
{वसुधारामदोन्नादो महाशङ्खनिधिप्रियः}
{नमद्वसुमतीमाली महापद्मनिधिः प्रभुः}

\twolineshloka
{सर्वसद्गुरुसंसेव्यः शोचिष्केशहृदाश्रयः}
{ईशानमूर्धा देवेन्द्रशिखः पवननन्दनः}

\twolineshloka
{प्रत्युग्रनयनो दिव्यो दिव्यास्त्रशतपर्वधृक्}
{ऐरावतादिसर्वाशावारणो वारणप्रियः}

\twolineshloka
{वज्राद्यस्त्रपरीवारो गणचण्डसमाश्रयः}
{जयाजयपरिकरो विजयाविजयावहः}

\twolineshloka
{अजयार्चितपादाब्जो नित्यानन्दवनस्थितः}
{विलासिनीकृतोल्लासः शौण्डी सौन्दर्यमण्डितः}

\twolineshloka
{अनन्तानन्तसुखदः सुमङ्गलसुमङ्गलः}
{ज्ञानाश्रयः क्रियाधार इच्छाशक्तिनिषेवितः}

\twolineshloka
{सुभगासंश्रितपदो ललिताललिताश्रयः}
{कामिनीपालनः कामकामिनीकेलिलालितः}

\twolineshloka
{सरस्वत्याश्रयो गौरीनन्दनः श्रीनिकेतनः}
{गुरुगुप्तपदो वाचासिद्धो वागीश्वरीपतिः}

\twolineshloka
{नलिनीकामुको वामारामो ज्येष्ठामनोरमः}
{रौद्रीमुद्रितपादाब्जो हुम्बीजस्तुङ्गशक्तिकः}

\twolineshloka
{विश्वादिजननत्राणः स्वाहाशक्तिः सकीलकः}
{अमृताब्धिकृतावासो मदघूर्णितलोचनः}

\twolineshloka
{उच्छिष्टोच्छिष्टगणको गणेशो गणनायकः}
{सार्वकालिकसंसिद्धिर्नित्यसेव्यो दिगम्बरः}

\twolineshloka
{अनपायोऽनन्तदृष्टिरप्रमेयोऽजरामरः}
{अनाविलोऽप्रतिहतिरच्युतोऽमृतमक्षरः}

\twolineshloka
{अप्रतर्क्योऽक्षयोऽजय्योऽनाधारोऽनामयोऽमलः}
{अमेयसिद्धिरद्वैतमघोरोऽग्निसमाननः}

\twolineshloka
{अनाकारोऽब्धिभूम्यग्निबलघ्नोऽव्यक्तलक्षणः}
{आधारपीठमाधार आधाराधेयवर्जितः}

\twolineshloka
{आखुकेतन आशापूरक आखुमहारथः}
{इक्षुसागरमध्यस्थ इक्षुभक्षणलालसः}

\twolineshloka
{इक्षुचापातिरेकश्रीरिक्षुचापनिषेवितः}
{इन्द्रगोपसमानश्रीरिन्द्रनीलसमद्युतिः}

\twolineshloka
{इन्दीवरदलश्याम इन्दुमण्डलमण्डितः}
{इध्मप्रिय इडाभाग इडावानिन्दिराप्रियः}

\twolineshloka
{इक्ष्वाकुविघ्नविध्वंसी इतिकर्तव्यतेप्सितः}
{ईशानमौलिरीशान ईशानप्रिय ईतिहा}

\twolineshloka
{ईषणात्रयकल्पान्त ईहामात्रविवर्जितः}
{उपेन्द्र उडुभृन्मौलिरुडुनाथकरप्रियः}

\twolineshloka
{उन्नतानन उत्तुङ्ग उदारस्त्रिदशाग्रणीः}
{ऊर्जस्वानूष्मलमद ऊहापोहदुरासदः}

\twolineshloka
{ऋग्यजुःसामनयन ऋद्धिसिद्धिसमर्पकः}
{ऋजुचित्तैकसुलभो ऋणत्रयविमोचनः}

\twolineshloka
{लुप्तविघ्नः स्वभक्तानां लुप्तशक्तिः सुरद्विषाम्}
{लुप्तश्रीर्विमुखार्चानां लूताविस्फोटनाशनः}

\twolineshloka
{एकारपीठमध्यस्थ एकपादकृतासनः}
{एजिताखिलदैत्यश्रीरेधिताखिलसंश्रयः}

\twolineshloka
{ऐश्वर्यनिधिरैश्वर्यमैहिकामुष्मिकप्रदः}
{ऐरम्मदसमोन्मेष ऐरावतसमाननः}

\twolineshloka
{ओङ्कारवाच्य ओङ्कार ओजस्वानोषधीपतिः}
{औदार्यनिधिरौद्धत्यधैर्य औन्नत्यनिःसमः}

\twolineshloka
{अङ्कुशः सुरनागानामङ्कुशाकारसंस्थितः}
{अः समस्तविसर्गान्तपदेषु परिकीर्तितः}

\twolineshloka
{कमण्डलुधरः कल्पः कपर्दी कलभाननः}
{कर्मसाक्षी कर्मकर्ता कर्माकर्मफलप्रदः}

\twolineshloka
{कदम्बगोलकाकारः कूष्माण्डगणनायकः}
{कारुण्यदेहः कपिलः कथकः कटिसूत्रभृत्}

\twolineshloka
{खर्वः खड्गप्रियः खड्गः खान्तान्तःस्थः खनिर्मलः}
{खल्वाटशृङ्गनिलयः खट्वाङ्गी खन्दुरासदः}

\twolineshloka
{गुणाढ्यो गहनो गद्यो गद्यपद्यसुधार्णवः}
{गद्यगानप्रियो गर्जो गीतगीर्वाणपूर्वजः}

\twolineshloka
{गुह्याचाररतो गुह्यो गुह्यागमनिरूपितः}
{गुहाशयो गुडाब्धिस्थो गुरुगम्यो गुरुर्गुरुः}

\twolineshloka
{घण्टाघर्घरिकामाली घटकुम्भो घटोदरः}
{ङकारवाच्यो ङाकारो ङकाराकारशुण्डभृत्}

\twolineshloka
{चण्डश्चण्डेश्वरश्चण्डी चण्डेशश्चण्डविक्रमः}
{चराचरपिता चिन्तामणिश्चर्वणलालसः}

\twolineshloka
{छन्दश्छन्दोद्भवश्छन्दो दुर्लक्ष्यश्छन्दविग्रहः}
{जगद्योनिर्जगत्साक्षी जगदीशो जगन्मयः}

\twolineshloka
{जप्यो जपपरो जाप्यो जिह्वासिंहासनप्रभुः}
{स्रवद्गण्डोल्लसद्धानझङ्कारिभ्रमराकुलः}

\twolineshloka
{टङ्कारस्फारसंरावष्टङ्कारमणिनूपुरः}
{ठद्वयीपल्लवान्तस्थसर्वमन्त्रेषु सिद्धिदः}

\twolineshloka
{डिण्डिमुण्डो डाकिनीशो डामरो डिण्डिमप्रियः}
{ढक्कानिनादमुदितो ढौङ्को ढुण्ढिविनायकः}

\twolineshloka
{तत्त्वानां प्रकृतिस्तत्त्वं तत्त्वम्पदनिरूपितः}
{तारकान्तरसंस्थानस्तारकस्तारकान्तकः}

\twolineshloka
{स्थाणुः स्थाणुप्रियः स्थाता स्थावरं जङ्गमं जगत्}
{दक्षयज्ञप्रमथनो दाता दानं दमो दया}

\twolineshloka
{दयावान्दिव्यविभवो दण्डभृद्दण्डनायकः}
{दन्तप्रभिन्नाभ्रमालो दैत्यवारणदारणः}

\twolineshloka
{दंष्ट्रालग्नद्वीपघटो देवार्थनृगजाकृतिः}
{धनं धनपतेर्बन्धुर्धनदो धरणीधरः}

\twolineshloka
{ध्यानैकप्रकटो ध्येयो ध्यानं ध्यानपरायणः}
{ध्वनिप्रकृतिचीत्कारो ब्रह्माण्डावलिमेखलः}

\twolineshloka
{नन्द्यो नन्दिप्रियो नादो नादमध्यप्रतिष्ठितः}
{निष्कलो निर्मलो नित्यो नित्यानित्यो निरामयः}

\threelineshloka
{परं व्योम परं धाम परमात्मा परं पदम्}
{परात्परः पशुपतिः पशुपाशविमोचनः}
{पूर्णानन्दः परानन्दः पुराणपुरुषोत्तमः}

\twolineshloka
{पद्मप्रसन्नवदनः प्रणताज्ञाननाशनः}
{प्रमाणप्रत्ययातीतः प्रणतार्तिनिवारणः}

\threelineshloka
{फणिहस्तः फणिपतिः फूत्कारः फणितप्रियः}
{बाणार्चिताङ्घ्रियुगलो बालकेलिकुतूहली}
{ब्रह्म ब्रह्मार्चितपदो ब्रह्मचारी बृहस्पतिः}

\twolineshloka
{बृहत्तमो ब्रह्मपरो ब्रह्मण्यो ब्रह्मवित्प्रियः}
{बृहन्नादाग्र्यचीत्कारो ब्रह्माण्डावलिमेखलः}

\twolineshloka
{भ्रूक्षेपदत्तलक्ष्मीको भर्गो भद्रो भयापहः}
{भगवान् भक्तिसुलभो भूतिदो भूतिभूषणः}

\twolineshloka
{भव्यो भूतालयो भोगदाता भ्रूमध्यगोचरः}
{मन्त्रो मन्त्रपतिर्मन्त्री मदमत्तो मनोरमः}

\twolineshloka
{मेखलाहीश्वरो मन्दगतिर्मन्दनिभेक्षणः}
{महाबलो महावीर्यो महाप्राणो महामनाः}

\twolineshloka
{यज्ञो यज्ञपतिर्यज्ञगोप्ता यज्ञफलप्रदः}
{यशस्करो योगगम्यो याज्ञिको याजकप्रियः}

\twolineshloka
{रसो रसप्रियो रस्यो रञ्जको रावणार्चितः}
{राज्यरक्षाकरो रत्नगर्भो राज्यसुखप्रदः}

\twolineshloka
{लक्षो लक्षपतिर्लक्ष्यो लयस्थो लड्डुकप्रियः}
{लासप्रियो लास्यपरो लाभकृल्लोकविश्रुतः}

\twolineshloka
{वरेण्यो वह्निवदनो वन्द्यो वेदान्तगोचरः}
{विकर्ता विश्वतश्चक्षुर्विधाता विश्वतोमुखः}

\twolineshloka
{वामदेवो विश्वनेता वज्रिवज्रनिवारणः}
{विवस्वद्बन्धनो विश्वाधारो विश्वेश्वरो विभुः}

\twolineshloka
{शब्दब्रह्म शमप्राप्यः शम्भुशक्तिगणेश्वरः}
{शास्ता शिखाग्रनिलयः शरण्यः शम्बरेश्वरः}

\twolineshloka
{षडृतुकुसुमस्रग्वी षडाधारः षडक्षरः}
{संसारवैद्यः सर्वज्ञः सर्वभेषजभेषजम्}

\twolineshloka
{सृष्टिस्थितिलयक्रीडः सुरकुञ्जरभेदकः}
{सिन्दूरितमहाकुम्भः सदसद्भक्तिदायकः}

\twolineshloka
{साक्षी समुद्रमथनः स्वयंवेद्यः स्वदक्षिणः}
{स्वतन्त्रः सत्यसङ्कल्पः सामगानरतः सुखी}

\twolineshloka
{हंसो हस्तिपिशाचीशो हवनं हव्यकव्यभुक्}
{हव्यं हुतप्रियो हृष्टो हृल्लेखामन्त्रमध्यगः}

\twolineshloka
{क्षेत्राधिपः क्षमाभर्ता क्षमाक्षमपरायणः}
{क्षिप्रक्षेमकरः क्षेमानन्दः क्षोणीसुरद्रुमः}

\twolineshloka
{धर्मप्रदोऽर्थदः कामदाता सौभाग्यवर्धनः}
{विद्याप्रदो विभवदो भुक्तिमुक्तिफलप्रदः}

\twolineshloka
{आभिरूप्यकरो वीरश्रीप्रदो विजयप्रदः}
{सर्ववश्यकरो गर्भदोषहा पुत्रपौत्रदः}

\twolineshloka
{मेधादः कीर्तिदः शोकहारी दौर्भाग्यनाशनः}
{प्रतिवादिमुखस्तम्भो रुष्टचित्तप्रसादनः}

\twolineshloka
{पराभिचारशमनो दुःखहा बन्धमोक्षदः}
{लवस्त्रुटिः कला काष्ठा निमेषस्तत्परक्षणः}

\twolineshloka
{घटी मुहूर्तः प्रहरो दिवा नक्तमहर्निशम्}
{पक्षो मासर्त्वयनाब्दयुगं कल्पो महालयः}

\twolineshloka
{राशिस्तारा तिथिर्योगो वारः करणमंशकम्}
{लग्नं होरा कालचक्रं मेरुः सप्तर्षयो ध्रुवः}

\twolineshloka
{राहुर्मन्दः कविर्जीवो बुधो भौमः शशी रविः}
{कालः सृष्टिः स्थितिर्विश्वं स्थावरं जङ्गमं जगत्}

\twolineshloka
{भूरापोऽग्निर्मरुद्व्योमाहङ्कृतिः प्रकृतिः पुमान्}
{ब्रह्मा विष्णुः शिवो रुद्र ईशः शक्तिः सदाशिवः}

\twolineshloka
{त्रिदशाः पितरः सिद्धा यक्षा रक्षांसि किन्नराः}
{सिद्धविद्याधरा भूता मनुष्याः पशवः खगाः}

\twolineshloka
{समुद्राः सरितः शैला भूतं भव्यं भवोद्भवः}
{साङ्ख्यं पातञ्जलं योगं पुराणानि श्रुतिः स्मृतिः}

\twolineshloka
{वेदाङ्गानि सदाचारो मीमांसा न्यायविस्तरः}
{आयुर्वेदो धनुर्वेदो गान्धर्वं काव्यनाटकम्}

\twolineshloka
{वैखानसं भागवतं मानुषं पाञ्चरात्रकम्}
{शैवं पाशुपतं कालामुखं भैरवशासनम्}

\twolineshloka
{शाक्तं वैनायकं सौरं जैनमार्हतसंहिता}
{सदसद्व्यक्तमव्यक्तं सचेतनमचेतनम्}

\twolineshloka
{बन्धो मोक्षः सुखं भोगो योगः सत्यमणुर्महान्}
{स्वस्ति हुम्फट् स्वधा स्वाहा श्रौषट् वौषट् वषण्णमः}

\twolineshloka
{ज्ञानं विज्ञानमानन्दो बोधः संवित्समोऽसमः}
{एक एकाक्षराधार एकाक्षरपरायणः}

\twolineshloka
{एकाग्रधीरेकवीर एकोऽनेकस्वरूपधृक्}
{द्विरूपो द्विभुजो द्व्यक्षो द्विरदो द्वीपरक्षकः}

\twolineshloka
{द्वैमातुरो द्विवदनो द्वन्द्वहीनो द्वयातिगः}
{त्रिधामा त्रिकरस्त्रेता त्रिवर्गफलदायकः}

\twolineshloka
{त्रिगुणात्मा त्रिलोकादिस्त्रिशक्तीशस्त्रिलोचनः}
{चतुर्विधवचोवृत्तिपरिवृत्तिप्रवर्तकः}

\twolineshloka
{चतुर्बाहुश्चतुर्दन्तश्चतुरात्मा चतुर्भुजः}
{चतुर्विधोपायमयश्चतुर्वर्णाश्रमाश्रयः}

\twolineshloka
{चतुर्थीपूजनप्रीतश्चतुर्थीतिथिसम्भवः}
{पञ्चाक्षरात्मा पञ्चात्मा पञ्चास्यः पञ्चकृत्तमः}

\twolineshloka
{पञ्चाधारः पञ्चवर्णः पञ्चाक्षरपरायणः}
{पञ्चतालः पञ्चकरः पञ्चप्रणवमातृकः}

\twolineshloka
{पञ्चब्रह्ममयस्फूर्तिः पञ्चावरणवारितः}
{पञ्चभक्षप्रियः पञ्चबाणः पञ्चशिवात्मकः}

\twolineshloka
{षट्कोणपीठः षट्चक्रधामा षड्ग्रन्थिभेदकः}
{षडङ्गध्वान्तविध्वंसी षडङ्गुलमहाह्रदः}

\twolineshloka
{षण्मुखः षण्मुखभ्राता षट्शक्तिपरिवारितः}
{षड्वैरिवर्गविध्वंसी षडूर्मिभयभञ्जनः}

\twolineshloka
{षट्तर्कदूरः षट्कर्मा षड्गुणः षड्रसाश्रयः}
{सप्तपातालचरणः सप्तद्वीपोरुमण्डलः}

\twolineshloka
{सप्तस्वर्लोकमुकुटः सप्तसप्तिवरप्रदः}
{सप्ताङ्गराज्यसुखदः सप्तर्षिगणवन्दितः}

\twolineshloka
{सप्तच्छन्दोनिधिः सप्तहोत्रः सप्तस्वराश्रयः}
{सप्ताब्धिकेलिकासारः सप्तमातृनिषेवितः}

\twolineshloka
{सप्तच्छन्दो मोदमदः सप्तच्छन्दो मखप्रभुः}
{अष्टमूर्तिर्ध्येयमूर्तिरष्टप्रकृतिकारणम्}

\twolineshloka
{अष्टाङ्गयोगफलभृदष्टपत्राम्बुजासनः}
{अष्टशक्तिसमानश्रीरष्टैश्वर्यप्रवर्धनः}

\twolineshloka
{अष्टपीठोपपीठश्रीरष्टमातृसमावृतः}
{अष्टभैरवसेव्योऽष्टवसुवन्द्योऽष्टमूर्तिभृत्}

\threelineshloka
{अष्टचक्रस्फुरन्मूर्तिरष्टद्रव्यहविःप्रियः}
{अष्टश्रीरष्टसामश्रीरष्टैश्वर्यप्रदायकः}
{नवनागासनाध्यासी नवनिध्यनुशासितः}

\twolineshloka
{नवद्वारपुरावृत्तो नवद्वारनिकेतनः}
{नवनाथमहानाथो नवनागविभूषितः}

\twolineshloka
{नवनारायणस्तुल्यो नवदुर्गानिषेवितः}
{नवरत्नविचित्राङ्गो नवशक्तिशिरोद्धृतः}

\twolineshloka
{दशात्मको दशभुजो दशदिक्पतिवन्दितः}
{दशाध्यायो दशप्राणो दशेन्द्रियनियामकः}

\twolineshloka
{दशाक्षरमहामन्त्रो दशाशाव्यापिविग्रहः}
{एकादशमहारुद्रैःस्तुतश्चैकादशाक्षरः}

\twolineshloka
{द्वादशद्विदशाष्टादिदोर्दण्डास्त्रनिकेतनः}
{त्रयोदशभिदाभिन्नो विश्वेदेवाधिदैवतम्}

\twolineshloka
{चतुर्दशेन्द्रवरदश्चतुर्दशमनुप्रभुः}
{चतुर्दशाद्यविद्याढ्यश्चतुर्दशजगत्पतिः}

\twolineshloka
{सामपञ्चदशः पञ्चदशीशीतांशुनिर्मलः}
{तिथिपञ्चदशाकारस्तिथ्या पञ्चदशार्चितः}

\twolineshloka
{षोडशाधारनिलयः षोडशस्वरमातृकः}
{षोडशान्तपदावासः षोडशेन्दुकलात्मकः}

\twolineshloka
{कलासप्तदशी सप्तदशसप्तदशाक्षरः}
{अष्टादशद्वीपपतिरष्टादशपुराणकृत्}

\twolineshloka
{अष्टादशौषधीसृष्टिरष्टादशविधिः स्मृतः}
{अष्टादशलिपिव्यष्टिसमष्टिज्ञानकोविदः}

\twolineshloka
{अष्टादशान्नसम्पत्तिरष्टादशविजातिकृत्}
{एकविंशः पुमानेकविंशत्यङ्गुलिपल्लवः}

\twolineshloka
{चतुर्विंशतितत्त्वात्मा पञ्चविंशाख्यपूरुषः}
{सप्तविंशतितारेशः सप्तविंशतियोगकृत्}

\twolineshloka
{द्वात्रिंशद्भैरवाधीशश्चतुस्त्रिंशन्महाह्रदः}
{षट्त्रिंशत्तत्त्वसम्भूतिरष्टत्रिंशत्कलात्मकः}

\threelineshloka
{पञ्चाशद्विष्णुशक्तीशः पञ्चाशन्मातृकालयः}
{द्विपञ्चाशद्वपुःश्रेणी त्रिषष्ट्यक्षरसंश्रयः}
{पञ्चाशदक्षरश्रेणीपञ्चाशद्रुद्रविग्रहः}

\twolineshloka
{चतुःषष्टिमहासिद्धियोगिनीवृन्दवन्दितः}
{नमदेकोनपञ्चाशन्मरुद्वर्गनिरर्गलः}

\twolineshloka
{चतुःषष्ट्यर्थनिर्णेता चतुःषष्टिकलानिधिः}
{अष्टषष्टिमहातीर्थक्षेत्रभैरववन्दितः}

\twolineshloka
{चतुर्नवतिमन्त्रात्मा षण्णवत्यधिकप्रभुः}
{शतानन्दः शतधृतिः शतपत्रायतेक्षणः}

\twolineshloka
{शतानीकः शतमखः शतधारावरायुधः}
{सहस्रपत्रनिलयः सहस्रफणिभूषणः}

\twolineshloka
{सहस्रशीर्षा पुरुषः सहस्राक्षः सहस्रपात्}
{सहस्रनामसंस्तुत्यः सहस्राक्षबलापहः}

\twolineshloka
{दशसाहस्रफणिभृत्फणिराजकृतासनः}
{अष्टाशीतिसहस्राद्यमहर्षिस्तोत्रपाठितः}

\twolineshloka
{लक्षाधारः प्रियाधारो लक्षाधारमनोमयः}
{चतुर्लक्षजपप्रीतश्चतुर्लक्षप्रकाशकः}

\twolineshloka
{चतुरशीतिलक्षाणां जीवानां देहसंस्थितः}
{कोटिसूर्यप्रतीकाशः कोटिचन्द्रांशुनिर्मलः}

\twolineshloka
{शिवोद्भवाद्यष्टकोटिवैनायकधुरन्धरः}
{सप्तकोटिमहामन्त्रमन्त्रितावयवद्युतिः}

\twolineshloka
{त्रयस्त्रिंशत्कोटिसुरश्रेणीप्रणतपादुकः}
{अनन्तदेवतासेव्यो ह्यनन्तशुभदायकः}

\twolineshloka
{अनन्तनामानन्तश्रीरनन्तोऽनन्तसौख्यदः}
{अनन्तशक्तिसहितो ह्यनन्तमुनिसंस्तुतः}

\dnsub{फलश्रुतिः}

\twolineshloka
{इति वैनायकं नाम्नां सहस्रमिदमीरितम्}
{इदं ब्राह्मे मुहूर्ते यः पठति प्रत्यहं नरः}

\twolineshloka
{करस्थं तस्य सकलमैहिकामुष्मिकं सुखम्}
{आयुरारोग्यमैश्वर्यं धैर्यं शौर्यं बलं यशः}

\twolineshloka
{मेधा प्रज्ञा धृतिः कान्तिः सौभाग्यमभिरूपता}
{सत्यं दया क्षमा शान्तिर्दाक्षिण्यं धर्मशीलता}

\twolineshloka
{जगत्संवननं विश्वसंवादो वेदपाटवम्}
{सभापाण्डित्यमौदार्यं गाम्भीर्यं ब्रह्मवर्चसम्}

\twolineshloka
{ओजस्तेजः कुलं शीलं प्रतापो वीर्यमार्यता}
{ज्ञानं विज्ञानमास्तिक्यं स्थैर्यं विश्वासता तथा}

\twolineshloka
{धनधान्यादिवृद्धिश्च सकृदस्य जपाद्भवेत्}
{वश्यं चतुर्विधं विश्वं जपादस्य प्रजायते}

\twolineshloka
{राज्ञो राजकलत्रस्य राजपुत्रस्य मन्त्रिणः}
{जप्यते यस्य वश्यार्थे स दासस्तस्य जायते}

\twolineshloka
{धर्मार्थकाममोक्षाणामनायासेन साधनम्}
{शाकिनीडाकिनीरक्षोयक्षग्रहभयापहम्}

\twolineshloka
{साम्राज्यसुखदं सर्वसपत्नमदमर्दनम्}
{समस्तकलहध्वंसि दग्धबीजप्ररोहणम्}

\twolineshloka
{दुःस्वप्नशमनं क्रुद्धस्वामिचित्तप्रसादनम्}
{षड्वर्गाष्टमहासिद्धित्रिकालज्ञानकारणम्}

\twolineshloka
{परकृत्यप्रशमनं परचक्रप्रमर्दनम्}
{संग्राममार्गे सर्वेषामिदमेकं जयावहम्}

\twolineshloka
{सर्ववन्ध्यत्वदोषघ्नं गर्भरक्षैककारणम्}
{पठ्यते प्रत्यहं यत्र स्तोत्रं गणपतेरिदम्}

\twolineshloka
{देशे तत्र न दुर्भिक्षमीतयो दुरितानि च}
{न तद्गेहं जहाति श्रीर्यत्रायं जप्यते स्तवः}

\twolineshloka
{क्षयकुष्ठप्रमेहार्शभगन्दरविषूचिकाः}
{गुल्मं प्लीहानमाध्मानमतिसारं महोदरम्}

\twolineshloka
{कासं श्वासमुदावर्तं शूलं शोफामयोदरम्}
{शिरोरोगं वमिं हिक्कां गण्डमालामरोचकम्}

\twolineshloka
{वातपित्तकफद्वन्द्वत्रिदोषजनितज्वरम्}
{आगन्तुविषमं शीतमुष्णं चैकाहिकादिकम्}

\twolineshloka
{इत्याद्युक्तमनुक्तं वा रोगदोषादिसम्भवम्}
{सर्वं प्रशमयत्याशु स्तोत्रस्यास्य सकृज्जपः}

\twolineshloka
{प्राप्यतेऽस्य जपात्सिद्धिः स्त्रीशूद्रैः पतितैरपि}
{सहस्रनाममन्त्रोऽयं जपितव्यः शुभाप्तये}

\twolineshloka
{महागणपतेः स्तोत्रं सकामः प्रजपन्निदम्}
{इच्छया सकलान् भोगानुपभुज्येह पार्थिवान्}

\twolineshloka
{मनोरथफलैर्दिव्यैर्व्योमयानैर्मनोरमैः}
{चन्द्रेन्द्रभास्करोपेन्द्रब्रह्मशर्वादिसद्मसु}

\twolineshloka
{कामरूपः कामगतिः कामदः कामदेश्वरः}
{भुक्त्वा यथेप्सितान्भोगानभीष्टैः सह बन्धुभिः}

\twolineshloka
{गणेशानुचरो भूत्वा गणो गणपतिप्रियः}
{नन्दीश्वरादिसानन्दैर्नन्दितः सकलैर्गणैः}

\twolineshloka
{शिवाभ्यां कृपया पुत्रनिर्विशेषं च लालितः}
{शिवभक्तः पूर्णकामो गणेश्वरवरात्पुनः}

\twolineshloka
{जातिस्मरो धर्मपरः सार्वभौमोऽभिजायते}
{निष्कामस्तु जपन्नित्यं भक्त्या विघ्नेशतत्परः}

\twolineshloka
{योगसिद्धिं परां प्राप्य ज्ञानवैराग्यसंयुतः}
{निरन्तरे निराबाधे परमानन्दसंज्ञिते}

\twolineshloka
{विश्वोत्तीर्णे परे पूर्णे पुनरावृत्तिवर्जिते}
{लीनो वैनायके धाम्नि रमते नित्यनिर्वृते}

\twolineshloka
{यो नामभिर्हुतैर्दत्तैः पूजयेदर्चयेन्नरः}
{राजानो वश्यतां यान्ति रिपवो यान्ति दासताम्}

\twolineshloka
{तस्य सिध्यन्ति मन्त्राणां दुर्लभाश्चेष्टसिद्धयः}
{मूलमन्त्रादपि स्तोत्रमिदं प्रियतमं मम}

\twolineshloka
{नभस्ये मासि शुक्लायां चतुर्थ्यां मम जन्मनि}
{दूर्वाभिर्नामभिः पूजां तर्पणं विधिवच्चरेत्}

\twolineshloka
{अष्टद्रव्यैर्विशेषेण कुर्याद्भक्तिसुसंयुतः}
{तस्येप्सितं धनं धान्यमैश्वर्यं विजयो यशः}

\twolineshloka
{भविष्यति न सन्देहः पुत्रपौत्रादिकं सुखम्}
{इदं प्रजपितं स्तोत्रं पठितं श्रावितं श्रुतम्}

\twolineshloka
{व्याकृतं चर्चितं ध्यातं विमृष्टमभिवन्दितम्}
{इहामुत्र च विश्वेषां विश्वैश्वर्यप्रदायकम्}

\twolineshloka
{स्वच्छन्दचारिणाप्येष येन सन्धार्यते स्तवः}
{स रक्ष्यते शिवोद्भूतैर्गणैरध्यष्टकोटिभिः}

\twolineshloka
{लिखितं पुस्तकस्तोत्रं मन्त्रभूतं प्रपूजयेत्}
{तत्र सर्वोत्तमा लक्ष्मीः सन्निधत्ते निरन्तरम्}

\fourlineindentedshloka
{दानैरशेषैरखिलैर्व्रतैश्च}
{तीर्थैरशेषैरखिलैर्मखैश्च}
{न तत्फलं विन्दति यद्गणेश-}
{सहस्रनामस्मरणेन सद्यः}

\fourlineindentedshloka
{एतन्नाम्नां सहस्रं पठति दिनमणौ प्रत्यहं प्रोज्जिहाने}
{सायं मध्यन्दिने वा त्रिषवणमथवा सन्ततं वा जनो यः}
{स स्यादैश्वर्यधुर्यः प्रभवति वचसां कीर्तिमुच्चैस्तनोति}
{दारिद्र्यं हन्ति विश्वं वशयति सुचिरं वर्धते पुत्रपौत्रैः}

\twolineshloka
{अकिञ्चनोऽप्येकचित्तो नियतो नियतासनः}
{प्रजपंश्चतुरो मासान् गणेशार्चनतत्परः}

\twolineshloka
{दरिद्रतां समुन्मूल्य सप्तजन्मानुगामपि}
{लभते महतीं लक्ष्मीमित्याज्ञा पारमेश्वरी}

\fourlineindentedshloka
{आयुष्यं वीतरोगं कुलमतिविमलं सम्पदश्चार्तिनाशः}
{कीर्तिर्नित्यावदाता भवति खलु नवा कान्तिरव्याजभव्या}
{पुत्राः सन्तः कलत्रं गुणवदभिमतं यद्यदन्यच्च तत्तत्}
{नित्यं यः स्तोत्रमेतत् पठति गणपतेस्तस्य हस्ते समस्तम्}

\twolineshloka
{गणञ्जयो गणपतिर्हेरम्बो धरणीधरः}
{महागणपतिर्बुद्धिप्रियः क्षिप्रप्रसादनः}

\twolineshloka
{अमोघसिद्धिरमृतमन्त्रश्चिन्तामणिर्निधिः}
{सुमङ्गलो बीजमाशापूरको वरदः कलः}

\twolineshloka
{काश्यपो नन्दनो वाचासिद्धो ढुण्ढिर्विनायकः}
{मोदकैरेभिरत्रैकविंशत्या नामभिः पुमान्}

\twolineshloka
{उपायनं ददेद्भक्त्या मत्प्रसादं चिकीर्षति}
{वत्सरं विघ्नराजोऽस्य तथ्यमिष्टार्थसिद्धये}

\twolineshloka
{यः स्तौति मद्गतमना ममाराधनतत्परः}
{स्तुतो नाम्ना सहस्रेण तेनाहं नात्र संशयः}

\fourlineindentedshloka
{नमो नमः सुरवरपूजिताङ्घ्रये}
{नमो नमो निरुपममङ्गलात्मने}
{नमो नमो विपुलदयैकसिद्धये}
{नमो नमः करिकलभाननाय ते}

\fourlineindentedshloka
{किङ्किणीगणरचितचरणः}
{प्रकटितगुरुमितचारुकरणः}
{मदजललहरीकलितकपोलः}
{शमयतु दुरितं गणपतिनाम्ना}

{॥इति श्रीगणेशपुराणे उपासनाखण्डे ईश्वरगणेशसंवादे गणेशसहस्रनामस्तोत्रं नाम षट्चत्वारिंशोऽध्यायः॥}
