% !TeX program = XeLaTeX
% !TeX root = ../../shloka.tex
\sect{ध्यान-स्तोत्राणि}
\dnsub{गणेश-ध्यानम्}
\fourlineindentedshloka*
{ओंकार-सन्निभमिभाननमिन्दुभालम्}{मुक्ताग्रबिन्दुममलद्युतिमेकदन्तम्}
{लम्बोदरं कलचतुर्भुजमादिदेवम्}{ध्यायेन्महागणपतिं मतिसिद्धिकान्तम्}


\dnsub{विष्णु-ध्यानम्}
\fourlineindentedshloka*
{शान्ताकारं भुजगशयनं पद्मनाभं सुरेशम्}
{विश्वाधारं गगनसदृशं मेघवर्णं शुभाङ्गम्}
{लक्ष्मीकान्तं कमलनयनं योगिहृद्‌ध्यानगम्यम्}
{वन्दे विष्णुं भवभयहरं सर्वलोकैकनाथम्}


\dnsub{लक्ष्मी-ध्यानम्}
\fourlineindentedshloka*
{लक्ष्मीं क्षीरसमुद्रराजतनयां श्रीरङ्गधामेश्वरीम्}
{दासीभूतसमस्तदेववनितां लोकैकदीपाङ्कुराम्}
{श्रीमन्मन्दकटाक्षलब्धविभवब्रह्मेन्द्रगङ्गाधराम्}
{त्वां त्रैलोक्यकुटुम्बिनीं सरसिजां वन्दे मुकुन्दप्रियाम्}

\dnsub{राम-ध्यानम्}
\fourlineindentedshloka*
{वैदेहीसहितं सुरद्रुमतले हैमे महामण्डपे}
{मध्ये पुष्पकमासने मणिमये वीरासने सुस्थितम्}
{अग्रे वाचयति प्रभञ्जनसुते तत्त्वं मुनिभ्यः परम्}
{व्याख्यान्तं भरतादिभिः परिवृतं रामं भजे श्यामलम्}

\fourlineindentedshloka*
{वामे भूमिसुता पुरश्च हनुमान् पश्चात् सुमित्रासुतः}
{शत्रुघ्नो भरतश्च पार्श्वदलयोर्वाय्वादिकोणेषु च}
{सुग्रीवश्च विभीषणश्च युवराट् तारासुतो जाम्बवान्}
{मध्ये नीलसरोजकोमलरुचिं रामं भजे श्यामलम्}

\dnsub{सीता-ध्यानम्}
\fourlineindentedshloka*
{वामाङ्गे रघुनायकस्य रुचिरे या संस्थिता शोभना}
{या विप्राधिपयानरम्यनयना या विप्रपालानना}
{विद्युत्पुञ्जविराजमानवसना भक्तार्तिसङ्खण्डना}
{श्रीमद्राघवपादपद्मयुगलन्यस्तेक्षणा साऽवतु}

\dnsub{हनुमत्-ध्यानम्}
\fourlineindentedshloka*
{उष्ट्रारूढ-सुवर्चलासहचरन् सुग्रीवमित्राञ्जना-}
{सूनो वायुकुमार केसरितनूजाऽक्षादिदैत्यान्तक}
{सीतशोकहराग्निनन्दन सुमित्रासम्भवप्राणद}
{श्रीभीमाग्रज शम्भुपुत्र हनुमान् सूर्यास्य तुभ्यं नमः}

\fourlineindentedshloka*
{खड्गं खेटक-भिण्डिपाल-परशुं पाश-त्रिशूल-द्रुमान्}
{चक्रं शङ्ख-गदा-फलाङ्कुश-सुधाकुम्भान् हलं पर्वतम्}
{टङ्कं पर्वतकार्मुकाहिडमरूनेतानि दिव्यायुधान्}
{एवं विंशतिबाहुभिश्च दधतं ध्यायेद्धनूमत्प्रभुम्}

\dnsub{सदाशिव-ध्यानम्}
\fourlineindentedshloka*
{वन्दे शम्भुमुमापतिं सुरगुरुं वन्दे जगत्कारणम्}
{वन्दे पन्नगभूषणं मृगधरं वन्दे पशूनां पतिम्}
{वन्दे सूर्यशशाङ्कवह्निनयनं वन्दे मुकुन्दप्रियम्}
{वन्दे भक्तजनाश्रयं च वरदं वन्दे शिवं शङ्करम्}

\dnsub{सुब्रह्मण्य-ध्यानम्}
\fourlineindentedshloka*
{सिन्दूरारुणमिन्दुकान्तिवदनं केयूरहारादिभिर्-}
{दिव्यैराभरणैर्विभूषिततनुं स्वर्गादिसौख्यप्रदम्}
{अम्भोजाभय-शक्ति-कुक्कुटधरं रक्ताङ्गरागोज्ज्वलम्}
{सुब्रह्मण्यमुपास्महे प्रणमतां भीतिप्रणाशोद्यतम्}

\fourlineindentedshloka*
{षड्वक्त्रं शिखिवाहनं त्रिनयनं चित्राम्बरालङ्कृतम्}
{वज्रं शक्तिमसिं त्रिशूलमभयं खेटं धनुश्चक्रकम्}
{पाशं कुक्कुटमङ्कुशं च वरदं दोर्भिर्दधानं सदा}
{ध्यायेदीप्सित-सिद्धिदं शिवसुतं स्कन्दं सुराराधितम्}

\dnsub{वल्ली-ध्यानम्}
\fourlineindentedshloka*
{श्यामां पङ्कजधारिणीं मणिलसत् ताटङ्ककर्णोज्ज्वलाम्}
{दक्षे लम्बकरां किरीटमकुटां तुङ्गस्तनोत्कञ्चुकाम्}
{अन्योन्येक्षणसंयुगां शरवणोद्भूतस्य सव्ये स्थिताम्}
{गुञ्जामाल्यधराम् प्रवालवसनां वल्लीश्वरीं भावये}

\dnsub{देवसेना-ध्यानम्}
\fourlineindentedshloka*
{पीतामुत्फलधारिणीं शशिसुतां पीताम्बरालङ्कृताम्}
{वामे लम्बकरां महेन्द्रतनयां मन्दारमालाधराम्}
{दैवार्चितपादपद्मयुगलां स्कन्दस्य वामे स्थिताम्}
{सेनां दिव्यविभूषितां त्रिनयनां देवीं त्रिभङ्गीं भजे}

\dnsub{दण्डायुधपाणि-ध्यानम्}
\fourlineindentedshloka*
{कल्पद्रुमं प्रणमतां कमलारुणाभम्}
{स्कन्दं भुजद्वयमनामयमेकवक्त्रम्}
{कात्यायनीप्रियसुतं कटिबद्धवामम्}
{कौपीनदण्डधरदक्षिणहस्तमीडे}
\closesection
