\sect{श्रीमहाभारत-ध्यानम्}
\label{sec:start}

\twolineshloka*
{शुक्लाम्बरधरं विष्णुं शशिवर्णं चतुर्भुजम्}
{प्रसन्नवदनं ध्यायेत् सर्वविघ्नोपशान्तये}

\twolineshloka*
{वागीशाद्याः सुमनसः सर्वार्थानामुपक्रमे}
{यं नत्वा कृतकृत्याः स्युस्तं नमामि गजाननम्}

\dnsub{श्री-गुरु-प्रार्थना}

\twolineshloka*
{गुरुर्ब्रह्मा गुरुर्विष्णुर्गुरुर्देवो महेश्वरः}
{गुरुः साक्षात् परं ब्रह्म तस्मै श्री-गुरवे नमः}

\twolineshloka*
{सदाशिवसमारम्भां शङ्कराचार्यमध्यमाम्}
{अस्मदाचार्यपर्यन्तां वन्दे गुरुपरम्पराम्}

\twolineshloka*
{अखण्डमण्डलाकारं व्याप्तं येन चराचरम्}
{तत्पदं दर्शितं येन तस्मै श्री-गुरवे नमः}

\dnsub{श्री-सरस्वती-प्रार्थना}
\fourlineindentedshloka*
{दोर्भिर्युक्ता चतुर्भिः स्फटिकमणिनिभैरक्षमालां दधाना}
{हस्तेनैकेन पद्मं सितमपि च शुकं पुस्तकं चापरेण}
{भासा कुन्देन्दुशङ्खस्फटिकमणिनिभा भासमानाऽसमाना}
{सा मे वाग्देवतेयं निवसतु वदने सर्वदा सुप्रसन्ना}


\dnsub{प्रार्थना}

\twolineshloka
{अशुभानि निराचष्टे तनोति शुभसन्ततिम्}
{स्मृतमात्रेण यत् पुंसां ब्रह्म तन्मङ्गलं विदुः}

\twolineshloka
{भारताध्ययनात् पुण्यादपि पादमधीयतः}
{श्रद्दधानस्य पूयन्ते सर्वपापान्यशेषतः}

\twolineshloka
{सरस्वतीपदं वन्दे श्रियः पतिमुमापतिम्}
{त्विषां पतिं गणपतिं बृहस्पतिमुखानृषीन्}

\twolineshloka
{आद्यं पुरुषमीशानं पुरुहूतं  पुरुष्टुतम्}
{ऋतमेकाक्षरं ब्रह्म व्यक्ताव्यक्तं सनातनम्}

\twolineshloka
{असच्च सच्चैव च यद् विश्वं सदसतः परम्}
{परावराणां स्रष्टारं पुराणं परमव्ययम्}

\twolineshloka
{मङ्गल्यं मङ्गलं विष्णुं वरेण्यमनघं शुचिम्}
{नमस्कृत्य हृषीकेशं चराचरगुरुं हरिम्}

\resetShloka

\dnsub{श्री-व्यास-नमस्क्रिया}

\twolineshloka
{महर्षेः सर्वलोकेषु पूजितस्य महात्मनः}
{प्रवक्ष्यामि मतं कृत्स्नं व्यासस्यामिततेजसः}

\twolineshloka
{व्यासं वसिष्ठनप्तारं शक्तेः पौत्रमकल्मषम्}
{पराशरात्मजं वन्दे शुकतातं तपोनिधिम्}

\fourlineindentedshloka
{अभ्रश्यामः पिङ्गजटाबद्धकलापः}
{प्रांशुर्दण्डी कृष्णमृगत्वक्परिधानः}
{साक्षाल्लोकान्पावयमानः कविमुख्यः}
{पाराशर्यः पर्वसु रूपं विवृणोतु}

\fourlineindentedshloka
{पाराशर्यवचः सरोजममलं गीतार्थगन्धोत्कटम्}
{नानाख्यानककेसरं हरिकथासम्बोधनाबोधितम्}
{लोके सज्जनषट्पदैरहरहः पेपीयमानं मुदा}
{भूयाद्भारतपङ्कजं कलिमलप्रध्वंसि नः श्रेयसे}

\twolineshloka
{जयति पराशरसूनुः सत्यवतीहृदयनन्दनो व्यासः}
{यस्यास्यकमलगलितं वाङ्मयममृतं जगत् पिबति}

\dnsub{श्री-महाभारत-प्रार्थना}

\fourlineindentedshloka
{धर्मो विवर्धति युधिष्ठिरकीर्तनेन} 
{पापं प्रणश्यति वृकोदरकीर्तनेन}
{शत्रुर्विनश्यति धनञ्जयकीर्तनेन} 
{माद्रीसुतौ कथयतां न भवन्ति रोगाः}

\annofourlineindentedshloka
{युधिष्ठिरो धर्ममयो महाद्रुमः}
{स्कन्धोऽर्जुनो भीमसेनोऽस्य शाखाः}
{माद्रीसुतौ पुष्पफले समृद्धे} 
{मूलं कृष्णो ब्रह्म च ब्राह्मणाश्च}
{१-१-१२९}

\annotwolineshloka
{नारायणं नमस्कृत्य नरं चैव नरोत्तमम्}
{देवीं सरस्वतीं व्यासं ततो जयमुदीरयेत्}{१-१-१}

\annotwolineshloka
{नमो धर्माय महते नमः कृष्णाय वेधसे}
{ब्राह्मणेभ्यो नमस्कृत्य धर्मान् वक्ष्यामि शाश्वतान्}{१-१-३}


\resetShloka
\centerline{\textbf{ॐ श्री-गुरुभ्यो नमः।}}

\resetShloka
\dnsub{मङ्गलश्लोकाः}

\fourlineindentedshloka
{स्वस्ति प्रजाभ्यः परिपालयन्ताम्}
{न्यायेन मार्गेण महीं महीशाः}
{गोब्राह्मणेभ्यः शुभमस्तु नित्यम्}
{लोकाः समस्ताः सुखिनो भवन्तु}

\twolineshloka
{काले वर्षतु पर्जन्यः पृथिवी सस्यशालिनी}
{देशोऽयं क्षोभरहितो ब्राह्मणाः सन्तु निर्भयाः}

\twolineshloka
{अपुत्राः पुत्रिणः सन्तु पुत्रिणः सन्तु पौत्रिणः}
{अधनाः सधनाः सन्तु जीवन्तु शरदां शतम्}

\fourlineindentedshloka
{पाराशर्यवचः सरोजममलं गीतार्थगन्धोत्कटम्}
{नानाख्यानककेसरं हरिकथासम्बोधनाबोधितम्}
{लोके सज्जनषट्पदैरहरहः पेपीयमानं मुदा}
{भूयाद्भारतपङ्कजं कलिमलप्रध्वंसि नः श्रेयसे}



\annotwolineshloka
{धर्मशास्त्रमिदं पुण्यमर्थशास्त्रमिदं परम्}
{मोक्षशास्त्रमिदं प्रोक्तं व्यासेनामितबुद्धिना}{१-६२-२५}

\twolineshloka
{भारतं सर्वशास्त्राणामुत्तमं भरतर्षभ}
{सम्प्रत्याचक्षते चेदं तथा श्रोष्यन्ति चापरे}

\annotwolineshloka
{भरतानां महज्जन्म शृण्वतामनसूयताम्}
{नास्ति व्याधिभयं तेषां परलोकभयं कुतः}{१-६२-२८}

\annotwolineshloka
{शरीरेण कृतं पापं वाचा च मनसैव च}
{सर्वं सन्त्यजति क्षिप्रं य इदं शृणुयान्नरः}{१-६२-२९}

\annotwolineshloka
{इदं हि वेदैः समितं पवित्रमपि चोत्तमम्}
{श्राव्यं श्रुतिसुखं चैव पावनं शीलवर्धनम्}{१-६२-५२}

% \annotwolineshloka
% {य इदं भारतं राजन् वाचकाय प्रयच्छति}
% {तेन सर्वा मही दत्ता भवेत् सागरमेखला}{१-६२-५३}

\annotwolineshloka
{कीर्तिं प्रथयता लोके पाण्डवानां महात्मनाम्}
{अन्येषां क्षत्रियाणां च भूरिद्रविणतेजसाम्}{१८-५-३८}

\annotwolineshloka
{कार्ष्णं वेदमिमं सर्वं शृणुयाद् यः समाहितः}
{ब्रह्महत्यादिपापानां कोटिस्तस्य विनश्यति}{१८-५-४१}

\annotwolineshloka
{महत्त्वाद् भारवत्त्वाच्च महाभारतमुच्यते}
{निरुक्तमस्य यो वेद सर्वपापैः प्रमुच्यते}{१८-५-४५}

\annotwolineshloka
{अष्टादशपुराणानि धर्मशास्त्राणि सर्वशः}
{वेदाः साङ्कास्तथैकत्र भारतं चैकतः स्थितम्}{१८-५-४६}

\annotwolineshloka
{श्रूयतां सिंहनादोऽयमृषेस्तस्य महात्मनः}
{अष्टादशपुराणानां कर्तुर्वेदमहोदधेः}{१८-५-४७}

\annotwolineshloka
{धर्मे चार्थे च कामे च मोक्षे च भरतर्षभ}
{यदिहास्ति तदन्यत्र यन्नेहास्ति न कुत्रचित्}{१८-५-५०}

\annotwolineshloka
{अनागतश्च मोक्षश्च कृष्णद्वैपायनः प्रभुः}
{सन्दर्भं भारतस्यास्य कृतवान् धर्मकाम्यया}{१८-५-६६}

\annotwolineshloka
{भारतश्रवणे राजन् पारणे च नृपोत्तम}
{सदा यत्नवता भाव्यं श्रेयस्तु परमिच्छता}{१८-६-८८}

\annotwolineshloka
{भारतं शृणुयान्नित्यं भारतं परिकीर्तयेत्}
{भारतं भवने यस्य तस्य हस्तगतो जयः}{१८-६-८९}

\annotwolineshloka
{भारतं परमं पुण्यं भारते विविधाः कथाः}
{भारतं सेव्यते देवैर्भारतं परमं पदम्}{१८-६-९०}

\annotwolineshloka
{भारतं सर्वशास्त्राणामुत्तमं भरतर्षभ}
{भारतात् प्राप्यते मोक्षस्तत्त्वमेतद्ब्रवीमि तत्}{१८-६-९१}

\annotwolineshloka
{महाभारतमाख्यानं क्षितिं गां च सरस्वतीम्}
{ब्राह्मणान् केशवं चैव कीर्तयन् नावसीदति}{१८-६-९२}

\annotwolineshloka
{वेदे रामायणे पुण्ये भारते भरतर्षभ}
{आदौ चान्ते च मध्ये च हरिः सर्वत्र गीयते}{१८-६-९३}

\annotwolineshloka
{यत्र विष्णुकथा दिव्याः श्रुतयश्च सनातनाः}
{तच्छ्रोतव्यं मनुष्येण परं पदमिहेच्छता}{१८-६-९४}

\annotwolineshloka
{एतत् पवित्रं परममेतद् धर्मनिदर्शनम्}
{एतत् सर्वगुणोपेतं श्रोतव्यं भूतिमिच्छता}{१८-६-९५}

\annotwolineshloka
{कायिकं वाचिकं चैव मनसा समुपार्जितम्}
{तत् सर्वं नाशमायाति तमः सूर्योदये यथा}{१८-६-९६}

\annotwolineshloka
{यदह्ना कुरुते पापं ब्राह्मणस्त्विन्द्रियैश्चरन्}
{महाभारतमाख्याय सन्ध्यां मुच्यति पश्चिमाम्}{१-२-३९४}

\annotwolineshloka
{यद्रात्रौ कुरुते पापं कर्मणा मनसा गिरा}
{महाभारतमाख्याय पूर्वां सन्ध्यां प्रमुच्यते}{१-२-३९५}

\annofourlineindentedshloka
{यो गोशतं कनकशृङ्गमयं ददाति}
{विप्राय वेदविदुषे च बहुश्रुताय}
{पुण्यां च भारतकथां शृणुयाच्च नित्यं}
{तुल्यं फलं भवति तस्य च तस्य चैव}{१-२-३९६}


\hyperref[sec:start]{\closesection}