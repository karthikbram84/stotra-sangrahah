% !TeX program = XeLaTeX
% !TeX root = ../../shloka.tex

\sect{आपदुद्धारक-द्वादशमुख-हनुमान् स्तोत्रम्}
ॐ अस्य श्री-आपदुद्धारक-द्वादशमुख-हनुमान् स्तोत्र-महामन्त्रस्य विभीषण ऋषिः। अनुष्टुप् छन्दः।\\
श्री-द्वादशमुख-प्रचण्ड-हनुमान् देवता।\\
 मारुतात्मज इति बीजम्। अञ्जनासूनुरिति शक्तिः।\\
वायुपुत्रेति कीलकम्। श्रीहनुमत्प्रसादसिद्धिद्वारा सर्वापन्निवारणार्थे जपे विनियोगः।

\dnsub{ध्यानम्}
\fourlineindentedshloka*
{उष्ट्रारूढ-सुवर्चलासहचरन् सुग्रीवमित्राञ्जना-}
{सूनो वायुकुमार केसरितनूजाक्षादिदैत्यान्तक}
{सीतशोकहराग्निनन्दन सुमित्रासम्भवप्राणद}
{श्रीभीमाग्रज शम्भुपुत्र हनुमान् सूर्यास्य तुभ्यं नमः}

\fourlineindentedshloka*
{खड्गं खेटक-भिन्दिपाल-परशुं पाश-त्रिशूल-द्रुमान्}
{चक्रं शङ्ख-गदा-फलाङ्कुश-सुधाकुम्भान् हलं पर्वतम्}
{टङ्कं पर्वतकार्मुकाहिडमरूनेतानि दिव्यायुधान्}
{एवं विंशतिबाहुभिश्च दधतं ध्यायेद्धनूमत्प्रभुम्}

\dnsub{स्तोत्रम्}

\twolineshloka
{ॐ नमो भगवते तुभ्यं नमो मारुतसूनवे}
{नमः श्रीरामभक्ताय श्यामास्याय च ते नमः}

\twolineshloka
{नमो वानरवीराय सुग्रीवसख्यकारिणे}
{लङ्काविदाहकायाथ हेलासागरतारिणे}

\twolineshloka
{सीताशोकविनाशाय राममुद्राधराय च}
{रावणस्य कुलच्छेदकारिणे ते नमो नमः}

\twolineshloka
{मेघनादमखध्वंसकारिणे ते नमो नमः}
{अशोकवनविध्वंसकारिणे भयहारिणे}

\twolineshloka
{वायुपुत्राय वीराय आकाशोदरगामिने}
{वनपालशिरश्छेत्रे लङ्काप्रासादभञ्जिने}

\twolineshloka
{ज्वलत्कनकवर्णाय दीर्घलाङ्गूलधारिणे}
{सौमित्रिजयदात्रे च रामदूताय ते नमः}

\twolineshloka
{अक्षस्य वधकर्त्रे च ब्रह्मशक्तिनिवारिणे}
{लक्ष्मणाङ्गमहाशक्ति-घात-क्षत-विनाशिने}

\twolineshloka
{रक्षोघ्नाय रिपुघ्नाय भूतघ्नाय च ते नमः}
{ऋक्षवानरवीरौघ-प्राणदायक ते नमः}

\twolineshloka
{परसैन्यबलघ्नाय शस्त्रास्त्रविघनाय च}
{विषघ्नाय द्विषघ्नाय ज्वरघ्नाय च ते नमः}

\twolineshloka
{महाभयरिपुघ्नाय भक्तत्राणैककारिणे}
{परप्रेरितमन्त्राणां यन्त्राणां स्तम्भकारिणे}

\twolineshloka
{पयः-पाषाण-तरण-कारणाय नमो नमः}
{बालार्कमण्डलग्रासकारिणे भवतारिणे}

\twolineshloka
{नखायुधाय भीमाय दन्तायुधधराय च}
{रिपुमायाविनाशाय रामाज्ञालोकरक्षिणे}

\twolineshloka
{प्रतिग्रामस्थितायाथ रक्षोभूतवधार्थिने}
{करालशैलशस्त्राय द्रुमशस्त्राय ते नमः}

\twolineshloka
{बालैकब्रह्मचर्याय रुद्रमूर्तिधराय च}
{विहङ्गमाय शर्वाय वज्रदेहाय ते नमः}

\twolineshloka
{कौपीनवाससे तुभ्यं रामभक्तिरताय च}
{दक्षिणाशाभास्कराय शतचन्द्रोदयात्मने}

\twolineshloka
{कृत्या-क्षत-व्यथघ्नाय सर्वक्लेशहराय च}
{स्वाम्याज्ञा-पार्थसङ्ग्राम-सङ्ख्ये सञ्जयधारिणे}

\twolineshloka
{भक्तानां दिव्यवादेषु सङ्ग्रामे जयदायिने}
{किलकिल्याबूबुरोच्चघोरशब्दकराय च}

\threelineshloka
{सर्पाग्निव्याधिसंस्तम्भकारिणे वनचारिणे}
{सदा वनफलाहार-सत्तृप्ताय विशेषतः}
{महार्णव-शिला-बद्ध-सेतवे ते नमो नमः}

\twolineshloka
{वादे विवादे सङ्ग्रामे भये घोरे महावने}
{सिंहव्याघ्रादि चौरेभ्यः स्तोत्रपाठाद्भयं न हि}

\twolineshloka
{दिव्ये भूतभये व्याधौ गृहे स्थावरजङ्गमे}
{राजशस्त्रभये चोग्रबाधा ग्रहभयेषु च}

\threelineshloka
{जले सर्वे महावृष्टौ दुर्भिक्षे प्राणसम्प्लवे}
{पठेत् स्तोत्रं प्रमुच्येत भयेभ्यः सर्वतो नरः}
{तस्य क्वापि भयं नास्ति हनुमत् स्तवपाठतः}

\twolineshloka
{सर्वथा वै त्रिकालं च पठनीयमिमं स्तवम्}
{सर्वान् कामानवाप्नोति नात्र कार्या विचारणा}

\twolineshloka
{विनतायाः स्वमातुश्च दासीत्वस्य निवृत्तये}
{सुधार्णं यातुकामाय महापौरुषशालिने}

\twolineshloka
{विभीषणकृतं स्तोत्रं तार्क्ष्येण समुदीरितम्}
{ये पठन्ति सदा भक्त्या सिद्धयस्तत्करे स्थिताः}

॥इति श्री-सुदर्शनसंहितायां श्री-विभीषणगरुडसंवादे श्री-विभीषणकृतम्~आपदुद्धारक श्री-द्वादशमुख-हनुमान् स्तोत्रं सम्पूर्णम्॥
