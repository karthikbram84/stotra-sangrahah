% !TeX program = XeLaTeX
% !TeX root = ../../shloka.tex

\sect{अक्रूरकृत-दशावतारस्तुतिः}

\twolineshloka
{नमः कारणमत्स्याय प्रलयाब्धिचराय च}
{हयशीर्ष्णे नमस्तुभ्यं मधुकैटभमृत्यवे}%॥१७॥

\twolineshloka
{अकूपाराय बृहते नमो मन्दरधारिणे}
{क्षित्युद्धारविहाराय नमः शूकरमूर्तये}%॥१८॥

\twolineshloka
{नमस्तेऽद्भुतसिंहाय साधुलोकभयापह}
{वामनाय नमस्तुभ्यं क्रान्तत्रिभुवनाय च}%॥१९॥

\twolineshloka
{नमो भृगूणां पतये दृप्तक्षत्रवनच्छिदे}
{नमस्ते रघुवर्याय रावणान्तकराय च}%॥२०॥

\twolineshloka
{नमस्ते वासुदेवाय नमः सङ्कर्षणाय च}
{प्रद्युम्नायानिरुद्धाय सात्वतां पतये नमः}%॥२१॥

\twolineshloka
{नमो बुद्धाय शुद्धाय दैत्यदानवमोहिने}
{म्लेच्छप्रायक्षत्रहन्त्रे नमस्ते कल्किरूपिणे}%॥२२॥

॥इति~श्रीमद्भागवते महापुराणे दशमस्कन्धे चत्वारिंशेऽध्याये श्री-अक्रूरकृत दशावतारस्तुतिः सम्पूर्णा॥ 
%
%\closesection
%\setlength{\shlokaspaceskip}{6pt}

%\fourlineindentedshloka*
%{कस्तूरीतिलकं ललाटपटले वक्षःस्थले कौस्तुभम्}
%{नासाग्रे वरमौक्तिकं करतले वेणुः करे कङ्कणम्}
%{सर्वाङ्गे हरिचन्दनं सुललितं कण्ठे च मुक्तावली}
%{गोपस्त्रीपरिवेष्टितो विजयते गोपालचूडामणिः}
%

%\fourlineindentedshloka*
%{अस्ति स्वस्तरुणीकराग्रविगलत् कल्पप्रसूनाप्लुतम्}
%{वस्तुप्रस्तुतवेणुनादलहरी निर्वाणनिर्व्याकुलम्}
%{स्रस्तस्रस्तनिबद्धनीविविलसत् गोपीसहस्रावृतम्}
%{हस्तन्यस्तनतापवर्गमखिलोदारं किशोराकृति}
%
%
%\setlength{\shlokaspaceskip}{24pt}