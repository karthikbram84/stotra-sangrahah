% !TeX program = XeLaTeX
% !TeX root = ../../shloka.tex

\sect{बालमुकुन्दाष्टकम्}

\twolineshloka
{करारविन्देन पदारविन्दं मुखारविन्दे विनिवेशयन्तम्‌}
{वटस्य पत्रस्य पुटे शयानं बालं मुकुन्दं मनसा स्मरामि}% १}%

\twolineshloka
{संहृत्य लोकान् वटपत्रमध्ये शयानमाद्यन्तविहीनरूपम्}
{सर्वेश्वरं सर्वहितावतारं बालं मुकुन्दं मनसा स्मरामि}% २}%

\twolineshloka
{इन्दीवरश्यामलकोमलाङ्गं इन्द्रादिदेवार्चितपादपद्मम्‌}
{सन्तानकल्पद्रुममाश्रितानां बालं मुकुन्दं मनसा स्मरामि}% ३}%

\twolineshloka
{लम्बालकं लम्बितहारयष्टिं शृङ्गारलीलाङ्कितदन्तपङ्क्तिम्}
{बिम्बाधरं चारुविशालनेत्रं बालं मुकुन्दं मनसा स्मरामि}% ४}%

\twolineshloka
{शिक्ये निधायाद्यपयोदधीनि बहिर्गतायां व्रजनायिकायाम्‌}
{भुक्त्वा यथेष्टं कपटेन सुप्तं बालं मुकुन्दं मनसा स्मरामि}% ५}%

\twolineshloka
{कलिन्दजान्तस्थितकालियस्य फणाग्ररङ्गे नटनप्रियन्तम्‌}
{तत्पुच्छहस्तं शरदिन्दुवक्त्रं बालं मुकुन्दं मनसा स्मरामि}% ६}%

\twolineshloka
{उलूखले बद्धमुदारशौर्यं उत्तुङ्गयुग्मार्जुन-भङ्गलीलम्‌}
{उत्फुल्लपद्मायत-चारुनेत्रं बालं मुकुन्दं मनसा स्मरामि}% ७}%

\twolineshloka
{आलोक्य मातुर्मुखमादरेण स्तन्यं पिबन्तं सरसीरुहाक्षम्‌}
{सच्चिन्मयं देवमनन्तरूपं बालं मुकुन्दं मनसा स्मरामि}% २}%

॥इति श्री-बालमुकुन्दाष्टकं सम्पूर्णम्‌॥

\twolineshloka*
{आकुञ्चितं जानु करं च वामं न्यस्य क्षितौ दक्षिणहस्तपद्मे}
{आलोकयन्तं नवनीतखण्डं बालं मुकुन्दं मनसा स्मरामि}% ३.९२}%