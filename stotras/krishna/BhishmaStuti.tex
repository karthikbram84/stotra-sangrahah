% !TeX program = XeLaTeX
% !TeX root = ../../shloka.tex
\sect{भीष्मस्तुतिः}

\uvacha{श्री-भीष्म उवाच}

\twolineshloka
{इति मतिरुपकल्पिता वितृष्णा भगवति सात्वतपुङ्गवे विभूम्नि}
{स्वसुखमुपगते क्वचिद्विहर्तुं प्रकृतिमुपेयुषि यद्भवप्रवाहः}

\twolineshloka
{त्रिभुवनकमनं तमालवर्णं रविकरगौरवराम्बरं दधाने}
{वपुरलककुलावृताननाब्जं विजयसखे रतिरस्तु मेऽनवद्या}

\twolineshloka
{युधि तुरगरजोविधूम्रविष्वक्कचलुलितश्रमवार्यलङ्कृतास्ये}
{मम निशितशरैर्विभिद्यमान त्वचि विलसत्कवचेऽस्तु कृष्ण आत्मा}

\twolineshloka
{सपदि सखिवचो निशम्य मध्ये निजपरयोर्बलयो रथं निवेश्य}
{स्थितवति परसैनिकायुरक्ष्णा हृतवति पार्थसखे रतिर्ममास्तु}

\twolineshloka
{व्यवहितपृतनामुखं निरीक्ष्य स्वजनवधाद्विमुखस्य दोषबुद्ध्या}
{कुमतिमहरदात्मविद्यया यश्चरणरतिः परमस्य तस्य मेऽस्तु}

\twolineshloka
{स्वनिगममपहाय मत्प्रतिज्ञामृतमधिकर्तुमवप्लुतो रथस्थः}
{धृतरथचरणोऽभ्ययाच्चलद्गुर्हरिरिव हन्तुमिभं गतोत्तरीयः}

\twolineshloka
{शितविशिखहतो विशीर्णदंशः क्षतजपरिप्लुत आततायिनो मे}
{प्रसभमभिससार मद्वधार्थं स भवतु मे भगवान्गतिर्मुकुन्दः}

\twolineshloka
{विजयरथकुटुम्ब आत्ततोत्रे धृतहयरश्मिनि तच्छ्रियेक्षणीये}
{भगवति रतिरस्तु मे मुमूर्षोर्यमिह निरीक्ष्य हता गताः स्वरूपम्}

\twolineshloka
{ललितगतिविलासवल्गुहास प्रणयनिरीक्षणकल्पितोरुमानाः}
{कृतमनुकृतवत्य उन्मदान्धाः प्रकृतिमगन्किल यस्य गोपवध्वः}

\twolineshloka
{मुनिगणनृपवर्यसङ्कुलेऽन्तः सदसि युधिष्ठिरराजसूय एषाम्}
{अर्हणमुपपेद ईक्षणीयो मम दृशिगोचर एष आविरात्मा}

\twolineshloka
{तमिममहमजं शरीरभाजां हृदि हृदि धिष्ठितमात्मकल्पितानाम्}
{प्रतिदृशमिव नैकधार्कमेकं समधिगतोऽस्मि विधूतभेदमोहः}

\uvacha{सूत उवाच}
\twolineshloka*
{कृष्ण एवं भगवति मनोवाग्दृष्टिवृत्तिभिः}
{आत्मन्यात्मानमावेश्य सोऽन्तःश्वास उपारमत्}

{॥इति श्रीमद्भागवते महापुराणे पारमहंस्यां संहितायां प्रथमे स्कन्धे नवमेऽध्याये श्री-भीष्मस्तुतिः सम्पूर्णः॥}
