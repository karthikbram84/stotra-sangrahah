% !TeX program = XeLaTeX
% !TeX root = ../../shloka.tex
\sect{ध्रुवस्तुतिः}

\uvacha{ध्रुव उवाच}
\fourlineindentedshloka
{योऽन्तः प्रविश्य मम वाचमिमां प्रसुप्ताम्}
{सञ्जीवयत्यखिलशक्तिधरः स्वधाम्ना}
{अन्यांश्च हस्तचरणश्रवणत्वगादीन्}
{प्राणान्नमो भगवते पुरुषाय तुभ्यम्}

\fourlineindentedshloka
{एकस्त्वमेव भगवन्निदमात्मशक्त्या}
{मायाख्ययोरुगुणया महदाद्यशेषम्}
{सृष्ट्वाऽनुविश्य पुरुषस्तदसद्गुणेषु}
{नानेव दारुषु विभावसुवद्विभासि}

\fourlineindentedshloka
{त्वद्दत्तया वयुनयेदमचष्ट विश्वम्}
{सुप्तप्रबुद्ध इव नाथ भवत्प्रपन्नः}
{तस्यापवर्ग्यशरणं तव पादमूलं}
{विस्मर्यते कृतविदा कथमार्तबन्धो}

\fourlineindentedshloka
{नूनं विमुष्टमतयस्तव मायया ते}
{ये त्वां भवाप्ययविमोक्षणमन्यहेतोः}
{अर्चन्ति कल्पकतरुं कुणपोपभोग्यम्}
{इच्छन्ति यत्स्पर्शजं निरयेऽपि नॄणाम्}

\fourlineindentedshloka
{या निर्वृतिस्तनुभृतां तव पादपद्म}
{ध्यानाद्भवज्जनकथाश्रवणेन वा स्यात्}
{सा ब्रह्मणि स्वमहिमन्यपि नाथ मा भूत्}
{किं त्वन्तकासिलुलितात्पततां विमानात्}

\fourlineindentedshloka
{भक्तिं मुहुः प्रवहतां त्वयि मे प्रसङ्गो}
{भूयादनन्त महताममलाशयानाम्}
{येनाञ्जसोल्बणमुरुव्यसनं भवाब्धिम्}
{नेष्ये भवद्गुणकथामृतपानमत्तः}

\fourlineindentedshloka
{ते न स्मरन्त्यतितरां प्रियमीश मर्त्यम्}
{ये चान्वदः सुतसुहृद्गृहवित्तदाराः}
{ये त्वब्जनाभ भवदीयपदारविन्द}
{सौगन्ध्यलुब्धहृदयेषु कृतप्रसङ्गाः}

\fourlineindentedshloka
{तिर्यङ्नगद्विजसरीसृपदेवदैत्य}
{मर्त्यादिभिः परिचितं सदसद्विशेषम्}
{रूपं स्थविष्ठमज ते महदाद्यनेकम्}
{नातः परं परम वेद्मि न यत्र वादः}

\fourlineindentedshloka
{कल्पान्त एतदखिलं जठरेण गृह्णन्}
{शेते पुमान्स्वदृगनन्तसखस्तदङ्के}
{यन्नाभिसिन्धुरुहकाञ्चनलोकपद्म}
{गर्भे द्युमान्भगवते प्रणतोऽस्मि तस्मै}

\fourlineindentedshloka
{त्वं नित्यमुक्तपरिशुद्धविबुद्ध आत्मा}
{कूटस्थ आदिपुरुषो भगवांस्त्र्यधीशः}
{यद्बुद्ध्यवस्थितिमखण्डितया स्वदृष्ट्या}
{द्रष्टा स्थितावधिमखो व्यतिरिक्त आस्से}

\fourlineindentedshloka
{यस्मिन्विरुद्धगतयो ह्यनिशं पतन्ति}
{विद्यादयो विविधशक्तय आनुपूर्व्यात्}
{तद्ब्रह्म विश्वभवमेकमनन्तमाद्यम्}
{आनन्दमात्रमविकारमहं प्रपद्ये}

\fourlineindentedshloka
{सत्याशिषो हि भगवंस्तव पादपद्मम्}
{आशीस्तथानुभजतः पुरुषार्थमूर्तेः}
{अप्येवमर्य भगवान्परिपाति दीनान्}
{वाश्रेव वत्सकमनुग्रहकातरोऽस्मान्}

{॥इति श्रीमद्भागवते महापुराणे पारमहंस्यां संहितायां चतुर्थे स्कन्धे नवमेऽध्याये श्री-ध्रुवस्तुतिः सम्पूर्णः॥}
