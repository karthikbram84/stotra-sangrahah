% !TeX program = XeLaTeX
% !TeX root = ../../shloka.tex

\sect{कृष्णद्वादशनामस्तोत्रम्}
\twolineshloka*
{शृणुध्वं मुनयः सर्वे गोपालस्य महात्मनः}
{अनन्तस्याप्रमेयस्य नामद्वाशकं स्तवम्}

\twolineshloka*
{अर्जुनाय पुरा गीतं गोपालेन महात्मना}
{द्वारकायां प्रार्थयते यशोदायाश्च सन्निधौ}

\dnsub{ध्यानम्}
\twolineshloka*
{जानुभ्यामपि धावन्तं बाहुभ्यामतिसुन्दरम्}
{सकुण्डलालकं बालं गोपालं चिन्तयेदुषः}

\dnsub{स्तोत्रम्}
\twolineshloka
{प्रथमं तु हरिं विद्यात् द्वितीयं केशवं तथा}
{तृतीयं पद्मनाभं तु चतुर्थं वामनं तथा}% १}%

\twolineshloka
{पञ्चमं वेदगर्भं च षष्ठं तु मधुसूदनं}
{सप्तमं वासुदेवं च वराहं चाष्टमं तथा}% २}%

\twolineshloka
{नवमं पुण्डरीकाक्षं दशमं तु जनार्दनम्}
{कृष्णमेकादशं प्रोक्तं द्वादशं श्रीधरं तथा}% ३}%

\twolineshloka
{एतद्द्वादशनामानि मया प्रोक्तानि फाल्गुन}
{कालत्रये पठेद्यस्तु तस्य पुण्यफलं शृणु}% ४}%

\twolineshloka
{चान्द्रायणसहस्रस्य कन्यादानशतस्य च}
{अश्वमेधसहस्रस्य फलमाप्नोति मानवः}% ५}%

॥इति श्री-कृष्णद्वादशनामस्तोत्रं सम्पूर्णम्‌॥