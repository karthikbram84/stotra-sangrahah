% !TeX program = XeLaTeX
% !TeX root = ../../shloka.tex

\sect{कनकधारास्तवम्}
\fourlineindentedshloka
{अङ्गं हरेः पुलकभूषणमाश्रयन्ती}
{भृङ्गाङ्गनेव मुकुलाभरणं तमालम्}
{अङ्गीकृताखिलविभूतिरपाङ्गलीला}
{माङ्गल्यदास्तु मम मङ्गलदेवतायाः}

\fourlineindentedshloka
{मुग्धा मुहुर्विदधती वदने मुरारेः}
{प्रेमत्रपाप्रणिहितानि गतागतानि}
{माला दृशोर्मधुकरीव महोत्पले या}
{सा मे श्रियं दिशतु सागरसम्भवायाः}

\fourlineindentedshloka
{आमीलिताक्षमधिगम्य मुदा मुकुन्दम्}
{आनन्दकन्दमनिमेषमनङ्गतन्त्रम्}
{आकेकरस्थितकनीनिकपक्ष्मनेत्रम्}
{भूत्यै भवेन्मम भुजङ्गशयाङ्गनायाः}

\fourlineindentedshloka
{बाह्वन्तरे मधुजितः श्रितकौस्तुभे या}
{हारावलीव हरिनीलमयी विभाति}
{कामप्रदा भगवतोऽपि कटाक्षमाला}
{कल्याणमावहतु मे कमलालयायाः}

\fourlineindentedshloka
{कालाम्बुदालिललितोरसि कैटभारेः}
{धाराधरे स्फुरति या तडिदङ्गनेव}
{मातुः समस्तजगतां महनीयमूर्तिः}
{भद्राणि मे दिशतु भार्गवनन्दनायाः}

\fourlineindentedshloka
{प्राप्तं पदं प्रथमतः खलु यत्प्रभावात्}
{माङ्गल्यभाजि मधुमाथिनि मन्मथेन}
{मय्यापतेत्तदिह मन्थरमीक्षणार्धम्}
{मन्दालसं च मकरालयकन्यकायाः}

\fourlineindentedshloka
{विश्वामरेन्द्रपदवीभ्रमदानदक्षम्}
{आनन्दहेतुरधिकं मुरविद्विषोऽपि}
{ईषन्निषीदतु मयि क्षणमीक्षणार्द्धम्}
{इन्दीवरोदरसहोदरमिन्दिरायाः}

\fourlineindentedshloka
{इष्टा विशिष्टमतयोऽपि यया दयार्द्र-}
{दृष्ट्या त्रिविष्टपपदं सुलभं लभन्ते}
{दृष्टिः प्रहृष्टकमलोदरदीप्तिरिष्टाम्}
{पुष्टिं कृषीष्ट मम पुष्करविष्टरायाः}

\fourlineindentedshloka
{दद्याद्दयानुपवनो द्रविणाम्बुधाराम्}
{अस्मिन्नकिञ्चनविहङ्गशिशौ विषण्णे}
{दुष्कर्मघर्ममपनीय चिराय दूरम्}
{नारायणप्रणयिनीनयनाम्बुवाहः}

\fourlineindentedshloka
{गीर्देवतेति गरुडध्वजसुन्दरीति}
{शाकम्भरीति शशिशेखरवल्लभेति}
{सृष्टिस्थितिप्रलयकेलिषु संस्थितायै}
{तस्यै नमस्त्रिभुवनैकगुरोस्तरुण्यै}

\fourlineindentedshloka
{श्रुत्यै नमोऽस्तु शुभकर्मफलप्रसूत्यै}
{रत्यै नमोऽस्तु रमणीयगुणार्णवायै}
{शक्त्यै नमोऽस्तु शतपत्रनिकेतनायै}
{पुष्ट्यै नमोऽस्तु पुरुषोत्तमवल्लभायै}

\fourlineindentedshloka
{नमोऽस्तु नालीकनिभाननायै}
{नमोऽस्तु दुग्धोदधिजन्मभूम्यै}
{नमोऽस्तु सोमामृतसोदरायै}
{नमोऽस्तु नारायणवल्लभायै}

\fourlineindentedshloka
{नमोऽस्तु हेमाम्बुजपीठिकायै}
{नमोऽस्तु भूमण्डलनायिकायै}
{नमोऽस्तु देवादिदयापरायै}
{नमोऽस्तु शार्ङ्गायुधवल्लभायै}

\fourlineindentedshloka
{नमोऽस्तु देव्यै भृगुनन्दनायै}
{नमोऽस्तु विष्णोरुरसि स्थितायै}
{नमोऽस्तु लक्ष्म्यै कमलालयायै}
{नमोऽस्तु दामोदरवल्लभायै}

\fourlineindentedshloka
{नमोऽस्तु कान्त्यै कमलेक्षणायै}
{नमोऽस्तु भूत्यै भुवनप्रसूत्यै}
{नमोऽस्तु देवादिभिरर्चितायै}
{नमोऽस्तु नन्दात्मजवल्लभायै}

\fourlineindentedshloka
{सम्पत्कराणि सकलेन्द्रियनन्दनानि}
{साम्राज्यदानविभवानि सरोरुहाक्षि}
{त्वद्वन्दनानि दुरिताहरणोद्यतानि}
{मामेव मातरनिशं कलयन्तु मान्ये}

\fourlineindentedshloka
{यत्कटाक्षसमुपासनाविधिः}
{सेवकस्य सकलार्थसम्पदः}
{सन्तनोति वचनाङ्गमानसैः}
{त्वां मुरारिहृदयेश्वरीं भजे}

\fourlineindentedshloka
{सरसिजनिलये सरोजहस्ते}
{धवलतमांशुकगन्धमाल्यशोभे}
{भगवति हरिवल्लभे मनोज्ञे}
{त्रिभुवनभूतिकरि प्रसीद मह्यम्}

\fourlineindentedshloka
{दिग्घस्तिभिः कनककुम्भमुखावसृष्ट-}
{स्वर्वाहिनी विमलचारुजलाप्लुताङ्गीम्}
{प्रातर्नमामि जगतां जननीमशेष-}
{लोकाधिनाथगृहिणीम् अमृताब्धिपुत्रीम्}

\fourlineindentedshloka
{कमले कमलाक्षवल्लभे त्वं}
{करुणापूरतरङ्गितैरपाङ्गैः}
{अवलोकय मामकिञ्चनानां}
{प्रथमं पात्रमकृत्रिमं दयायाः}

\fourlineindentedshloka
{स्तुवन्ति ये स्तुतिभिरमीभिरन्वहम्}
{त्रयीमयीं त्रिभुवनमातरं रमाम्}
{गुणाधिका गुरुतरभाग्यभागिनो}
{भवन्ति ते भुवि बुधभाविताशयाः}

\fourlineindentedshloka*
{देवि प्रसीद जगदीश्वरि लोकमातः}
{कल्याणगात्रि कमलेक्षणजीवनाथे}
{दारिद्र्यभीतिहृदयं शरणागतं माम्}
{आलोकय प्रतिदिनं सदयैरपाङ्गैः}
॥इति  श्रीमच्छङ्कराचार्यविरचितं श्री-कनकधारास्तवं सम्पूर्णम्॥