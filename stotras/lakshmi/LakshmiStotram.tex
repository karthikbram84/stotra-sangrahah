% !TeX program = XeLaTeX
% !TeX root = ../../shloka.tex

\sect{लक्ष्मीस्तोत्रम् विष्णुपुराणान्तर्गतम्}

\uvacha{श्री-पराशर उवाच}
\twolineshloka
{सिंहासनगतः शक्रः सम्प्राप्य त्रिदिवं पुनः}
{देवराज्ये स्थितो देवीं तुष्टावाब्जकरां ततः}

\uvacha{इन्द्र उवाच}
\twolineshloka
{नमस्ये सर्वलोकानां जननीमब्जसम्भवाम्}
{श्रियमुन्निद्रपद्माक्षीं विष्णुवक्षःस्थलस्थिताम्}

\twolineshloka
{पद्मालयां पद्मकरां पद्मपत्रनिभेक्षणाम्}
{वन्दे पद्ममुखीं देवीं पद्मनाभप्रियामहम्}

\twolineshloka
{त्वं सिद्धिस्त्वं स्वधा स्वाहा सुधा त्वं लोकपावनी}
{सन्ध्या रात्रिः प्रभा भूतिर्मेधा श्रद्धा सरस्वती}

\twolineshloka
{यज्ञविद्या महाविद्या गुह्यविद्या च शोभने}
{आत्मविद्या च देवि त्वं विमुक्तिफलदायिनी}

\twolineshloka
{आन्वीक्षिकी त्रयी वार्ता दण्डनीतिस्त्वमेव च}
{सौम्यासौम्यैर्जगद्रूपैस्त्वयैतद्देवि पूरितम्}

\twolineshloka
{का त्वन्या त्वामृते देवि सर्वयज्ञमयं वपुः}
{अध्यास्ते देवदेवस्य योगचिन्त्यं गदाभृतः}

\twolineshloka
{त्वया देवि परित्यक्तं सकलं भुवनत्रयम्}
{विनष्टप्रायमभवत्त्वयेदानीं समेधितम्}

\twolineshloka
{दाराः पुत्रास्तथाऽऽगारसुहृद्धान्यधनादिकम्}
{भवत्येतन्महाभागे नित्यं त्वद्वीक्षणान्नृणाम्}

\twolineshloka
{शरीरारोग्यमैश्वर्यमरिपक्षक्षयः सुखम्}
{देवि त्वद्दृष्टिदृष्टानां पुरुषाणां न दुर्लभम्}

\twolineshloka
{त्वं माता सर्वलोकानां देवदेवो हरिः पिता}
{त्वयैतद्विष्णुना चाम्ब जगद्व्याप्तं चराचरम्}

\twolineshloka
{मा नः कोशस्तथा गोष्ठं मा गृहं मा परिच्छदम्}
{मा शरीरं कलत्रं च त्यजेथाः सर्वपावनि}

\twolineshloka
{मा पुत्रान्मा सुहृद्वर्गं मा पशून्मा विभूषणम्}
{त्यजेथा मम देवस्य विष्णोर्वक्षःस्थलालये}

\twolineshloka
{सत्त्वेन सत्यशौचाभ्यां तथा शीलादिभिर्गुणैः}
{त्यज्यन्ते ते नराः सद्यः सन्त्यक्ता ये त्वयाऽमले}

\twolineshloka
{त्वया विलोकिताः सद्यः शीलाद्यैरखिलैर्गुणैः}
{कुलैश्वर्यैश्च युज्यन्ते पुरुषा निर्गुणा अपि}

\twolineshloka
{स श्लाघ्यः स गुणी धन्यः स कुलीनः स बुद्धिमान्}
{स शूरः स च विक्रान्तो यस्त्वया देवि वीक्षितः}

\twolineshloka
{सद्यो वैगुण्यमायान्ति शीलाद्याः सकला गुणाः}
{पराङ्मुखी जगद्धात्री यस्य त्वं विष्णुवल्लभे}

\twolineshloka
{न ते वर्णयितुं शक्ता गुणाञ्जिह्वाऽपि वेधसः}
{प्रसीद देवि पद्माक्षि माऽस्मांस्त्याक्षीः कदाचन}

\uvacha{श्री-पराशर उवाच}
\twolineshloka
{एवं श्रीः संस्तुता सम्यक् प्राह हृष्टा शतक्रतुम्}
{शृण्वतां सर्वदेवानां सर्वभूतस्थिता द्विज}

\uvacha{श्रीरुवाच}
\twolineshloka
{परितुष्टाऽस्मि देवेश स्तोत्रेणानेन ते हरे}
{वरं वृणीष्व यस्त्विष्टो वरदाऽहं तवाऽऽगता}

\uvacha{इन्द्र उवाच}
\twolineshloka
{वरदा यदि मे देवि वरार्हो यदि वाऽप्यहम्}
{त्रैलोक्यं न त्वया त्याज्यमेष मेऽस्तु वरः परः}

\twolineshloka
{स्तोत्रेण यस्तथैतेन त्वां स्तोष्यत्यब्धिसम्भवे}
{स त्वया न परित्याज्यो द्वितीयोऽस्तु वरो मम}

\uvacha{श्रीरुवाच}
\twolineshloka
{त्रैलोक्यं त्रिदशश्रेष्ठ न सन्त्यक्ष्यामि वासव}
{दत्तो वरो मयाऽयं ते स्तोत्राराधनतुष्टया}

\twolineshloka
{यश्च सायं तथा प्रातः स्तोत्रेणानेन मानवः}
{मां स्तोष्यति न तस्याहं भविष्यामि पराङ्मुखी}

॥इति श्री-विष्णुपुराणे श्री-लक्ष्मीस्तोत्रं सम्पूर्णम्॥
