% !TeX program = XeLaTeX
% !TeX root = ../../shloka.tex

\sect{महालक्ष्म्यष्टकम्}
\uvacha{इन्द्र उवाच}
\twolineshloka
{नमस्तेऽस्तु महामाये श्रीपीठे सुरपूजिते}
{शङ्खचक्रगदाहस्ते महालक्ष्मि नमोऽस्तु ते}

\twolineshloka
{नमस्ते गरुडारूढे कोलासुरभयङ्करि}
{सर्वपापहरे देवि महालक्ष्मि नमोऽस्तु ते}

\twolineshloka
{सर्वज्ञे सर्ववरदे सर्वदुष्टभयङ्करि}
{सर्वदुःखहरे देवि महालक्ष्मि नमोऽस्तु ते}

\twolineshloka
{सिद्धिबुद्धिप्रदे देवि भुक्तिमुक्तिप्रदायिनि}
{मन्त्रमूर्ते सदा देवि महालक्ष्मि नमोऽस्तु ते}

\twolineshloka
{आद्यन्तरहिते देवि आद्यशक्तिमहेश्वरि}
{योगजे योगसम्भूते महालक्ष्मि नमोऽस्तु ते}

\twolineshloka
{स्थूलसूक्ष्ममहारौद्रे महाशक्ति महोदरे}
{महापापहरे देवि महालक्ष्मि नमोऽस्तु ते}

\twolineshloka
{पद्मासनस्थिते देवि परब्रह्मस्वरूपिणि}
{परमेशि जगन्मातर्महालक्ष्मि नमोऽस्तु ते}

\twolineshloka
{श्वेताम्बरधरे देवि नानालङ्कारभूषिते}
{जगत्स्थिते जगन्मातर्महालक्ष्मि नमोऽस्तु ते}

%\dnsub{फलश्रुतिः}
\twolineshloka*
{महालक्ष्म्यष्टकं स्तोत्रं यः पठेद्भक्तिमान्नरः}
{सर्वसिद्धिमवाप्नोति राज्यं प्राप्नोति सर्वदा}

\twolineshloka*
{एककाले पठेन्नित्यं महापापविनाशनम्}
{द्विकालं यः पठेन्नित्यं धनधान्यसमन्वितः}

\twolineshloka*
{त्रिकालं यः पठेन्नित्यं महाशत्रुविनाशनम्}
{महालक्ष्मीर्भवेन्नित्यं प्रसन्ना वरदा शुभा}

॥इति श्रीमद्पद्मपुराणे श्री-महालक्ष्म्यष्टकं सम्पूर्णम्॥