% !TeX program = XeLaTeX
% !TeX root = ../../shloka.tex
\sect{नर्मदाष्टकम्}

\fourlineindentedshloka
{सबिन्दु-सिन्धु-सुस्खलत्-तरङ्ग-भङ्ग-रञ्जितम्}
{द्विषत्सु पाप-जात-जातकादि-वारि-संयुतम्}
{कृतान्त-दूत-काल-भूत-भीति-हारि-वर्मदे}
{त्वदीय-पाद-पङ्कजं नमामि देवि नर्मदे} %|| १ ||

\fourlineindentedshloka
{त्वदम्बु-लीन-दीन-मीन-दिव्य-सम्प्रदायकम्}
{कलौ मलौघ-भार-हारि-सर्व-तीर्थ-नायकम्}
{सुमच्छ-कच्छ-नक्र-चक्र-वाक-चक्र-शर्मदे}
{त्वदीय-पाद-पङ्कजं नमामि देवि नर्मदे} %|| २ ||

\fourlineindentedshloka
{महा-गभीर-नीर-पूर-पाप-धूत-भूतलम्}
{ध्वनत्-समस्त-पातकारि-दारिताप-दाछलम्}
{जगल्लये महाभये मृकण्डु-सूनु-हर्म्यदे}
{त्वदीय-पादपङ्कजं नमामि देवि नर्मदे} %|| ३ ||

\fourlineindentedshloka
{गतं तदैव मे भयं त्वदम्बु वीक्षितं यदा}
{मृकण्डु-सूनु-शौनका-सुरारि-सेवितं सदा}
{पुनर्-भवाब्धि-जन्मजं भवाब्धि-दुःख-वर्मदे}
{त्वदीय-पाद-पङ्कजं नमामि देवि नर्मदे} %|| ४ ||

\fourlineindentedshloka
{अलक्ष्य-लक्ष-किन्नरामरासुरादि-पूजितम्}
{सुलक्ष-नीर-तीर-धीर-पक्षि-लक्ष-कूजितम्}
{वसिष्ठ-शिष्ट-पिप्पलादि-कर्दमादि-शर्मदे}
{त्वदीय-पादपङ्कजं नमामि देवि नर्मदे} %|| ५ ||

\fourlineindentedshloka
{सनत्कुमार-नाचिकेत-कश्यपात्रि-षत्पदैः}
{धृतं स्वकीय-मानसेषु नारदादि-षत्पदैः}
{रवीन्दु-रन्ति-देव-देवराज-कर्म-शर्मदे}
{त्वदीय-पाद-पङ्कजं नमामि देवि नर्मदे} %|| ६ ||

\fourlineindentedshloka
{अलक्ष-लक्ष-लक्ष-पाप-लक्ष-सारसायुधम्}
{ततस्तु जीव-जन्तु-तन्तु-भुक्ति-मुक्ति-दायकम्}
{विरिञ्चि-विष्णु-शङ्कर-स्वकीय-धाम-वर्मदे}
{त्वदीय-पाद-पङ्कजं नमामि देवि नर्मदे} %|| ७ ||

\fourlineindentedshloka
{अहो धृतं स्वनं श्रुतं महेशि-केश-जातटे}
{किरात-सूत-बाडबेषु पण्डिते शठे नटे}
{दुरन्त-पाप-तापहारि-सर्व-जन्तु-शर्मदे}
{त्वदीय-पाद-पङ्कजं नमामि देवि नर्मदे} %|| ८ ||

\fourlineindentedshloka
{इदं तु नर्मदाष्टकं त्रिकालमेव ये सदा}
{पठन्ति ते निरन्तरं न यान्ति दुर्गतिं कदा}
{सुलभ्यदेहदुर्लभं महेशधामगौरवम्}
{पुनर्भवा नरा न वै विलोकयन्ति रौरवम्} %|| ९ ||

॥इति श्रीमत्परमहंसपरिव्राजकाचार्यस्य श्री-गोविन्द-भगवत्पूज्य-पाद-शिष्यस्य 
श्रीमच्छङ्करभगवतः कृतौ श्री-नर्मदाष्टकम् सम्पूर्णम्॥