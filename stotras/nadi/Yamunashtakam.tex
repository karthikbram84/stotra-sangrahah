% !TeX program = XeLaTeX
% !TeX root = ../../shloka.tex
\sect{यमुनाष्टकम्}

\fourlineindentedshloka
{मुरारिकायकालिमाललामवारिधारिणी}
{तृणीकृतत्रिविष्टपा त्रिलोकशोकहारिणी}
{मनोऽनुकूलकूलकुञ्जपुञ्जधूतदुर्मदा}
{धुनोतु नो मनोमलं कलिन्दनन्दिनी सदा}% ॥ १॥

\fourlineindentedshloka
{मलापहारिवारिपूरिभूरिमण्डितामृता}
{भृशं प्रवातकप्रपञ्चनातिपण्डितानिशा}
{सुनन्दनन्दिनाङ्गसङ्गरागरञ्जिता हिता}
{धुनोतु नो मनोमलं कलिन्दनन्दिनी सदा}% ॥ २॥

\fourlineindentedshloka
{लसत्तरङ्गसङ्गधूतभूतजातपातका}
{नवीनमाधुरीधुरीणभक्तिजातचातका}
{तटान्तवा  सदा  सहंससंसृताह्निकामदा} 
{धुनोतु नो मनोमलं कलिन्दनन्दिनी सदा}% ॥ ३॥

\fourlineindentedshloka
{विहाररासस्वेदभेदधीरतीरमारुता}
{गता गिरामगोचरे यदीयनीरचारुता}
{प्रवाहसाहचर्यपूतमेदिनीनदीनदा}
{धुनोतु नो मनोमलं कलिन्दनन्दिनी सदा}% ॥ ४॥

\fourlineindentedshloka
{तरङ्गसङ्गसैकतान्तरातितं सदासिता}
{शरन्निशाकरांशुमञ्जुमञ्जरी सभाजिता}
{भवार्चनाप्रचारुणाम्बुनाधुना विशारदा}
{धुनोतु नो मनोमलं कलिन्दनन्दिनी सदा}% ॥ ५॥

\fourlineindentedshloka
{जलान्तकेलिकारिचारुराधिकाङ्गरागिणी}
{स्वभर्तुरन्यदुर्लभाङ्गताङ्गतांशभागिनी}
{स्वदत्तसुप्तसप्तसिन्धुभेदिनातिकोविदा}
{धुनोतु नो मनोमलं कलिन्दनन्दिनी सदा}% ॥ ६॥

\fourlineindentedshloka
{जलच्युताच्युताङ्गरागलम्पटालिशालिनी}
{विलोलराधिकाकचान्तचम्पकालिमालिनी}
{सदावगाहनावतीर्णभर्तृभृत्यनारदा}
{धुनोतु नो मनोमलं कलिन्दनन्दिनी सदा}% ॥ ७॥

\fourlineindentedshloka
{सदैव नन्दिनन्दकेलिशालिकुञ्जमञ्जुला}
{तटोत्थफुल्लमल्लिकाकदम्बरेणुसूज्ज्वला}
{जलावगाहिनां नृणां भवाब्धिसिन्धुपारदा}
{धुनोतु नो मनोमलं कलिन्दनन्दिनी सदा}% ॥ ८॥

॥इति श्रीमच्छङ्कराचार्यविरचितं श्री-यमुनाष्टकं सम्पूर्णम्॥
