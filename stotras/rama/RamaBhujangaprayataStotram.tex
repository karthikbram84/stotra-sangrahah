% !TeX program = XeLaTeX
% !TeX root = ../../shloka.tex

\sect{रामभुजङ्गप्रयातस्तोत्रम्}

\fourlineindentedshloka
{विशुद्धं परं सच्चिदानन्द-रूपम्}
{गुणाधारमाधार-हीनं वरेण्यम्}
{महान्तं विभान्तं गुहान्तं गुणान्तम्}
{सुखान्तं स्वयं धाम रामं प्रपद्ये}%१

\fourlineindentedshloka
{शिवं नित्यमेकं विभुं तारकाख्यम्}
{सुखाकारमाकार-शून्यं सुमान्यम्}
{महेशं कलेशं सुरेशं परेशं}
{नरेशं निरीशं महीशं प्रपद्ये}%२

\fourlineindentedshloka
{यदाऽवर्णयत् कर्ण-मूलेऽन्त-काले}
{शिवो राम रामेति रामेति काश्याम्}
{तदेकं परं तारक-ब्रह्म-रूपम्}
{भजेऽहं भजेऽहं भजेऽहं भजेऽहम्}%३

\fourlineindentedshloka
{महारत्न-पीठे शुभे कल्प-मूले}
{सुखासीनमादित्य-कोटि-प्रकाशम्}
{सदा जानकी-लक्ष्मणोपेतमेकम्}
{सदा रामचन्द्रं भजेऽहं भजेऽहम्}%४

\fourlineindentedshloka
{क्वणद्रत्न-मञ्जीर-पादारविन्दम्}
{लसन्मेखला-चारु-पीताम्बराढ्यम्}
{महारत्न-हारोल्लसत्-कौस्तुभाङ्गम्}
{नदच्चञ्चरी-मञ्जरी-लोल-मालम्}%५

\fourlineindentedshloka
{लसच्चन्द्रिका-स्मेर-शोणाधराभम्}
{समुद्यत्-पतङ्गेन्दु-कोटि-प्रकाशम्}
{नमद्-ब्रह्म-रुद्रादि-कोटीर-रत्न-}
{स्फुरत्-कान्ति-नीराजना-ऽऽराधिताङ्घ्रिम्}%६

\fourlineindentedshloka
{पुरः प्राञ्जलीनाञ्जनेयादि-भक्तान्}
{स्व-चिन्मुद्रया भद्रया बोधयन्तम्}
{भजेऽहं भजेऽहं सदा रामचन्द्रम्}
{त्वदन्यं न मन्ये न मन्ये न मन्ये}%७

\fourlineindentedshloka
{यदा मत्समीपं कृतान्तः समेत्य}
{प्रचण्ड-प्रतापैर्-भटैर्-भीषयेन्माम्}
{तदाऽऽविष्करोषि त्वदीयं स्वरूपम्}
{तदापत्-प्रणाशं स-कोदण्ड-बाणम्}%८

\fourlineindentedshloka
{निजे मानसे मन्दिरे सन्निधेहि}
{प्रसीद प्रसीद प्रभो रामचन्द्र}
{स-सौमित्रिणा कैकयी-नन्दनेन}
{स्व-शक्त्याऽनुभक्त्या च संसेव्यमान}%९

\fourlineindentedshloka
{स्वभक्ताग्रगण्यैः कपीशैर्-महीशैर्-}
{अनीकैरनेकैश्च राम प्रसीद}
{नमस्ते नमोऽस्त्वीश राम प्रसीद}
{प्रशाधि प्रशाधि प्रकाशं प्रभो माम्}%१०

\fourlineindentedshloka
{त्वमेवासि दैवं परं मे यदेकम्}
{सुचैतन्यमेतत् त्वदन्यं न मन्ये}
{यतोऽभूदमेयं वियद्-वायु-तेजो-}
{जलोर्व्यादि-कार्यं चरं चाचरं च}%११

\fourlineindentedshloka
{नमः सच्चिदानन्द-रूपाय तस्मै}
{नमो देव-देवाय रामाय तुभ्यम्}
{नमो जानकी-जीवितेशाय तुभ्यम्}
{नमः पुण्डरीकायताक्षाय तुभ्यम्}%१२

\fourlineindentedshloka
{नमो भक्ति-युक्तानुरक्ताय तुभ्यम्}
{नमः पुण्य-पुञ्जैक-लभ्याय तुभ्यम्}
{नमो वेद-वेद्याय चाऽऽद्याय पुंसे}
{नमः सुन्दरायेन्दिरा-वल्लभाय}%१३

\fourlineindentedshloka
{नमो विश्व-कर्त्रे नमो विश्व-हर्त्रे}
{नमो विश्व-भोक्त्रे नमो विश्व-मात्रे}
{नमो विश्व-नेत्रे नमो विश्व-जेत्रे}
{नमो विश्व-पित्रे नमो विश्व-मात्रे}%१४

\fourlineindentedshloka
{नमस्ते नमस्ते समस्त-प्रपञ्च-}
{प्रभाग-प्रवीण प्रमाण-प्रवीण}
{मदीयं मनस्-त्वत्-पद-द्वन्द्व-सेवाम्}
{विधातुं प्रवृत्तं सुचैतन्य-सिद्ध्यै}%१५

\fourlineindentedshloka
{शिलाऽपि त्वदङ्घ्रि-क्षमा-सङ्गि-रेणु-}
{प्रसादाद्धि चैतन्यमाऽधत्त राम}
{नरस्-त्वत्पद-द्वन्द्व-सेवा-विधानात्}
{सुचैतन्यमेतीति किं चित्रमद्य}%१६

\fourlineindentedshloka
{पवित्रं चरित्रं विचित्रं त्वदीयम्}
{नरा ये स्मरन्त्यन्वहं रामचन्द्र}
{भवन्तं भवान्तं भरन्तं भजन्तो}
{लभन्ते कृतान्तं न पश्यन्त्यतोऽन्ते}%१७

\fourlineindentedshloka
{स पुण्यः स गण्यः शरण्यो ममायम्}
{नरो वेद यो देव-चूडामणिं त्वाम्}
{सदाकारमेकं चिदानन्द-रूपम्}
{मनो-वागगम्यं परं धाम राम}%१८

\fourlineindentedshloka
{प्रचण्ड-प्रताप-प्रभावाभिभूत-}
{प्रभूतारि-वीर प्रभो रामचन्द्र}
{बलं ते कथं वर्ण्यतेऽतीव बाल्ये}
{यतोऽखण्डि चण्डीश-कोदण्ड-दण्डः}%१९

\fourlineindentedshloka
{दशग्रीवमुग्रं सपुत्रं समित्रम्}
{सरिद्-दुर्ग-मध्यस्थ-रक्षो-गणेशम्}
{भवन्तं विना राम वीरो नरो वा-}
{ऽसुरो वाऽमरो वा जयेत् कस्-त्रिलोक्याम्}%२०

\fourlineindentedshloka
{सदा राम रामेति रामामृतं ते}
{सदाराममानन्द-निष्यन्द-कन्दम्}
{पिबन्तं नमन्तं सुदन्तं हसन्तम्}
{हनूमन्तमन्तर्भजे तं नितान्तम्}%२१

\fourlineindentedshloka
{सदा राम रामेति रामामृतं ते}
{सदाराममानन्द-निष्यन्द-कन्दम्}
{पिबन्नन्वहं नन्वहं नैव मृत्योर्-}
{बिभेमि प्रसादादसादात् तवैव}%२२

\fourlineindentedshloka
{असीता-समेतैरकोदण्ड-भूषैर्-}
{असौमित्रि-वन्द्यैरचण्ड-प्रतापैः}
{अलङ्केश-कालैरसुग्रीव-मित्रैर्-}
{अरामाभिधेयैरलं दैवतैर्नः}%२३

\fourlineindentedshloka
{अवीरासन-स्थैरचिन्मुद्रिकाढ्यैर्-}
{अभक्ताञ्जनेयादि-तत्त्व-प्रकाशैः}
{अमन्दार-मूलैरमन्दार-मालैर्-}
{अरामाभिधेयैरलं दैवतैर्नः}%२४

\fourlineindentedshloka
{असिन्धु-प्रकोपैरवन्द्य-प्रतापैर्-}
{अबन्धु-प्रयाणैरमन्द-स्मिताढ्यैः}
{अदण्ड-प्रवासैरखण्ड-प्रबोधैर्-}
{अरामाभिधेयैरलं दैवतैर्नः}%२५

\fourlineindentedshloka
{हरे राम सीतापते रावणारे}
{खरारे मुरारेऽसुरारे परेति}
{लपन्तं नयन्तं सदा-कालमेवम्}
{समालोकयाऽऽलोकयाऽशेष-बन्धो}%२६

\fourlineindentedshloka
{नमस्ते सुमित्रा-सुपुत्राभिवन्द्य}
{नमस्ते सदा कैकयी-नन्दनेड्य}
{नमस्ते सदा वानराधीश-वन्द्य}
{नमस्ते नमस्ते सदा रामचन्द्र}%२७

\fourlineindentedshloka
{प्रसीद प्रसीद प्रचण्ड-प्रताप}
{प्रसीद प्रसीद प्रचण्डारि-काल}
{प्रसीद प्रसीद प्रपन्नानुकम्पिन्}
{प्रसीद प्रसीद प्रभो रामचन्द्र}%२८

\fourlineindentedshloka
{भुजङ्ग-प्रयातं परं वेद-सारम्}
{मुदा रामचन्द्रस्य भक्त्या च नित्यम्}
{पठन् सन्ततं चिन्तयन् स्वान्तरङ्गे}
{स एव स्वयं रामचन्द्रः स धन्यः}%२९

{॥इति श्रीमत्परमहंसपरिव्राजकाचार्यस्य श्री-गोविन्द-भगवत्पूज्य-पाद-शिष्यस्य 
श्रीमच्छङ्करभगवतः कृतौ श्री-रामभुजङ्गप्रयातस्तोत्रं सम्पूर्णम्॥}

\closesection
