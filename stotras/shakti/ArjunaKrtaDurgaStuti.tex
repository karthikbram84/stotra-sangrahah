% !TeX program = XeLaTeX
% !TeX root = ../../shloka.tex

\sect{दुर्गा-स्तुतिः (अर्जुन-कृता)}

\uvacha{सञ्जय उवाच}

\twolineshloka
{धार्तराष्ट्रबलं दृष्ट्वा युद्धाय समुपस्थितम्}
{अर्जुनस्य हितार्थाय कृष्णो वचनमब्रवीत्}

\uvacha{श्रीभगवानुवाच}

\twolineshloka
{शुचिर्भूत्वा महाबाहो सङ्ग्रामाभिमुखे स्थितः}
{पराजयाय शत्रूणां दुर्गास्तोत्रमुदीरय}

\uvacha{सञ्जय उवाच}

\twolineshloka
{एवमुक्तोऽर्जुनः सङ्ख्ये वासुदेवेन धीमता}
{अवतीर्य रथात्पार्थः स्तोत्रमाह कृताञ्जलिः}

\uvacha{अर्जुन उवाच}

\twolineshloka
{नमस्ते सिद्धसेनानि आर्ये मन्दरवासिनि}
{कुमारि कालि कापालि कपिले कृष्णपिङ्गले}

\twolineshloka
{भद्रकालि नमस्तुभ्यं महाकालि नमोऽस्तु ते}
{चण्डि चण्डे नमस्तुभ्यं तारिणि वरवर्णिनि}

\twolineshloka
{कात्यायनि महाभागे करालि विजये जये}
{शिखिपिच्छध्वजधरे नानाभरणभूषिते}

\twolineshloka
{अट्टशूलप्रहरणे खड्गखेटकधारिणि}
{गोपेन्द्रस्यानुजे ज्येष्ठे नन्दगोपकुलोद्भवे}

\twolineshloka
{महिषासृक्प्रिये नित्यं कौशिकि पीतवासिनि}
{अट्टहासे कोकमुखे नमस्तेऽस्तु रणप्रिये}

\twolineshloka
{उमे शाकम्भरि श्वेते कृष्णे कैटभनाशिनि}
{हिरण्याक्षि विरूपाक्षि सुधूम्राक्षि नमोऽस्तु ते}

\twolineshloka
{वेदश्रुति महापुण्ये ब्रह्मण्ये जातवेदसि}
{जम्बूकटकचैत्येषु नित्यं सन्निहितालये}

\twolineshloka
{त्वं ब्रह्मविद्या विद्यानां महानिद्रा च देहिनाम्}
{स्कन्दमातर्भगवति दुर्गे कान्तारवासिनि}

\twolineshloka
{स्वाहाकारः स्वधा चैव कला काष्ठा सरस्वती}
{सावित्री वेदमाता च तथा वेदान्त उच्यते}

\twolineshloka
{स्तुताऽसि त्वं महादेवि विशुद्धेनान्तरात्मना}
{जयो भवतु मे नित्यं त्वत्प्रसादाद्रणाजिरे}

\twolineshloka
{कान्तारभयदुर्गेषु भक्तानां चाऽऽलयेषु च}
{नित्यं वससि पाताले युद्धे जयसि दानवान्}

\twolineshloka
{त्वं जम्भनी मोहिनी च माया ह्रीः श्रीस्तथैव च}
{सन्ध्या प्रभावती चैव सावित्री जननी तथा}

\twolineshloka
{तुष्टिः पुष्टिर्धृतिर्दीप्तिश्चन्द्रादित्यविवर्धिनी}
{भूतिर्भूतिमतां सङ्ख्ये वीक्ष्यसे सिद्धचारणैः}

\uvacha{सञ्जय उवाच}

\twolineshloka
{ततः पार्थस्य विज्ञाय भक्तिं मानववत्सला}
{अन्तरिक्षगतोवाच गोविन्दस्याग्रतः स्थिता}

\uvacha{देव्युवाच}

\twolineshloka
{स्वल्पेनैव तु कालेन शत्रूञ्जेष्यसि पाण्डव}
{नरस्त्वमसि दुर्धर्ष नारायणसहायवान्}

\twolineshloka
{अजेयस्त्वं रणेऽरीणामपि वज्रभृतः स्वयम्}
{इत्येवमुक्त्वा वरदा क्षणेनान्तरधीयत}

\twolineshloka
{लब्ध्वा वरं तु कौन्तेयो मेने विजयमात्मनः}
{आरुरोह ततः पार्थो रथं परमसम्मतम्}

\twolineshloka
{कृष्णार्जुनावेकरथौ दिव्यौ शङ्खौ प्रदध्मतुः}
{य इदं पठते स्तोत्रं कल्य उत्थाय मानवः}

\twolineshloka
{यक्षरक्षःपिशाचेभ्यो न भयं विद्यते सदा}
{न चापि रिपवस्तेभ्यः सर्पाद्या ये च दंष्ट्रिणः}

\twolineshloka
{न भयं विद्यते तस्य सदा राजकुलादपि}
{विवादे जयमाप्नोति बद्धो मुच्यति बन्धनात्}

\twolineshloka
{दुर्गं तरति चावश्यं तथा चोरैर्विमुच्यते}
{सङ्ग्रामे विजयेन्नित्यं लक्ष्मीं प्राप्नोति केवलाम्}

\twolineshloka
{आरोग्यबलसम्पन्नो जीवेद्वर्षशतं तथा}
{एतद्दृष्टं प्रसादात्तु मया व्यासस्य धीमतः}

\twolineshloka
{मोहादेतौ न जानन्ति नरनारायणावृषी}
{तव पुत्रा दुरात्मानः सर्वे मन्युवशानुगाः}


\threelineshloka
{प्राप्तकालमिदं वाक्यं कालपाशेन कुण्ठिताः}
{द्वैपायनो नारदश्च कण्वो रामस्तथाऽनघः}
{अवारयंस्तव सुतं न चासौ तद्गृहीतवान्}

\twolineshloka
{यत्र धर्मो द्युतिः कान्तिर्यत्र ह्रीः श्रीस्तथा मतिः}
{यतो धर्मस्ततः कृष्णो यतः कृष्णस्ततो जयः}

॥इति श्रीमन्महाभरते भीष्मपर्वणि श्रीमद्भगवद्गीतापर्वणि
त्रयोविंशोऽध्यायः॥
