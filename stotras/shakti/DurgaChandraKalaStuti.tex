% !TeX program = XeLaTeX
% !TeX root = ../../shloka.tex

\sect{दुर्गाचन्द्रकलास्तुतिः}

\twolineshloka
{वेधोहरीश्वरस्तुत्यां विहर्त्रीं विन्ध्यभूधरे}
{हरप्राणेश्वरीं वन्दे हन्त्रीं विबुधविद्विषाम्}%  ॥१॥%  1 \hspace{-3ex}

\fourlineindentedshloka
{अभ्यर्थनेन सरसीरुहसम्भवस्य}
{त्यक्त्वोदिता भगवदक्षिपिधानलीलाम्}
{विश्वेश्वरी विपदपाकरणे पुरस्तात्}
{माता ममास्तु मधुकैटभयोर्निहन्त्री} %॥२॥\hspace{-3ex}


\fourlineindentedshloka
{प्राङ्निर्जरेषु निहितैर्निजशक्तिलेश}
{एकीभवद्भिरुदिताऽखिललोकगुप्त्यै}
{सम्पन्नशस्त्रनिकरा च तदायुधस्थैः}
{माता ममास्तु महिषान्तकरी पुरस्तात्} %॥३॥\hspace{-3ex}

\fourlineindentedshloka
{प्रालेयशैलतनयातनुकान्तिसम्पत}
{कोशोदिता कुवलयच्छविचारुदेहा}
{नारायणी नमदभीप्सितकल्पवल्ली}
{सुप्रीतिमावहतु शुम्भनिशुम्भहन्त्री} %॥४॥\hspace{-3ex}


\fourlineindentedshloka
{विश्वेश्वरीति महिषान्तकरीति यस्य}
{नारायणीत्यपि च नामभिरङ्कितानि}
{सूक्तानि पङ्कजभुवा च सुरर्षिभिश्च}
{दृष्टानि पावकमुखैश्च शिवां भजे ताम्}
%॥५॥\hspace{-3ex}


\fourlineindentedshloka
{उत्पत्तिदैत्यहननस्तवनात्मकानि}
{संरक्षकाण्यखिलभूतहिताय यस्याः}
{सूक्तान्यशेषनिगमान्तविदः पठन्त}
{तां विश्वमातरमजस्रमभिष्टवीमि} %॥६॥\hspace{-3ex}

\fourlineindentedshloka
{ये वैप्रचित्तपुनरुत्थितशुम्भमुख्य}
{दुर्भिक्षघोरसमयेन च कारितासु}
{आविष्कृतास्त्रिजगदार्तिषु रूपभेदाः}
{तैरम्बिका समभिरक्षतु मां विपद्भ्यः} %॥७॥\hspace{-3ex}


\fourlineindentedshloka
{सूक्तं यदीयमरविन्दभवादिदृष्टम्}
{आवर्त्य देव्यनुपदं सुरथः समाधिः}
{द्वावप्यवापतुरभीष्टमनन्यलभ्यं}
{तामादिदेवतरुणीं प्रणमामि मूर्ध्ना} %॥८॥\hspace{-3ex}

\fourlineindentedshloka
{माहिष्मतीतनुभवं च रुरुं चन्तुम्}
{आविष्कृतैर्निजरसादवतारभेदैः}
{अष्टादशाहतनवाहतकोटिसंख्यैः}
{अम्बा सदा समभिरक्षतु मां विपद्भ्यः} %॥९॥\hspace{-3ex}

\fourlineindentedshloka
{एतच्चरित्रमखिलं लिखितं हि ययाः}
{सम्पूजितं सदन एव निवेशितं वा}
{दुर्गं च तारयति दुस्तरमप्यशेषं}
{श्रेयः प्रयच्छति च सर्वमुमां भजे ताम्} %॥१०॥\hspace{-3ex}


\fourlineindentedshloka
{यत्पूजनस्तुतिनमस्कृतिभिर्भवन्त}
{प्रीताः पितामहरमेशहरास्त्रयोऽपि}
{तेषामपि स्वकगुणैर्ददती वपूंषि}
{तामीश्वरस्य तरुणीं शरणं प्रपद्ये} %॥११॥\hspace{-3ex}


\fourlineindentedshloka
{कान्तारमध्यदृढलग्नतयाऽवसन्नाः}
{मग्नाश्च वारिधिजले रिपुभिश्च रुद्धाः}
{यस्याः प्रपद्य चरणौ विपदस्तरन्ति}
{सा मे सदाऽस्तु हृदि सर्वजगत्सवित्री} %॥१२॥\hspace{-3ex}


\fourlineindentedshloka
{बन्धे वधे महति मृत्युभये प्रसक्}
{वित्तक्षये च विविधे य महोपतापे}
{यत्पादपूजनमिह प्रतिकारमाहुः}
{सा मे समस्तजननी शरणं भवानी} %॥१३॥\hspace{-3ex}

\fourlineindentedshloka
{बाणासुरप्रहितपन्नगबन्धमोक्}
{तद्बाहुदर्पदलनादुषया च योगः}
{प्राद्युम्निना द्रुतमलभ्यत यत्प्रसादा}
{सा मे शिवा सकलमप्यशुभं क्षिणोतु} %॥१४॥\hspace{-3ex}

\fourlineindentedshloka
{पापः पुलस्त्यतनयः पुनरुत्थितो म्}
{अद्यापि हर्तुमयमागत इत्युदीतम्}
{यत्सेवनेन भयमिन्दिरयाऽवधूतं}
{तामादिदेवतरुणीं शरणं गतोऽस्मि} %॥१५॥\hspace{-3ex}

\fourlineindentedshloka
{यद् ध्यानजं सुखमवाप्यमनन्तपुण्य}
{साक्षात्तमच्युतपरिग्रहमाश्ववापुः}
{गोपाङ्गनाः किल यदर्चनपुण्यमात्रा}
{सा मे सदा भगवती भवतु प्रसन्ना} %॥१६॥\hspace{-3ex}

\fourlineindentedshloka
{रात्रिं प्रपद्य इति मन्त्रविदः प्रपन्न्}
{उद्बोध्य मृत्यवधिमन्यफलैः प्रलोभ्य}
{बुद्ध्वा च तद्विमुखतां प्रतनं नयन्तीम्}
{आकाशमादिजननीं जगतां भजे ताम्} %॥१७॥\hspace{-3ex}

\twolineshloka
{देशकालेषु दुष्टेषु दुर्गाचन्द्रकलास्तुतिः}
{सन्ध्ययोरनुसन्धेया सर्वापद्विनिवृत्तये}% ॥१८॥% 18}

॥इति श्रीदुर्गाचन्द्रकलास्तुतिः सम्पूर्णा॥
