% !TeX program = XeLaTeX
% !TeX root = ../../shloka.tex
\sect{कल्याणवृष्टिस्तोत्रम्}

\fourlineindentedshloka
{कल्याणवृष्टिभिरिवामृतपूरिताभिर्-}
{लक्ष्मीस्वयंवरणमङ्गलदीपिकाभिः}
{सेवाभिरम्ब तव पादसरोजमूले}
{नाकारि किं मनसि भक्तिमतां जनानाम्}% ॥१॥

\fourlineindentedshloka
{एतावदेव जननि स्पृहणीयमास्ते}
{त्वद्वन्दनेषु सलिलस्थगिते च नेत्रे}
{सान्निध्यमुद्यदरुणायुतसोदरस्य}
{त्वद्विग्रहस्य परया सुधया प्लुतस्य}% ॥२॥

\fourlineindentedshloka
{ईशित्वभावकलुषाः कति नाम सन्ति}
{ब्रह्मादयः प्रतियुगं प्रलयाभिभूताः}
{एकः स एव जननि स्थिरसिद्धिरास्ते}
{यः पादयोस्तव सकृत्प्रणतिं करोति}% ॥३॥

\fourlineindentedshloka
{लब्ध्वा सकृत् त्रिपुरसुन्दरि तावकीनम्}
{कारुण्यकन्दलितकान्तिभरं कटाक्षम्}
{कन्दर्पकोटिसुभगास्त्वयि भक्तिभाजः}
{सम्मोहयन्ति तरुणीर्भुवनत्रयेषु}% ॥४॥

\fourlineindentedshloka
{ह्रीङ्कारमेव तव नाम गृणन्ति वेदा}
{मातस्त्रिकोणनिलये त्रिपुरे त्रिनेत्रे}
{त्वत्संस्मृतौ यमभटादिभयं विहाय}
{दीव्यन्ति नन्दनवने सह लोकपालैः}% ॥५॥

\fourlineindentedshloka
{हन्तुः पुरामधिगलं परिपूर्यमाणः}
{क्रूरः कथं न भविता गरलस्य वेगः}
{आश्वासनाय किल मातरिदं तवार्धम्}
{देहस्य शश्वदमृताप्लुतशीतलस्य}% ॥६॥

\fourlineindentedshloka
{सर्वज्ञतां सदसि वाक्पटुतां प्रसूते}
{देवि त्वदङ्घ्रिसरसीरुहयोः प्रणामः}
{किं च स्फुरन्मकुटमुज्ज्वलमातपत्रं}
{द्वे चामरे च वसुधां महतीं ददाति}% ॥७॥

\fourlineindentedshloka
{कल्पद्रुमैरभिमतप्रतिपादनेषु}
{कारुण्यवारिधिभिरम्ब भवत्कटाक्षैः}
{आलोकय त्रिपुरसुन्दरि मामनाथम्}
{त्वय्येव भक्तिभरितं त्वयि दत्तदृष्टिम्}% ॥८॥

\fourlineindentedshloka
{हन्तेतरेष्वपि मनांसि निधाय चान्ये}
{भक्तिं वहन्ति किल पामरदैवतेषु}
{त्वामेव देवि मनसा वचसा स्मरामि}
{त्वामेव नौमि शरणं जगति त्वमेव}% ॥९॥

\fourlineindentedshloka
{लक्ष्येषु सत्स्वपि तवाक्षिविलोकनानाम्}
{आलोकय त्रिपुरसुन्दरि मां कथञ्चित्}
{नूनं मयाऽपि सदृशं करुणैकपात्रम्}
{जातो जनिष्यति जनो न च जायते च}% ॥१०॥

\fourlineindentedshloka
{ह्रीं ह्रीमिति प्रतिदिनं जपतां जनानाम्}
{किं नाम दुर्लभमिह त्रिपुराधिवासे}
{मालाकिरीटमदवारणमाननीयान्स्-}
{तान्सेवते वसुमती स्वयमेव लक्ष्मीः}% ॥११॥

\fourlineindentedshloka
{सम्पत्कराणि सकलेन्द्रियनन्दनानि}
{साम्राज्यदानकुशलानि सरोरुहाक्षि}
{त्वद्वन्दनानि दुरितौघहरोद्यतानि}
{मामेव मातरनिशं कलयन्तु नान्यम्}% ॥१२॥

\fourlineindentedshloka
{कल्पोपसंहरण-कल्पित-ताण्डवस्य}
{देवस्य खण्डपरशोः परमेश्वरस्य}
{पाशाङ्कुशैक्षव-शरासन-पुष्पबाणा}
{सा साक्षिणी विजयते तव मूर्तिरेका}% ॥१३॥

\fourlineindentedshloka
{लग्नं सदा भवतु मातरिदं तवार्धम्}
{तेजः परं बहुल-कुङ्कुम-पङ्क-शोणम्}
{भास्वत्किरीटममृतांशुकलावतंसम्}
{मध्ये त्रिकोणमुदितं परमामृतार्द्रम्}% ॥१४॥

\fourlineindentedshloka
{ह्रीङ्कारमेव तव नाम तदेव रूपम्}
{त्वन्नाम सुन्दरि सरोजनिवासमूले}
{त्वत्तेजसा परिणतं वियदादि भूतम्}
{सौख्यं तनोति सरसीरुहसम्भवादेः}% ॥१५॥

\fourlineindentedshloka
{ह्रीङ्कारत्रयसम्पुटेन महता मन्त्रेण सन्दीपितम्}
{स्तोत्रं यः प्रतिवासरं तव पुरो मातर्जपेन्मन्त्रवित्}
{तस्य क्षोणिभुजो भवन्ति वशगा लक्ष्मीश्चिरस्थायिनी}
{वाणी निर्मलसूक्तिभारभरिता जागर्ति दीर्घं वयः}% ॥१६॥
 
॥इति श्रीमत्परमहंसपरिव्राजकाचार्यस्य श्री-गोविन्द-भगवत्पूज्य-पाद-शिष्यस्य 
श्रीमच्छङ्करभगवतः कृतौ श्री-कल्याणवृष्टिस्तोत्रं  सम्पूर्णम्॥