% !TeX program = XeLaTeX
% !TeX root = ../../shloka.tex

\sect{दुर्गास्तोत्रम् (युधिष्ठिरकृतम्)}

\uvacha{वैशम्पायन उवाच}

\twolineshloka
{विराटनगरं रम्यं गच्छमानो युधिष्ठिरः}
{अस्तुवन्मनसा देवीं दुर्गां त्रिभुवनेश्वरीम्}% १॥

\twolineshloka
{यशोदागर्भसम्भूतां नारायणवरप्रियाम्}
{नन्दगोपकुले जातां मङ्गल्यां कुलवर्धनीम्}% २॥

\twolineshloka
{कंसविद्रावणकरीमसुराणां क्षयङ्करीम्}
{शिलातटविनिक्षिप्तामाकाशं प्रति गामिनीम्}% ३॥

\twolineshloka
{वासुदेवस्य भगिनीं दिव्यमाल्यविभूषिताम्}
{दिव्याम्बरधरां देवीं खड्गखेटकधारिणीम्}% ४॥

\twolineshloka
{भारावतरणे पुण्ये ये स्मरन्ति सदा शिवाम्}
{तान् वै तारयसे पापात्पङ्के गामिव दुर्बलाम्}% ५॥

\twolineshloka
{स्तोतुं प्रचक्रमे भूयो विविधैः स्तोत्रसम्भवैः}
{आमन्त्र्य दर्शनाकाङ्क्षी राजा देवीं सहानुजः}% ६॥

\twolineshloka
{नमोऽस्तु वरदे कृष्णे कुमारि ब्रह्मचारिणि}
{बालार्कसदृशाकारे पूर्णचन्द्रनिभानने}% ७॥

\twolineshloka
{चतुर्भुजे चतुर्वक्त्रे पीनश्रोणिपयोधरे}
{मयूरपिच्छवलये केयूराङ्गदधारिणि}% ८॥

\twolineshloka
{भासि देवि यथा पद्मा नारायणपरिग्रहः}
{स्वरूपं ब्रह्मचर्यं च विशदं तव खेचरि}% ९॥

\twolineshloka
{कृष्णच्छविसमा कृष्णा सङ्कर्षणसमानना}
{बिभ्रती विपुलौ बाहू शक्रध्वजसमुच्छ्रयौ}% १०॥

\twolineshloka
{पात्री च पङ्कजी घण्टी स्त्री विशुद्धा च या भुवि}
{पाशं धनुर्महाचक्रं विविधान्यायुधानि च}% ११॥

\twolineshloka
{कुण्डलाभ्यां सुपूर्णाभ्यां कर्णाभ्यां च विभूषिता}
{चन्द्रविस्पर्धिना देवि मुखेन त्वं विराजसे}% १२॥

\twolineshloka
{मुकुटेन विचित्रेण केशबन्धेन शोभिना}
{भुजङ्गाभोगवासेन श्रोणिसूत्रेण राजता}% १३॥

\twolineshloka
{विभ्राजसे चाबद्धेन भोगेनेवेह मन्दरः}
{ध्वजेन शिखिपिच्छानामुच्छ्रितेन विराजसे}% १४॥

\twolineshloka
{कौमारं व्रतमास्थाय त्रिदिवं पावितं त्वया}
{तेन त्वं स्तूयसे देवि त्रिदशैः पूज्यसेऽपि च}% १५॥

\twolineshloka
{त्रैलोक्यरक्षणार्थाय महिषासुरनाशिनि}
{प्रसन्ना मे सुरश्रेष्ठे दयां कुरु शिवा भव}% १६॥

\twolineshloka
{जया त्वं विजया चैव सङ्ग्रामे च जयप्रदा}
{ममापि विजयं देहि वरदा त्वं च साम्प्रतम्}% १७॥

\twolineshloka
{विन्ध्ये चैव नगश्रेष्ठे तव स्थानं हि शाश्वतम्}
{कालि कालि महाकालि शीधुमांसपशुप्रिये}% १८॥

\twolineshloka
{कृतानुयात्रा भूतैस्त्वं वरदा कामचारिणी}
{भारावतारे ये च त्वां संस्मरिष्यन्ति मानवाः}% १९॥

\twolineshloka
{प्रणमन्ति च ये त्वां हि प्रभाते तु नरा भुवि}
{न तेषां दुर्लभं किञ्चित्पुत्रतो धनतोऽपि वा}% २०॥

\threelineshloka
{दुर्गात्तारयसे दुर्गे तत् त्वं दुर्गा स्मृता जनैः}
{कान्तारेष्ववसन्नानां मग्नानां च महार्णवे}
{दस्युभिर्वा निरुद्धानां त्वं गतिः परमा नृणाम्}% २१॥

\twolineshloka
{जलप्रतरणे चैव कान्तारेष्वटवीषु च}
{ये स्मरन्ति महादेवि न च सीदन्ति ते नराः}% २२॥

\twolineshloka
{त्वं कीर्तिः श्रीर्धृतिः सिद्धिर्ह्रीर्विद्या सन्ततिर्मतिः}
{सन्ध्या रात्रिः प्रभा निद्रा ज्योत्स्ना कान्तिः क्षमा दया}% २३॥

\twolineshloka
{नृणां च बन्धनं मोहं पुत्रनाशं धनक्षयम्}
{व्याधिं मृत्युं भयं चैव पूजिता नाशयिष्यसि}% २४॥

\twolineshloka
{सोऽहं राज्यात्परिभ्रष्टः शरणं त्वां प्रपन्नवान्}
{प्रणतश्च यथा मूर्ध्ना तव देवि सुरेश्वरि}% २५॥

\twolineshloka
{त्राहि मां पद्मपत्राक्षि सत्ये सत्या भवस्व नः}
{शरणं भव मे दुर्गे शरण्ये भक्तवत्सले}% २६॥

\twolineshloka
{एवं स्तुता हि सा देवी दर्शयामास पाण्डवम्}
{उपगम्य तु राजानमिदं वचनमब्रवीत्}% २७॥

\uvacha{देव्युवाच}

\twolineshloka
{शृणु राजन्महाबाहो मदीयं वचनं प्रभो}
{भविष्यत्यचिरादेव सङ्ग्रामे विजयस्तव}% २८॥

\twolineshloka
{मम प्रसादान्निर्जित्य हत्वा कौरववाहिनीम्}
{राज्यं निष्कण्टकं कृत्वा भोक्ष्यसे मेदिनीं पुनः}% २९॥

\twolineshloka
{भ्रातृभिः सहितो राजन्प्रीतिं प्राप्स्यसि पुष्कलाम्}
{मत्प्रसादाच्च ते सौख्यमारोग्यं च भविष्यति}% ३०॥

\twolineshloka
{ये च सङ्कीर्तयिष्यन्ति लोके विगतकल्मषाः}
{तेषां तुष्टा प्रदास्यामि राज्यमायुर्वपुः सुतम्}% ३१॥

\twolineshloka
{प्रवासे नगरे वाऽपि सङ्ग्रामे शत्रुसङ्कटे}
{अटव्यां दुर्गकान्तारे सागरे गहने गिरौ}% ३२॥

\twolineshloka
{ये स्मरिष्यन्ति मां राजन् यथाऽहं भवता स्मृता}
{न तेषां दुर्लभं किञ्चिदस्मिन् लोके भविष्यति}% ३३॥

\twolineshloka
{इदं स्तोत्रवरं भक्त्या शृणुयाद्वा पठेत वा}
{तस्य सर्वाणि कार्याणि सिद्धिं यास्यन्ति पाण्डवाः}% ३४॥

\twolineshloka
{मत्प्रसादाच्च वः सर्वान्विराटनगरे स्थितान्}
{न प्रज्ञास्यन्ति कुरवो नरा वा तन्निवासिनः}% ३५॥

\twolineshloka
{इत्युक्त्वा वरदा देवी युधिष्ठिरमरिन्दमम्}
{रक्षां कृत्वा च पाण्डूनां तत्रैवान्तरधीयत}% ३६॥

॥इति श्रीमन्महाभारते विराटपर्वणि पाण्डवप्रवेशपर्वणि अष्टमोऽध्यायः॥