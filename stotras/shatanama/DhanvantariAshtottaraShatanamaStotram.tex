% !TeX program = XeLaTeX
% !TeX root = ../../shloka.tex
\sect{धन्वन्तर्यष्टोत्तरशतनामस्तोत्रम्}


\dnsub{ध्यानम्}
\fourlineindentedshloka*
{शङ्खं चक्रं जलौकां दधदमृतघटं चारुदोर्भिश्चतुर्भिः}
{सूक्ष्मस्वच्छातिहृद्यांशुक-परिविलसन्मौलिमम्भोजनेत्रम्}
{कालाम्भोदोज्ज्वलाङ्गं कटितटविलसच्चारूपीताम्बराढ्यम्}
{वन्दे धन्वन्तरिं तं निखिलगदवनप्रौढदावाग्निलीलम्}

\dnsub{स्तोत्रम्}
\twolineshloka
{धन्वन्तरिः सुधापूर्णकलशाढ्यकरो हरिः}
{जरामृतित्रस्तदेवप्रार्थनासाधकः प्रभुः}

\twolineshloka
{निर्विकल्पो निस्समानो मन्दस्मितमुखाम्बुजः}
{आञ्जनेयप्रापिताद्रिः पार्श्वस्थविनतासुतः}

\twolineshloka
{निमग्नमन्दरधरः कूर्मरूपी बृहत्तनुः}
{नीलकुञ्चितकेशान्तः परमाद्भुतरूपधृत्}

\twolineshloka
{कटाक्षवीक्षणाश्वस्तवासुकिः सिंहविक्रमः}
{स्मर्तृहृद्रोगहरणो महाविष्ण्वंशसम्भवः}

\twolineshloka
{प्रेक्षणीयोत्पलश्याम आयुर्वेदाधिदैवतम्}
{भेषजग्रहणानेहस्स्मरणीयपदाम्बुजः}

\twolineshloka
{नवयौवनसम्पन्नः किरीटान्वितमस्तकः}
{नक्रकुण्डलसंशोभिश्रवणद्वयशष्कुलिः}

\twolineshloka
{दीर्घपीवरदोर्दण्डः कम्बुग्रीवोऽम्बुजेक्षणः}
{चतुर्भुजः शङ्खधरश्चक्रहस्तो वरप्रदः}

\twolineshloka
{सुधापात्रोपरिलसदाम्रपत्रलसत्करः}
{शतपद्याढ्यहस्तश्च कस्तूरीतिलकाञ्चितः}

\twolineshloka
{सुकपोलः सुनासश्च सुन्दरभ्रूलताञ्चितः}
{स्वङ्गुलीतलशोभाढ्यो गूढजत्रुर्महाहनुः}

\twolineshloka
{दिव्याङ्गदलसद्बाहुः केयूरपरिशोभितः}
{विचित्ररत्नखचितवलयद्वयशोभितः}

\twolineshloka
{समोल्लसत्सुजातांसश्चाङ्गुलीयविभूषितः}
{सुधागन्धरसास्वादमिलद्भृङ्गमनोहरः}

\twolineshloka
{लक्ष्मीसमर्पितोत्फुल्लकञ्जमालालसद्गलः}
{लक्ष्मीशोभितवक्षस्को वनमालाविराजितः}

\twolineshloka
{नवरत्नमणीकॢप्तहारशोभितकन्धरः}
{हीरनक्षत्रमालादिशोभारञ्जितदिङ्मुखः}

\twolineshloka
{विरजोऽम्बरसंवीतो विशालोराः पृथुश्रवाः}
{निम्ननाभिः सूक्ष्ममध्यः स्थूलजङ्घो निरञ्जनः}

\twolineshloka
{सुलक्षणपदाङ्गुष्ठः सर्वसामुद्रिकान्वितः}
{अलक्तकारक्तपादो मूर्तिमद्वार्धिपूजितः}

\twolineshloka
{सुधार्थान्योन्यसंयुध्यद्देवदैतेयसान्त्वनः}
{कोटिमन्मथसङ्काशः सर्वावयवसुन्दरः}

\twolineshloka
{अमृतास्वादनोद्युक्तदेवसङ्घपरिष्टुतः}
{पुष्पवर्षणसंयुक्तगन्धर्वकुलसेवितः}

\twolineshloka
{शङ्खतूर्यमृदङ्गादिसुवादित्राप्सरोवृतः}
{विष्वक्सेनादियुक्पार्श्वः सनकादिमुनिस्तुतः}

\twolineshloka
{साश्चर्यसस्मितचतुर्मुखनेत्रसमीक्षितः}
{साशङ्कसम्भ्रमदितिदनुवंश्यसमीडितः}

\twolineshloka
{नमनोन्मुखदेवादिमौलीरत्नलसत्पदः}
{दिव्यतेजःपुञ्जरूपः सर्वदेवहितोत्सुकः}

\twolineshloka
{स्वनिर्गमक्षुब्धदुग्धवाराशिर्दुन्दुभिस्वनः}
{गन्धर्वगीतापदानश्रवणोत्कमहामनाः}

\twolineshloka
{निष्किञ्चनजनप्रीतो भवसम्प्राप्तरोगहृत्}
{अन्तर्हितसुधापात्रो महात्मा मायिकाग्रणीः}

\twolineshloka
{क्षणार्धमोहिनीरूपः सर्वस्त्रीशुभलक्षणः}
{मदमत्तेभगमनः सर्वलोकविमोहनः}

\twolineshloka
{स्रंसन्नीवीग्रन्थिबन्धासक्तदिव्यकराङ्गुलिः}
{रत्नदर्वीलसद्धस्तो देवदैत्यविभागकृत्}

\twolineshloka
{सङ्ख्यातदेवतान्यासो दैत्यदानववञ्चकः}
{देवामृतप्रदाता च परिवेषणहृष्टधीः}

\twolineshloka
{उन्मुखोन्मुखदैत्येन्द्रदन्तपङ्कितविभाजकः}
{पुष्पवत्सुविनिर्दिष्टराहुरक्षःशिरोहरः}

\twolineshloka
{राहुकेतुग्रहस्थानपश्चाद्गतिविधायकः}
{अमृतालाभनिर्विण्णयुध्यद्देवारिसूदनः}

\twolineshloka
{गरुत्मद्वाहनारूढः सर्वेशस्तोत्रसंयुतः}
{स्वस्वाधिकारसन्तुष्टशक्रवह्न्यादिपूजितः}

\twolineshloka
{मोहिनीदर्शनायातस्थाणुचित्तविमोहकः}
{शचीस्वाहादिदिक्पालपत्नीमण्डलसन्नुतः}

\twolineshloka
{वेदान्तवेद्यमहिमा सर्वलोकैकरक्षकः}
{राजराजप्रपूज्याङ्घ्रिश्चिन्तितार्थप्रदायकः}

\twolineshloka
{धन्वन्तरेर्भगवतो नाम्नामष्टोत्तरं शतम्}
{यः पठेत्सततं भक्त्या नीरोगः सुखभाग्भवेत्}

{॥इति बृहद्ब्रह्मानन्दोपनिषदान्तर्गतं श्री-धन्वन्तर्यष्टोत्तरशतनामस्तोत्रं सम्पूर्णम्॥}