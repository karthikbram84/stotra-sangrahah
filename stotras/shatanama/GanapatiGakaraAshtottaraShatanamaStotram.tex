% !TeX program = XeLaTeX
% !TeX root = ../../shloka.tex

\sect{गणपति गकार अष्टोत्तरशतनामस्तोत्रम्}

\twolineshloka
{गकाररूपो गम्बीजो गणेशो गणवन्दितः}
{गणनीयो गणोगण्यो गणनातीतसद्गुणः}% १

\twolineshloka
{गगनादिकसृद्गङ्गासुतो गङ्गासुतार्चितः}
{गङ्गाधरप्रीतिकरो गवीशेड्यो गदापहः}% २

\twolineshloka
{गदाधरनुतो गद्यपद्यात्मककवित्वदः}
{गजास्यो गजलक्ष्मीवान् गजवाजिरथप्रदः}% ३

\twolineshloka
{गञ्जानिरतशिक्षाकृद्गणितज्ञो गणोत्तमः}
{गण्डदानाञ्चितो गन्ता गण्डोपलसमाकृतिः}% ४

\twolineshloka
{गगनव्यापको गम्यो गमानादिविवर्जितः}
{गण्डदोषहरो गण्डभ्रमद्भ्रमरकुण्डलः}% ५

\twolineshloka
{गतागतज्ञो गतिदो गतमृत्युर्गतोद्भवः}
{गन्धप्रियो गन्धवाहो गन्धसिन्धूरबृन्दगः}% ६

\twolineshloka
{गन्धादिपूजितो गव्यभोक्ता गर्गादिसन्नुतः}
{गरिष्ठो गरभिद्गर्वहरो गरलिभूषणः}% ७

\twolineshloka
{गविष्ठो गर्जितारावो गभीरहृदयो गदी}
{गलत्कुष्ठहरो गर्भप्रदो गर्भार्भरक्षकः}% ८

\twolineshloka
{गर्भाधारो गर्भवासि-शिशुज्ञान-प्रदायकः}
{गरुत्मत्तुल्यजवनो गरुडध्वजवन्दितः}% ९

\twolineshloka
{गयेडितो गयाश्राद्धफलदश्च गयाकृतिः}
{गदाधरावतारी च गन्धर्वनगरार्चितः}% १०

\twolineshloka
{गन्धर्वगानसन्तुष्टो गरुडाग्रजवन्दितः}
{गणरात्रसमाराध्यो गर्हणस्तुति-साम्यधीः}% ११

\twolineshloka
{गर्ताभनाभिर्गव्यूतिदीर्घतुण्डो गभस्तिमान्}
{गर्हिताचारदूरश्च गरुडोपलभूषितः}% १२

\twolineshloka
{गजारिविक्रमो गन्धमूषवाजी गतश्रमः}
{गवेषणीयो गहनो गहनस्थमुनिस्तुतः}% १३

\twolineshloka
{गवयच्छिद्गण्डकभिद्गह्वरापथवारणः}
{गजदन्तायुधो गर्जद्रिपुघ्नो गजकर्णिकः}% १४

\twolineshloka
{गजचर्मामयच्छेत्ता गणाध्यक्षो गणार्चितः}%
{गणिकानर्तनप्रीतो गच्छन् गन्धफली प्रियः}% १५

\twolineshloka
{गन्धकादिरसाधीशो गणकानन्ददायकः}%
{गरभादिजनुर्हर्ता गण्डकीगाहनोत्सुकः}% १६

\twolineshloka
{गण्डूषीकृतवाराशिः गरिमालघिमादिदः}%
{गवाक्षवत्सौधवासी गर्भितो गर्भिणीनुतः}% १७

\twolineshloka
{गन्धमादनशैलाभो गण्डभेरुण्डविक्रमः}%
{गदितो गद्गदारावसंस्तुतो गह्वरीपतिः}% १८

\twolineshloka
{गजेशो गरीयान् गद्येड्यो गतभीर्गदितागमः}%
{गर्हणीय गुणाभावो गङ्गादिकशुचिप्रदः}% १९

\twolineshloka
{गणनातीत-विद्या-श्री-बलायुष्यादि-दायकः}%
{एवं श्रीगणनाथस्य नाम्नामष्टोत्तरं शतम्}% २०

\twolineshloka
{पठनाच्छ्रवणात् पुंसां श्रेयः प्रेमप्रदायकम्}%
{पूजान्ते यः पठेन्नित्यं प्रीतः सन् तस्य विघ्नराट्}% २१

\twolineshloka
{यं यं कामयते कामं तं तं शीघ्रं प्रयच्छति}%
{दूर्वयाभ्यर्चयन् देवमेकविंशतिवासरान्}% २२

\twolineshloka
{एकविंशतिवारं यो नित्यं स्तोत्रं पठेद्यदि}%
{तस्य प्रसन्नो विघ्नेशः सर्वान् कामान् प्रयच्छति}% २३

॥ इति श्री-गणपति गकार अष्टोत्तरशतनामस्तोत्रं सम्पूर्णम्॥
