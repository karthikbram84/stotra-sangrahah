% !TeX program = XeLaTeX
% !TeX root = ../../shloka.tex

\sect{गौर्यष्टोत्तरशतनामस्तोत्रम्}
\twolineshloka
{गौरी गणेशजननी गिरिराजतनूद्भवा}
{गुहाम्बिका जगन्माता गङ्गाधरकुटुम्बिनी}
\twolineshloka
{वीरभद्रप्रसूर्विश्वव्यापिनी विश्वरूपिणी}
{अष्टमूर्त्यात्मिका कष्टदारिद्र्यशमनी शिवा}
\twolineshloka
{शाम्भवी शङ्करी बाला भवानी भद्रदायिनी}
{माङ्गल्यदायिनी सर्वमङ्गला मञ्जुभाषिणी}
\twolineshloka
{महेश्वरी महामाया मन्त्राराध्या महाबला}
{हेमाद्रिजा हैमवती पार्वती पापनाशिनी}
\twolineshloka
{नारायणांशजा नित्या निरीशा निर्मलाऽम्बिका}
{मृडानी मुनिसंसेव्या मानिनी मेनकात्मजा}
\twolineshloka
{कुमारी कन्यका दुर्गा कलिदोषनिषूदिनी}
{कात्यायनी कृपापूर्णा कल्याणी कमलार्चिता}
\twolineshloka
{सती सर्वमयी चैव सौभाग्यदा सरस्वती}
{अमलाऽमरसंसेव्या अन्नपूर्णाऽमृतेश्वरी}
\twolineshloka
{अखिलागमसंसेव्या सुखसच्चित्सुधारसा}
{बाल्याराधितभूतेशा भानुकोटिसमद्युतिः}
\twolineshloka
{हिरण्मयी परा सूक्ष्मा शीतांशुकृतशेखरा}
{हरिद्राकुङ्कुमाराध्या सर्वकालसुमङ्गली}
\twolineshloka
{सर्वबोधप्रदा सामशिखा वेदान्तलक्षणा}
{कर्मब्रह्ममयी कामकलना काङ्क्षितार्थदा}
\twolineshloka
{चन्द्रार्कायुतताटङ्का चिदम्बरशरीरिणी}
{श्रीचक्रवासिनी देवी कला कामेश्वरप्रिया}
\twolineshloka
{मारारातिप्रियार्धाङ्गी मार्कण्डेयवरप्रदा}
{पुत्रपौत्रप्रदा पुण्या पुरुषार्थप्रदायिनी}
\twolineshloka
{सत्यधर्मरता सर्वसाक्षिणी सर्वरूपिणी}
{श्यामला बगला चण्डी मातृका भगमालिनी}
\twolineshloka
{शूलिनी विरजा स्वाहा स्वधा प्रत्यङ्गिराम्बिका}
{आर्या दाक्षायणी दीक्षा सर्ववस्तूत्तमोत्तमा}
\threelineshloka
{शिवाभिधाना श्रीविद्या प्रणवार्थस्वरूपिणी}
{ह्रीङ्कारी नादरूपा च त्रिपुरा त्रिगुणेश्वरी}
{सुन्दरी स्वर्णगौरी च षोडशाक्षरदेवता}

{॥इति श्री-गौर्यष्टोत्तरशतनामस्तोत्रं सम्पूर्णम्॥}