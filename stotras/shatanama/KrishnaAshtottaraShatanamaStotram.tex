%!TeX program = Xelatex
%!TeX root = ../../shloka.tex
\sect{कृष्णाष्टोत्तरशतनामस्तोत्रम्}
ॐ अस्य श्रीकृष्णाष्टोत्तरशतनामस्तोत्रस्य श्रीशेष ऋषिः।\\
अनुष्टुप्-छन्दः। श्रीकृष्णो देवता।\\
 श्रीकृष्णप्रीत्यर्थे श्री-कृष्णाष्टोत्तरशतनामजपे विनियोगः।

\dnsub{ध्यानम्}
\fourlineindentedshloka*
{शिखिमुकुटविशेषं नीलपद्माङ्गदेशम्}
{विधुमुखकृतकेशं कौस्तुभापीतवेशम्}
{मधुररवकलेशं शं भजे भ्रातृशेषम्}
{व्रजजनवनितेशं माधवं राधिकेशम्}

\uvacha{श्रीशेष उवाच}

\twolineshloka*
{वसुन्धरे वरारोहे जनानामस्ति मुक्तिदम्}
{सर्वमङ्गलमूर्धन्यमणिमाद्यष्टसिद्धिदम्}

\twolineshloka*
{महापातककोटिघ्नं सर्वतीर्थफलप्रदम्}
{समस्तजपयज्ञानां फलदं पापनाशनम्}

\twolineshloka*
{शृणु देवि प्रवक्ष्यामि नाम्नामष्टोत्तर शतम्}
{सहस्रनाम्नां पुण्यानां त्रिरावृत्या तु यत्फलम्}

\twolineshloka*
{एकावृत्या तु कृष्णस्य नामैकं तत्प्रयच्छति}
{तस्मात्पुण्यतरं चैतत्स्तोत्रं पातकनाशनम्}

\twolineshloka*
{नाम्नामष्टोत्तरशतस्याहमेव ऋषिः प्रिये}
{छन्दोऽनुष्टुब्देवता तु योगः कृष्णप्रियावहः}

\dnsub{स्तोत्रम्}
\twolineshloka
{श्रीकृष्णः कमलानाथो वासुदेवः सनातनः}
{वसुदेवात्मजः पुण्यो लीलामानुषविग्रहः}

\twolineshloka
{श्रीवत्सकौस्तुभधरो यशोदावत्सलो हरिः}
{चतुर्भुजात्तचक्रासिगदाशङ्खाम्बुजायुधः}

\twolineshloka
{देवकीनन्दनः श्रीशो नन्दगोपप्रियात्मजः}
{यमुनावेगसंहारी बलभद्रप्रियानुजः}

\twolineshloka
{पूतनाजीवितहरः शकटासुरभञ्जनः}
{नन्दव्रजजनानन्दी सच्चिदानन्दविग्रहः}

\twolineshloka
{नवनीतविलिप्ताङ्गो नवनीतनटोऽनघः}
{नवनीतनवाहारो मुचुकुन्दप्रसादकः}

\twolineshloka
{षोडशस्त्रीसहस्रेशस्त्रिभङ्गी मधुराकृतिः}
{शुकवागमृताब्धीन्दुर्गोविन्दो योगिनां पतिः}

\twolineshloka
{वत्सवाटचरोऽनन्तो धेनुकासुरभञ्जनः}
{तृणीकृततृणावर्तो यमलार्जुनभञ्जनः}

\twolineshloka
{उत्तालतालभेत्ता च तमालश्यामलाकृतिः}
{गोपगोपीश्वरो योगी कोटिसूर्यसमप्रभः}

\twolineshloka
{इलापतिः परञ्ज्योतिर्यादवेन्द्रो यदूद्वहः}
{वनमाली पीतवासाः पारिजातापहारकः}

\twolineshloka
{गोवर्धनाचलोद्धर्ता गोपालः सर्वपालकः}
{अजो निरञ्जनः कामजनकः कञ्जलोचनः}

\twolineshloka
{मधुहा मथुरानाथो द्वारकानायको बली}
{वृन्दावनान्तसञ्चारी तुलसीदामभूषणः}

\twolineshloka
{स्यमन्तकमणेर्हर्ता नरनारायणात्मकः}
{कुब्जाकृष्णाम्बरधरो मायी परमपूरुषः}

\twolineshloka
{मुष्टिकासुरचाणूरमल्लयुद्धविशारदः}
{संसारवैरी कंसारिर्मुरारिर्नरकान्तकः}

\twolineshloka
{अनादिब्रह्मचारी च कृष्णाव्यसनकर्षकः}
{शिशुपालशिरश्छेत्ता दुर्योधनकुलान्तकः}

\twolineshloka
{विदुराक्रूरवरदो विश्वरूपप्रदर्शकः}
{सत्यवाक् सत्यसङ्कल्पः सत्यभामारतो जयी}

\twolineshloka
{सुभद्रापूर्वजो विष्णुर्भीष्ममुक्तिप्रदायकः}
{जगद्गुरुर्जगन्नाथो वेणुनादविशारदः}

\twolineshloka
{वृषभासुरविध्वंसी बाणासुरकरान्तकः}
{युधिष्ठिरप्रतिष्ठाता बर्हिबर्हावतंसकः}

\twolineshloka
{पार्थसारथिरव्यक्तो गीतामृतमहोदधिः}
{कालीयफणिमाणिक्यरञ्जितश्रीपदाम्बुजः}

\twolineshloka
{दामोदरो यज्ञभोक्ता दानवेन्द्रविनाशकः}
{नारायणः परब्रह्म पन्नगाशनवाहनः}

\twolineshloka
{जलक्रीडासमासक्तगोपीवस्त्रापहारकः}
{पुण्यश्लोकस्तीर्थपादो वेदवेद्यो दयानिधिः}

\twolineshloka
{सर्वतीर्थात्मकः सर्वग्रहरूपी परात्परः}
{इत्येवं कृष्णदेवस्य नाम्नामष्टोत्तरं शतम्}

\twolineshloka
{कृष्णेन कृष्णभक्तेन श्रुत्वा गीतामृतं पुरा}
{स्तोत्रं कृष्णप्रियकरं कृतं तस्मान्मया श्रुतम्}

\twolineshloka
{कृष्णप्रेमामृतं नाम परमानन्ददायकम्}
{अत्युपद्रवदुःखघ्नं परमायुष्यवर्धनम्}

\twolineshloka
{दानं व्रतं तपस्तीर्थं यत्कृतं त्विह जन्मनि}
{पठतां शृण्वतां चैव कोटिकोटिगुणं भवेत्}

\twolineshloka
{पुत्त्रप्रदमपुत्त्राणामगतीनां गतिप्रदम्}
{धनावहं दरिद्राणां जयेच्छूनां जयावहम्}

\twolineshloka
{शिशूनां गोकुलानां च पुष्टिदं पुण्यवर्धनम्}
{बालरोगग्रहादीनां शमनं शान्तिकारकम्}

\twolineshloka
{अन्ते कृष्णस्मरणदं भवतापत्रयापहम्}
{असिद्धसाधकं भद्रे जपादिकरमात्मनाम्}

\twolineshloka
{कृष्णाय यादवेन्द्राय ज्ञानमुद्राय योगिने}
{नाथाय रुक्मिणीशाय नमो वेदान्तवेदिने}

\twolineshloka
{इमं मन्त्रं महादेवि जपन्नेव दिवानिशम्}
{सर्वग्रहानुग्रहभाक् सर्वप्रियतमो भवेत्}

\twolineshloka
{पुत्रपौत्रैः परिवृतः सर्वसिद्धिसमृद्धिमान्}
{निषेव्यभोगानन्तेऽपि कृष्णसायुज्यमाप्युनात्}

{॥ इति श्रीब्रह्माण्डे महापुराणे वायुप्रोक्ते मध्यभागे तृतीय उपोद्घातपादे भार्गवचरिते षट्त्रिंशत्तमोऽध्यायान्तर्गत श्रीकृष्णाष्टोत्तरशतनामस्तोत्रं सम्पूर्णम् ॥}