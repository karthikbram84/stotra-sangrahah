% !TeX program = XeLaTeX
% !TeX root = ../../shloka.tex
\sect{सीताष्टोत्तरशतनामस्तोत्रम्}

\dnsub{ध्यानम्}
\fourlineindentedshloka*
{वामाङ्गे रघुनायकस्य रुचिरे या संस्थिता शोभना}
{या विप्राधिपयानरम्यनयना या विप्रपालानना}
{विद्युत्पुञ्जविराजमानवसना भक्तार्तिसङ्खण्डना}
{श्रीमद्राघवपादपद्मयुगलन्यस्तेक्षणा साऽवतु}

\dnsub{स्तोत्रम्}
\twolineshloka
{श्रीसीता जानकी देवी वैदेही राघवप्रिया}
{रमाऽवनिसुता रामा राक्षसान्तप्रकारिणी}

\twolineshloka
{रत्नगुप्ता मातुलुङ्गी मैथिली भक्ततोषदा}
{पद्माक्षजा कञ्जनेत्रा स्मितास्या नूपुरस्वना}

\twolineshloka
{वैकुण्ठनिलया मा श्रीर्मुक्तिदा कामपूरणी}
{नृपात्मजा हेमवर्णा मृदुलाङ्गी सुभाषिणी}

\twolineshloka
{कुशाम्बिका दिव्यदा च लवमाता मनोहरा}
{हनुमद्वन्दितपदा मुग्धा केयूरधारिणी}

\twolineshloka
{अशोकवनमध्यस्था रावणादिकमोहिनी}
{विमानसंस्थिता सुभ्रूः सुकेशी रशनान्विता}

\twolineshloka
{रजोरूपा सत्त्वरूपा तामसी वह्निवासिनी}
{हेममृगासक्तचित्ता वाल्मीक्याश्रमवासिनी}

\twolineshloka
{पतिव्रता महामाया पीतकौशेयवासिनी}
{मृगनेत्रा च बिम्बोष्ठी धनुर्विद्याविशारदा}

\twolineshloka
{सौम्यरूपा दशरथस्नुषा चामरवीजिता}
{सुमेधादुहिता दिव्यरूपा त्रैलोक्यपालिनी}

\twolineshloka
{अन्नपूर्णा महालक्ष्मीर्धीर्लज्जा च सरस्वती}
{शान्तिः पुष्टिः क्षमा गौरी प्रभाऽयोध्यानिवासिनी}

\twolineshloka
{वसन्तशीतला गौरी स्नानसन्तुष्टमानसा}
{रमानामभद्रसंस्था हेमकुम्भपयोधरा}

\twolineshloka
{सुरार्चिता धृतिः कान्तिः स्मृतिर्मेधा विभावरी}
{लघूदरा वरारोहा हेमकङ्कणमण्डिता}

\twolineshloka
{द्विजपत्न्यर्पितनिजभूषा राघवतोषिणी}
{श्रीरामसेवानिरता रत्नताटङ्कधारिणी}

\twolineshloka
{रामवामाङ्गसंस्था च रामचन्द्रैकरञ्जनी}
{सरयूजलसङ्क्रीडाकारिणी राममोहिनी}

\twolineshloka
{सुवर्णतुलिता पुण्या पुण्यकीर्तिः कलावती}
{कलकण्ठा कम्बुकण्ठा रम्भोरुर्गजगामिनी}

\twolineshloka
{रामार्पितमना रामवन्दिता रामवल्लभा}
{श्रीरामपदचिह्नाङ्का रामरामेतिभाषिणी}

\twolineshloka
{रामपर्यङ्कशयना रामाङ्घ्रिक्षालिनी वरा}
{कामधेन्वन्नसन्तुष्टा मातुलुङ्गकरे धृता}

\twolineshloka
{दिव्यचन्दनसंस्था श्रीर्मूलकासुरमर्दिनी}
{एवमष्टोत्तरशतं सीतानाम्नां सुपुण्यदम्}

\dnsub{फलश्रुतिः}
\twolineshloka
{ये पठन्ति नरा भूम्यां ते धन्याः स्वर्गगामिनः}
{अष्टोत्तरशतं नाम्नां सीतायाः स्तोत्रमुत्तमम्}

\twolineshloka
{जपनीयं प्रयत्नेन सर्वदा भक्तिपूर्वकम्}
{सन्ति स्तोत्राण्यनेकानि पुण्यदानि महान्ति च}

\twolineshloka
{नानेन सदृशानीह तानि सर्वाणि भूसुर}
{स्तोत्राणामुत्तमं चेदं भुक्तिमुक्तिप्रदं नृणाम्}

\twolineshloka
{एवं सुतीक्ष्ण ते प्रोक्तमष्टोत्तरशतं शुभम्}
{सीतानाम्नां पुण्यदं च श्रवणान्मङ्गलप्रदम्}

\twolineshloka
{नरैः प्रातः समुत्थाय पठितव्यं प्रयत्नतः}
{सीतापूजनकालेऽपि सर्ववाञ्छितदायकम्}

{॥इति श्री-आनन्दरामायणे श्रीसीताष्टोत्तरशतनामस्तोत्रं सम्पूर्णम्॥}

