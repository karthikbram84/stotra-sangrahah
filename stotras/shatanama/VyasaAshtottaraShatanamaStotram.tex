\sect{॥व्यासाष्टोत्तरशतनामस्तोत्रम्॥}

\twolineshloka
{नारायणकुलोद्भूतो नारायणपरो वरः}
{नारायणावतारश्च नारायणवशंवदः}

\twolineshloka
{स्वयम्भूवंशसम्भूतो वसिष्ठकुलदीपकः}
{शक्तिपौत्रः पापहन्ता पराशरसुतोऽमलः}

\twolineshloka
{द्वैपायनो मातृभक्तः शिष्टः सत्यवतीसुतः}
{स्वयमुद्भूतवेदश्च चतुर्वेदविभागकृत्}

\twolineshloka
{महाभारतकर्ता च ब्रह्मसूत्रप्रजापतिः}
{अष्टादशपुराणानां कर्ता श्यामः प्रशिष्यकः}

\twolineshloka
{शुकतातः पिङ्गजटः प्रांशुर्दण्डी मृगाजिनः}
{वश्यवाग् ज्ञानदाता च शङ्करायुःप्रदः शुचिः}

\twolineshloka
{मातृवाक्यकरो धर्मी कर्मी तत्त्वार्थदर्शकः}
{सञ्जयज्ञानदाता च प्रतिस्मृत्युपदेशकः}

\twolineshloka
{सर्वधर्मोपदेष्टा च मृतदर्शनपण्डितः}
{विचक्षणः प्रहृष्टात्मा पर्वपूज्यः प्रभुर्मुनिः}

\twolineshloka
{वीरो विश्रुतविज्ञानः प्राज्ञश्चाज्ञाननाशनः}
{ब्राह्मकृत् पाद्मकृद् धीरो विष्णुकृच्छिवकृत् तथा}

\twolineshloka
{श्रीभागवतकर्ता च भविष्यरचनादरः}
{नारदाख्यस्य कर्ता च मार्कण्डेयकरोऽग्निकृत्}

\twolineshloka
{ब्रह्मवैवर्तकर्ता च लिङ्गकृच्च वराहकृत्}
{स्कान्दकर्ता वामनकृत् कूर्मकर्ता च मत्स्यकृत्}

\twolineshloka
{गरुडाख्यस्य कर्ता च ब्रह्माण्डाख्यपुराणकृत्}
{उपपुराणानां कर्ता पुराणः पुरुषोत्तमः}

\twolineshloka
{काशिवासी ब्रह्मनिधिर्गीतादाता महामतिः}
{सर्वज्ञः सर्वसिद्धिश्च सर्वशास्त्रप्रवर्तकः}

\twolineshloka
{सर्वाश्रयः सर्वहितः सर्वः सर्वगुणाश्रयः}
{विशुद्धः शुद्धिकृद् दक्षो विष्णुभक्तः शिवार्चकः}

\twolineshloka
{देवीभक्तः स्कन्दरुचिर्गणेशादृच्च योगवित्}
{पैलाचार्य ऋचः कर्ता शाकल्यार्यश्च याजुषः}

\twolineshloka
{जैमिन्यार्यः सामकर्ता सुमन्त्वार्योऽप्यथर्वकृत्}
{रोमहर्षणसूतार्यो लोकाचार्यो महामुनिः}

\twolineshloka
{व्यासकाशीरतिर्विश्वपूज्यो विश्वेशपूजकः}
{शान्तः शान्ताकृतिः शान्तचित्तः शान्तिप्रदस्तथा}

॥इति व्यासाष्टोत्तरशतनामस्तोत्रं सम्पूर्णम्॥