% !TeX program = XeLaTeX
% !TeX root = ../../shloka.tex

\sect{अर्धनारीश्वर अष्टकम्}

\fourlineindentedshloka
{चाम्पेयगौरार्ध-शरीरकायै}{कर्पूरगौरार्ध-शरीरकाय}
{धम्मिल्लकायै च जटाधराय}{नमः शिवायै च नमः शिवाय}

\fourlineindentedshloka
{कस्तूरिकाकुङ्कुमचर्चितायै}{चितारजःपुञ्जविचर्चिताय}
{कृतस्मरायै विकृतस्मराय}{नमः शिवायै च नमः शिवाय}

\fourlineindentedshloka
{झणत्क्वणत्कङ्कण-नूपुरायै}{पादाब्जराजत्-फणिनूपुराय}
{हेमाङ्गदायै च भुजङ्गदाय}{नमः शिवायै च नमः शिवाय}

\fourlineindentedshloka
{विशालनीलोत्पललोचनायै}{विकासिपङ्केरुहलोचनाय}
{समेक्षणायै विषमेक्षणाय}{नमः शिवायै च नमः शिवाय}

\fourlineindentedshloka
{मन्दारमालाकलितालकायै}{कपालमालाङ्कितकन्धराय}
{दिव्याम्बरायै च दिगम्बराय}{नमः शिवायै च नमः शिवाय}

\fourlineindentedshloka
{अम्भोधरश्यामलकुन्तलायै}{तटित्प्रभाताम्रजटाधराय}
{निरीश्वरायै निखिलेश्वराय}{नमः शिवायै च नमः शिवाय}

\fourlineindentedshloka
{प्रपञ्चसृष्ट्युन्मुखलास्यकायै}{समस्तसंहारकताण्डवाय}
{जगज्जनन्यै जगदेकपित्रे}{नमः शिवायै च नमः शिवाय}

\fourlineindentedshloka
{प्रदीप्तरत्नोज्ज्वलकुण्डलायै}{स्फुरन्महापन्नगभूषणाय}
{शिवान्वितायै च शिवान्विताय}{नमः शिवायै च नमः शिवाय}

\fourlineindentedshloka*
{एतत्पठेदष्टकमिष्टदं यो}{भक्त्या स मान्यो भुवि दीर्घजीवी}
{प्राप्नोति सौभाग्यमनन्तकालम्}{भूयात् सदा तस्य समस्तसिद्धिः}

{॥इति श्रीमत्परमहंसपरिव्राजकाचार्यस्य श्री-गोविन्द-भगवत्पूज्य-पाद-शिष्यस्य 
श्रीमच्छङ्करभगवतः कृतौ श्री-अर्धनारीश्वर अष्टकं सम्पूर्णम्॥}
%
%
%\closesection
%
%\fourlineindentedshloka*
%{वन्दे शम्भुमुमापतिं सुरगुरुं वन्दे जगत्कारणम्}
%{वन्दे पन्नगभूषणं मृगधरं वन्दे पशूनां पतिम्}
%{वन्दे सूर्यशशाङ्कवह्निनयनं वन्दे मुकुन्दप्रियम्}
%{वन्दे भक्तजनाश्रयं च वरदं वन्दे शिवं शङ्करम्}
