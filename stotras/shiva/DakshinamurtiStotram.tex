% !TeX program = XeLaTeX
% !TeX root = ../../shloka.tex

\sect{दक्षिणामूर्तिस्तोत्रम्}
\fourlineindentedshloka
{उपासकानां यदुपासनीयम्}
{उपात्तवासं वटशाखिमूले}
{तद्धाम दाक्षिण्यजुषा स्वमूर्त्या}
{जागर्तु चित्ते मम बोधरूपम्}

\fourlineindentedshloka
{अद्राक्षमक्षीणदयानिधानम्}
{आचार्यमाद्यं वटमूलभागे}
{मौनेन मन्दस्मितभूषितेन}
{महर्षि लोकस्य तमो नुदन्तम्}

\fourlineindentedshloka
{विद्राविताशेष-तमोगुणेन}
{मुद्राविशेषेण मुहुर्मुनीनाम्}
{निरस्य मायां दयया विधत्ते}
{देवो महांस्तत्त्वमसीति बोधम्}

\fourlineindentedshloka
{अपारकारुण्यसुधातरङ्गैः}
{अपाङ्गपातैरवलोकयन्तम्}
{कठोरसंसारनिदाघतप्तान्}
{मुनीनहं नौमि गुरुं गुरूणाम्}

\fourlineindentedshloka
{ममाद्यदेवो वटमूलवासी}
{कृपाविशेषात्कृतसन्निधानः}
{ओङ्काररूपामुपदिश्य विद्याम्}
{आविद्यकध्वान्तमपाकरोतु}

\fourlineindentedshloka
{कलाभिरिन्दोरिव कल्पिताङ्गं}
{मुक्ताकलापैरिव बद्धमूर्तिम्}
{आलोकये देशिकमप्रमेयम्}
{अनाद्यविद्यातिमिरप्रभातम्}

\fourlineindentedshloka
{स्वदक्षजानुस्थितवामपादम्}
{पादोदरालङ्कृतयोगपट्टम्}
{अपस्मृतेराहितपादमङ्गे}
{प्रणौमि देवं प्रणिधानवन्तम्}

\fourlineindentedshloka
{तत्त्वार्थमन्तेवसतामृषीणाम्}
{युवाऽपि यः सन्नुपदेष्टुमीष्टे}
{प्रणौमि तं प्राक्तनपुण्यजालैः}
{आचार्यमाश्चर्यगुणाधिवासम्}

\fourlineindentedshloka
{एकेन मुद्रां परशुं करेण}
{करेण चान्येन मृगं दधानः}
{स्वजानुविन्यस्तकरः पुरस्तात्}
{आचार्यचूडामणिराविरस्तु}

\fourlineindentedshloka
{आलेपवन्तं मदनाङ्गभूत्या}
{शार्दूलकृत्त्या परिधानवन्तम्}
{आलोकये कञ्चनदेशिकेन्द्रम्}
{अज्ञानवाराकरबाडवाग्निम्}

\fourlineindentedshloka
{चारुस्मितं सोमकलावतंसम्}
{वीणाधरं व्यक्तजटाकलापम्}
{उपासते केचन योगिनस्त्वाम्}
{उपात्तनादानुभवप्रमोदम्}

\fourlineindentedshloka
{उपासते यं मुनयः शुकाद्याः}
{निराशिषो निर्ममताधिवासाः}
{तं दक्षिणामूर्तितनुं महेशम्}
{उपास्महे मोहमहार्तिशान्त्यै}

\fourlineindentedshloka
{कान्त्या निन्दितकुन्दकन्दलवपुर्न्यग्रोधमूले वसन्}
{कारुण्यामृतवारिभिर्मुनिजनं सम्भावयन् वीक्षितैः}
{मोहध्वान्तविभेदनं विरचयन् बोधेन तत्तादृशा}
{देवस्तत्त्वमसीति बोधयतु मां मुद्रावता पाणिना}

\fourlineindentedshloka
{अगौरगात्रैरललाटनेत्रैः}
{अशान्तवेषैरभुजङ्गभूषैः}
{अबोधमुद्रैरनपास्तनिद्रैः}
{अपूर्णकामैरमरैरलं नः}

\fourlineindentedshloka
{दैवतानि कति सन्ति चावनौ}
{नैव तानि मनसो मतानि मे}
{दीक्षितं जडधियामनुग्रहे}
{दक्षिणाभिमुखमेव दैवतम्}

\fourlineindentedshloka
{मुदिताय मुग्धशशिनावतंसिने}
{भसितावलेपरमणीयमूर्तये}
{जगदीन्द्रजालरचनापटीयसे}
{महसे नमोऽस्तु वटमूलवासिने}

\fourlineindentedshloka
{व्यालम्बिनीभिः परितो जटाभिः}
{कलावशेषेण कलाधरेण}
{पश्यल्ललाटेन मुखेन्दुना च}
{प्रकाशसे चेतसि निर्मलानाम्}

\fourlineindentedshloka
{उपासकानां त्वमुमासहायः}
{पूर्णेन्दुभावं प्रकटीकरोषि}
{यदद्य ते दर्शनमात्रतो मे}
{द्रवत्यहो मानसचन्द्रकान्तः}

\fourlineindentedshloka
{यस्ते प्रसन्नामनुसन्दधानो}
{मूर्तिं मुदा मुग्धशशाङ्कमौलेः}
{ऐश्वर्यमायुर्लभते च विद्याम्}
{अन्ते च वेदान्तमहारहस्यम्}
॥इति श्रीमच्छङ्कराचार्यविरचितं श्री-दक्षिणामूर्तिस्तोत्रं सम्पूर्णम्॥