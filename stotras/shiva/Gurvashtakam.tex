% !TeX program = XeLaTeX
% !TeX root = ../../shloka.tex

\sect{गुर्वष्टकम्}

\fourlineindentedshloka
{शरीरं सुरूपं तथा वा कलत्रं}
{यशश्चारु चित्रं धनं मेरुतुल्यम्}
{मनश्चेन्न लग्नं गुरोरङ्घ्रिपद्मे}
{ततः किं ततः किं ततः किं ततः किम्}% ||१||
 

 \fourlineindentedshloka
{कलत्रं धनं पुत्रपौत्रादिसर्वं}
{गृहं बान्धवाः सर्वमेतद्धि जातम्}
{मनश्चेन्न लग्नं गुरोरङ्घ्रिपद्मे}
{ततः किं ततः किं ततः किं ततः किम्}% ||२||
 

\fourlineindentedshloka 
{षडङ्गादिवेदो मुखे शास्त्रविद्या}
{कवित्वादि गद्यं सुपद्यं करोति}
{मनश्चेन्न लग्नं गुरोरङ्घ्रिपद्मे}
{ततः किं ततः किं ततः किं ततः किम्}% ||३||
 

\fourlineindentedshloka 
{विदेशेषु मान्यः स्वदेशेषु धन्यः}
{सदाचारवृत्तेषु मत्तो न चान्यः}
{मनश्चेन्न लग्नं गुरोरङ्घ्रिपद्मे}
{ततः किं ततः किं ततः किं ततः किम्}% ||४||
 

\fourlineindentedshloka 
{क्षमामण्डले भूपभूपालबृन्दैः}
{सदा सेवितं यस्य पादारविन्दम्}
{मनश्चेन्न लग्नं गुरोरङ्घ्रिपद्मे}
{ततः किं ततः किं ततः किं ततः किम्}% ||५||
 
\fourlineindentedshloka 
{यशो मे गतं दिक्षु दानप्रतापात्}
{जगद्वस्तुसर्वं करे यत्प्रसादात्}
{मनश्चेन्न लग्नं गुरोरङ्घ्रिपद्मे}
{ततः किं ततः किं ततः किं ततः किम्}% ||६||
 
\fourlineindentedshloka 
{न भोगे न योगे न वा वाजिराजौ}
{न कन्तामुखे नैव वित्तेषु चित्तम्}
{मनश्चेन्न लग्नं गुरोरङ्घ्रिपद्मे}
{ततः किं ततः किं ततः किं ततः किम्}% ||७||
 

\fourlineindentedshloka 
{अरण्ये न वा स्वस्य गेहे न कार्ये}
{न देहे मनो वर्तते मे त्वनर्घ्ये}
{मनश्चेन्न लग्नं गुरोरङ्घ्रिपद्मे}
{ततः किं ततः किं ततः किं ततः किम्}% ||८||
 

\fourlineindentedshloka*
{गुरोरष्टकं यः पठेत्पुण्यदेही}
{यतिर्भूपतिर्ब्रह्मचारी च गेही}
{लभेद्वाञ्छितार्थं पदं ब्रह्मसंज्ञं}
{गुरोरुक्तवाक्ये मनो यस्य लग्नम्}% ||९||
 

॥इति श्रीमत्परमहंसपरिव्राजकाचार्यस्य श्री-गोविन्द-भगवत्पूज्य-पाद-शिष्यस्य 
श्रीमच्छङ्करभगवतः कृतौ श्री-गुर्वष्टकं सम्पूर्णम्॥