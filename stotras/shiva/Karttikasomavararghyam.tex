% !TeX program = XeLaTeX
% !TeX root = ../../shloka.tex

\sect{कार्त्तिकसोमवारार्घ्यम्}

\twolineshloka
{सोमवारे दिवा स्थित्वा निराहारो महेश्वर}
{नक्तं भोक्ष्यामि देवेश अर्पयामि सदाशिव}
\nobreak\hfill{}---साम्बशिवाय नमः इदमर्घ्यम्। (त्रिः)

\twolineshloka
{नक्तं च सोमवारे च सोमनाथ जगत्पते}
{अनन्तकोटिसौभाग्यं अक्षय्यं कुरु शङ्कर }
\nobreak\hfill{}---साम्बशिवाय नमः इदमर्घ्यम्। (त्रिः)

\twolineshloka
{नमः सोमविभूषाय सोमायामिततेजसे}
{इदमर्घ्यं प्रदास्यामि सोमो यच्छतु मे शिवम्}
\nobreak\hfill{}---साम्बशिवाय नमः इदमर्घ्यम्। (त्रिः)

\twolineshloka
{आकाशदिग्शरीराय ग्रहनक्षत्रमालिने}
{इदमर्घ्यं प्रदास्यामि सुप्रीतो वरदो भव}
\nobreak\hfill{}---साम्बशिवाय नमः इदमर्घ्यम्। (त्रिः)


\twolineshloka
{अम्बिकायै नमस्तुभ्यं नमस्ते देवि पार्वति}
{अनघे वरदे देवि गृहाणार्घ्यं प्रसीद मे}
\nobreak\hfill{}---पार्वत्यै नमः इदमर्घ्यम्। (त्रिः)


\twolineshloka
{सुब्रह्मण्य महाभाग कार्त्तिकेय सुरेश्वर}
{इदमर्घ्यं प्रदास्यामि सुप्रीतो वरदो भव}
\nobreak\hfill{}---सुब्रह्मण्याय नमः इदमर्घ्यम्। (त्रिः)

\twolineshloka
{नन्दिकेश महाभाग शिवध्यानपरायण}
{शैलादये नमस्तुभ्यं गृहाणार्घ्यमिदं प्रभो}
\nobreak\hfill{}---नन्दिकेश्वराय नमः इदमर्घ्यम्। (त्रिः)

\twolineshloka
{[नीलकण्ठ-पदाम्भोज-परिस्फुरित-मानस}
{शम्भोः सेवाफलं देहि चण्डेश्वर नमोऽस्तु ते}
\nobreak\hfill{}---नन्दिकेश्वराय नमः इदमर्घ्यम्। (त्रिः)]

