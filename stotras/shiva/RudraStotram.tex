% !TeX program = XeLaTeX
% !TeX root = ../../shloka.tex

\sect{श्रीकृष्णार्जुनकृत-रुद्रस्तोत्रम्}

\uvacha{कृष्णार्जुनावूचतुः}

\addtocounter{shlokacount}{54}

\twolineshloka
{नमो भवाय शर्वाय रुद्राय वरदाय च}
{पशूनां पतये नित्यमुग्राय च कपर्दिने}%५५

\twolineshloka
{महादेवाय भीमाय त्र्यम्बकाय च शान्तये}
{ईशानाय मखघ्नाय नमोऽस्त्वन्धकघातिने}%५६

\twolineshloka
{कुमारगुरवे तुभ्यं नीलग्रीवाय वेधसे}
{पिनाकिने हविष्याय सत्याय विभवे सदा}%५७

\twolineshloka
{विलोहिताय धूम्राय व्याधायानपराजिते}
{नित्यं नीलशिखण्डाय शूलिने दिव्यचक्षुषे}%५८

\twolineshloka
{होत्रे पोत्रे त्रिनेत्राय व्याधाय वसुरेतसे}
{अचिन्त्यायाम्बिकाभर्त्रे सर्वदेवस्तुताय च}%५९

\twolineshloka
{वृषध्वजाय मुण्डाय जटिने ब्रह्मचारिणे}
{तप्यमानाय सलिले ब्रह्मण्यायाजिताय च}%६०

\twolineshloka
{विश्वात्मने विश्वसृजे विश्वमावृत्य तिष्ठते}
{नमो नमस्ते सेव्याय भूतानां प्रभवे सदा}%६१

\twolineshloka
{ब्रह्मवक्त्राय सर्वाय शङ्कराय शिवाय च}
{नमोऽस्तु वाचस्पतये प्रजानां पतये नमः}%६२

\threelineshloka
{अभिगम्याय काम्याय स्तुत्यायार्याय सर्वदा}
{नमोऽस्तु देवदेवाय महाभूतधराय च}
{नमो विश्वस्य पतये पत्तीनां पतये नमः}%६३

\threelineshloka
{नमो विश्वस्य पतये महतां पतये नमः}
{नमः सहस्रशिरसे सहस्रभुजमृत्यवे}
{सहस्रनेत्रपादाय नमोऽसङ्ख्येयकर्मणे}%६४

\twolineshloka
{नमो हिरण्यवर्णाय हिरण्यकवचाय च}
{भक्तानुकम्पिने नित्यं सिध्यतां नो वरः प्रभो}%६५

\uvacha{सञ्जय उवाच}

\twolineshloka
{एवं स्तुत्वा महादेवं वासुदेवः सहार्जुनः}
{प्रसादयामास भवं तदा ह्यस्त्रोपलब्धये}%६६

॥इति श्रीमन्महाभारते द्रोणपर्वणि प्रतिज्ञापर्वणि अशीतितमोऽध्याये श्रीकृष्णार्जुनकृतं रुद्रस्तोत्रं सम्पूर्णम्॥