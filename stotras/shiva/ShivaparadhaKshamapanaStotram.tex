% !TeX program = XeLaTeX
% !TeX root = ../../shloka.tex
\sect{शिवापराधक्षमापन स्तोत्रम्}
\setlength{\shlokaspaceskip}{10pt}
\fourlineindentedshloka
{आदौ कर्मप्रसङ्गात्कलयति कलुषं मातृकुक्षौ स्थितं माम्}
{विण्मूत्रामेध्यमध्ये कथयति नितरां जाठरो जातवेदाः}
{यद्यद्वै तत्र दुःखं व्यथयति नितरां शक्यते केन वक्तुम्}
{क्षन्तव्यो मेऽपराधः शिव शिव शिव भो श्री-महादेव शम्भो} %1

\fourlineindentedshloka
{बाल्ये दुःखातिरेको मललुलितवपुः स्तन्यपाने पिपासा}
{नो शक्तश्चेन्द्रियेभ्यो भवगुणजनिता जन्तवो मां तुदन्ति}
{नानारोगादिदुःखाद्रुदनपरवशः शङ्करं न स्मरामि}
{क्षन्तव्यो मेऽपराधः शिव शिव शिव भो श्री-महादेव शम्भो} %2

\fourlineindentedshloka
{प्रौढोऽहं यौवनस्थो विषयविषधरैः पञ्चभिर्मर्मसन्धौ}
{दष्टो नष्टोऽविवेकः सुतधनयुवतिस्वादुसौख्ये निषण्णः}
{शैवीचिन्ताविहीनं मम हृदयमहो मानगर्वाधिरूढम्}
{क्षन्तव्यो मेऽपराधः शिव शिव शिव भो श्री-महादेव शम्भो} %3

\fourlineindentedshloka
{वार्धक्ये चेन्द्रियाणां विगतगतिमतिश्चाधिदैवादितापैः}
{पापै रोगैर्वियोगैस्त्वनवसितवपुः प्रौढहीनं च दीनम्}
{मिथ्यामोहाभिलाषैर्भ्रमति मम मनो धूर्जटेर्ध्यानशून्यम्}
{क्षन्तव्यो मेऽपराधः शिव शिव शिव भो श्री-महादेव शम्भो} %4

\fourlineindentedshloka
{नो शक्यं स्मार्तकर्म प्रतिपदगहनप्रत्यवायाकुलाख्यम्}
{श्रौते वार्ता कथं मे द्विजकुलविहिते ब्रह्ममार्गेऽसुसारे}
{ज्ञातो धर्मो विचारैः श्रवणमननयोः किं निदिध्यासितव्यम्}
{क्षन्तव्यो मेऽपराधः शिव शिव शिव भो श्री-महादेव शम्भो} %5

\fourlineindentedshloka
{स्नात्वा प्रत्यूषकाले स्नपनविधिविधौ नाहृतं गाङ्गतोयम्}
{पूजार्थं वा कदाचिद्बहुतरगहनात्खण्डबिल्वीदलानि}
{नानीता पद्ममाला सरसि विकसिता गन्धधूपैस्त्वदर्थम्}
{क्षन्तव्यो मेऽपराधः शिव शिव शिव भो श्री-महादेव शम्भो} %6

\fourlineindentedshloka
{दुग्धैर्मध्वाज्युतैर्दधिसितसहितैः स्नापितं नैव लिङ्गम्}
{नो लिप्तं चन्दनाद्यैः कनकविरचितैः पूजितं न प्रसूनैः}
{धूपैः कर्पूरदीपैर्विविधरसयुतैर्नैव भक्ष्योपहारैः}
{क्षन्तव्यो मेऽपराधः शिव शिव शिव भो श्री-महादेव शम्भो} %7

\fourlineindentedshloka
{ध्यात्वा चित्ते शिवाख्यं प्रचुरतरधनं नैव दत्तं द्विजेभ्यो}
{हव्यं ते लक्षसङ्ख्यैर्हुतवहवदने नार्पितं बीजमन्त्रैः}
{नो तप्तं गङ्गातीरे व्रतजननियमै रुद्रजाप्यैर्न वेदैः}
{क्षन्तव्यो मेऽपराधः शिव शिव शिव भो श्री-महादेव शम्भो} %8

\fourlineindentedshloka
{स्थित्वा स्थाने सरोजे प्रणवमयमरुत्कुम्भके सूक्ष्ममार्गे}
{शान्ते स्वान्ते प्रलीने प्रकटितविभवे ज्योतिरूपेऽपराख्ये}
{लिङ्गज्ञे ब्रह्मवाक्ये सकलतनुगतं शङ्करं न स्मरामि}
{क्षन्तव्यो मेऽपराधः शिव शिव शिव भो श्री-महादेव शम्भो} %9


\fourlineindentedshloka
{नग्नो निःसङ्गशुद्धस्त्रिगुणविरहितो ध्वस्तमोहान्धकारो}
{नासाग्रे न्यस्तदृष्टिर्विदितभवगुणो नैव दृष्टः कदाचित्}
{उन्मन्याऽवस्थया त्वां विगतकलिमलं शङ्करं न स्मरामि}
{क्षन्तव्यो मेऽपराधः शिव शिव शिव भो श्री-महादेव शम्भो} %10

\fourlineindentedshloka
{चन्द्रोद्भासितशेखरे स्मरहरे गङ्गाधरे शङ्करे}
{सर्पैर्भूषितकण्ठकर्णयुगले नेत्रोत्थवैश्वानरे}
{दन्तित्वकृतसुन्दराम्बरधरे त्रैलोक्यसारे हरे}
{मोक्षार्थं कुरु चित्तवृत्तिमचलामन्यैस्तु किं कर्मभिः} %11


\fourlineindentedshloka
{किं वाऽनेन धनेन वाजिकरिभिः प्राप्तेन राज्येन किम्}
{किं वा पुत्रकलत्रमित्रपशुभिर्देहेन गेहेन किम्}
{ज्ञात्वैतत्क्षणभङ्गुरं सपदि रे त्याज्यं मनो दूरतः}
{स्वात्मार्थं गुरुवाक्यतो भज मन श्रीपार्वतीवल्लभम्} %12

\fourlineindentedshloka
{आयुर्नश्यति पश्यतां प्रतिदिनं याति क्षयं यौवनम्}
{प्रत्यायान्ति गताः पुनर्न दिवसाः कालो जगद्भक्षकः}
{लक्ष्मीस्तोयतरङ्गभङ्गचपला विद्युच्चलं जीवितम्}
{तस्मात्त्वां शरणागतं शरणद त्वं रक्ष रक्षाधुना} %13


\fourlineindentedshloka
{वन्दे देवमुमापतिं सुरगुरुं वन्दे जगत्कारणम्}
{वन्दे पन्नगभूषणं मृगधरं वन्दे पशूनां पतिम्}
{वन्दे सूर्यशशाङ्कवह्निनयनं वन्दे मुकुन्दप्रियम्}
{वन्दे भक्तजनाश्रयं च वरदं वन्दे शिवं शङ्करम्} %14

\fourlineindentedshloka
{गात्रं भस्मसितं च हसितं हस्ते कपालं सितम्}
{खट्वाङ्गं च सितं सितश्च वृषभः कर्णे सिते कुण्डले}
{गङ्गाफेनसिता जटा पशुपतेश्चन्द्रः सितो मूर्धनि}
{सोऽयं सर्वसितो ददातु विभवं पापक्षयं सर्वदा} %15

\fourlineindentedshloka
{करचरणकृतं वाक्कायजं कर्मजं वा}
{श्रवणनयनजं वा मानसं वाऽपराधम्}
{विहितमविहितं वा सर्वमेतत्क्ष्मस्व}
{जय जय करुणाब्धे श्री-महादेव शम्भो} %16


॥इति श्रीमच्छङ्कराचार्यविरचितं श्री-शिवापराधक्षमापणस्तोत्रं सम्पूर्णम्॥
\setlength{\shlokaspaceskip}{24pt}