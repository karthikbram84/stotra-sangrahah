% !TeX program = XeLaTeX
% !TeX root = ../../shloka.tex

\sect{सुब्रह्मण्यभुजङ्गम्}
\fourlineindentedshloka
{सदा बालरूपाऽपि विघ्नाद्रिहन्त्री}
{महादन्तिवक्त्राऽपि पञ्चास्यमान्या}
{विधीन्द्रादिमृग्या गणेशाभिधा मे}
{विधत्तां श्रियं काऽपि कल्याणमूर्तिः}

\fourlineindentedshloka
{न जानामि शब्दं न जानामि चार्थम्}
{न जानामि पद्यं न जानामि गद्यम्}
{चिदेका षडास्या हृदि द्योतते मे}
{मुखान्निस्सरन्ते गिरश्चापि चित्रम्}

\fourlineindentedshloka
{मयूराधिरूढं महावाक्यगूढम्}
{मनोहारिदेहं महच्चित्तगेहम्}
{महीदेवदेवं महावेदभावम्}
{महादेवबालं भजे लोकपालम्}

\fourlineindentedshloka
{यदा सन्निधानं गता मानवा मे}
{भवाम्भोधिपारं गतास्ते तदैव}
{इति व्यञ्जयन् सिन्धुतीरे य आस्ते}
{तमीडे पवित्रं पराशक्तिपुत्रम्}

\fourlineindentedshloka
{यथाब्धेस्तरङ्गा लयं यान्ति तुङ्गास्-}
{तथैवापदः सन्निधौ सेवतां मे}
{इतीवोर्मिपङ्क्तीर्नृणां दर्शयन्तम्}
{सदा भावये हृत्सरोजे गुहं तम्}

\fourlineindentedshloka
{गिरौ मन्निवासे नरा येऽधिरूढास्-}
{तदा पर्वते राजते तेऽधिरूढाः}
{इतीव ब्रुवन् गन्धशैलाधिरूढः}
{स देवो मुदे मे सदा षण्मुखोऽस्तु}

\fourlineindentedshloka
{महाम्भोधितीरे महापापचोरे}
{मुनीन्द्रानुकूले सुगन्धाख्यशैले}
{गुहायां वसन्तं स्वभासा लसन्तम्}
{जनार्तिं हरन्तं श्रयामो गुहं तम्}

\fourlineindentedshloka
{लसत्स्वर्णगेहे नृणां कामदोहे}
{सुमस्तोमसञ्छन्नमाणिक्यमञ्चे}
{समुद्यत्सहस्रार्कतुल्यप्रकाशम्}
{सदा भावये कार्त्तिकेयं सुरेशम्}

\fourlineindentedshloka
{रणद्धंसके मञ्जुलेऽत्यन्तशोणे}
{मनोहारिलावण्यपीयूषपूर्णे}
{मनःषट्पदो मे भवक्लेशतप्तः}
{सदा मोदतां स्कन्द ते पादपद्मे}

\fourlineindentedshloka
{सुवर्णाभदिव्याम्बरैर्भासमानाम्}
{क्वणत्किङ्किणीमेखलाशोभमानाम्}
{लसद्धेमपट्टेन विद्योतमानाम्}
{कटिं भावये स्कन्द ते दीप्यमानाम्}

\fourlineindentedshloka
{पुलिन्देशकन्याघनाभोगतुङ्ग-}
{स्तनालिङ्गनासक्तकाश्मीररागम्}
{नमस्याम्यहं तारकारे तवोरः}
{स्वभक्तावने सर्वदा सानुरागम्}

\fourlineindentedshloka
{विधौ कॢप्तदण्डान् स्वलीलाधृताण्डान्}
{निरस्तेभशुण्डान् द्विषत्कालदण्डान्}
{हतेन्द्रारिषण्डाञ्जगत्त्राणशौण्डान्}
{सदा ते प्रचण्डान् श्रये बाहुदण्डान्}

\fourlineindentedshloka
{सदा शारदाः षण्मृगाङ्का यदि स्युः}
{समुद्यन्त एव स्थिताश्चेत् समन्तात्}
{सदा पूर्णबिम्बाः कलङ्कैश्च हीनास्-}
{तदा त्वन्मुखानां ब्रुवे स्कन्द साम्यम्}

\fourlineindentedshloka
{स्फुरन्मन्दहासैः सहंसानि चञ्चत्}
{कटाक्षावलीभृङ्गसङ्घोज्ज्वलानि}
{सुधास्यन्दिबिम्बाधराणीशसूनो}
{तवाऽऽलोकये षण्मुखाम्भोरुहाणि}

\fourlineindentedshloka
{विशालेषु कर्णान्तदीर्घेष्वजस्रम्}
{दयास्यन्दिषु द्वादशस्वीक्षणेषु}
{मयीषत् कटाक्षः सकृत् पातितश्चेद्-}
{भवेत् ते दयाशील का नाम हानिः}

\fourlineindentedshloka
{सुताङ्गोद्भवो मेऽसि जीवेति षड्धा}
{जपन् मन्त्रमीशो मुदा जिघ्रते यान्}
{जगद्भारभृद्भ्यो जगन्नाथ तेभ्यः}
{किरीटोज्ज्वलेभ्यो नमो मस्तकेभ्यः}

\fourlineindentedshloka
{स्फुरद्रत्नकेयूरहाराभिरामश्-}
{चलत्कुण्डलश्रीलसद्गण्डभागः}
{कटौ पीतवासाः करे चारुशक्तिः}
{पुरस्तान्ममास्तां पुरारेस्तनूजः}

\fourlineindentedshloka
{इहाऽऽयाहि वत्सेति हस्तान् प्रसार्या-}
{ऽऽह्वयत्यादराच्छङ्करे मातुरङ्कात्}
{समुत्पत्य तातं श्रयन्तं कुमारम्}
{हराश्लिष्टगात्रं भजे बालमूर्तिम्}

\fourlineindentedshloka
{कुमारेशसूनो गुह स्कन्द सेना-}
{पते शक्तिपाणे मयूराधिरूढ}
{पुलिन्दात्मजाकान्त भक्तार्तिहारिन्}
{प्रभो तारकारे सदा रक्ष मां त्वम्}

\fourlineindentedshloka
{प्रशान्तेन्द्रिये नष्टसंज्ञे विचेष्टे}
{कफोद्गारिवक्त्रे भयोत्कम्पिगात्रे}
{प्रयाणोन्मुखे मय्यनाथे तदानीम्}
{द्रुतं मे दयालो भवाग्रे गुह त्वम्}

\fourlineindentedshloka
{कृतान्तस्य दूतेषु चण्डेषु कोपाद्-}
{दहच्छिन्धि भिन्धीति मां तर्जयत्सु}
{मयूरं समारुह्य मा भीरिति त्वम्}
{पुरः शक्तिपाणिर्ममाऽऽयाहि शीघ्रम्}

\fourlineindentedshloka
{प्रणम्यासकृत् पादयोस्ते पतित्वा}
{प्रसाद्य प्रभो प्रार्थयेऽनेकवारम्}
{न वक्तुं क्षमोऽहं तदानीं कृपाब्धे}
{न कार्याऽन्तकाले मनागप्युपेक्षा}

\fourlineindentedshloka
{सहस्राण्डभोक्ता त्वया शूरनामा}
{हतस्तारकः सिंहवक्त्रश्च दैत्यः}
{ममान्तर्हृदिस्थं मनःक्लेशमेकम्}
{न हंसि प्रभो किं करोमि क्व यामि}

\fourlineindentedshloka
{अहं सर्वदा दुःखभारावसन्नो -}
{भवान् दीनबन्धुस्त्वदन्यं न याचे}
{भवद्भक्तिरोधं सदा कॢप्तबाधम्}
{ममाधिं द्रुतं नाशयोमासुत त्वम्}

\fourlineindentedshloka
{अपस्मारकुष्ठक्षयार्शःप्रमेह-}
{ज्वरोन्मादगुल्मादिरोगा महान्तः}
{पिशाचाश्च सर्वे भवत्पत्रभूतिम्}
{विलोक्य क्षणात् तारकारे द्रवन्ते}

\fourlineindentedshloka
{दृशि स्कन्दमूर्तिः श्रुतौ स्कन्दकीर्तिर्-}
{मुखे मे पवित्रं सदा तच्चरित्रम्}
{करे तस्य कृत्यं वपुस्तस्य भृत्यम्}
{गुहे सन्तु लीना ममाशेषभावाः}

\fourlineindentedshloka
{मुनीनामुताहो नृणां भक्तिभाजाम्}
{अभीष्टप्रदाः सन्ति सर्वत्र देवाः}
{नृणामन्त्यजानामपि स्वार्थदाने}
{गुहाद्देवमन्यं न जाने न जाने}

\fourlineindentedshloka
{कलत्रं सुता बन्धुवर्गः पशुर्वा}
{नरो वाऽथ नारी गृहे ये मदीयाः}
{यजन्तो नमन्तः स्तुवन्तो भवन्तम्}
{स्मरन्तश्च ते सन्तु सर्वे कुमार}

\fourlineindentedshloka
{मृगाः पक्षिणो दंशका ये च दुष्टास्-}
{तथा व्याधयो बाधका ये मदङ्गे}
{भवच्छक्तितीक्ष्णाग्रभिन्नाः सुदूरे}
{विनश्यन्तु ते चूर्णितक्रौञ्चशैल}

\fourlineindentedshloka
{जनित्री पिता च स्वपुत्रापराधम्}
{सहेते न किं देवसेनाधिनाथ}
{अहं चातिबालो भवाँल्लोकतातः}
{क्षमस्वापराधं समस्तं महेश}

\fourlineindentedshloka
{नमः केकिने शक्तये चापि तुभ्यम्}
{नमश्छाग तुभ्यं नमः कुक्कुटाय}
{नमः सिन्धवे सिन्धुदेशाय तुभ्यम्}
{पुनः स्कन्दमूर्ते नमस्ते नमोऽस्तु}

\fourlineindentedshloka
{जयाऽऽनन्दभूमन् जयापारधामन्}
{जयामोघकीर्ते जयाऽऽनन्दमूर्ते}
{जयाऽऽनन्दसिन्धो जयाशेषबन्धो}
{जय त्वं सदा मुक्तिदानेशसूनो}

\fourlineindentedshloka
{भुजङ्गाख्यवृत्तेन कॢप्तं स्तवं यः}
{पठेद्भक्तियुक्तो गुहं सम्प्रणम्य}
{स पुत्रान् कलत्रं धनं दीर्घमायुर्-}
{लभेत् स्कन्दसायुज्यमन्ते नरः सः}

॥इति  श्रीमत्परमहंसपरिव्राजकाचार्यस्य श्री-गोविन्द-भगवत्पूज्य-पाद-शिष्यस्य 
श्रीमच्छङ्करभगवतः कृतौ श्री-सुब्रह्मण्यभुजङ्गं सम्पूर्णम्॥
