% !TeX program = XeLaTeX
% !TeX root = ../../shloka.tex

\sect{सुब्रह्मण्यपञ्चरत्नम्}
\fourlineindentedshloka
{षडाननं चन्दनलेपिताङ्गं}
{महोरसं दिव्यमयूरवाहनम्}
{रुद्रस्य सूनुं सुरलोकनाथं}
{ब्रह्मण्यदेवं शरणं प्रपद्ये}

\fourlineindentedshloka
{जाज्वल्यमानं सुरवृन्दवन्द्यं}
{कुमार-धारातट-मन्दिरस्थम्}
{कन्दर्परूपं कमनीयगात्रं}
{ब्रह्मण्यदेवं शरणं प्रपद्ये}

\fourlineindentedshloka
{द्विषड्भुजं द्वादशदिव्यनेत्रं}
{त्रयीतनुं शूलमसीदधानम्}
{शेषावतारं कमनीयरूपं}
{ब्रह्मण्यदेवं शरणं प्रपद्ये}

\fourlineindentedshloka
{सुरारिघोराहवशोभमानं}
{सुरोत्तमं शक्तिधरं कुमारम्}
{सुधार-शक्त्यायुध-शोभिहस्तं}
{ब्रह्मण्यदेवं शरणं प्रपद्ये}

\fourlineindentedshloka
{इष्टार्थसिद्धिप्रदमीशपुत्रं}
{मिष्टान्नदं भूसुरकामधेनुम्}
{गङ्गोद्भवं सर्वजनानुकूलं}
{ब्रह्मण्यदेवं शरणं प्रपद्ये}

\fourlineindentedshloka*
{यः श्लोकपञ्चकमिदं पठतीह भक्त्या}
{ब्रह्मण्यदेव-विनिवेशित-मानसः सन्}
{प्राप्नोति भोगमखिलं भुवि यद्यदिष्टम्}
{अन्ते स गच्छति मुदा गुहसाम्यमेव}
॥इति श्री-सुब्रह्मण्यपञ्चरत्नं सम्पूर्णम्॥

%\closesection
%\bigskip
%\twolineshloka*
%{उमाकोमलहस्ताब्जसम्भावितललाटकम्}
%{हिरण्यकुण्डलं वन्दे कुमारं पुष्करस्रजम्}