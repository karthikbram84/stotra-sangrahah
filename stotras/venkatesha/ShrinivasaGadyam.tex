% !TeX program = XeLaTeX
% !TeX root = ../../shloka.tex

\sect{श्रीनिवास गद्यम्}
\begin{flushleft}
श्रीमदखिल-महीमण्डल-मण्डन-धरणिधर-मण्डलाखण्डलस्य'
निखिल-सुरासुर-वन्दित-वराहक्षेत्र-विभूषणस्य' शेषाचल-\\गरुडाचल-वृषभाचल-नारायणाचलाञ्जनाचलादि शिखरिमालाकुलस्य' नादमुख-बोधनिधि-वीधिगुण-साभरण-सत्त्वनिधि-तत्त्वनिधि-भक्तिगुणपूर्ण-श्रीशैलपूर्ण-गुणवशंवद-परमपुरुष-कृपापूर-विभ्रमदतुङ्गशृङ्ग-गलद्गगनगङ्गासमालिङ्गितस्य' सीमातिग गुण रामानुजमुनि नामाङ्कित बहु भूमाश्रय सुरधामालय
वनरामायत वनसीमापरिवृत विशङ्कटतट निरन्तर विजृम्भित भक्तिरस 
निर्झरानन्तार्याहार्य प्रस्रवणधारापूर विभ्रमद-सलिल\-भरभरित महातटाक मण्डितस्य' कलिकर्दम मलमर्दन कलितोद्यम विलसद्यम
नियमादिम मुनिगणनिषेव्यमाण प्रत्यक्षीभवन्निजसलिल मज्जन
नमज्जन निखिलपापनाशन पापनाशन तीर्थाध्यासितस्य'
मुरारिसेवक जरादिपीडित निरार्तिजीवन निराश भूसुर वरातिसुन्दर सुराङ्गनारति कराङ्गसौष्ठव कुमारताकृति कुमारतारक समापनोदय तनूनपातक महापदामय विहापनोदित सकलभुवन विदित कुमारधाराभिधान-तीर्थाधिष्ठितस्य' धरणितल गत सकल हतकलिल शुभसलिल गतबहुल विविधमल हति चतुर रुचिरतर विलोकनमात्र विदलित विविधमहापातक स्वामिपुष्करिणी समेतस्य' बहुसङ्कट
नरकावट पतदुत्कट कलिकङ्कट कलुषोद्भट जनपातक विनिपातक
रुचिनाटक करहाटक कलशाहृत कमलारत शुभमज्जन जल सज्जन भरित निजदुरित हतिनिरत जनसतत निरर्गलपेपीयमान सलिल
सम्भृत विशङ्कट कटाहतीर्थ विभूषितस्य' एवमादिम भूरिमञ्जिम
सर्वपातक गर्वहातक सिन्धुडम्बर हारिशम्बर विविधविपुल पुण्यतीर्थनिवहनिवासस्य' श्रीमतो वेङ्कटाचलस्य शिखरशेखर-%
महाकल्पशाखी' खर्वीभवदति गर्वीकृत गुरुमेर्वीशगिरि मुखोर्वीधर कुलदर्वीकर दयितोर्वीधर शिखरोर्वी' सतत सदूर्वीकृति चरणघन गर्वचर्वण निपुण तनुकिरणमसृणित गिरिशिखरशेखरतरुनिकर
तिमिरः' वाणीपतिशर्वाणी दयितेन्द्राणीश्वर मुख नाणीयोरसवेणी निभशुभवाणी नुतमहिमाणी' यस्तर कोणी भवदखिलभुवनभवनोदरः' वैमानिकगुरु भूमाधिक गुण रामानुज कृतधामाकर करधामारि दरललामाच्छकनक दामायित निजरामालय' नवकिसलयमय तोरणमालायित वनमालाधरः' कालाम्बुद मालानिभ नीलालक
जालावृत बालाब्ज सलीलामल फालाङ्गसमूलामृत धाराद्वयावधीरण' धीरललिततर विशदतर घन घनसारमयोर्ध्वपुण्ड्ररेखाद्वयरुचिरः' सुविकस्वर दलभास्वर कमलोदर गतमेदुर नवकेसर ततिभासुर परिपिञ्जर कनकाम्बर कलितादर ललितोदर तदालम्ब जम्भरिपु मणिस्तम्भ गम्भीरिमदम्भस्तम्भ समुज्जृम्भमान पीवरोरुयुगल
तदालम्ब पृथुल कदली मुकुल मदहरणजङ्घाल जङ्घायुगलः'
नव्यदल भव्यगल पीतमल शोणिमल सन्मृदुल सत्किसलयाश्रुजल-%
कारि बल शोणतल पदकमल निजाश्रय बलबन्दीकृत शरदिन्दुमण्डली विभ्रमदादभ्र शुभ्र पुनर्भवाधिष्ठिताङ्गुलीगाढ निपीडित पद्मापनः' जानुतलावधि लम्बि विडम्बित वारण शुण्डादण्ड विजृम्भित नीलमणिमय कल्पकशाखा विभ्रमदायि मृणाललतायत समुज्ज्वलतर कनकवलय वेल्लितैकतर बाहुदण्डयुगलः' युगपदुदित कोटि खरकर हिमकर मण्डल जाज्वल्यमान सुदर्शन पाञ्चजन्य समुत्तुङ्गित शृङ्गापर बाहु युगलः' अभिनवशाण समुत्तेजित महामहा नीलखण्ड मतखण्डन निपुण नवीन परितप्त कार्तस्वर कवचित महनीय पृथुल सालग्राम परम्परा गुम्भित नाभिमण्डल पर्यन्त लम्बमान प्रालम्बदीप्ति समालम्बित विशाल वक्षःस्थलः' गङ्गाझर तुङ्गाकृति भङ्गावलि भङ्गावह सौधावलि बाधावह धारानिभ हारावलि दूराहत गेहान्तर मोहावह महिम मसृणित महातिमिरः' पिङ्गाकृति भृङ्गारु निभाङ्गार दलाङ्गामल निष्कासित दुष्कार्यघ निष्कावलि दीपप्रभ नीपच्छवि तापप्रद कनकमालिका पिशङ्गित सर्वाङ्गः' नवदलित दलवलित मृदुललित कमलतति मदविहति चतुरतर पृथुलतर सरसतर कनकसरमय रुचिकण्ठिका कमनीयकण्ठः' वाताशनाधिपति शयन कमन परिचरण रतिसमेताखिल फणधरतति मतिकरकनकमय नागाभरण परिवीताखिलाङ्गावगमित शयन भूताहिराज जातातिशयः' रविकोटी परिपाटी धरकोटी रपताटी कितवाटी रसधाटी धर मणिगणकिरण विसरण सततविधुत तिमिरमोह गर्भगेहः' अपरिमित विविधभुवन भरिताखण्ड ब्रह्माण्डमण्डल पिचण्डिलः' आर्यधुर्यानन्तार्य पवित्र खनित्रपात पात्रीकृत निजचुबुक गतव्रणकिण विभूषणवहनसूचित श्रितजनवत्सलतातिशयः' मड्डुडिण्डिम ढमरु झर्झर काहली पटहावली मृदुमर्द्दलाशि मृदङ्ग दुन्दुभि ढक्किकामुक हृद्य वाद्यक मधुरमङ्गल नादमेदुर विसृमर सरस गानरस रुचिर सन्तत सन्तन्यमान नित्योत्सव पक्षोत्सव मासोत्सव संवत्सरोत्सवादि विविधोत्सव कृतानन्दः' श्रीमदानन्दनिलय विमानवासः' सतत पद्मालया पदपद्मरेणु सञ्चितवक्षःस्थल पटवासः' श्रीश्रीनिवासः' सुप्रसन्नो विजयताम्॥१॥

नाटारभि भूपाल बिलहरि मायामालव गौला असावेरी' सावेरी
शुद्धसावेरी देवगान्धारी' धन्यासी बेगड हिन्दुस्थानी कापी तोडी नाटकुरञ्जी' श्रीराग सहन अठाण सारङ्गी दर्बारु पन्तुवराली वराली' कल्याणी पूर्वीकल्याणी यमुनाकल्याणी हुसेनी जञ्झोटी कौमारी'
कन्नड खरहरप्रिया कलहंस नादनामक्रिया मुखारी' तोडी पुन्नागवराली
काम्भोजी भैरवी' यदुकुलकाम्भोजी आनन्दभैरवी शङ्कराभरण मोहन
रेगुप्ती सौराष्ट्री' नीलाम्बरी गुणक्रिया मेघगर्जनी' हंसध्वनि शोकवराली मध्यमावती जेञ्जुरुटी सुरटी' द्विजावन्ती मलयाम्बरी कापि
परशुधनासरी देशिकतोडी' आहिरी वसन्तगौली सन्तु केदारगौला कनकाङ्गी रत्नाङ्गी गानमूर्ति' वनस्पति वाचस्पति दानवती मानरूपी सेनापति' हनुमत्तोडी धेनुका नाटकप्रिया कोकिलप्रिया रूपवती गायकप्रिया' वकुलाभरण चक्रवाक सूर्यकान्त हाटकाम्बरी
झङ्कारध्वनि' नटभैरवी गीर्वाणी हरिकाम्भोजी धीरशङ्कराभरण नागानन्दिनी यागप्रिया' विसृमर सरस गानरसेत्यादि सन्तत
सन्तन्यमान नित्योत्सव पक्षोत्सव मासोत्सव संवत्सरोत्सवादि
विविधोत्सव कृतानन्दः' श्रीमदानन्दनिलयवासः' सतत पद्मालया पदपद्मरेणु सञ्चितवक्षःस्थल पटवासः' श्रीश्रीनिवासः' सुप्रसन्नो विजयताम्॥२॥

श्री-अलर्मेल्मङ्गासमेत श्रीश्रीनिवास स्वामी' सुप्रीतः सुप्रसन्नो वरदो
भूत्वा' पनस पाटली पालाश बिल्व पुन्नाग चूत कदली चन्दन
चम्पक मञ्जुल मन्दार हिन्तुलादि तिलक मातुलुङ्ग नारिकेल
क्रौञ्चाशोक माधूकामलक हिन्दुक नागकेतक पूर्णकुन्द पूर्ण गन्ध रस
कन्द वन वञ्जुल खर्जूर साल कोविदार हिन्ताल पनस विकट
वैकसवरुण तरुधमरण विचुलङ्काश्वत्थ यक्ष वसुध वर्माध मन्त्रिणी'
तिन्त्रिणी बोध न्यग्रोध घटपटल जम्बूमतल्ली वसति वासती जीवनी
पोषणी प्रमुख निखिल सन्दोह तमाल माला महित विराजमान चषक
मयूर हंस भारद्वाज कोकिल चक्रवाक कपोत गरुड नारायण नानाविध पक्षिजाति समूह ब्रह्म-क्षत्रिय-वैश्य-शूद्र-नानाजात्युद्भव देवता
निर्माण' माणिक्य-वज्र-वैडूर्य-गोमेधिक-पुष्यराग-पद्मरागेन्द्र प्रवालमौक्तिक-स्फटिक-हेम-रत्नखचित धगद्धगायमान रथगज
तुरग पदादि सेवा समूह' भेरी-मद्दल-मुरवक-झल्लरी-शङ्ख-काहल नृत्यगीत-तालवाद्य-कुम्भवाद्य-पञ्चमुखवाद्य अहमीमार्गन्नटीवाद्य किटिकुन्तलवाद्य सुरटीचौण्डोवाद्य तिमिलकवितालवाद्य
तक्कराग्रवाद्य घण्टाताडन ब्रह्मताल समताल कोट्टरीताल ढक्करीताल ऎक्काल' धारावाद्य पटह कांस्यवाद्य भरतनाट्यालङ्कार किन्नर किम्पुरुष
रुद्रवीणा मुखवीणा वायुवीणा' तुम्बुरुवीणा गान्धर्ववीणा नारदवीणा' स्वरमण्डल रावणहस्तवीणास्तक्रियालङ्क्रियालङ्कृतानेक-\\विधवाद्य वापीकूपतटाकादि गङ्गा यमुना रेवा वरुणा शोणनदी शोभनदी'
सुवर्णमुखी वेगवती वेत्रवती क्षीरनदी बाहुनदी गरुडनदी कावेरी ताम्रपर्णी प्रमुखा महापुण्यनद्यः' सजलतीर्थैः सहोभयकूलङ्गत सदाप्रवाह ऋग्यजुःसामाथर्वण वेदशास्त्रेतिहासपुराण-सकलविद्याघोष भानुकोटिप्रकाश चन्द्रकोटिसमान नित्यकल्याण परम्परोत्तरोत्तराभिवृद्धिर्भूयादिति' भवन्तो महान्तोऽनुगृह्णन्तु। 
ब्रह्मण्यो राजा धार्मिकोऽस्तु। देशोऽयं निरुपद्रवोऽस्तु।
सर्वे साधुजनाः सुखिनो विलसन्तु। समस्तसन्मङ्गलानि सन्तु। उत्तरोत्तराभिवृद्धिरस्तु। सकलकल्याणसमृद्धिरस्तु॥३॥
\centerline{॥हरिः ॐ॥}
\centerline{॥इति श्री-श्रीशैलरङ्गाचार्यविरचितं श्री-श्रीनिवासगद्यं सम्पूर्णम्॥}
\end{flushleft}