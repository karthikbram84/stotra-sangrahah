% !TeX program = XeLaTeX
% !TeX root = ../../shloka.tex
\sect{वेङ्कटेश करावलम्बस्तोत्रम्}
\fourlineindentedshloka
{श्रीशेषशैल-सुनिकेतन दिव्यमूर्ते}
{नारायणाच्युत हरे नलिनायताक्ष}
{लीलाकटाक्ष-परिरक्षित-सर्वलोक}
{श्रीवेङ्कटेश मम देहि करावलम्बम्}

\fourlineindentedshloka
{ब्रह्मादिवन्दितपदाम्बुज शङ्खपाणे}
{श्रीमत्सुदर्शन-सुशोभित-दिव्यहस्त}
{कारुण्यसागर शरण्य सुपुण्यमूर्ते}
{श्रीवेङ्कटेश मम देहि करावलम्बम्}

\fourlineindentedshloka
{वेदान्त-वेद्य भवसागर-कर्णधार}
{श्रीपद्मनाभ कमलार्चितपादपद्म}
{लोकैक-पावन परात्पर पापहारिन्}
{श्रीवेङ्कटेश मम देहि करावलम्बम्}

\fourlineindentedshloka
{लक्ष्मीपते निगमलक्ष्य निजस्वरूप}
{कामादिदोष-परिहारक बोधदायिन्}
{दैत्यादिमर्दन जनार्दन वासुदेव}
{श्रीवेङ्कटेश मम देहि करावलम्बम्}

\fourlineindentedshloka
{तापत्रयं हर विभो रभसा मुरारे}
{संरक्ष मां करुणया सरसीरुहाक्ष}
{मच्छिष्य इत्यनुदिनं परिरक्ष विष्णो}
{श्रीवेङ्कटेश मम देहि करावलम्बम्}

\fourlineindentedshloka
{श्री-जातरूपनवरत्न-लसत्किरीट}
{कस्तूरिकातिलकशोभिललाटदेश}
{राकेन्दुबिम्ब-वदनाम्बुज वारिजाक्ष}
{श्रीवेङ्कटेश मम देहि करावलम्बम्}

\fourlineindentedshloka
{वन्दारुलोक-वरदान-वचोविलास}
{रत्नाढ्यहार-परिशोभित-कम्बुकण्ठ}
{केयूररत्न-सुविभासि-दिगन्तराल}
{श्रीवेङ्कटेश मम देहि करावलम्बम्}

\fourlineindentedshloka
{दिव्याङ्गदाञ्चित-भुजद्वय मङ्गलात्मन्}
{केयूरभूषण-सुशोभित-दीर्घबाहो}
{नागेन्द्र-कङ्कण-करद्वय कामदायिन्}
{श्रीवेङ्कटेश मम देहि करावलम्बम्}

\fourlineindentedshloka
{स्वामिन् जगद्धरणवारिधिमध्यमग्नम्}
{मामुद्धराद्य कृपया करुणापयोधे}
{लक्ष्मीं च देहि मम धर्म-समृद्धिहेतुम्}
{श्रीवेङ्कटेश मम देहि करावलम्बम्}

\fourlineindentedshloka
{दिव्याङ्गरागपरिचर्चित-कोमलाङ्ग}
{पीताम्बरावृततनो तरुणार्क-दीप्ते}
{सत्काञ्चनाभ-परिधान-सुपट्टबन्ध}
{श्रीवेङ्कटेश मम देहि करावलम्बम्}

\fourlineindentedshloka
{रत्नाढ्यदाम-सुनिबद्ध-कटि-प्रदेश}
{माणिक्यदर्पण-सुसन्निभ-जानुदेश}
{जङ्घाद्वयेन परिमोहित सर्वलोक}
{श्रीवेङ्कटेश मम देहि करावलम्बम्}

\fourlineindentedshloka
{लोकैकपावन-सरित्परिशोभिताङ्घ्रे}
{त्वत्पाददर्शन दिने च ममाघमीश}
{हार्दं तमश्च सकलं लयमाप भूमन्}
{श्रीवेङ्कटेश मम देहि करावलम्बम्}

\fourlineindentedshloka
{कामादि-वैरि-निवहोऽच्युत मे प्रयातः}
{दारिद्र्यमप्यपगतं सकलं दयालो}
{दीनं च मां समवलोक्य दयार्द्र-दृष्ट्या}
{श्रीवेङ्कटेश मम देहि करावलम्बम्}

\fourlineindentedshloka
{श्रीवेङ्कटेश-पदपङ्कज-षट्पदेन}
{श्रीमन्नृसिंहयतिना रचितं जगत्याम्}
{ये तत्पठन्ति मनुजाः पुरुषोत्तमस्य}
{ते प्राप्नुवन्ति परमां पदवीं मुरारेः}
॥इति~श्री-शृङ्गेरि-जगद्गुरुणा~श्री-नृसिंहभारती-स्वामिना रचितं श्री-वेङ्कटेश-करावलम्बस्तोत्रं सम्पूर्णम्॥