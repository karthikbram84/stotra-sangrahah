% !TeX program = XeLaTeX
% !TeX root = ../../shloka.tex

\sect{वेङ्कटेश स्तोत्रम्}
\twolineshloka
{कमलाकुच-चूचुक-कुङ्कुमतो नियतारुणितातुल-नीलतनो}
{कमलायतलोचन लोकपते विजयी भव वेङ्कटशैलपते}

\twolineshloka
{सचतुर्मुख-षण्मुख-पञ्चमुख-प्रमुखाखिलदैवतमौलिमणे}
{शरणागतवत्सल सारनिधे परिपालय मां वृषशैलपते}

\twolineshloka
{अतिवेलतया तव दुर्विषहैरनुवेलकृतैरपराधशतैः}
{भरितं त्वरितं वृषशैलपते परया कृपया परिपाहि हरे}

\twolineshloka
{अधिवेङ्कटशैलमुदारमते जनताभिमताधिकदानरतात्}
{परदेवतया गदितान्निगमैः कमलादयितान्न परं कलये}

\twolineshloka
{कलवेणुरवावशगोपवधू शतकोटिवृतात्स्मरकोटिसमात्}
{प्रतिवल्लविकाभिमतात्सुखदाद् वसुदेवसुतान्न परं कलये}

\twolineshloka
{अभिरामगुणाकर दाशरथे जगदेकधनुर्धर धीरमते}
{रघुनायक राम रमेश विभो वरदो भव देव दयाजलधे}

\twolineshloka
{अवनीतनया-कमनीयकरं रजनीकरचारुमुखाम्बुरुहम्}
{रजनीचरराजतमोमिहिरं महनीयमहं रघुराम मये}

\twolineshloka
{सुमुखं सुहृदं सुलभं सुखदं स्वनुजं च सुखायममोघशरम्}
{अपहाय रघूद्वहमन्यमहं न कथञ्चन कञ्चन जातु भजे}

\fourlineindentedshloka
{विना वेङ्कटेशं न नाथो न नाथः}
{सदा वेङ्कटेशं स्मरामि स्मरामि}
{हरे वेङ्कटेश प्रसीद प्रसीद}
{प्रियं वेङ्कटेश प्रयच्छ प्रयच्छ}% (एवं त्रिः)

\fourlineindentedshloka
{अहं दूरतस्ते पदाम्भोजयुग्म}
{प्रणामेच्छयाऽऽगत्य सेवां करोमि}
{सकृत्सेवया नित्यसेवाफलं त्वम्}
{प्रयच्छ प्रयच्छ प्रभो वेङ्कटेश}

\twolineshloka
{अज्ञानिना मया दोषानशेषान् विहितान् हरे}
{क्षमस्व त्वं क्षमस्व त्वं शेषशैल-शिखामणे}
॥इति श्री-वेङ्कटेश-स्तोत्रं सम्पूर्णम्॥
