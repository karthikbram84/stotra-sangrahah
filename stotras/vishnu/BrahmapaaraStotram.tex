% !TeX program = XeLaTeX
% !TeX root = ../../shloka.tex

\sect{ब्रह्मपारस्तोत्रम्}

\addtocounter{shlokacount}{53}

\uvacha{प्रचेतस ऊचुः}
\twolineshloka
{ब्रह्मपारं मुने श्रोतुमिच्छामः परमं स्तवम्}
{जपता कण्डुना देवो येनाऽऽराध्यत केशवः} %॥५४॥

\uvacha{सोम उवाच}
\fourlineindentedshloka
{पारं परं विष्णुरपारपारः}
{परः परेभ्यः परमार्थरूपी}
{स ब्रह्मपारः परपारभूतः}
{परः पराणामपि पारपारः} %॥५५॥

\fourlineindentedshloka
{स कारणं कारणतस्ततोऽपि}
{तस्यापि हेतुः परहेतुहेतुः}
{कार्येषु चैवं सह कर्मकर्तृ-}
{रूपैरशेषैरवतीह सर्वम्} %॥५६॥

\fourlineindentedshloka
{ब्रह्म प्रभुर्ब्रह्म स सर्वभूतो}
{ब्रह्म प्रजानां पतिरच्युतोऽसौ} 
{ब्रह्माव्ययं नित्यमजं स विष्णुः}
{अपक्षयाद्यैरखिलैरसङ्गिः} %॥५७॥

\twolineshloka
{ब्रह्माक्षरमजं नित्यं यथाऽसौ पुरुषोत्तमः}
{तथा रागादयो दोषाः प्रयान्तु प्रशमं मम} %॥५८॥


\uvacha{सोम उवाच}
\twolineshloka
{एतद्ब्रह्म पराख्यं वै संस्तवं परमं जपन्}
{अवाप परमां सिद्धिं समाराध्य स केशवम्} %॥५९॥

\twolineshloka*
{इमं स्तवं यः पठति शृणुयाद्वाऽपि नित्यशः}
{स कामदोषैरखिलैर्मुक्तः प्राप्नोति वाञ्छितम्} %॥६०॥

॥इति श्रीविष्णुपुराणे प्रथमेंऽशे पञ्चदशोऽध्याये ब्रह्मपारस्तोत्रं सम्पूर्णम्॥ 