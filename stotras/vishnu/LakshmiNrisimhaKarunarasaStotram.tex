% !TeX program = XeLaTeX
% !TeX root = ../../shloka.tex

\sect{लक्ष्मी-नृसिंह-करुणारस-स्तोत्रं}


\fourlineindentedshloka
{श्रीमत्-पयोनिधि-निकेतन चक्रपाणे}
{भोगीन्द्र-भोग-मणि-राजित-पुण्य-मूर्ते}
{योगीश शाश्वत शरण्य भवाब्धि-पोत}
{लक्ष्मी-नृसिंह मम देहि करावलम्बम्}

\fourlineindentedshloka
{ब्रह्मेन्द्र-रुद्र-मरुदर्क-किरीट-कोटि-}
{सङ्घट्टिताङ्घ्रि-कमलामल-कान्ति-कान्त}
{लक्ष्मी-लसत्-कुच-सरोरुह-राजहंस}
{लक्ष्मी-नृसिंह मम देहि करावलम्बम्}

\fourlineindentedshloka
{संसार-दाव-दहनाकर-भी-करोरु-}
{ज्वालावलीभिरतिदग्ध-तनूरुहस्य}
{त्वत्-पाद-पद्म-सरसी-शरणागतस्य}
{लक्ष्मी-नृसिंह मम देहि करावलम्बम्}

\fourlineindentedshloka
{संसार-जाल-पतितस्य जगन्निवास}
{सर्वेन्द्रियार्थ-बडिशाग्र-झषोपमस्य}
{प्रोत्कम्पित-प्रचुर-तालुक-मस्तकस्य}
{लक्ष्मी-नृसिंह मम देहि करावलम्बम्}

\fourlineindentedshloka
{संसार-कूपमतिघोरमगाध-मूलं}
{सम्प्राप्य दुःख-शत-सर्प-समाकुलस्य}
{दीनस्य देव कृपया पदमागतस्य}
{लक्ष्मी-नृसिंह मम देहि करावलम्बम्}

\fourlineindentedshloka
{संसार-भीकर-करीन्द्र-कराभिघात-}
{निष्पीड्यमान-वपुषः सकलार्ति-नाश}
{प्राण-प्रयाण-भव-भीति-समाकुलस्य}
{लक्ष्मी-नृसिंह मम देहि करावलम्बम्}

\fourlineindentedshloka
{संसार-सर्प-विष-दिग्ध-महोग्र-तीव्र-}
{दंष्ट्राग्र-कोटि-परिदष्ट-विनष्ट-मूर्तेः}
{नागारि-वाहन सुधाब्धि-निवास शौरे}
{लक्ष्मी-नृसिंह मम देहि करावलम्बम्}

\fourlineindentedshloka
{संसार-वृक्षमघ-बीजमनन्त-कर्म-}
{शाखा-युतं करण-पत्रमनङ्ग-पुष्पम्}
{आरुह्य दुःख-फलितं पततं दयालो}
{लक्ष्मी-नृसिंह मम देहि करावलम्बम्}

\fourlineindentedshloka
{संसार-सागर-विशाल-कराल-काल-}
{नक्र-ग्रह-ग्रसित-निग्रह-विग्रहस्य}
{व्यग्रस्य राग-निचयोर्मि-निपीडितस्य}
{लक्ष्मी-नृसिंह मम देहि करावलम्बम्}

\fourlineindentedshloka
{संसार-सागर-निमज्जनमुह्यमानम्}
{दीनं विलोकय विभो करुणा-निधे माम्}
{प्रह्लाद-खेद-परिहार-परावतार}
{लक्ष्मी-नृसिंह मम देहि करावलम्बम्}

\fourlineindentedshloka
{संसार-घोर-गहने चरतो मुरारे}
{मारोग्र-भीकर-मृग-प्रचुरार्दितस्य}
{आर्तस्य मत्सर-निदाघ-सुदुःखितस्य}
{लक्ष्मी-नृसिंह मम देहि करावलम्बम्}

\fourlineindentedshloka
{बद्\mbox{}ध्वा गले यम-भटा बहु तर्जयन्तः}
{कर्षन्ति यत्र भव-पाश-शतैर्युतं माम्}
{एकाकिनं परवशं चकितं दयालो}
{लक्ष्मी-नृसिंह मम देहि करावलम्बम्}

\fourlineindentedshloka
{लक्ष्मीपते कमलनाभ सुरेश विष्णो}
{यज्ञेश यज्ञ मधुसूदन विश्वरूप}
{ब्रह्मण्य केशव जनार्दन वासुदेव}
{लक्ष्मी-नृसिंह मम देहि करावलम्बम्}

\fourlineindentedshloka
{एकेन चक्रमपरेण करेण शङ्खम्}
{अन्येन सिन्धु-तनयाम् अवलम्ब्य तिष्ठन्}
{वामेतरेण वरदाभय-पद्म-चिह्नं}
{लक्ष्मी-नृसिंह मम देहि करावलम्बम्}

\fourlineindentedshloka
{अन्धस्य मे हृत-विवेक-महाधनस्य}
{चोरैर्-महाबलिभिरिन्द्रिय-नामधेयैः}
{मोहान्धकार-कुहरे विनिपातितस्य}
{लक्ष्मी-नृसिंह मम देहि करावलम्बम्}

\fourlineindentedshloka
{प्रह्लाद-नारद-पराशर-पुण्डरीक-}
{व्यासादि-भागवत-पुङ्गव-हृन्निवास}
{भक्तानुरक्त-परिपालन-पारिजात}
{लक्ष्मी-नृसिंह मम देहि करावलम्बम्}

\fourlineindentedshloka
{लक्ष्मी-नृसिंह-चरणाब्ज-मधुव्रतेन}
{स्तोत्रं कृतं शुभकरं भुवि शङ्करेण}
{ये तत् पठन्ति मनुजा हरि-भक्ति-युक्ताः}
{ते यान्ति तत्-पद-सरोजमखण्ड-रूपम्}


॥इति श्रीमत्परमहंसपरिव्राजकाचार्यस्य श्री-गोविन्द-भगवत्पूज्य-पाद-शिष्यस्य 
श्रीमच्छङ्करभगवतः कृतौ श्री-लक्ष्मी-नृसिंह-करुणारस-स्तोत्रं सम्पूर्णम्‌॥
