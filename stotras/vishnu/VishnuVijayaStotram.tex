% !TeX program = XeLaTeX
% !TeX root = ../../shloka.tex

\sect{विष्णुभुजङ्गप्रयातस्तोत्रम्‌}
\uvacha{देवा ऊचुः}
\addtoshlokacount{88}

\fourlineindentedshloka
{नताः स्म विष्णुं जगदादिभूतं}
{सुरासुरेंद्रं जगतां प्रपालकम्}
{यन्नाभिपद्मात्किल पद्मयोनिर्}
{बभूव तं वै शरणं गताः स्मः} %॥८९॥

\fourlineindentedshloka
{नमोनमो मत्स्यवपुर्द्धराय}
{नमोस्तु ते कच्छपरूपधारिणे}
{नमः प्रकुर्मश्च नृसिंहरूपिणे}
{तथा पुनर्वामनरूपिणे नमः} %॥९०॥

\fourlineindentedshloka
{नमोस्तु ते क्षत्रविनाशनाय}
{रामाय रामाय दशास्यनाशिने}
{प्रलंबहंत्रे शितिवाससे नमो}
{नमोस्तु बुद्धाय च दैत्यमोहिने} %॥९१॥

\fourlineindentedshloka
{म्लेच्छांतकायापि च कल्किनाम्ने}
{नमः पुनः क्रोडवपुर्धराय}
{जगद्धितार्थं च युगेयुगे भवान्}
{बिभर्ति रूपं त्वसुराभवाय} %॥९२॥

\fourlineindentedshloka
{निषूदितोऽयं ह्यधुना किल त्वया}
{दैत्यो हिरण्याक्ष इति प्रगल्भः}
{यश्चेंद्रमुख्यान्किललोकपालान्}
{संहेलया चैव तिरश्चकार} %॥९३॥

\fourlineindentedshloka
{स वै त्वया देवहितार्थमेव}
{निपातितो देववर प्रसीद}
{त्वमस्य विश्वस्य विसर्गकर्ता}
{ब्राह्मेण रूपेण च देवदेव} %॥९४॥

\fourlineindentedshloka
{पाता त्वमेवास्य युगेयुगे च}
{रूपाणि धत्से सुमनोहराणि}
{त्वमेव कालाग्निहरश्च भूत्वा}
{विश्वं क्षयं नेष्यसि चांतकाले} %॥९५॥

\fourlineindentedshloka
{अतो भवानेव च विश्वकारणं}
{न ते परं जीवमजीवमीश}
{यत्किंच भूतं च भविष्यरूपं}
{प्रवर्त्तमानं च तथैव रूपम्} %॥९६॥

\fourlineindentedshloka
{सर्वं त्वमेवासि चराचराख्यं}
{न भाति विश्वं त्वदृते च किंचित्}
{अस्तीति नास्तीति च भेदनिष्ठं}
{त्वय्येव भातं सदसत्स्वरूपम्} %॥९७॥

\fourlineindentedshloka
{ततो भवंतं कतमोपि देव}
{न ज्ञातुमर्हत्यविपक्वबुद्धिः}
{ऋते भवत्पादपरायणं जनं}
{तेनागता स्मः शरणं शरण्यम्} %॥९८॥

\uvacha{व्यास उवाच}

\twolineshloka
{ततो विष्णुः प्रसन्नात्मा उवाच त्रिदिवौकसः}
{तुष्टोस्मि देवा भद्रं वो युष्मत्स्तोत्रेण सांप्रतम्} %॥९९॥

\twolineshloka
{य इदं प्रपठेद्भक्त्या विजयस्तोत्रमादरात्}
{न तस्य दुर्लभं देवास्त्रिषुलोकेषु किंचन} %॥१००॥

\twolineshloka
{गवां शतसहस्रस्य सम्यग्दत्तस्य यत्फलम्}
{तत्फलं समवाप्नोति कीर्तनाच्छ्रवणान्नरः} %॥१०१॥

\twolineshloka
{सर्वकामप्रदं नित्यं देवदेवस्य कीर्तनम्}
{अतः परं महाज्ञानं न भूतं न भविष्यति} %॥१०२॥

॥इति श्रीपाद्मपुराणे प्रथमे सृष्टिखंडे देवासुरसंग्रामसमाप्तौ विजयस्तोत्रंनाम पंचसप्ततितमोऽध्यायः॥ %७५।